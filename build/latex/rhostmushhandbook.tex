%% Generated by Sphinx.
\def\sphinxdocclass{report}
\documentclass[letterpaper,10pt,english]{sphinxmanual}
\ifdefined\pdfpxdimen
   \let\sphinxpxdimen\pdfpxdimen\else\newdimen\sphinxpxdimen
\fi \sphinxpxdimen=.75bp\relax

\PassOptionsToPackage{warn}{textcomp}
\usepackage[utf8]{inputenc}
\ifdefined\DeclareUnicodeCharacter
% support both utf8 and utf8x syntaxes
  \ifdefined\DeclareUnicodeCharacterAsOptional
    \def\sphinxDUC#1{\DeclareUnicodeCharacter{"#1}}
  \else
    \let\sphinxDUC\DeclareUnicodeCharacter
  \fi
  \sphinxDUC{00A0}{\nobreakspace}
  \sphinxDUC{2500}{\sphinxunichar{2500}}
  \sphinxDUC{2502}{\sphinxunichar{2502}}
  \sphinxDUC{2514}{\sphinxunichar{2514}}
  \sphinxDUC{251C}{\sphinxunichar{251C}}
  \sphinxDUC{2572}{\textbackslash}
\fi
\usepackage{cmap}
\usepackage[T1]{fontenc}
\usepackage{amsmath,amssymb,amstext}
\usepackage{babel}



\usepackage{times}
\expandafter\ifx\csname T@LGR\endcsname\relax
\else
% LGR was declared as font encoding
  \substitutefont{LGR}{\rmdefault}{cmr}
  \substitutefont{LGR}{\sfdefault}{cmss}
  \substitutefont{LGR}{\ttdefault}{cmtt}
\fi
\expandafter\ifx\csname T@X2\endcsname\relax
  \expandafter\ifx\csname T@T2A\endcsname\relax
  \else
  % T2A was declared as font encoding
    \substitutefont{T2A}{\rmdefault}{cmr}
    \substitutefont{T2A}{\sfdefault}{cmss}
    \substitutefont{T2A}{\ttdefault}{cmtt}
  \fi
\else
% X2 was declared as font encoding
  \substitutefont{X2}{\rmdefault}{cmr}
  \substitutefont{X2}{\sfdefault}{cmss}
  \substitutefont{X2}{\ttdefault}{cmtt}
\fi


\usepackage[Bjarne]{fncychap}
\usepackage{sphinx}

\fvset{fontsize=\small}
\usepackage{geometry}


% Include hyperref last.
\usepackage{hyperref}
% Fix anchor placement for figures with captions.
\usepackage{hypcap}% it must be loaded after hyperref.
% Set up styles of URL: it should be placed after hyperref.
\urlstyle{same}

\addto\captionsenglish{\renewcommand{\contentsname}{Table of Contents}}

\usepackage{sphinxmessages}
\setcounter{tocdepth}{1}



\title{RhostMUSH Handbook}
\date{Mar 13, 2021}
\release{}
\author{wstarter}
\newcommand{\sphinxlogo}{\vbox{}}
\renewcommand{\releasename}{}
\makeindex
\begin{document}

\pagestyle{empty}
\sphinxmaketitle
\pagestyle{plain}
\sphinxtableofcontents
\pagestyle{normal}
\phantomsection\label{\detokenize{index::doc}}


\sphinxAtStartPar
The RhostMUSH source tree offers many abilities and options
not normally found in any other flavor of mush.  This doesn’t
make it better than other servers (though we think so \sphinxstyleemphasis{j/k})
but it does give you a wider selection of configurability,
which, as you know, is the best part of setting up a mush.
(yea, right)


\chapter{RhostMUSH offers the following over other mushes}
\label{\detokenize{index:rhostmush-offers-the-following-over-other-mushes}}\begin{itemize}
\item {} 
\sphinxAtStartPar
A high\sphinxhyphen{}performance duel\sphinxhyphen{}quota system.

\item {} 
\sphinxAtStartPar
A complete rewrite of key areas and referbishments of all the other areas.

\item {} 
\sphinxAtStartPar
An on\sphinxhyphen{}line recover tool for accidental db destruction.

\item {} 
\sphinxAtStartPar
Multi\sphinxhyphen{}wizard architecture for better control of staff.

\item {} 
\sphinxAtStartPar
Multi\sphinxhyphen{}power system to tweek abilities of players.

\item {} 
\sphinxAtStartPar
Multi\sphinxhyphen{}zone system where people can belong to multiple zones.

\item {} 
\sphinxAtStartPar
Built in error correction for db corruption or other misuse.

\item {} 
\sphinxAtStartPar
Built in accounting for those annoying twinks who try to hack.

\item {} 
\sphinxAtStartPar
Many new functions and commands not seen anywhere else.

\item {} 
\sphinxAtStartPar
A lot more that could drag this document out for pages.

\end{itemize}


\section{Installing RhostMUSH}
\label{\detokenize{install:installing-rhostmush}}\label{\detokenize{install:id1}}\label{\detokenize{install::doc}}
\begin{sphinxShadowBox}
\sphinxstyletopictitle{Table of Contents}
\begin{itemize}
\item {} 
\sphinxAtStartPar
\phantomsection\label{\detokenize{install:id7}}{\hyperref[\detokenize{install:rhostmush-requirements}]{\sphinxcrossref{RhostMUSH Requirements}}}
\begin{itemize}
\item {} 
\sphinxAtStartPar
\phantomsection\label{\detokenize{install:id8}}{\hyperref[\detokenize{install:system-requirements}]{\sphinxcrossref{System Requirements}}}
\begin{itemize}
\item {} 
\sphinxAtStartPar
\phantomsection\label{\detokenize{install:id9}}{\hyperref[\detokenize{install:supported-platforms}]{\sphinxcrossref{Supported Platforms}}}

\item {} 
\sphinxAtStartPar
\phantomsection\label{\detokenize{install:id10}}{\hyperref[\detokenize{install:unsupported-platforms}]{\sphinxcrossref{Unsupported Platforms}}}

\item {} 
\sphinxAtStartPar
\phantomsection\label{\detokenize{install:id11}}{\hyperref[\detokenize{install:untested-platforms}]{\sphinxcrossref{Untested Platforms}}}

\item {} 
\sphinxAtStartPar
\phantomsection\label{\detokenize{install:id12}}{\hyperref[\detokenize{install:known-platform-issues}]{\sphinxcrossref{Known Platform Issues}}}

\end{itemize}

\item {} 
\sphinxAtStartPar
\phantomsection\label{\detokenize{install:id13}}{\hyperref[\detokenize{install:software-requirements}]{\sphinxcrossref{Software Requirements}}}
\begin{itemize}
\item {} 
\sphinxAtStartPar
\phantomsection\label{\detokenize{install:id14}}{\hyperref[\detokenize{install:optional-packages}]{\sphinxcrossref{Optional Packages}}}

\end{itemize}

\item {} 
\sphinxAtStartPar
\phantomsection\label{\detokenize{install:id15}}{\hyperref[\detokenize{install:hosting-requirements}]{\sphinxcrossref{Hosting Requirements}}}

\end{itemize}

\item {} 
\sphinxAtStartPar
\phantomsection\label{\detokenize{install:id16}}{\hyperref[\detokenize{install:obtaining-rhostmush}]{\sphinxcrossref{Obtaining RhostMUSH}}}

\end{itemize}
\end{sphinxShadowBox}


\subsection{RhostMUSH Requirements}
\label{\detokenize{install:rhostmush-requirements}}\label{\detokenize{install:id2}}

\subsubsection{System Requirements}
\label{\detokenize{install:system-requirements}}\label{\detokenize{install:id3}}\begin{itemize}
\item {} 
\sphinxAtStartPar
Minimum 1 GB (memory and swap combined) to compile (functions.c is huge)

\item {} 
\sphinxAtStartPar
Any Unix flavor should be fine.  Linux, BSD, Mac OSX, Solaris, Tru64, AIX, etc.

\item {} 
\sphinxAtStartPar
(BETA ONLY) cygwin under Windows.  It requires the entire base development set and Requirements below.

\item {} 
\sphinxAtStartPar
Disk:  100 MB or more (depending on size of db and how many backups you wish to maintain)

\item {} 
\sphinxAtStartPar
Memory: 12\sphinxhyphen{}100 MB (depending on size of mush and what size buffers you select and packages you include)

\end{itemize}


\paragraph{Supported Platforms}
\label{\detokenize{install:supported-platforms}}\begin{itemize}
\item {} 
\sphinxAtStartPar
SunOS (all platforms)

\item {} 
\sphinxAtStartPar
Solaris (all platforms)

\item {} 
\sphinxAtStartPar
Linux (all platforms except redhat 5.x mentioned below)

\item {} 
\sphinxAtStartPar
AIX (all platforms)

\item {} 
\sphinxAtStartPar
Ultrix (all platforms)

\item {} 
\sphinxAtStartPar
iBSD (all platforms)

\item {} 
\sphinxAtStartPar
FreeBSD (all platforms)

\item {} 
\sphinxAtStartPar
OpenBSD (all platforms)

\item {} 
\sphinxAtStartPar
NetBSD (all platforms)

\item {} 
\sphinxAtStartPar
IRIX (all platforms)

\item {} 
\sphinxAtStartPar
HPUX (32 bit systems only)

\end{itemize}

\sphinxAtStartPar
If it’s not listed here, it probably still will compile clean.


\paragraph{Unsupported Platforms}
\label{\detokenize{install:unsupported-platforms}}\begin{itemize}
\item {} 
\sphinxAtStartPar
Win32/Win16

\item {} 
\sphinxAtStartPar
Alpha systems.

\end{itemize}


\paragraph{Untested Platforms}
\label{\detokenize{install:untested-platforms}}\begin{itemize}
\item {} 
\sphinxAtStartPar
HP\sphinxhyphen{}UX (64 bit systems)

\item {} 
\sphinxAtStartPar
VMS (all platforms)

\end{itemize}


\paragraph{Known Platform Issues}
\label{\detokenize{install:known-platform-issues}}
\sphinxAtStartPar
On Alpha boxes running Redhat 5.0, structure pointers are slaughtered with
the built\sphinxhyphen{}in gcc package (up to and including 2.8.x).  Because of this some
config options may not work fully or cause the server to crash.  This is only
a known issue with config options and only on this platform.


\subsubsection{Software Requirements}
\label{\detokenize{install:software-requirements}}\label{\detokenize{install:id4}}
\sphinxAtStartPar
RhostMUSH is a Linux or Unix based server software that runs as a daemon on the host.
In order to build this software, you will need the bare minimum of the ability to run ‘make’ commands.

\sphinxAtStartPar
Package requirements are as follows:
\begin{itemize}
\item {} 
\sphinxAtStartPar
glibc and gcc/clang (compiling the code)

\item {} 
\sphinxAtStartPar
git (to clone the source and maintain patches)

\item {} 
\sphinxAtStartPar
bash/ksh/dash (or compatible shell \sphinxhyphen{} for use with build menu)

\item {} 
\sphinxAtStartPar
libcrypt (for password encryption \sphinxhyphen{} this is usually standard on unix based systems)

\end{itemize}


\paragraph{Optional Packages}
\label{\detokenize{install:optional-packages}}
\sphinxAtStartPar
RhostMUSH also offers optional linking and library attachments.
For some of these libraries it will attempt to do auto\sphinxhyphen{}detection,
but in a worse case scenario, there exists override hashes in the menu to disable options it thinks exist that do not.

\sphinxAtStartPar
Optional packages are as follows:
\begin{itemize}
\item {} 
\sphinxAtStartPar
openssl dev libraries/headers (for MUX password compatibility, and digest() and advanced cryptology functionality.

\item {} 
\sphinxAtStartPar
mysql client \& mysql\_config (required for mysql capabilities)

\item {} 
\sphinxAtStartPar
sqlite3 libraries (required for sqlite capabilities)

\item {} 
\sphinxAtStartPar
ruby/perl/python/etc (for custom interactive dynamic custom functions with the execscript() feature)

\item {} 
\sphinxAtStartPar
libpcre (if you wish to use system pcre libraries instead of the built\sphinxhyphen{}in ones)

\end{itemize}


\subsubsection{Hosting Requirements}
\label{\detokenize{install:hosting-requirements}}\label{\detokenize{install:id5}}
\sphinxAtStartPar
You will need a stable host and access to open a single port number of your choice on the firewall.
Most games choose a number between 1025 and 9999, by convention.
Please make sure your debug\_id matches the port number + 5.
So if your port is 1234, your debug\_id will be 12345.
The debug\_id is for use in the API daemon that runs Rhost as a container to keep track of heap, stack, and execution location.


\subsection{Obtaining RhostMUSH}
\label{\detokenize{install:obtaining-rhostmush}}\label{\detokenize{install:id6}}
\sphinxAtStartPar
It is assumed that you have gotten to this point with the following command:

\begin{sphinxVerbatim}[commandchars=\\\{\}]
\PYG{n}{git} \PYG{n}{clone} \PYG{n}{https}\PYG{p}{:}\PYG{o}{/}\PYG{o}{/}\PYG{n}{github}\PYG{o}{.}\PYG{n}{com}\PYG{o}{/}\PYG{n}{RhostMUSH}\PYG{o}{/}\PYG{n}{trunk} \PYG{n}{Rhost}
\end{sphinxVerbatim}

\sphinxAtStartPar
If you did NOT get it this way, your file permissions may not be properly set up.  Please type:

\begin{sphinxVerbatim}[commandchars=\\\{\}]
\PYG{n}{chmod} \PYG{o}{+}\PYG{n}{rx} \PYG{n+nb}{bin}\PYG{o}{/}\PYG{o}{*}\PYG{o}{.}\PYG{n}{sh} \PYG{n}{src}\PYG{o}{/}\PYG{o}{*}\PYG{o}{.}\PYG{n}{sh} \PYG{n}{game}\PYG{o}{/}\PYG{o}{*}\PYG{o}{.}\PYG{n}{sh} \PYG{n}{game}\PYG{o}{/}\PYG{n}{Startmush} \PYG{n}{game}\PYG{o}{/}\PYG{n}{db\PYGZus{}}\PYG{o}{*}
\end{sphinxVerbatim}

\sphinxAtStartPar
This makes sure all the build scripts are properly made executable.
This will result in ‘permission denied’ or similar results when running a script.


\section{What RhostMUSH is about and what’s so great about RhostMUSH}
\label{\detokenize{features:what-rhostmush-is-about-and-what-s-so-great-about-rhostmush}}\label{\detokenize{features::doc}}
\sphinxAtStartPar
RhostMUSH was founded in 1989, originally by Natasha Davis (Nyctasia) and as
a branch from the original release of TinyMUD code.  It was her desire to make
a game that was flexible, with multiple levels of progression and highly
customizeable.  She lost time and interest and passed the game to
Steve Shivers (Seawolf), Mike McDermott (Thorin), and Jace Hoppel (Ashen\sphinxhyphen{}Shugar)

\sphinxAtStartPar
Through their work, the stability improved, we fixed it to be multi\sphinxhyphen{}platform
and as bug free as we could possibly make it.  We introduced several methods both
in game and in source that allowed consistent memory bounds checking and
various alerts for any mischievous naughtyness in\sphinxhyphen{}game or possibilities of any
hacks, patches, or alterations in the code causing leaks or issues.

\sphinxAtStartPar
While not perfect, it has allowed us to have an absolutely outstanding
turn around for any bugs sent our way, which anyone who uses RhostMUSH will
attest to.

\sphinxAtStartPar
Over the years, others have joined the RhostMUSH team, including Ambrosia
(who is the current dev lead), Lensman, Kage (who kindly provided the
UTF8/unicode port), Jeff/Loki, Rook, Noltar, and Odin.

\sphinxAtStartPar
We also have had hundreds of people who have offered (and provided) help,
patches, suggestions, bug fixes, and alternations all on their own and
every single one will have had their name mentioned in the RHOST.CHANGES
file in the readme directory.  It’s far too large to have in the online
help files.

\sphinxAtStartPar
RhostMUSH today provides an amazing tool that allows nearly entire
customization in\sphinxhyphen{}game of every single feature available in Rhost without
having the requirement to modify the hardcode.  This includes but is
not limited to:


\subsection{Recycle bin}
\label{\detokenize{features:recycle-bin}}\begin{quote}

\sphinxAtStartPar
Yup, you guessed it.  RhostMUSH has a windows like recycle bin.
This means the objects you @nuke and @destroy become ‘destroyed’
but not recycled until they are @purged.  If you use the Myrddin
CRON in the Mushcode directory, it by default sets up a job
to purge anything over 30 days old, which should be more than
sufficient for any needs.  The goodness of this?  You can recover
nuked things from any period of time, as long as they were not
@purged first.

\sphinxAtStartPar
Commands: @purge, @nuke, @destroy, @recover, @reclist
\end{quote}


\subsection{@snapshot}
\label{\detokenize{features:snapshot}}\begin{quote}

\sphinxAtStartPar
Live image snapshots to unload or load to and from
disk.  As many snapshots as you want, as often as you want.
It essentially does a flatfile dump of a dbref\#.  Great for
backups or cross\sphinxhyphen{}Rhost portability.

\sphinxAtStartPar
Command: @snapshot
\end{quote}


\subsection{Wizard and Immortals by default}
\label{\detokenize{features:wizard-and-immortals-by-default}}\begin{itemize}
\item {} 
\sphinxAtStartPar
are spoofable.  Meaning all their @pemits by default will not
trigger NOSPOOF.  If you do not wish this, set the SPOOF flag
this applies to anyone below their level.

\item {} 
\sphinxAtStartPar
override all locks.  There’s two flags to disable this.
NO\_OVERRIDE to stop overriding all locks (including attribs)
and NO\_USELOCK to just stop overriding uselocks.
This applies to anything their level and lower.

\item {} 
\sphinxAtStartPar
optionally cloak from all non\sphinxhyphen{}immortals/God player.
This can be highly abused if not careful and there
is a @depower to disable cloaking and/or dark that will
disable this.

\item {} 
\sphinxAtStartPar
immortals can optionally supercloak from even wizards.
this can not be disabled, and you must consider that immortals
should be treated as the God player (\#1) since they are
effectively \#1 in nearly every way.

\end{itemize}


\subsection{Titles and Captions to a player’s name}
\label{\detokenize{features:titles-and-captions-to-a-player-s-name}}\begin{quote}

\sphinxAtStartPar
@caption and @titlecaption
\end{quote}


\subsection{Have an alternate name with locks for NPC obfuscation}
\label{\detokenize{features:have-an-alternate-name-with-locks-for-npc-obfuscation}}\begin{quote}

\sphinxAtStartPar
@altname
@lock/altname
\end{quote}


\subsection{Have multiple player aliases}
\label{\detokenize{features:have-multiple-player-aliases}}\begin{quote}

\sphinxAtStartPar
As well as a method to reserve player names per player w/o revealing who has what name.

\sphinxAtStartPar
@protect
\end{quote}


\subsection{Actively control how dark works both game\sphinxhyphen{}wide and individually}
\label{\detokenize{features:actively-control-how-dark-works-both-game-wide-and-individually}}\begin{quote}

\sphinxAtStartPar
@depower dark

\sphinxAtStartPar
@admin allow\_whodark, sweep\_dark, command\_dark, lcon\_checks\_dark,
secure\_dark, see\_owned\_dark, idle\_wiz\_dark, player\_dark

\sphinxAtStartPar
@toggle snuffdark

\sphinxAtStartPar
@flagdef to redefine who and what can set the DARK flag
\end{quote}


\subsection{Make config file changes in\sphinxhyphen{}game without having to reboot or have shell access}
\label{\detokenize{features:make-config-file-changes-in-game-without-having-to-reboot-or-have-shell-access}}\begin{quote}

\sphinxAtStartPar
@admin admin\_object
\end{quote}


\subsection{Execute any binary or script as a localized function}
\label{\detokenize{features:execute-any-binary-or-script-as-a-localized-function}}\begin{quote}

\sphinxAtStartPar
EXECSCRIPT (power), SIDEFX (flag)
\end{quote}


\subsection{Customized percent substitutions (like \%n, \%\#, etc)}
\label{\detokenize{features:customized-percent-substitutions-like-n-etc}}\begin{quote}

\sphinxAtStartPar
@admin sub\_include, @hook
\end{quote}


\subsection{Redefine percent substitutions (like \%n, \%\#, etc)}
\label{\detokenize{features:redefine-percent-substitutions-like-n-etc}}\begin{quote}

\sphinxAtStartPar
@admin sub\_override, @hook
\end{quote}


\subsection{Localize command and function overrides in a sandbox}
\label{\detokenize{features:localize-command-and-function-overrides-in-a-sandbox}}\begin{quote}

\sphinxAtStartPar
@icmd, @lfunction, subeval(), sandbox()
\end{quote}


\subsection{Multiple Zones}
\label{\detokenize{features:multiple-zones}}\begin{quote}

\sphinxAtStartPar
Have multiple zones which can optionally belong to multiple targets (multiple zones per target allowable!)

\sphinxAtStartPar
@zone, zones, lzone(), zonecmd()
\end{quote}


\subsection{Optionally control, enable, or disable sideeffects}
\label{\detokenize{features:optionally-control-enable-or-disable-sideeffects}}\begin{quote}

\sphinxAtStartPar
@admin sideeffects, SIDEFX (flag)
\end{quote}


\subsection{Have 31 cross\sphinxhyphen{}interactive realities for locations}
\label{\detokenize{features:have-31-cross-interactive-realities-for-locations}}\begin{quote}

\sphinxAtStartPar
This works as a truly independant and self\sphinxhyphen{}contained environment.
A room can have 31 ‘layers’, each ‘layer’ is a reality in
the same physical space.  These layers can work independently
or allow interaction with other layers for vast customization.
This affects all methods within the game including all matching,
looking, \$commands, listens, movement, interaction, pretty
much every single aspect of mushing.

\sphinxAtStartPar
REALITY LEVELS
\end{quote}


\subsection{Override any command with softcode}
\label{\detokenize{features:override-any-command-with-softcode}}\begin{quote}

\sphinxAtStartPar
@admin access (check ignore)

\sphinxAtStartPar
Master room \$commands to then override the hardcode
\end{quote}


\subsection{The abilility to raise or lower permissions on the various}
\label{\detokenize{features:the-abilility-to-raise-or-lower-permissions-on-the-various}}\begin{quote}

\sphinxAtStartPar
staff bitlevels for each player.

\sphinxAtStartPar
@power, @depower, TOGGLES, FLAGS
\end{quote}


\subsection{Customize new commands on the connect screen}
\label{\detokenize{features:customize-new-commands-on-the-connect-screen}}\begin{quote}

\sphinxAtStartPar
@admin file\_object2
\end{quote}


\subsection{Softcode any txt file (like connect.txt)}
\label{\detokenize{features:softcode-any-txt-file-like-connect-txt}}\begin{quote}

\sphinxAtStartPar
and have it evaluate in\sphinxhyphen{}game.  It evaluates as the object it is on.

\sphinxAtStartPar
@admin file\_object
\end{quote}


\subsection{Advanced tracing methods for debugging your code including labels!}
\label{\detokenize{features:advanced-tracing-methods-for-debugging-your-code-including-labels}}\begin{quote}

\sphinxAtStartPar
Commands: @label

\sphinxAtStartPar
Functions: parenmatch(), trace()

\sphinxAtStartPar
Toggles: CPUTIME

\sphinxAtStartPar
Flags: TRACE

\sphinxAtStartPar
Attributes: TRACE\_GREP, TRACE, TRACE\_COLOR, TRACE\_COLOR\_\textless{}attr\textgreater{}

\sphinxAtStartPar
Substitutions: \%\_
\end{quote}


\subsection{Built in pretty\sphinxhyphen{}printing of attributes with the parenmatch() function}
\label{\detokenize{features:built-in-pretty-printing-of-attributes-with-the-parenmatch-function}}
\sphinxAtStartPar
Example Code Output:

\begin{sphinxVerbatim}[commandchars=\\\{\}]
\PYG{n+nd}{@emit} \PYG{p}{[}\PYG{n}{add}\PYG{p}{(}\PYG{l+m+mi}{1}\PYG{p}{,}\PYG{n}{sub}\PYG{p}{(}\PYG{l+m+mi}{2}\PYG{p}{,}\PYG{n}{div}\PYG{p}{(}\PYG{l+m+mi}{3}\PYG{p}{,}\PYG{l+m+mi}{4}\PYG{p}{)}\PYG{p}{)}\PYG{p}{,}\PYG{l+m+mi}{5}\PYG{p}{)}\PYG{p}{]}\PYG{p}{;}\PYG{n+nd}{@emit} \PYG{p}{[}\PYG{n}{extract}\PYG{p}{(}\PYG{n}{get}\PYG{p}{(}\PYG{n}{me}\PYG{o}{/}\PYG{n}{foo}\PYG{p}{)}\PYG{p}{,}\PYG{l+m+mi}{3}\PYG{p}{,}\PYG{l+m+mi}{1}\PYG{p}{)}

\PYG{n}{Example} \PYG{n}{Pretty} \PYG{n}{Print}\PYG{p}{:}
\PYG{n+nd}{@emit} \PYG{p}{[}
   \PYG{n}{add}\PYG{p}{(}
      \PYG{l+m+mi}{1}\PYG{p}{,}\PYG{n}{sub}\PYG{p}{(}
         \PYG{l+m+mi}{2}\PYG{p}{,}\PYG{n}{div}\PYG{p}{(}
            \PYG{l+m+mi}{3}\PYG{p}{,}\PYG{l+m+mi}{4}
         \PYG{p}{)}
      \PYG{p}{)}\PYG{p}{,}\PYG{l+m+mi}{5}
   \PYG{p}{)}
\PYG{p}{]}\PYG{p}{;}\PYG{n+nd}{@emit} \PYG{p}{[}
   \PYG{n}{extract}\PYG{p}{(}
      \PYG{n}{get}\PYG{p}{(}
         \PYG{n}{me}\PYG{o}{/}\PYG{n}{foo}
      \PYG{p}{)}\PYG{p}{,}\PYG{l+m+mi}{3}\PYG{p}{,}\PYG{l+m+mi}{1}
   \PYG{p}{)}
\PYG{p}{]}
\end{sphinxVerbatim}


\subsection{Plenty more not mentioned!}
\label{\detokenize{features:plenty-more-not-mentioned}}
\sphinxAtStartPar
The flexibility to customize RhostMUSH is what is most daunting.
Don’t fret, you don’t need to do it to run RhostMUSH successfully.
In fact, the default configuration is mostly compatible with
MUSH and will work correctly out of the box for most needs.  For those
wishing to play, of course the sky is the limit of what you want to
do.


\subsection{Advanced features of RhostMUSH}
\label{\detokenize{features:advanced-features-of-rhostmush}}

\subsubsection{Debugging/Tracing}
\label{\detokenize{features:debugging-tracing}}\begin{itemize}
\item {} 
\sphinxAtStartPar
Debugging in Rhost allows for advanced features like expressing where and
when to do debugging via a trace() function, with toggled labels, and the
ability to grep content from trace output.  There also exists features to
color\sphinxhyphen{}match parenthesis, braces, and brackets in\sphinxhyphen{}game as well as pretty print
the output of commands and functions.
\begin{itemize}
\item {} 
\sphinxAtStartPar
help trace

\item {} 
\sphinxAtStartPar
help \%\_

\item {} 
\sphinxAtStartPar
help trace()

\item {} 
\sphinxAtStartPar
help parenmatch()

\item {} 
\sphinxAtStartPar
help parenstr()

\end{itemize}

\end{itemize}


\subsubsection{Zoning}
\label{\detokenize{features:zoning}}\begin{itemize}
\item {} 
\sphinxAtStartPar
Zoning in Rhost allows the same functionality of Penn and MUX, though the
syntax is different.  It also allows the ability to belong to multiple
zones at the same time and take advantage of mulitple zones at once.
This allows for increased levels of complexity.
\begin{itemize}
\item {} 
\sphinxAtStartPar
help zones

\item {} 
\sphinxAtStartPar
help @zone

\item {} 
\sphinxAtStartPar
help zonecmd()

\item {} 
\sphinxAtStartPar
help lzone()

\item {} 
\sphinxAtStartPar
help @Lock type twink

\item {} 
\sphinxAtStartPar
help @lock type zone

\end{itemize}

\end{itemize}


\subsubsection{Reality Levels}
\label{\detokenize{features:reality-levels}}\begin{itemize}
\item {} 
\sphinxAtStartPar
Reality levels allows for the ability to have a sandboxed ‘existance’
in each physical location across the entirity of the mush.  Each
reality is its own sandbox and can either stand alone or work
dependently with other realities.  A person can belong to multiple
realities at the same time, and realities is geared to a method for
send and receive.  Each ‘method’ requires to be in the given reality
to affect it.
\begin{itemize}
\item {} 
\sphinxAtStartPar
help reality levels

\item {} 
\sphinxAtStartPar
wizhelp chkreality

\item {} 
\sphinxAtStartPar
wizhelp reaity level

\item {} 
\sphinxAtStartPar
help @Lock type user

\end{itemize}

\end{itemize}


\subsubsection{Function and Command Overriding}
\label{\detokenize{features:function-and-command-overriding}}\begin{itemize}
\item {} 
\sphinxAtStartPar
Functions and commands can both be overridden with softcode.  To
override a hardcoded command you first set the command ignore.
There are various levels of ignoring so that you could have it
ignored from mortals but have it executed fine for non\sphinxhyphen{}mortals.
This allows you to use the actual physical command within a
softcode override.  You may also use @Hook for altering how
a command works.   Functions are overridden by setting the
function in question ignored, then writing a softcode alternative
that is then executed and fetched appropriately.

\sphinxAtStartPar
Commands:
\begin{itemize}
\item {} 
\sphinxAtStartPar
wizhelp @admin

\item {} 
\sphinxAtStartPar
wizhelp access

\item {} 
\sphinxAtStartPar
wizhelp permissions

\item {} 
\sphinxAtStartPar
wizhelp @Hook

\item {} 
\sphinxAtStartPar
wizhelp hook setup

\end{itemize}

\sphinxAtStartPar
Functions:
\begin{itemize}
\item {} 
\sphinxAtStartPar
wizhelp @admin

\item {} 
\sphinxAtStartPar
wizhelp function\_access

\item {} 
\sphinxAtStartPar
wizhelp @function

\item {} 
\sphinxAtStartPar
help @lfunction

\item {} 
\sphinxAtStartPar
wizhelp bypass()

\end{itemize}

\end{itemize}


\subsubsection{The Recycle Bin}
\label{\detokenize{features:the-recycle-bin}}\begin{itemize}
\item {} 
\sphinxAtStartPar
Rhost has a recycle bin that works a bit like a windows recycle bin.
Whenever you destroy something within the mush, it is stacked onto
the recycle bin and marked unavailable within the mush.  This marks
the dbref\# as garbage in any sense of the word.  However, the object
is not able to be reused until purged.  Once purged, it is put onto
a free list that can then be reassigned to a new object.
\begin{itemize}
\item {} 
\sphinxAtStartPar
wizhelp @nuke

\item {} 
\sphinxAtStartPar
wizhelp @destroy

\item {} 
\sphinxAtStartPar
wizhelp @toad

\item {} 
\sphinxAtStartPar
wizhelp @turtle

\item {} 
\sphinxAtStartPar
wizhelp @purge

\item {} 
\sphinxAtStartPar
wizhelp @recover

\item {} 
\sphinxAtStartPar
wizhelp @reclist

\end{itemize}

\end{itemize}


\subsubsection{Percent Substitution Adding/Overriding}
\label{\detokenize{features:percent-substitution-adding-overriding}}\begin{itemize}
\item {} 
\sphinxAtStartPar
Rhost allows the ability to both override percent substitutions as
well as creating new ones.  This is done with @Hook and admin
params and issues softcode overriding.  Due to how it is evaluated
there is no risk of recursion.
\begin{itemize}
\item {} 
\sphinxAtStartPar
wizhelp @hook

\item {} 
\sphinxAtStartPar
wizhelp hook\_cmd

\item {} 
\sphinxAtStartPar
wizhelp sub\_include

\item {} 
\sphinxAtStartPar
wizhelp sub\_override

\end{itemize}

\end{itemize}


\subsubsection{Hooking}
\label{\detokenize{features:hooking}}\begin{itemize}
\item {} 
\sphinxAtStartPar
Hooking allows you to have advanced methods to manipulate commands
including adding customized switches to them via softcode.
\begin{itemize}
\item {} 
\sphinxAtStartPar
wizhelp @hook

\item {} 
\sphinxAtStartPar
wizhelp hook\_cmd

\item {} 
\sphinxAtStartPar
wizhelp hook\_obj

\item {} 
\sphinxAtStartPar
wizhelp hook setup

\end{itemize}

\end{itemize}


\subsubsection{Command based uselocks}
\label{\detokenize{features:command-based-uselocks}}\begin{itemize}
\item {} 
\sphinxAtStartPar
This allows you to have unique uselocks per \$command.  This is done
through the use of the USELOCK attribute flag, then you set up
a matching attribute name with a prefix of a \textasciitilde{} to specify how
the lock is to be evaluated.  This works in the same manner
as an evaluation lock.  To be able to use the USELOCK attribute flag
you must be empowered to do so with the ‘ATRUSE’ @toggle.  You may
also use the secure\_atruselock config parameter to globally enable
this and not require the toggle to be set.
\begin{itemize}
\item {} 
\sphinxAtStartPar
wizhelp atruse toggle

\item {} 
\sphinxAtStartPar
help attribute uselocks

\end{itemize}

\end{itemize}


\subsubsection{Differentating between command and listen locks}
\label{\detokenize{features:differentating-between-command-and-listen-locks}}\begin{itemize}
\item {} 
\sphinxAtStartPar
We distinguish between commands and listens with uselocks by passing
an optional argument to all locks that are uselocks.  This optional
argument is 0 for a default lock, 1 for a command lock and 2 for
a listen lock.
\begin{itemize}
\item {} 
\sphinxAtStartPar
help @lock type uselock

\end{itemize}

\end{itemize}


\subsubsection{Wizard auto\sphinxhyphen{}overiding and how to disable it}
\label{\detokenize{features:wizard-auto-overiding-and-how-to-disable-it}}\begin{itemize}
\item {} 
\sphinxAtStartPar
By default wizards override all locks, including attribute locks,
can see all dark exits, and bypass pagelocks.  This can be
troublesome, and even abusive, so there’s ways to disable this.
\begin{itemize}
\item {} 
\sphinxAtStartPar
wizhelp @depower (for those abusing it)

\item {} 
\sphinxAtStartPar
wizhelp no\_override (disable overiding locks)

\item {} 
\sphinxAtStartPar
wizhelp no\_uselock (disable just uselock overriding)

\item {} 
\sphinxAtStartPar
wizhelp pagelock toggle (disable pagelock overriding)

\end{itemize}

\end{itemize}


\subsubsection{Advanced FLAG and TOGGLE control}
\label{\detokenize{features:advanced-flag-and-toggle-control}}\begin{itemize}
\item {} 
\sphinxAtStartPar
Flags and toggles can be controlled to have multiple permissions
and enable/disable targets of how the flags are allowed to be
set.  This is done through commands in\sphinxhyphen{}game or you can use
conf file options to do so.
\begin{itemize}
\item {} 
\sphinxAtStartPar
wizhelp @flagdef

\item {} 
\sphinxAtStartPar
wizhelp @toggledef

\item {} 
\sphinxAtStartPar
@admin @flagdef alternatives
\begin{itemize}
\item {} 
\sphinxAtStartPar
wizhelp flag\_access\_set

\item {} 
\sphinxAtStartPar
wizhelp flag\_access\_unset

\item {} 
\sphinxAtStartPar
wizhelp flag\_access\_see

\item {} 
\sphinxAtStartPar
wizhelp flag\_access\_type

\end{itemize}

\item {} 
\sphinxAtStartPar
@admin @toggledef alternatives
\begin{itemize}
\item {} 
\sphinxAtStartPar
wizhelp toggle\_access\_set

\item {} 
\sphinxAtStartPar
wizhelp toggle\_access\_unset

\item {} 
\sphinxAtStartPar
wizhelp toggle\_access\_see

\item {} 
\sphinxAtStartPar
wizhelp toggle\_access\_type

\end{itemize}

\end{itemize}

\end{itemize}


\subsubsection{Advanced site control}
\label{\detokenize{features:advanced-site-control}}\begin{itemize}
\item {} 
\sphinxAtStartPar
We allow advanced site control by not only blocking various sites
but we can specify how many times a player can be connected at the
same time as well as how many times sites are able to connect at
the same time.  This is done through normal site manipulation.
\begin{itemize}
\item {} 
\sphinxAtStartPar
wizhelp forbid\_host

\item {} 
\sphinxAtStartPar
wizhelp register\_host

\item {} 
\sphinxAtStartPar
wizhelp noguest\_host

\item {} 
\sphinxAtStartPar
wizhelp @list (site option)

\end{itemize}

\end{itemize}


\subsubsection{Auto\sphinxhyphen{}Registration}
\label{\detokenize{features:auto-registration}}\begin{itemize}
\item {} 
\sphinxAtStartPar
Autoregistration is the method that a player can auto\sphinxhyphen{}register
by providing their email on the connect screen.  It will email
them a password and an optional document that the administrator
provides.  This is well described in the wizhelp.
\begin{itemize}
\item {} 
\sphinxAtStartPar
wizhelp autoregistration

\end{itemize}

\end{itemize}


\subsubsection{Which bit level is best?}
\label{\detokenize{features:which-bit-level-is-best}}\begin{itemize}
\item {} 
\sphinxAtStartPar
This is something that should be discussed by you and the staff
of your game.  As a good rule of thumb, only provide the bitlevel
that is required to do the job.  Too much power is always risky.
In essence, each tier of bit can do everything the previous bitlevel
can do, and then additional stuff on top of it.  The highest bitlevel
is \#1 itself, being bitlevel 7.  Then immortal, which should be
considered the \#1 character in most cases and is bitlevel 6.  Then
the royalty character, which is equal to wizard on penn, mux, or
other codebases.  For most things, this is the bitlevel you want
to assign players.  The exception will likely be game owners or
people who control the master room code.

\sphinxAtStartPar
There’s a bunch of readme files and online wizhelp that goes into
detail of the various bitlevels and what each can do.
\begin{itemize}
\item {} 
\sphinxAtStartPar
wizhelp control

\end{itemize}

\end{itemize}


\subsubsection{What are the limits for size/growth for RhostMUSH?}
\label{\detokenize{features:what-are-the-limits-for-size-growth-for-rhostmush}}\begin{quote}
\begin{itemize}
\item {} 
\sphinxAtStartPar
While using QDBM, there’s really no set limits for most things.
The limits that we have are as followed:

\end{itemize}
\begin{quote}

\sphinxAtStartPar
LBUF \sphinxhyphen{} 64K.  It is recommended to only use 32K as there is some issues with networking with 64K lbufs.
\begin{quote}
\begin{itemize}
\item {} 
\sphinxAtStartPar
Compile time option with the menu configurator

\end{itemize}

\sphinxAtStartPar
SBUF \sphinxhyphen{} 64 characters (if configured \textendash{} it’s suggested you do).
\begin{itemize}
\item {} 
\sphinxAtStartPar
Compile time option with the menu configurator

\end{itemize}

\sphinxAtStartPar
MBUF \sphinxhyphen{} 200 characters.  Not able to be changed.

\sphinxAtStartPar
MAX CONNECTIONS \sphinxhyphen{} Limited by the total number of open sockets and descriptors on the account and server running on.  There are various tools to limit connection DoS attemps and other such nastiness.  This is well documented in the netrhost.conf file.
\begin{itemize}
\item {} 
\sphinxAtStartPar
wizhelp max\_players

\item {} 
\sphinxAtStartPar
\textasciitilde{}/game/netrhost.conf

\end{itemize}

\sphinxAtStartPar
MEMORY \sphinxhyphen{} no limit.  Generally runs between 8\sphinxhyphen{}50M depending on the size of the mush and the LBUF size specified.

\sphinxAtStartPar
CPU  \sphinxhyphen{} no limit, but has built in cpu abort in code.  The netrhost.conf file documents this well for customizing.  the default values are usually good enough.
\begin{itemize}
\item {} 
\sphinxAtStartPar
wizhelp max\_cpu\_cycles

\item {} 
\sphinxAtStartPar
wizhelp cpuintervalchk

\item {} 
\sphinxAtStartPar
wizhelp cputimechk

\item {} 
\sphinxAtStartPar
wizhelp cpu\_secure\_lvl

\item {} 
\sphinxAtStartPar
wizhelp heavy\_cpu\_max

\item {} 
\sphinxAtStartPar
\textasciitilde{}/game/netrhost.conf

\end{itemize}

\sphinxAtStartPar
DISK \sphinxhyphen{} no limit.  Generally will be 75\sphinxhyphen{}200M depending on size, number of backups and if you leave your compiled object files in.

\sphinxAtStartPar
DB Size \sphinxhyphen{} (20000 default) There is no limit on the number of objects the db can have.  By default it’s soft limited to 20K objects, which can be changed by a netrhost.conf file change.  We have had this up past 1.5 million objects, and other than a second or two of lag for complex searches we had no real problem.
\begin{itemize}
\item {} 
\sphinxAtStartPar
wizhelp maximum\_size

\item {} 
\sphinxAtStartPar
help @quota

\item {} 
\sphinxAtStartPar
wizhelp @quota

\item {} 
\sphinxAtStartPar
wizhelp @limit

\end{itemize}

\sphinxAtStartPar
Attribute Size \sphinxhyphen{} 10K as a hard limit.  750 as a soft limit.  You can increase this but it can’t exceed 10000 attributes.  This is to avoid DoS style attacks.
\begin{itemize}
\item {} 
\sphinxAtStartPar
wizhelp vlimit

\item {} 
\sphinxAtStartPar
wizhelp @limit

\end{itemize}
\end{quote}
\end{quote}
\end{quote}


\subsubsection{Sqlite and MySQL/Maria setup and why use it?}
\label{\detokenize{features:sqlite-and-mysql-maria-setup-and-why-use-it}}\begin{itemize}
\item {} 
\sphinxAtStartPar
Both of these can be configured separately or conjointly to
run in parallel.  This can be done through the RhostMUSH
configuration utility.  You generally want to use SQL for
external data storage or accessing a central repository
of data to share between multiple projects.  Like, for
example between a wiki, a forum, and the mush.

\end{itemize}


\subsubsection{Executing outside scripts and binaries within RhostMUSH}
\label{\detokenize{features:executing-outside-scripts-and-binaries-within-rhostmush}}\begin{itemize}
\item {} 
\sphinxAtStartPar
Rhost has execscript() which allows executing outside binaries
or scripts as a native function.  All effort has been done to
avoid any type of DoS based issue or hang by doing this, however
the guidelines presented should be followed before doing so.
\begin{itemize}
\item {} 
\sphinxAtStartPar
wizhelp execscript

\item {} 
\sphinxAtStartPar
wizhelp power execscript

\item {} 
\sphinxAtStartPar
help sidefx

\item {} 
\sphinxAtStartPar
wizhelp writing scripts

\end{itemize}

\end{itemize}


\subsubsection{Pulling external data into RhostMUSH}
\label{\detokenize{features:pulling-external-data-into-rhostmush}}\begin{itemize}
\item {} 
\sphinxAtStartPar
You are capable of pulling external data in to RhostMUSH using several methods.  These are by using:
\begin{itemize}
\item {} 
\sphinxAtStartPar
SQL (mysql or sqlite)

\item {} 
\sphinxAtStartPar
execscript

\item {} 
\sphinxAtStartPar
cron (wizhelp signal)

\end{itemize}

\end{itemize}


\subsubsection{Integrating a unix cron right into RhostMUSH}
\label{\detokenize{features:integrating-a-unix-cron-right-into-rhostmush}}\begin{itemize}
\item {} 
\sphinxAtStartPar
The unix cron can be used to integrate with Rhost fairly
easilly by use of signals.  By using SIGUSR1 you can specify
Rhost to execute code in\sphinxhyphen{}game, which part of that could be
to pick up a pre\sphinxhyphen{}designed list of commands that the unix cron
has set up.
\begin{itemize}
\item {} 
\sphinxAtStartPar
wizhelp signal

\item {} 
\sphinxAtStartPar
wizhelp signal\_cron

\end{itemize}

\end{itemize}


\subsubsection{Signal handling, how it works, and when and why use it}
\label{\detokenize{features:signal-handling-how-it-works-and-when-and-why-use-it}}\begin{itemize}
\item {} 
\sphinxAtStartPar
Signals are used to do different things for the mush.  By default,
the following signals are recognized by the mush and will do
the following as defaults.
\begin{itemize}
\item {} 
\sphinxAtStartPar
SIGUSR1 \sphinxhyphen{} will do a reboot of the mush.  This is also customizable so that you can have it execute code in\sphinxhyphen{}mush if you want.

\item {} 
\sphinxAtStartPar
SIGUSR2 \sphinxhyphen{} will do a clean shutdown of the mush.

\item {} 
\sphinxAtStartPar
SIGTERM \sphinxhyphen{} will immediately scram the mush as cleanly and fast as possible.  It will avoid dumping anything to the database to speed up scramming, but write a TERM flat file to be loaded in if corruption.

\end{itemize}

\end{itemize}


\subsubsection{Setting up global parents, global @parents, global attribute formatting, and other global setups}
\label{\detokenize{features:setting-up-global-parents-global-parents-global-attribute-formatting-and-other-global-setups}}\begin{itemize}
\item {} 
\sphinxAtStartPar
Global parents are useful when you want to have a global ‘parent’
without actually having a defined \sphinxhref{mailto:'@parent}{‘@parent}’.  It always will be the
highest tier in a lookup.  The way lookups will go will be:

\begin{sphinxVerbatim}[commandchars=\\\{\}]
\PYG{n+nb+bp}{self}\PYG{o}{\PYGZhy{}}\PYG{o}{\PYGZgt{}}\PYG{n+nd}{@parent}\PYG{p}{(}\PYG{n}{s}\PYG{p}{)}\PYG{o}{\PYGZhy{}}\PYG{o}{\PYGZgt{}}\PYG{n+nd}{@zone}\PYG{p}{(}\PYG{n}{s}\PYG{p}{)}\PYG{o}{\PYGZhy{}}\PYG{o}{\PYGZgt{}}\PYG{n}{GlobalParent}
\end{sphinxVerbatim}

\sphinxAtStartPar
The type of the parent does not have to match the target.

\sphinxAtStartPar
These global parents can be defined either by using a global
generic parent or by using the type.  If a type is specified it
overrides the generic.  The following parameters are used:
\begin{itemize}
\item {} 
\sphinxAtStartPar
global\_parent\_obj     \sphinxhyphen{} The generic global parent (if defined)

\item {} 
\sphinxAtStartPar
global\_parent\_room    \sphinxhyphen{} The room global parent

\item {} 
\sphinxAtStartPar
global\_parent\_exit    \sphinxhyphen{} The exit global parent

\item {} 
\sphinxAtStartPar
global\_parent\_thing   \sphinxhyphen{} The thing global parent

\item {} 
\sphinxAtStartPar
global\_parent\_player  \sphinxhyphen{} The player global parent

\end{itemize}

\item {} 
\sphinxAtStartPar
Global @parents are different than global parents in that any new
item of similar type that is created is automatically assigned this
physical @parent.  It’s obviously more limiting since it sets
the literal physical parent defined.

\sphinxAtStartPar
The type of the parent does not have to match the target.

\sphinxAtStartPar
The following parameters are used:
\begin{itemize}
\item {} 
\sphinxAtStartPar
room\_parent           \sphinxhyphen{} The target that new rooms are @parented

\item {} 
\sphinxAtStartPar
exit\_parent           \sphinxhyphen{} The target that new exits are @parented

\item {} 
\sphinxAtStartPar
thing\_parent          \sphinxhyphen{} The target that new things are @parented

\item {} 
\sphinxAtStartPar
player\_parent         \sphinxhyphen{} The target that new players are @parented

\end{itemize}

\item {} 
\sphinxAtStartPar
Global attribute formatting is a method define a wrapper, of sorts,
where attributes like @desc, @odesc, @succ, and anything similar
can be processed through this.  All attributes will be either
\&FORMAT\textless{}attribute\textgreater{} or \&\textless{}attribute\textgreater{}FORMAT based on the current
configuration.  Example: \&FORMATDESC or \&DESCFORMAT localy, or
use the following global objects for global formatting.  Local
formatting has priority.

\sphinxAtStartPar
The type of the parent does not have to match the target.
\begin{itemize}
\item {} 
\sphinxAtStartPar
room\_attr\_default     \sphinxhyphen{} Target for room formatting

\item {} 
\sphinxAtStartPar
exit\_attr\_default     \sphinxhyphen{} Target for exit formatting

\item {} 
\sphinxAtStartPar
thing\_attr\_default    \sphinxhyphen{} Target for thing formatting

\item {} 
\sphinxAtStartPar
player\_attr\_default   \sphinxhyphen{} Target for player formatting

\end{itemize}

\end{itemize}


\subsubsection{RhostMUSH limitations and how to get around them}
\label{\detokenize{features:rhostmush-limitations-and-how-to-get-around-them}}\begin{quote}

\sphinxAtStartPar
While Rhost is insanely configurable and quite powerful, there are
limitations that exist within it.
\begin{itemize}
\item {} 
\sphinxAtStartPar
Function invocations.  Sometimes you will hit a ceiling on evaluation.
You may want to tweak values to allow more functions or commands
to execute.  The following controls that:
\begin{itemize}
\item {} 
\sphinxAtStartPar
function\_invocation\_limit {[}25000 default{]} \sphinxhyphen{} specifies the total functions you can execute per command.

\item {} 
\sphinxAtStartPar
function\_recursion\_limit {[}50{]} \sphinxhyphen{} specifies the total times a function can call itself over and over.  Rarely should this be increased and doing so can effect your stack depth.

\end{itemize}

\item {} 
\sphinxAtStartPar
Command queue limits.  Sometimes you want more to be queued up for
players or wizards.
\begin{itemize}
\item {} 
\sphinxAtStartPar
player\_queue\_limit  \sphinxhyphen{} Max number of entries a player can queue

\item {} 
\sphinxAtStartPar
wizard\_queue\_limit  \sphinxhyphen{} Max number of entries a wizard can queue

\end{itemize}

\item {} 
\sphinxAtStartPar
@limit is a wonderful way to lock down limitations per player or global.  Lots of power is available here.
\begin{itemize}
\item {} 
\sphinxAtStartPar
@limit

\item {} 
\sphinxAtStartPar
vattr\_limit\_checkwiz \sphinxhyphen{} Enable @limit checks for wizards

\item {} 
\sphinxAtStartPar
wizmax\_vattr\_limit   \sphinxhyphen{} Set wizard global VATTR limits

\item {} 
\sphinxAtStartPar
wizmax\_dest\_limit    \sphinxhyphen{} Set wizard global @destroy limits

\item {} 
\sphinxAtStartPar
max\_vattr\_limit      \sphinxhyphen{} Set player global VATTR limits

\item {} 
\sphinxAtStartPar
max\_dest\_limit       \sphinxhyphen{} Set player global @destroy limits

\end{itemize}

\item {} 
\sphinxAtStartPar
Lots of trace output can be cut off.  You can modify this with:
\begin{itemize}
\item {} 
\sphinxAtStartPar
trace\_output\_limit   \sphinxhyphen{} Set lines of trace output shown

\end{itemize}

\item {} 
\sphinxAtStartPar
To define how many commands a minute a player set SPAMMONITOR can use
\begin{itemize}
\item {} 
\sphinxAtStartPar
spam\_limit \textendash{} default 120

\end{itemize}

\item {} 
\sphinxAtStartPar
If you examine things and see ‘Output cut off’ messages, you may want
to increase your output limit, funny enough, the name of this is
similar
\begin{itemize}
\item {} 
\sphinxAtStartPar
output\_limit \sphinxhyphen{} You should set this no less than 4 times the current size of your LBUF.

\end{itemize}

\item {} 
\sphinxAtStartPar
Attributes names can not exceed 64 characters.  Sorry, it’s a hard limit

\item {} 
\sphinxAtStartPar
Sometime you may find a single may not work for an escape.  You can
in most cases use a \% instead or double escape the to make it work.
Also look at lit() as a solution.

\end{itemize}
\end{quote}


\subsubsection{Things other mushes can do that Rhost can not and how to emulate it}
\label{\detokenize{features:things-other-mushes-can-do-that-rhost-can-not-and-how-to-emulate-it}}

\paragraph{PennMUSH}
\label{\detokenize{features:pennmush}}\begin{itemize}
\item {} 
\sphinxAtStartPar
Attribute trees.  These are emulated as a base set and can be duplicated enough to at least port code.

\item {} 
\sphinxAtStartPar
lsearch() will have to be recoded to search()

\item {} 
\sphinxAtStartPar
align() will have to be recoded to printf()

\item {} 
\sphinxAtStartPar
Penn allows empty attributes.  Rhost does not.  Work will have to
be done to take this into consideration.

\item {} 
\sphinxAtStartPar
All *val() functions in Penn that are used will have to be remapped to a non\sphinxhyphen{}*val() function.  It should be as simple as function\_alias to the non\sphinxhyphen{}*val.  Example:

\begin{sphinxVerbatim}[commandchars=\\\{\}]
\PYG{n+nd}{@admin} \PYG{n}{function\PYGZus{}alias}\PYG{o}{=}\PYG{n}{hasattrval} \PYG{n+nb}{hasattr}
\end{sphinxVerbatim}

\item {} 
\sphinxAtStartPar
Penn’s hardcoded comsystem is emulated with the softcode comsys

\item {} 
\sphinxAtStartPar
Penn’s @mail system is workable with mail wrappers

\item {} 
\sphinxAtStartPar
Pueblo is not supported.

\item {} 
\sphinxAtStartPar
json is not supported.

\item {} 
\sphinxAtStartPar
ssl is not natively supported (yet), see section XIV

\item {} 
\sphinxAtStartPar
This uses @shutdown/restart, Rhost uses @reboot

\end{itemize}


\paragraph{MUX}
\label{\detokenize{features:mux}}\begin{itemize}
\item {} 
\sphinxAtStartPar
Mux has an async mysql database engine.  This is not possible with Rhost.  You’ll have to use the sync method instead.

\item {} 
\sphinxAtStartPar
UTF8 is supported but internally passed differently.

\item {} 
\sphinxAtStartPar
Mux’s hardcoded comsystem is emulated with the softcode comsys

\item {} 
\sphinxAtStartPar
MUX’s @mail system is workable with mail wrappers

\item {} 
\sphinxAtStartPar
Pueblo is not supported.

\item {} 
\sphinxAtStartPar
This uses @restart, Rhost uses @reboot

\end{itemize}


\subsubsection{Advanced guest setup}
\label{\detokenize{features:advanced-guest-setup}}\begin{itemize}
\item {} 
\sphinxAtStartPar
After you set up your guests, you can set unique names to each guest
if you so want after defining the dbref\#’s your guests use.  This is
done by defining them in the guest\_namelist parameter.  You can also
increase guests (or decrease them) between 0\sphinxhyphen{}31 guests.
\begin{itemize}
\item {} 
\sphinxAtStartPar
wizhelp guest\_namelist

\item {} 
\sphinxAtStartPar
wizhelp num\_guests

\end{itemize}

\end{itemize}


\subsubsection{Attribute permission masking and the joys of the power behind it}
\label{\detokenize{features:attribute-permission-masking-and-the-joys-of-the-power-behind-it}}\begin{itemize}
\item {} 
\sphinxAtStartPar
Attribute contentlocks can be set up so you can lock the actual
content that you can set (or even unset!) into an attribute.
The beauty of this is that you can specify case sensitive
information, lock different ways contents in attributes are set
based on who is setting it, or even on where it’s being set.
The sky’s the limit.
\begin{itemize}
\item {} 
\sphinxAtStartPar
global\_attrdefault    \sphinxhyphen{} Target for defining content locks

\end{itemize}

\end{itemize}


\subsubsection{The amazing @cluster and what it can do for you}
\label{\detokenize{features:the-amazing-cluster-and-what-it-can-do-for-you}}\begin{itemize}
\item {} 
\sphinxAtStartPar
Clusters is the way to virtually assign multiple objects into
a single physical object.  It essentially chains together two
or more objects to share attributes between them, so that any
attribute set on any object in that cluster can be set or fetched
as if it was a singular entity.  This allows some amazing ability
to distribute attribute content or even have a farm of a massive
amount of attributes without paying a hefty penalty on object bloat.
\begin{itemize}
\item {} 
\sphinxAtStartPar
help cluster  \textendash{} Gives a fantastic overview of how clusters work.

\end{itemize}

\end{itemize}


\subsubsection{What we plan for the future}
\label{\detokenize{features:what-we-plan-for-the-future}}\begin{itemize}
\item {} 
\sphinxAtStartPar
Things to look forward to the future with RhostMUSH.
\begin{itemize}
\item {} 
\sphinxAtStartPar
Full Unicode/UTF8 in Rhost 4.0

\item {} 
\sphinxAtStartPar
A fully featured tag system in Rhost 4.0

\item {} 
\sphinxAtStartPar
Built in Python API handler in Rhost 4.0

\item {} 
\sphinxAtStartPar
Hopefully a built in Ruby and Perl API in Rhost 4.0/4.1

\item {} 
\sphinxAtStartPar
Cross\sphinxhyphen{}Mush execution between mushes in Rhost 4.1

\item {} 
\sphinxAtStartPar
More as we think about it :)

\end{itemize}

\end{itemize}


\subsubsection{Additional features not covered otherwise}
\label{\detokenize{features:additional-features-not-covered-otherwise}}\begin{itemize}
\item {} 
\sphinxAtStartPar
+/\sphinxhyphen{} 5.4 million years can be utilized with the built in time functions
which includes timefmt(), secs(), convtime(), convsecs(), and moon().  Party on!

\item {} 
\sphinxAtStartPar
Changing permission levels in the middle of execution for evaluation.
\begin{itemize}
\item {} 
\sphinxAtStartPar
see help on the streval and ueval function’

\end{itemize}

\item {} 
\sphinxAtStartPar
Full features in\sphinxhyphen{}game customization of near every aspect of the game.

\end{itemize}


\subsection{Random notes and things to know about RhostMUSH}
\label{\detokenize{features:random-notes-and-things-to-know-about-rhostmush}}
\sphinxAtStartPar
Here are some things to know about RhostMUSH and what you may or may not
want to do.  Things here are not covered in other documents:

\sphinxAtStartPar
Admin toggles to configure the WHO, various things you’re used to, etc is in the
‘netrhost.conf’ file.  descriptions Notes in the autoconf.h file is in the
README.AUTOCONF file.


\subsubsection{Note on bits, their levels, and things they do}
\label{\detokenize{features:note-on-bits-their-levels-and-things-they-do}}\begin{quote}

\sphinxAtStartPar
IMMORTAL \sphinxhyphen{} They can do anything.  Treat this as \#1 and only give to
people you trust.  Period.   You don’t have to use this bit
if you do not want to and just assume \#1.

\sphinxAtStartPar
ROYALTY \sphinxhyphen{} Unlike PENN/MUX, this is \sphinxstyleemphasis{not} a sub\sphinxhyphen{}wizard, this is a
FULL wizard.  Plus, they can do a bit more.

\sphinxAtStartPar
COUNCILOR \sphinxhyphen{} Like royalty on PENN/MUX but they can modify.

\sphinxAtStartPar
ARCHITECT \sphinxhyphen{} Can’t do as much as councilor, but lot more than BUILDER.

\sphinxAtStartPar
GUILDMASTER \sphinxhyphen{} Very limited.  Sees dbrief\#’s, can ex things their
level and lower and @quota players.
\end{quote}


\subsubsection{You need to @pcreate your guest characters and set them GUEST}
\label{\detokenize{features:you-need-to-pcreate-your-guest-characters-and-set-them-guest}}\begin{quote}

\sphinxAtStartPar
It doesn’t create them on the fly but we considered this better.
You have 31 total you can have.  It defaults to 10 in the
netrhost.conf file.  You can rename the guests anything you want,
but before you do so, you must add the dbref\#’s to the param
guest\_namelist
\end{quote}


\subsubsection{@powers are INHERITED}
\label{\detokenize{features:powers-are-inherited}}\begin{quote}

\sphinxAtStartPar
Therefore, you need power\_objects enabled (@admin)
to make this work properly for non\sphinxhyphen{}plauyers.
A power is taken before a bit level ONLY if higher than that bit.
Yes, powers are multi\sphinxhyphen{}level.
\end{quote}


\subsubsection{@depowers are automatically checked first before anything else}
\label{\detokenize{features:depowers-are-automatically-checked-first-before-anything-else}}\begin{quote}

\sphinxAtStartPar
This is also meaningless on objects.
\end{quote}


\subsubsection{Zones are unique}
\label{\detokenize{features:zones-are-unique}}\begin{quote}

\sphinxAtStartPar
You can have things in multiple zones.
\end{quote}


\subsubsection{The db auto\sphinxhyphen{}repares itself when it can}
\label{\detokenize{features:the-db-auto-repares-itself-when-it-can}}\begin{quote}

\sphinxAtStartPar
It does this by purging anything
it can’t identify.  Dataloss is better than unrecoverable data.
Yes, any such ‘repairing’ is logged so you know if something is up.
\end{quote}


\subsubsection{You can get your connect.txt to parse ansi}
\label{\detokenize{features:you-can-get-your-connect-txt-to-parse-ansi}}\begin{quote}

\sphinxAtStartPar
See ansi\_txtfiles in wizhelp.

\sphinxAtStartPar
You can also override it with softcode if you so wanted.

\sphinxAtStartPar
See file\_object in wizhelp for more information on this.
\end{quote}


\subsubsection{Re\sphinxhyphen{}compiled binaries do not require an @shutdown}
\label{\detokenize{features:re-compiled-binaries-do-not-require-an-shutdown}}\begin{quote}

\sphinxAtStartPar
When re\sphinxhyphen{}compiling the binaries, all you have to remember is when done, issue

\sphinxAtStartPar
@reboot on the game and @readcache.

\sphinxAtStartPar
You do not need to @shutdown.
\end{quote}


\subsubsection{Softcode emulations of functions from other servers are included}
\label{\detokenize{features:softcode-emulations-of-functions-from-other-servers-are-included}}\begin{quote}

\sphinxAtStartPar
Load the file softfunctions into the mush once it’s set up.  This are
@functions that will alias the functions that PENN, MUX, and TinyMUSH have
that is either named differently or we don’t have for one reason or another.
\end{quote}


\section{Comparison of modern MUSH servers}
\label{\detokenize{differences:comparison-of-modern-mush-servers}}\label{\detokenize{differences::doc}}

\subsection{Differences to expect to the end user between PennMUSH, MUX, and RhostMUSH}
\label{\detokenize{differences:differences-to-expect-to-the-end-user-between-pennmush-mux-and-rhostmush}}
\sphinxAtStartPar
The largest end user experience will mostly resolve around some look and feel.
The general look and feel of how to set attributes, work with commands and
functions, and getting around a grid will be identical between the mush flavors.
@nuke and @destroy will work somewhat differently between the three codebases
and some effort should be looked at on how different it behaves.


\subsubsection{PennMUSH}
\label{\detokenize{differences:pennmush}}\begin{itemize}
\item {} 
\sphinxAtStartPar
The way the comsystem (hardcode) works with the latest PennMUSH has some
compatibility now with MUX’s comsystem.

\item {} 
\sphinxAtStartPar
The @mail system is different to MUX and Rhost.

\item {} 
\sphinxAtStartPar
Debugging code uses DEBUG and has an indented format.

\item {} 
\sphinxAtStartPar
The parser for code doesn’t require nested escaping like other codebases, but has issues with the pre\sphinxhyphen{}parser and nesting.

\item {} 
\sphinxAtStartPar
All standard movement, attribute setting, other should be similar

\item {} 
\sphinxAtStartPar
Penn allows empty attributes.

\item {} 
\sphinxAtStartPar
Penn supports attribute trees.

\item {} 
\sphinxAtStartPar
Penn has the standard set of bitlevel flags with on/off @powers

\end{itemize}


\subsubsection{MUX}
\label{\detokenize{differences:mux}}\begin{itemize}
\item {} 
\sphinxAtStartPar
The comsystem (hardcode) is unique to MUX/TinyMUSH3 and is not compatible with Penn.

\item {} 
\sphinxAtStartPar
The @mail system is unique to MUX/TinyMUSH3 and is not compatible with Penn.

\item {} 
\sphinxAtStartPar
Debugging uses TRACE and is the old standard for debugging.

\item {} 
\sphinxAtStartPar
The parser requires extra escaping for nested escaping but handles pre\sphinxhyphen{}parsing and nesting fine.

\item {} 
\sphinxAtStartPar
MUX does not allow empty attributes.

\item {} 
\sphinxAtStartPar
MUX does not support attribute trees.

\item {} 
\sphinxAtStartPar
MUX has the standard set of bitlevel flags with on/off @powers

\end{itemize}


\subsubsection{RhostMUSH}
\label{\detokenize{differences:rhostmush}}\begin{itemize}
\item {} 
\sphinxAtStartPar
The comsystem (softcode) is in the Mushcode directory and is compatible to both PennMUSH and MUX.

\item {} 
\sphinxAtStartPar
The mail system is unique to Rhost but there exists mail wrappers that allow MUX and Penn compatibility.

\item {} 
\sphinxAtStartPar
Debugging uses TRACE and is the old standard for debugging.  It allows advanced features like labeling and grepping for content.

\item {} 
\sphinxAtStartPar
The parser requires extra escaping for nested escaping but handles pre\sphinxhyphen{}parsing and nesting fine.

\item {} 
\sphinxAtStartPar
Rhost does not allow empty attributes.

\item {} 
\sphinxAtStartPar
Rhost marginally supports attribute trees.

\item {} 
\sphinxAtStartPar
Rhost has a multi\sphinxhyphen{}level set of bitlevel flags with multi\sphinxhyphen{}tier @powers, @depowers, and varying other tools for permissions.

\end{itemize}


\subsection{Comparison of features between RhostMUSH, PennMUSH, and MUX2}
\label{\detokenize{differences:comparison-of-features-between-rhostmush-pennmush-and-mux2}}
\sphinxAtStartPar
There are differences in the initial set up from Rhost to PennMUSH (and MUX2).

\sphinxAtStartPar
A lot of people use PennMUSH and if you are one of them, this may help you.
Some people use MUX2, hopefully this will help you as well.
\begin{description}
\item[{NOTE:  If you plan to use sideeffects, you must set the SIDEFX flag to allow}] \leavevmode
\sphinxAtStartPar
the target to use it

\item[{NOTE:  Rhost does support UNICODE/UTF8, but it’s currently not part of the main}] \leavevmode
\sphinxAtStartPar
branch as of this writing.  Please talk to Kage on the dev site for more
information on when it’ll be released.

\end{description}


\subsection{Organization of help and wizhelp}
\label{\detokenize{differences:organization-of-help-and-wizhelp}}
\sphinxAtStartPar
First, on PennMUSH, help is separated into ‘help’ and ‘wizhelp’.  This can
be confusing. If you want to ‘combine’ them, you may use the following snippit:

\sphinxAtStartPar
\$ahelp*:@pemit \%\#={[}switch({[}!!words(\%0){]}{[}match(/search,\%0*){]},0?,{[}setq(0,help){]}{[}setq(1,0){]},11,{[}setq(0,rest(\%0)){]}{[}setq(1,2){]},setq(0,trim(\%0))){]}{[}setq(a,textfile(help,\%q0,\%q1)){]}{[}setq(b,textfile(wizhelp,\%q0,\%q1)){]}{[}setq(2,){]}{[}ifelse(!strmatch(\%qa,No entry*),\%qa{[}setq(2,\%r){]}){]}{[}ifelse(!strmatch(\%qb,No entry*),\%q2\%qb{[}setq(2,\%r){]}){]}{[}ifelse(!\$r(2),No topics found for ‘\%q0’.){]}

\sphinxAtStartPar
This will display help from help and wizhelp for any matching topic.  It
should make life easier for you.  This also honors the /search switch.


\subsection{Mail, comsystem, guests, master rooms, etc..}
\label{\detokenize{differences:mail-comsystem-guests-master-rooms-etc}}
\sphinxAtStartPar
The comsystem is softcode and included in the readme directory. (comsys).
Just load it in.  The mail wrappers can be gotten from:
\begin{quote}

\sphinxAtStartPar
The Mushcode directory in the Rhost Distribution.
\end{quote}

\sphinxAtStartPar
Please see the other readme files on setting other things up like
guests, master rooms, and so forth.


\subsection{Comparisons of features}
\label{\detokenize{differences:comparisons-of-features}}
\sphinxAtStartPar
Comparisons were done based on the following:
RhostMUSH 3.9.5p2
MUX 2.12.0.2
PennMUSH 1.8.6p0


\subsection{@commands with equivalents}
\label{\detokenize{differences:commands-with-equivalents}}
\sphinxAtStartPar
The following are used for compatibility to keep in mind:
\begin{description}
\item[{@commands Pennmush—————\sphinxhyphen{}RhostMUSH———————\sphinxhyphen{}MUX2}] \leavevmode
\sphinxAtStartPar
@@                      @@                             @@
@ALLHALT                @halt/all                      @HALT/ALL
@ALLQUOTA               @quota/all                     @QUOTA/ALL
@ALIAS                  @alias/@protect                @ALIAS
N/A                     @apply\_marked                  @APPLY\_MARKED
@ASSERT                 @assert                        @ASSERT
@ATRCHOWN               @chown                         @CHOWN
@ATRLOCK                @lock                          @LOCK
@ATTRIBUTE              @attribute                     @ATTRIBUTE
N/A                     (see backup\_flat.sh)           @BACKUP
@BOOT                   @boot                          @BOOT
@BREAK                  @break                         @BREAK
N/A                     N/A                            @CCHARGE
N/A                     N/A                            @CCHOWN
@CEMIT                  (softcode)                     @CEMIT
@CHANNEL                N/A                            N/A
@CHAT                   N/A                            N/A
@CHOWNALL               @chownall                      @CHOWNALL
@CHZONE                 @zone                          @CHZONE
@CHZONEALL              @zone                          N/A
@CLOCK                  N/A                            N/A
@CLONE                  @clone                         @CLONE
N/A                     N/A                            @CCREATE
N/A                     N/A                            @CDESTROY
N/A                     N/A                            @CWHO
@COMMAND                N/A                            N/A
@CONFIG                 @admin                         @ADMIN
N/A                     N/A                            @COFLAGS
N/A                     N/A                            @CPFLAGS
N/A                     N/A                            @CSET
@CPATTR                 @cpattr                        @CPATTR
@CREATE                 @create                        @CREATE
N/A                     @cut                           @CUT
@DBCK                   @dbck                          @DBCK
@DECOMPILE              @decompile                     @DECOMPILE
@DESTROY                @destroy                       @DESTROY
@DIG                    @dig                           @DIG
@DISABLE                @disable                       @DISABLE
@DOING                  @doing                         @DOING
@DOLIST                 @dolist                        @DOLIST
@DRAIN                  @drain                         @DRAIN
@DUMP                   @dump                          @DUMP
@EDIT                   @edit                          @EDIT
@ELOCK                  @lock/enter                    @LOCK/ENTER
@EMIT                   @emit                          @EMIT
N/A                     N/A                            @EMAIL
@ENABLE                 @enable                        @ENABLE
@ENTRANCES              @entrances                     @ENTRANCES
@EUNLOCK                @unlock/enter                  @UNLOCK/ENTER
N/A                     @eval                          @EVAL
N/A                     @femit                         @FEMIT
N/A                     @fpose                         @FPOSE
N/A                     @fsay                          @FSAY
@FIND                   @find                          @FIND
@FIRSTEXIT              N/A                            N/A
@FLAG                   @flag                          @FLAG
@FORCE                  @force                         @FORCE
N/A                     folder                         @FOLDER
@FUNCTION               @function/@lfunction           @FUNCTION
@GREP                   @grep                          N/A
@HALT                   @halt                          @HALT
@HIDE                   @hide                          N/A
@HOOK                   @hook                          @HOOK
@INCLUDE                @include                       N/A
N/A                     @skip/ifelse                   @IF
@KICK                   @kick                          @KICK
N/A                     @last                          @LAST
@LEMIT                  @lemit                         @LEMIT
@LINK                   @link                          @LINK
@LIST                   @list                          @LIST
N/A                     @list\_file                     @LIST\_FILE
@LISTMOTD               @listmotd                      @LISTMOTD
@LOCK                   @lock                          @LOCK
@LOG                    @log                           @LOG
@LOGWIPE                N/A                            N/A
@LSET                   @set                           @SET
N/A                     @mark                          @MARK
N/A                     @mark\_all                      @MARK\_ALL
@MAIL                   mail                           @MAIL
@MALIAS                 wmail/alias                    @MALIAS
@MAPSQL                 N/A                            N/A
@MESSAGE                @pemit/@remit + parsestr()     N/A
@MONIKER                @extansi                       @MONIKER
@MOTD                   @motd                          @MOTD
@MVATTR                 @mvattr                        @MVATTR
@NAME                   @name                          @NAME
N/A                     @emit/noeval                   @NEMIT
N/A                     @pemit/noeval                  @NPEMIT
@NEWPASSWORD            @newpassword                   @NEWPASSWORD
@NOTIFY                 @notify                        @NOTIFY
@NSCEMIT                N/A                            N/A
@NSEMIT                 @emit                          @emit
@NSLEMIT                @lemit                         @LEMIT
@NSOEMIT                @oemit                         @OEMIT
@NSPEMIT                @pemit                         @PEMIT
@NSPROMPT               N/A                            N/A
@NSREMIT                @remit                         @REMIT
@NSZEMIT                @zemit                         N/A
@NUKE                   @destroy/@nuke                 @DESTROY/@NUKE
@OEMIT                  @oemit                         @OEMIT
@OPEN                   @open                          @OPEN
@PARENT                 @parent                        @PARENT
@PASSWORD               @password                      @PASSWORD
@PCREATE                @pcreate                       @PCREATE
@PEMIT                  @pemit                         @PEMIT
@POLL                   @doing/header                  @POLL
@POOR                   @poor                          @POOR
@POWER                  @power                         @POWER
@PROMPT                 N/A (@program?)                N/A (@program?)
N/A                     @program                       @PROGRAM
@PS                     @ps                            @PS
@PURGE                  @timewarp/dump 1               @TIMEWARP/DUMP 1
N/A                     @quitprogram                   @QUITPROGRAM
@QUOTA                  @quota                         @QUOTA
N/A                     N/A                            @QUERY
@READCACHE              @readcache                     @READCACHE
@RECYCLE                @purge                         N/A
N/A                     N/A                            @REFERENCE
N/A                     @robot                         @ROBOT
@REJECTMOTD             @rejectmotd                    @REJECTMOTD
@REMIT                  @remit                         @REMIT
@RESTART                @reboot                        @RESTART
@RETRY                  N/A                            N/A
@RWALL                  @wall/wiz                      @WALL/WIZ
@SCAN                   (see softcode)                 N/A
@SEARCH                 @search                        @SEARCH
@SELECT                 @switch/first                  @SWITCH/FIRST
@SET                    @set                           @SET
@SHUTDOWN               @shutdown                      @SHUTDOWN
@SITELOCK               @admin forbid\_host/forbid\_site @admin forbid\_site
@SLAVE                  N/A                            @STARTSLAVE
@SOCKSET                N/A                            N/A
@SQL                    (only if MySQL enabled)        @QUERY
@SQUOTA                 @quota                         N/A
@STATS                  @stats                         @STATS
@SWEEP                  @sweep                         @SWEEP
@SWITCH                 @switch                        @SWITCH
N/A                     @timewarp                      @TIMEWARP
@TELEPORT               @teleport                      @TELEPORT
N/A                     @timecheck                     @TIMECHECK
N/A                     @toad                          @TOAD
@TRIGGER                @trigger                       @TRIGGER
@ULOCK                  @lock/use                      @LOCK/USE
@UNDESTROY              @recover                       N/A
@UNLINK                 @unlink                        @UNLINK
@UNLOCK                 @unlock                        @UNLOCK
@UNRECYCLE              @recover                       N/A
@UPTIME                 @uptime                        @UPTIME
@UUNLOCK                @unlock/use                    @UNLOCK/USE
@VERB                   @verb                          @VERB
@VERSION                @version                       VERSION
@WAIT                   @wait                          @WAIT
@WALL                   @wall                          @WALL
@WARNINGS               N/A                            N/A
@WCHECK                 N/A                            N/A
@WEBPASSWD              N/A                            N/A
@WHEREIS                @whereis                       N/A
@WIPE                   @wipe                          @WIPE
@WIZMOTD                @wizmotd                       @WIZMOTD
@WIZWALL                @wall/wiz                      @WALL/WIZ
@ZEMIT                  @zemit                         N/A
N/A                     (softcode)                     ALLCOM
N/A                     (softcode)                     COMLIST
N/A                     (softcode)                     DELCOM
N/A                     (softcode)                     ADDCOM
N/A                     (softcode)                     COMTITLE
ANEWS                   @dynhelp                       N/A
ATTRIB\_SET              (@hook on S)                   (@hook on S)
BRIEF                   ex/brief                       EX/BRIEF
BUY                     N/A                            N/A
N/A                     N/A                            CLEARCOM
DESERT                  (see follow softcode)          N/A
DISMISS                 (see follow softcode)          N/A
DOING                   doing                          DOING
DROP                    drop                           DROP
EMPTY                   @tel/list lcon(target)=me      @tel/list lcon(target)=me
ENTER                   enter                          ENTER
EXAMINE                 examine                        EXAMINE
FOLLOW                  (see follow softcode)          N/A
GET                     get                            GET
GIVE                    give                           GIVE
GOTO                    goto                           GOTO
HELP                    help/wizhelp                   HELP/WIZHELP
HOME                    home                           HOME
HUH\_COMMAND             @admin global\_error\_obj        @admin global\_error\_obj
INFO                    INFO                           INFO
INVENTORY               inventory                      INVENTORY
KILL                    kill                           KILL
LEAVE                   leave                          LEAVE
LOGOUT                  logout                         LOGOUT
LOOK                    look                           LOOK
NEWS                    news                           NEWS
N/A                     outputprefix                   OUTPUTPREFIX
N/A                     outputsuffix                   OUTPUTSUFFIX
PAGE                    page/lpage/rpage/mrpage        PAGE
POSE                    pose                           POSE
N/A                     N/A                            PUEBLOCLIENT
QUIT                    quit                           QUIT
N/A                     N/A                            REPORT
SAY                     say                            SAY
SCORE                   score                          SCORE
SEMIPOSE                pose/nospace                   POSE/NOSPACE
SESSION                 session                        SESSION
SLAY                    slay                           SLAY
TEACH                   train                          TRAIN
THINK                   think                          THINK
UNFOLLOW                (see follow softcode)          N/A
UNIMPLEMENTED\_COMMAND   @admin global\_error\_obj        @admin global\_error\_obj
USE                     use                            USE
WARN\_ON\_MISSING         N/A                            N/A
WHISPER                 whisper                        WHISPER
WHO                     who                            WHO
WITH                    N/A                            N/A

\end{description}


\subsection{@commands unique to RhostMUSH}
\label{\detokenize{differences:commands-unique-to-rhostmush}}
\sphinxAtStartPar
Commands that exist in Rhost that have no PennMUSH equivelant:
@aflags                      @apply\_marked                 @areg
@blacklist                   @cluster                      @conncheck
@convert                     @cut                          @dbclean
@depower                     @dynhelp                      @eval
@femit                       @fixdb                        @fpose
@freeze                      @fsay                         @icmd
@last                        @lfunction                    @limit
@logrotate                   @mark                         @mark\_all
@money                       @pipe                         @program
@progreset                   @protect                      @quitprogram
@reclist                     @recover                      @register
@remote                      @robot                        @rxlevel
@skip                        @snapshot                     @snoop
@thaw                        @timewarp                     @toad
@toggle                      @toggledef                    @tor
@turtle                      @txlevel                      @whereall
grab                         join                          listen
mrpage                       newsdb                        rpage
smell                        taste                         touch
wielded                      worn                          +players


\subsection{@lock equivalents}
\label{\detokenize{differences:lock-equivalents}}\begin{description}
\item[{@locks PennMUSH—————RhostMUSH———————————\textendash{}MUX2}] \leavevmode
\sphinxAtStartPar
BASIC                  BASIC/DEFAULT                               DEFAULT
ENTER                  ENTER                                       ENTER
TELEPORT               TPORT                                       TPORT
USE                    USE                                         USE
PAGE                   PAGE                                        PAGE
ZONE                   ZONEWIZLOCK/ZONETOLOCK/TWINKLOCK            N/A
PARENT                 PARENT                                      PARENT
LINK                   LINK                                        LINK
OPEN                   OPEN                                        OPEN
MAIL                   mail/lock                                   MAIL
USER                   USER                                        USER
USER:\textless{}dynamicname\textgreater{}     lockencode()/lockdecode()/lockcheck()       N/A
SPEECH                 SPEECH                                      SPEECH
LISTEN                 USE (see listen argument)                   N/A
COMMAND                USE (commands are default)                  N/A
LEAVE                  LEAVE                                       LEAVE
DROP                   DROP                                        DROP
DROPIN                 DROPTO                                      N/A
GIVE                   GIVE                                        GIVE
FROM                   GIVETO                                      N/A
PAY                    N/A                                         N/A
RECEIVE                RECEIVE                                     RECEIVE
FOLLOW                 (See softcoded follow code)                 N/A
EXAMINE                See NO\_MODIFY/NO\_EXAMINE/TWINKLOCK          N/A
CHZONE                 ZONETOLOCK/ZONEWIZLOCK/TWINKLOCK            N/A
FORWARD                N/A                                         N/A
FILTER                 N/A                                         N/A
INFILTER               N/A                                         N/A
CONTROL                CONTROL                                     N/A
DROPTO                 DROPTO                                      N/A
DESTROY                See: @recover/@purge/INDESTRUCTIBLE/SAFE    N/A
INTERACT               N/A (See: Reality Levels)                   N/A (See: Reality Levels)
TAKE                   GETFROM                                     GETFROM
MAILFORWARD            mail/lock, mail/autofor                     N/A
N/A                    TELOUT                                      TELOUT
N/A                    DARK                                        VISIBLE

\end{description}


\subsection{@locks that only exist in RhostMUSH}
\label{\detokenize{differences:locks-that-only-exist-in-rhostmush}}
\sphinxAtStartPar
@locks that exist in Rhost that have no PennMUSH equivelant:
TELOUTLOCK                   TWINKLOCK                     DARKLOCK
ALTNAME                      CHOWN


\subsection{Flag and toggle equivalents}
\label{\detokenize{differences:flag-and-toggle-equivalents}}\begin{description}
\item[{FLAGS Pennmush—————\sphinxhyphen{}RhostMUSH——————————MUX2}] \leavevmode
\sphinxAtStartPar
ABODE                   ABODE                                  ABODE
N/A                     N/A                                    ASCII
ANSI                    ANSI                                   ANSI
AUDIBLE                 AUDIBLE                                AUDIBLE
(Not Needed)            (Not Needed)                           BLEED
N/A                     AUDITORIUM                             AUDITORIUM
N/A                     BLIND                                  BLIND
N/A                     COMMANDS                               COMMANDS
CHAN\_USEFIRSTMATCH      N/A                                    N/A
CHOWN\_OK                CHOWN\_OK                               CHOWN\_OK
CLOUDY                  N/A                                    N/A
COLOR                   ANSICOLOR                              N/A
CONNECTED               CONNECTED                              CONNECTED
DARK                    DARK                                   DARK
DEBUG                   TRACE                                  TRACE
DESTROY\_OK              DESTROY\_OK                             DESTROY\_OK
ENTER\_OK                ENTER\_OK                               ENTER\_OK
FIXED                   NO\_TEL                                 FIXED
FLOATING                FLOATING                               FLOATING
GAGGED                  FUBAR                                  GAGGED
GOING                   GOING                                  GOING
HALT                    HALT                                   HALT
HAVEN                   HAVEN                                  HAVEN
(see @flag)             (marker0\sphinxhyphen{}marker9)                      HEAD
HEAR\_CONNECT            MONITOR (@toggle)                      SITECON
HEAVY                   \sphinxhref{mailto:NO\_TEL/@lock-teleport}{NO\_TEL/@lock\sphinxhyphen{}teleport}                  N/A
N/A                     N/A                                    HTML
N/A                     FREE                                   IMMORTAL
N/A                     INHERIT                                INHERIT
JUMP\_OK                 JUMP\_OK                                JUMP\_OK
KEEPALIVE               KEEPALIVE (@toggle)                    KEEPALIVE
N/A                     KEY                                    KEY
LIGHT                   LIGHT                                  LIGHT
LINK\_OK                 LINK\_OK                                LINK\_OK
LISTEN\_PARENT           (@admin listen\_parents)                N/A
LOUD                    NO\_OVERRIDE/NO\_USELOCK                 N/A
(see @flag)             MARKER0\sphinxhyphen{}MARKER9                        MARKER0\sphinxhyphen{}MARKER9
MISTRUST                GUILDOBJ/NO\_GOBJ/BACKSTAGE/NOBACKSTAGE N/A
MONIKER                 EXTANSI (@toggle)                      N/A
MONITOR                 MONITOR                                MONITOR
MYOPIC                  MYOPIC                                 MYOPIC
NOACCENTS               ACCENTS (@toggle)                      ACCENTS
(Not Needed)            (Not Needed)                           NO\_BLEED
NOSPOOF                 NOSPOOF                                NOSPOOF
(See @ns\sphinxhyphen{}commands)      Auto\sphinxhyphen{}Enabled for Wiz+                  SPOOF
NO\_COMMAND              NO\_COMMAND                             NO\_COMMAND
NO\_LEAVE                @icmd \sphinxhref{mailto:leave/@lock-leave}{leave/@lock\sphinxhyphen{}leave}                @icmd \sphinxhref{mailto:leave/@lock-leave}{leave/@lock\sphinxhyphen{}leave}
NO\_TEL                  NO\_TEL                                 N/A
ON\sphinxhyphen{}VACATION             MARKER0\sphinxhyphen{}MARKER9                        VACATION
OPAQUE                  OPAQUE                                 OPAQUE
OPEN\_OK                 @lock\sphinxhyphen{}openfrom                         OPEN\_OK
ORPHAN                  NOGLOBPARENT (@toggle)                 N/A
N/A                     PARENT\_OK                              PARENT\_OK
PUPPET                  PUPPET                                 PUPPET
QUIET                   QUIET                                  QUIET
N/A                     ROBOT                                  ROBOT
ROYALTY                 GUILDMASTER/ARCHITECT/COUNCILOR        ROYALTY
(@see @flag)            MARKER0\sphinxhyphen{}9                              STAFF
SAFE                    SAFE                                   SAFE
N/A                     SLAVE                                  SLAVE
N/A                     MONITOR (@toggle)                      SITEMON
STICKY                  STICKY                                 STICKY
N/A                     SUSPECT                                SUSPECT
TERSE                   TERSE                                  TERSE
TRANSPARENT             TRANSPARENT                            TRANSPARENT
UNFINDABLE              UNFINDABLE                             UNFINDABLE
N/A                     N/A                                    UNICODE
(See @flag)             WANDERER                               UNINSPECTED
VERBOSE                 VERBOSE                                VERBOSE
VISUAL                  VISUAL                                 VISUAL
WIZARD                  WIZARD/IMMORTAL                        WIZARD
XTERM256                XTERMCOLOR                             COLOR256

\end{description}


\subsection{Flags and toggles that only exist in RhostMUSH}
\label{\detokenize{differences:flags-and-toggles-that-only-exist-in-rhostmush}}

\subsubsection{Flags}
\label{\detokenize{differences:flags}}
\sphinxAtStartPar
ALTQUOTA                     ANONYMOUS                     ARCHITECT
AUDITORIUM                   BLIND                         BOUNCE
CLOAK                        COUNCILOR                     FUBAR
GUILDMASTER                  GUILDOBJ                      IMMORTAL
INDESTRUCTABLE               NO\_ANSINAME                   NO\_BACKSTAGE
NO\_CODE                      NO\_CONNECT                    NO\_EXAMINE
NO\_FLASH                     NO\_GOBJ                       NO\_MODIFY
NO\_MOVE                      NO\_NAME                       NO\_OVERRIDE
NO\_PESTER                    NO\_POSSESS                    NO\_STOP
NO\_UNDERLINE                 NO\_USELOCK                    NO\_WALLS
NO\_YELL                      PRIVATE                       ROBOT
SCLOAK                       SEE\_OEMIT                     SHOWFAILCMD
SIDEFX                       SPAMMONITOR                   SPOOF
STOP                         WANDERER                      ZONECONTENTS
ZONEPARENT


\subsubsection{Toggles}
\label{\detokenize{differences:toggles}}
\sphinxAtStartPar
ATRUSE                       CHKREALITY                    CPUTIME
EXFULLWIZATTR                FORCEHALTED                   HIDEIDLE
IGNOREZONE                   IMMPROG                       LOGROOM
MAILVALIDATE                 MAIL\_LOCKDOWN                 MAIL\_NOPARSE
MAIL\_STRIPRETURN             MONITOR\_AREG                  MONITOR\_BAD
MONITOR\_CONN                 MONITOR\_CPU                   MONITOR\_DISREASON
MONITOR\_FAIL                 MONITOR\_SITE                  MONITOR\_STATS
MONITOR\_TIME                 MONITOR\_USERID                MONITOR\_VLIMIT
MORTALREALITY                NODEFAULT                     NOSHPROG
NOZONEPARENT                 NO\_ANSI\_EX                    NO\_ANSI\_EXIT
NO\_ANSI\_PLAYER               NO\_ANSI\_ROOM                  NO\_ANSI\_THING
NO\_FORMAT                    NO\_TIMESTAMP                  PAGELOCK
PROG                         PROG\_ON\_CONNECT               SAFELOG
SEE\_SUSPECT                  SILENTEFFECT                  SNUFFDARK
ZONECMDCHK                   ZONE\_AUTOADD                  ZONE\_AUTOADDALL


\subsection{@power equivalents}
\label{\detokenize{differences:power-equivalents}}\begin{description}
\item[{@powers Pennmush—————\sphinxhyphen{}RhostMUSH——————————————MUX2}] \leavevmode
\sphinxAtStartPar
Announce                FREE\_WALL                                          Announce
Boot                    BOOT                                               Boot
Builder                 @quota !WANDERER (flag)                            Builder
CAN\_DARK                @admin \sphinxhref{mailto:player\_dark/@depower}{player\_dark/@depower} dark                   N/A
Can\_spoof               N/A \sphinxhyphen{} Wizard+ auto\sphinxhyphen{}spoof                           N/A
Cemit                   N/A                                                N/A
N/A                     CHOWN\_OTHER                                        chown\_anything
N/A                     @lock/twink                                        control\_all
N/A                     WIZ\_WHO                                            expanded\_who
Chat\_Privs              N/A                                                comm\_all
DEBIT                   STEAL                                              Steal\_money
Functions               (See @lfunctions)                                  Wizard+ only
Guest                   GUEST (flag)                                       Guest
HOOK                    Wizard+ only                                       Wizard+ only
Halt                    HALT\_QUEUE/HALT\_QUEUE\_ALL                          Halt
Hide                    NOWHO                                              Hide
Idle                    @timeout player to \sphinxhyphen{}1                              Idle
N/A                     NO\_MODIFY (flag)                                   Immutable
Link\_Anywhere           N/A (security risk)                                N/A
Login                   LOGIN (flag)                                       LOGIN (flag)
Long\_Fingers            LONG\_FINGERS                                       Long\_fingers
MANY\_ATTRIBS            (@admin vlimit)                                    N/A
N/A                     MONITOR (@toggle)                                  Monitor
No\_Pay                  FREE (flag)                                        Free\_money
No\_Quota                FREE\_QUOTA                                         Free\_quota
Open\_Anywhere           N/A (security risk)                                N/A
N/A                     (Wiz+ Automatic)                                   Pass\_locks
PICK\_DBREFS             Wizard+ only                                       N/A
PUEBLO\_SEND             N/A                                                N/A
Pemit\_All               LONG\_FINGERS                                       N/A
Player\_Create           PCREATE                                            N/A
Poll                    N/A \textendash{} Softcode @doing/header                      Poll
N/A                     PROG (@toggle)                                     Prog
Queue                   SEE\_QUEUE/SEE\_QUEUE\_ALL/HALT\_QUEUE/HALT\_QUEUE\_ALL  N/A
Quotas                  CHANGE\_QUOTAS                                      N/A
SQL\_OK                  N/A                                                N/A
Search                  SEARCH\_ANY                                         Search
See\_All                 EXAMINE\_FULL                                       See\_all
N/A                     WHO\_UNFIND                                         See\_hidden
N/A                     SHUTDOWN                                           Siteadmin
See\_Queue               SEE\_QUEUE/SEE\_QUEUE\_ALL                            N/A
N/A                     STAT\_ANY                                           Stat\_any
Tport\_Anything          TEL\_ANYTHING                                       Tel\_anything
Tport\_Anywhere          TEL\_ANYWHERE                                       Tel\_anywhere
Unkillable              NOKILL                                             Unkillable

\end{description}


\subsubsection{@power unique to RhostMUSH}
\label{\detokenize{differences:power-unique-to-rhostmush}}
\sphinxAtStartPar
Depowers are unique in Rhost and PennMUSH has no equivelant.

\sphinxAtStartPar
Powers that exist in RhostMUSH that have no match in PennMUSH:
CHOWN\_ME                     WIZ\_WHO                       NOFORCE
FREE\_QUOTA                   JOIN\_PLAYER                   NO\_BOOT
STAT\_ANY                     WHO\_UNFIND                    SHUTDOWN
PURGE                        CHOWN\_ANYWHERE                CHOWN\_OTHER
GRAB\_PLAYER                  SECURITY                      WRAITH
HIDEBIT


\subsection{Functions equivalents}
\label{\detokenize{differences:functions-equivalents}}\begin{description}
\item[{Functions Pennmush—————\sphinxhyphen{}RhostMUSH——————\textendash{}MUX}] \leavevmode
\sphinxAtStartPar
@@                      @@                           @@
ABS                     ABS                          ABS/IABS
ACCENT                  ACCENT                       ACCENT
ACCNAME                 CNAME                        MONIKER
ACOS                    ACOS                         ACOS
ADD                     ADD                          ADD
AFTER                   AFTER                        AFTER
ALIAS                   get(\#db/alias)/LISTPROTECT   get(\#db/alias)
ALIGN                   PRINTF                       N/A
ALLOF                   OFPARSE                      N/A
ALPHAMAX                ALPHAMAX                     ALPHAMAX
ALPHAMIN                ALPHAMIN                     ALPHAMIN
AND                     AND                          AND/ANDBOOL
ANDFLAGS                ANDFLAGS                     ANDFLAGS
ANDLFLAGS               ANDFLAG                      N/A
ANDLPOWERS              @function (softfunctions)    N/A
ANSI                    ANSI                         ANSI
APOSS                   APOSS                        APOSS
ART                     ART                          ART
ASIN                    ASIN                         ASIN
ATAN                    ATAN                         ATAN
ATAN2                   ATAN2                        ATAN2
N/A                     ATTRCNT                      ATTRCNT
ATRLOCK                 HASFLAG(\#obj/attr,LOCK)      HASFLAG(\#obj/attr,LOCK)
ATTRIB\_SET              SET                          SET
BAND                    MASK                         BAND
BASECONV                PACK/UNPACK                  BASECONV
BEEP                    BEEP                         BEEP
BEFORE                  BEFORE                       BEFORE
N/A                     BITTYPE                      BITTYPE
BENCHMARK               CPUTIME (@toggle)            N/A
BNAND                   MASK                         BNAND
BNOT                    MASK                         N/A
BOR                     BOR                          BOR
BOUND                   BOUND/FBOUND                 N/A
BRACKETS                BRACKETS                     N/A
BXOR                    MASK                         BXOR
CAND                    CAND                         CAND/CANDBOOL
CAPSTR                  CAPSTR                       CAPSTR
CASE                    CASE                         CASE
CASEALL                 CASEALL                      N/A
CAT                     CAT                          CAT
CBUFFER                 N/A                          N/A
CBUFFERADD              N/A                          N/A
CDESC                   N/A                          N/A
CEIL                    CEIL                         CEIL
CEMIT                   N/A                          CEMIT
CENTER                  CENTER                       CENTER
CFLAGS                  N/A                          N/A
CHANNELS                N/A                          CHANNELS
CHECKPASS               CHECKPASS                    N/A
CHILDREN                CHILDREN                     CHILDREN
N/A                     N/A                          CHOOSE
CHR                     CHR                          CHR
CLFLAGS                 N/A                          N/A
CLOCK                   N/A                          N/A
CLONE                   CLONE                        N/A
CMDS                    CMDS                         CMDS
CMOGRIFIER              N/A                          N/A
CMSGS                   N/A                          N/A
COLORS                  COLORS                       N/A
N/A                     N/A                          COLORDEPTH
N/A                     @function (softfunctions)    COLUMNNS
N/A                     N/A                          COMALIAS
COMP                    COMP                         COMP
N/A                     N/A                          COMTITLE
CON                     CON                          CON
COND                    @function (softfunctions)    N/A
CONDALL                 @function (softfunctions)    N/A
CONFIG                  CONFIG                       CONFIG
CONN                    CONN                         CONN
convsecs(get(\#db/last)) convsecs(get(\#db/last))      CONNLAST
N/A                     N/A                          CONNLEFT
N/A                     N/A                          CONNMAX
N/A                     N/A                          CONNNUM
N/A                     N/A                          CONNRECORD
N/A                     N/A                          CONNTOTAL
CONTROLS                CONTROLS                     CONTROLS
CONVSECS                CONVSECS                     CONVSECS
CONVTIME                CONVTIME                     CONVTIME
CONVUTCSECS             CONVSECS                     CONVSECS
CONVUTCTIME             CONVTIME                     CONVTIME
COR                     COR                          COR/CORBOOL
COS                     COS                          COS
ALIGN                   PRINTF                       CPAD
N/A                     CRC32                        CRC32
COWNER                  N/A                          N/A
CREATE                  CREATE                       CREATE
CRECALL                 N/A                          N/A
CSECS                   N/A                          N/A
CSTATUS                 N/A                          N/A
CTIME                   N/A                          CTIME
CTITLE                  N/A                          N/A
CTU                     CTU                          CTU
CUSERS                  N/A                          N/A
CWHO                    N/A                          CWHO
DEC                     DEC/XDEC                     DEC
DECODE64                DECODE64                     N/A
DECOMPOSE               TRANSLATE                    TRANSLATE
DECRYPT                 DECRYPT                      DECRYPT
DEFAULT                 DEFAULT                      DEFAULT
N/A                     DESTROY                      DESTROY
DIE                     DICE                         DIE
DIG                     DIG                          CREATE(with ‘r’)
DIGEST                  DIGEST                       DIGEST
N/A                     TIMEFMT                      DIGITTIME
DIST2D                  DIST2D                       DIST2D
DIST3D                  DIST3D                       DIST3D
@function               @function                    DISTRIBUTE
DIV                     DIV                          IDIV
DOING                   DOING                        DOING
N/A                     N/A                          DUMPING
E                       E                            E
EDEFAULT                EDEFAULT                     EDEFAULT
EDIT                    PEDIT/EDIT                   EDIT
ELEMENT                 MATCH                        MATCH
ELEMENTS                ELEMENTSMUX/ELEMENTS         ELEMENTS
ELIST                   ELIST                        ITEMIZE
ELOCK                   ELOCK                        ELOCK
EMIT                    EMIT                         EMIT
ENCODE64                ENCODE64                     N/A
ENCRYPT                 ENCRYPT                      ENCRYPT
ENDTAG                  N/A                          N/A
ENTRANCES               ENTRANCES                    ENTRANCES
EQ                      EQ                           EQ
N/A                     ERROR                        ERROR
ESCAPE                  ESCAPE                       ESCAPE
ETIME                   @function (softfunctions)    N/A
ETIMEFMT                TIMEFMT                      ETIMEFMT
EVAL                    EVAL                         EVAL
EXIT                    EXIT                         EXIT
N/A                     EXP                          EXP
EXTRACT                 EXTRACT                      EXTRACT
\%+                      \%+                           FCOUNT
\%+                      \%+                           FDEPTH
FDIV                    FDIV                         FDIV
FILTER                  FILTER                       FILTER
FILTERBOOL              FILTER                       FILTERBOOL
FINDABLE                FINDABLE                     FINDABLE
FIRST                   FIRST                        FIRST
FIRSTOF                 OFPARSE                      N/A
FLAGS                   FLAGS                        FLAGS
FLIP                    REVERSE                      REVERSE
FLOOR                   FLOOR                        FLOOR
FLOORDIV                FLOORDIV                     FLOORDIV
FMOD                    FMOD                         FMOD
FN                      BYPASS                       N/A
FOLD                    FOLD                         FOLD
FOLDERSTATS             FOLDERLIST                   N/A
FOLLOWERS               N/A (softcode available)     N/A
FOLLOWING               N/A (softcode available)     N/A
FOREACH                 FOREACH                      FOREACH
FRACTION                N/A                          N/A
FULLALIAS               ALIAS + LISTPROTECT          N/A
FULLNAME                FULLNAME                     FULLNAME
FUNCTIONS               LISTFUNCTIONS                N/A
GET                     GET                          GET
GETPIDS                 PIDS                         N/A
GET\_EVAL                GET\_EVAL                     GET\_EVAL
GRAB                    GRAB                         GRAB
GRABALL                 GRABALL                      GRABALL
GREP                    GREP                         GREP
GREPI                   GREP                         GREPI
GT                      GT                           GT
GTE                     GTE                          GTE
HASATTR                 HASATTR                      HASATTR
HASATTRP                HASATTRP                     HASATTRP
HASATTRPVAL             HASATTRP                     HASATTRP
HASATTRVAL              HASATTR                      HASATTR
HASFLAG                 HASFLAG                      HASFLAG
HASPOWER                HASPOWER                     HASPOWER
N/A                     HASQUOTA                     HASQUOTA
HASTYPE                 HASTYPE                      HASTYPE
HEIGHT                  @function (softfunctions)    HEIGHT
HIDDEN                  HIDDEN                       N/A
HOME                    HOME                         HOME
HOST                    LOOKUP\_SITE                  HOST
HTML                    N/A                          N/A
IBREAK                  IBREAK                       N/A
IDLE                    IDLE                         IDLE
IF                      IFELSE                       IF
IFELSE                  IFELSE                       IFELSE
ILEV                    ILEV                         ILEV
INAME                   NAME                         NAME
INC                     INC/XINC                     INC
INDEX                   INDEX                        INDEX
INUM                    INUM/INUM2                   INUM
N/A                     INZONE                       INZONE
IPADDR                  LOOKUP\_SITE                  N/A
ISDAYLIGHT              TIMEFMT                      N/A
ISDBREF                 ISDBREF                      ISDBREF
ISINT                   ISINT                        ISINT
ISNUM                   ISNUM                        ISNUM
N/A                     N/A                          ISRAT
ISOBJID                 N/A                          N/A
ISREGEXP                N/A                          N/A
ISWORD                  ISWORD                       ISWORD
ITEMIZE                 ELIST                        ITEMIZE
ITEMS                   WORDS                        WORDS
ITER                    ITER                         ITER
ITEXT                   ITEXT                        ITEXT
LALIGN                  PRINTF                       N/A
STRMATH                 LADD                         LADD
N/A                     LAND                         LAND
LAST                    LAST                         LAST
N/A                     LASTCREATE                   LASTCREATE
LATTR                   LATTR                        LATTR
N/A                     LATTR                        LATTRCMDS
LATTRP                  LATTRP                       LATTRP
N/A                     LCMDS                        LCMDS
LCON                    LCON                         LCON
LCSTR                   LCSTR                        LCSTR
LDELETE                 LDELETE                      LDELETE
LEFT                    LEFT                         STRTRUNC
LEMIT                   LEMIT                        N/A
LETQ                    @function (softfunctions)    N/A
LEXITS                  LEXITS                       LEXITS
LFLAGS                  LFLAGS                       LFLAGS
LINK                    LINK                         LINK
N/A                     LIST (like iter())           LIST (like iter())
LINSERT                 INSERT                       INSERT
LIST                    LISTPOWERS, FLAGS, etc       N/A
LISTQ                   N/A                          N/A
LIT                     LIT                          LIT
LJUST                   LJUST                        LJUST
LLOCKFLAGS              N/A                          N/A
LLOCKS                  LOCKS                        LOCKS
LMATH                   STRFUNC                      N/A
LN                      LN                           LN
LNUM                    LNUM/LNUM2                   LNUM
LOC                     LOC                          LOC
LOCALIZE                LOCALIZE                     LOCALIZE
LOCATE                  LOCATE                       LOCATE
LOCK                    LOCK                         LOCK
N/A                     LOG2FILE                     LOG
LOCKFILTER              LOCKCHECK                    N/A
LOCKFLAGS               FLAGS                        FLAGS
LOCKOWNER               OWNER                        OWNER
LOCKS                   LOCK                         LOCK
LOG                     LOG                          LOG
LPARENT                 PARENTS                      LPARENT
LPIDS                   PIDS                         N/A
LPLAYERS                LCON                         LCON
N/A                     LOR                          LOR
LPORTS                  PORT                         PORTS
ALIGN                   PRINTF                       LPAD
LPOS                    LPOS                         LPOS
DIE                     DICE                         LRAND
N/A                     LROOMS                       LROOMS
LREPLACE                REPLACE                      REPLACE
LSEARCH                 SEARCH/SEARCHNG              SEARCH
LSEARCHR                revwords(search())           revwords(search())
LSET                    SET                          SET
LSTATS                  STATS                        STATS
LT                      LT                           LT
LTE                     LTE                          LTE
LTHINGS                 LCON                         LCON
LVCON                   LCON + STREVAL at mortal     N/A
LVEXITS                 LCON + STREVAL at mortal     N/A
LVPLAYERS               LCON + STREVAL at mortal     N/A
LVTHINGS                LCON + STREVAL at mortal     N/A
LWHO                    LWHO                         LWHO
LWHOID                  N/A                          N/A
MAIL                    MAILREAD/MAILSEND            MAIL
MAILDSTATS              MAILSIZE/MAILQUOTA           MAILSIZE
MAILFROM                MAILREAD                     MAILFROM
MAILFSTATS              FOLDERLIST/FOLDERCURRENT     N/A
MAILLIST                MAILREAD                     N/A
MAILSEND                MAILSEND                     N/A
MAILSTATS               MAILSIZE/MAILQUOTA           MAILSIZE
MAILSTATUS              MAILSIZE/MAILQUOTA           MAILSIZE
MAILSUBJECT             MAILREAD                     MAILSUBJ
MAILTIME                MAILREAD                     N/A
MALIAS                  MAILREAD                     N/A
MAP                     MAP                          MAP
MAPSQL                  N/A                          N/A
MATCH                   MATCH                        MATCH
MATCHALL                MATCHALL                     MATCHALL
MAX                     MAX                          MAX
MEAN                    AVG                          AVG
MEDIAN                  AVG                          AVG
MEMBER                  MEMBER                       MEMBER
MERGE                   MERGE                        MERGE
MESSAGE                 PARSESTR + PEMIT/REMIT       N/A
MID                     MID                          MID
MIN                     MIN                          MIN
MIX                     MIX                          MIX
MODULO                  MOD                          MOD
MONEY                   MONEY                        MONEY
MONIKER                 CNAME                        MONIKER
N/A                     N/A                          MOTD
MSECS                   MODIFYTIME + CONVTIME        MTIME + CONVTIME
MTIME                   MODIFYTIME                   MTIME
MUDNAME                 MUDNAME                      MUDNAME
MUDURL                  N/A                          N/A
MUL                     MUL                          MUL
MUNGE                   MUNGE                        MUNGE
MWHO                    LWHO + STREVAL at mortal     N/A
MWHOID                  N/A                          N/A
NAME                    NAME                         NAME
NAMEGRAB                @function (softfunctions)    N/A
NAMEGRABALL             @function (softfunctions)    N/A
NAMELIST                @function (softfunctions)    N/A
NAND                    NAND                         N/A
NATTR                   ATTRCNT                      ATTRCNT
NATTRP                  ATTRCNT                      ATTRCNT
NCAND                   !CAND                        NOT(CAND())
NCHILDREN               CHILDREN                     CHILDREN
NCON                    WORDS + LCON                 WORDS + LCON
NCOND                   @function (softfunctions)    N/A
NCONDALL                @function (softfunctions)    N/A
NCOR                    !COR                         NOT(COR())
NEARBY                  NEARBY                       NEARBY
NEQ                     NEQ                          NEQ
NEXITS                  WORDS + LEXITS               WORDS + LEXITS
NEXT                    NEXT                         NEXT
NEXTDBREF               N/A                          N/A
NLSEARCH                WORDS + SEARCH               WORDS + SEARCH
NMWHO                   WORDS+LWHO+STREVAL at mort   N/A
NOR                     NOR                          N/A
NOT                     NOT or !                     NOT
NPLAYERS                WORDS + LCON                 WORDS + LCON
NSCEMIT                 N/A                          N/A
NSEARCH                 WORDS + SEARCH               WORDS + SEARCH
NSEMIT                  EMIT                         EMIT
NSLEMIT                 LEMIT                        N/A
NSOEMIT                 OEMIT                        N/A
NSPEMIT                 PEMIT                        N/A
NSPROMPT                N/A (@program?)              N/A (@program?)
NSREMIT                 REMIT                        N/A
NSZEMIT                 ZEMIT                        N/A
NTHINGS                 WORDS + LCON                 WORDS + LCON
NULL                    NULL                         NULL
NUM                     NUM                          NUM
NUMVERSION              N/A                          N/A
NVCON                   WORDS+LCON+STREVAL at mort   N/A
NVEXITS                 WORDS+LEXITS+STREVAL at mo   N/A
NVPLAYERS               WORDS+LCON+STREVAL at mort   N/A
NVTHINGS                WORDS+LCON+STREVAL at mort   N/A
NWHO                    WORDS + LWHO                 WORDS + LWHO
OBJ                     OBJ                          OBJ
OBJEVAL                 OBJEVAL                      OBJEVAL
OBJID                   N/A                          N/A
OBJMEM                  SIZE                         OBJMEM
OEMIT                   OEMIT                        OEMIT
OPEN                    OPEN                         N/A
OR                      OR                           OR/ORBOOL
ORD                     ASC                          ORD
ORDINAL                 N/A                          N/A
ORFLAGS                 ORFLAGS                      ORFLAGS
ORLFLAGS                ORFLAG                       N/A
ORLPOWERS               N/A (easy to @function)      N/A
OWNER                   OWNER                        OWNER
PARENT                  PARENT                       PARENT
PCREATE                 CREATE                       CREATE
PEMIT                   PEMIT                        PEMIT
PFUN                    U + PARENT                   U + PARENT
PI                      PI                           PI
PIDINFO                 PID                          N/A
PLAYER                  before(grab(lwho(1),*:\%0),:) N/A
PLAYERMEM               SIZE                         PLAYMEM
PMATCH                  PMATCH                       PMATCH
POLL                    DOING                        POLL
PORTS                   PORT                         PORTS
POS                     POS                          POS
POSS                    POSS                         POSS
POWER                   POWER                        POWER
POWERS                  LPOWERS                      POWERS
PROMPT                  N/A (@program?)              N/A (@program?)
PUEBLO                  N/A                          N/A
QUOTA                   QUOTA                        N/A
R                       R                            R
RAND                    RAND                         RAND
RANDWORD                PICKRAND                     PICKRAND
RECV                    CHARIN                       N/A
REGEDIT                 REGEDIT                      N/A
REGEDITALL              REGEDITALL                   N/A
REGEDITALLI             REGEDITALLI                  N/A
REGEDITI                REGEDITI                     N/A
REGISTERS               N/A                          N/A
REGLATTR                LATTR                        N/A
REGLATTRP               LATTRP                       N/A
REGLMATCH               REGLMATCH                    N/A
REGLMATCHALL            REGLMATCHALL                 N/A
REGLMATCHALLI           REGLMATCHALLI                N/A
REGLMATCHI              REGLMATCHI                   N/A
REGMATCH                REGMATCH                     REGMATCH
REGMATCHI               REGMATCHI                    REGMATCHI
REGNATTR                WORDS + ATTR                 N/A
REGNATTRP               WORDS + ATTRP                N/A
REGRAB                  REGRAB                       REGRAB
REGRABALL               REGRABALL                    REGRABALL
REGRABALLI              REGRABALLI                   REGRABALLI
REGRABI                 REGRABI                      REGRABI
REGREP                  REGREP                       N/A
REGREPI                 REGREPI                      N/A
REGXATTR                ATTR                         N/A
REGXATTRP               ATTRP                        N/A
REMAINDER               REMAINDER                    REMAINDER
REMIT                   REMIT                        REMIT
REMOVE                  REMOVE                       REMOVE
RENDER                  N/A                          N/A
REPEAT                  REPEAT                       REPEAT
REST                    REST                         REST
RESTARTS                N/A                          RESTARTS
RESTARTTIME             REBOOTTIME                   RESTARTTIME
CONVTIME(RESTARTTIME))  CONVTIME(REBOOTTIME())       RESTARTSECS
RESWITCH                RESWITCH                     N/A
RESWITCHALL             RESWITCHALL                  N/A
RESWITCHALLI            RESWITCHALLI                 N/A
RESWITCHI               RESWITCHI                    N/A
REVWORDS                REVWORDS                     REVWORDS
RIGHT                   RIGHT                        RIGHT
RJUST                   RJUST                        RJUST
RLOC                    RLOC                         RLOC
N/A                     ROMAN                        ROMAN
RNUM                    RNUM                         N/A
ROOM                    ROOM                         ROOM
ROOT                    N/A                          N/A
ROUND                   ROUND                        ROUND
ALIGN                   PRINTF                       RPAD
S                       S                            S
SCAN                    N/A                          N/A
SCRAMBLE                SCRAMBLE                     SCRAMBLE
SECS                    SECS                         SECS
SECURE                  SECURE/SECUREX               SECURE
SENT                    CHAROUT                      N/A
SET                     SET                          SET
SETDIFF                 SETDIFF                      SETDIFF
SETINTER                SETINTER                     SETINTER
SETQ                    SETQ                         SETQ
SETR                    SETR                         SETR
SETUNION                SETUNION                     SETUNION
SHA0                    DIGEST                       DIGEST
DIGEST                  DIGEST                       SHA1
SHL                     SHL                          SHL
SHR                     SHR                          SHR
SHUFFLE                 SHUFFLE                      SHUFFLE
SIGN                    NCOMP(\%0,0)                  SIGN
SIN                     SIN                          SIN
SLEV                    N/A                          N/A
@function               @function                    SITEINFO
SORT                    SORT                         SORT
SORTBY                  SORTBY                       SORTBY
SORTKEY                 @function (softfunctions)    N/A
SOUNDEX                 SOUNDEX                      N/A
SOUNDSLIKE              SOUNDXLIKE                   N/A
SPACE                   SPACE                        SPACE
SPEAK                   PARSESTR                     N/A
SPEAKPENN               PARSESTR                     N/A
SPELLNUM                SPELLNUM                     SPELLNUM
SPLICE                  SPLICE                       SPLICE
SQL                     (if MYSQL enabled)           N/A (ASYNC db)
SQLESCAPE               (if MYSQL enabled)           N/A (ASYNC db)
SQRT                    SQRT                         SQRT
SQUISH                  SQUISH                       SQUISH
SSL                     N/A                          N/A
STARTTIME               STARTTIME                    STARTTIME
CONVTIME(STARTTIME))    CONVTIME(STARTTIME())        STARTSECS
N/A                     STATS                        STATS
STDDEV                  AVG                          AVG
STEP                    STEP                         STEP
STEXT                   N/A                          N/A
STRALLOF                OFPARSE                      N/A
STRCAT                  STRCAT                       STRCAT
N/A                     STRIP                        STRIP
STRDELETE               CREPLACE/DELETE              DELETE
STRFIRSTOF              OFPARSE                      N/A
STRINGSECS              @function (softfunctions)    N/A
STRINSERT               CREPLACE                     N/A
STRIPACCENTS            STRIPACCENTS                 STRIPACCENTS
STRIPANSI               STRIPANSI                    STRIPANSI
STRLEN                  STRLEN                       STRLEN
STRMATCH                STRMATCH                     STRMATCH
N/A                     STRLENRAW                    STRMEM
STRREPLACE              CREPLACE/REPLACE             REPLACE
SUB                     SUB                          SUB
N/A                     ESCAPEX                      SUBEVAL
SUBJ                    SUBJ                         SUBJ
N/A                     N/A                          SUCCESSES
SWITCH                  SWITCH                       SWITCH
SWITCHALL               SWITCHALL                    N/A
T                       T                            T
TABLE                   @function (softfunctions)    TABLE
TAG                     N/A                          N/A
TAGWRAP                 N/A                          N/A
TAN                     TAN                          TAN
TEL                     TEL                          TEL
TERMINFO                N/A                          TERMINFO
TESTLOCK                LOCKCHECK                    N/A
TEXTENTRIES             WORDS + TEXTFILE             WORDS + TEXTFILE
TEXTFILE                TEXTFILE                     TEXTFILE
TIME                    TIME                         TIME
TIMEFMT                 PTIMEFMT                     TIMEFMT
TIMESTRING              SINGLETIME/TIMEFMT           SINGLETIME
TR                      TR                           TR
TRIM                    TRIM                         TRIM
TRIMPENN                TRIM                         TRIM
TRIMTINY                TRIM                         TRIM
N/A                     N/A                          TRIGGER
TRUNC                   TRUNC                        TRUNC
TYPE                    TYPE                         TYPE
UCSTR                   UCSTR                        UCSTR
UDEFAULT                UDEFAULT                     UDEFAULT
UFUN                    U                            U
ULAMBDA                 U + \#lambda                  N/A
ULDEFAULT               ULDEFAULT                    N/A
ULOCAL                  ULOCAL                       ULOCAL
UNIQUE                  LISTDIFF/LISTUNION/LISTINTER N/A
UNSETQ                  N/A                          N/A
UPTIME                  N/A                          N/A
UTCTIME                 TIME                         TIME
V                       V                            V
VADD                    VADD                         VADD
VALID                   VALID                        VALID
VCROSS                  VCROSS                       VCROSS
VDIM                    VDIM                         VDIM
VDOT                    VDOT                         VDOT
VERSION                 VERSION                      VERSION
VISIBLE                 VISIBLE                      VISIBLE
VMAG                    VMAG                         VMAG
VMAX                    SORTLIST                     N/A
VMIN                    SORTLIST                     N/A
VMUL                    VMUL                         VMUL
VSUB                    VSUB                         VSUB
VUNIT                   VUNIT                        VUNIT
WHERE                   WHERE                        WHERE
WIDTH                   @function (softfunctions)    WIDTH
WILDGREP                GREP                         N/A
WILDGREPI               GREPI                        N/A
WIPE                    WIPE                         WIPE
WORDPOS                 WORDPOS                      WORDPOS
WORDS                   WORDS/MWORDS                 WORDS
WRAP                    WRAP                         WRAP
N/A                     MODIFYTIME                   WRITETIME
XATTR                   ATTR                         N/A
XATTRP                  ATTRP                        N/A
XCON                    XCON                         N/A
XEXITS                  LEXITS + EXTRACT             N/A
XGET                    XGET                         XGET
XMWHO                   LWHO+EXTRACT+STREVAL at mor  N/A
XMWHOID                 N/A                          N/A
XOR                     XOR                          XOR
XPLAYERS                XCON                         N/A
XTHINGS                 XCON                         N/A
XVCON                   XCON + STREVAL at mortal     N/A
XVEXITS                 LEXITS + STREVAL at mortal   N/A
XVPLAYERS               XCON + STREVAL at mortal     N/A
XVTHINGS                XCON + STREVAL at mortal     N/A
XWHO                    LWHO + EXTRACT               LWHO + EXTRACT
XWHOID                  N/A                          N/A
ZEMIT                   ZEMIT                        N/A
ZFUN                    ZFUN                         ZFUN
ZMWHO                   ZWHO + STREVAL at mortal     N/A
ZONE                    LZONE                        ZONE
ZWHO                    ZWHO                         ZWHO

\end{description}


\subsubsection{Functions that only exist in RhostMUSH}
\label{\detokenize{differences:functions-that-only-exist-in-rhostmush}}
\sphinxAtStartPar
Functions that exist in Rhost that do not have a match in PennMUSH:
AIINDEX                      AINDEX                        ANDCHR
ARRAY                        ATTRCNT                       BETWEEN
BITTYPE                      CANSEE                        CAPLIST
CHKGARBAGE                   CHKREALITY                    CHKTRACE
CHOMP                        CITER                         CLOAK
CLUSTER\_ADD                  CLUSTER\_ATTRCNT               CLUSTER\_DEFAULT
CLUSTER\_EDEFAULT             CLUSTER\_FLAGS                 CLUSTER\_GET
CLUSTER\_GET\_EVAL             CLUSTER\_GREP                  CLUSTER\_HASATTR
CLUSTER\_HASFLAG              CLUSTER\_LATTR                 CLUSTER\_REGREP
CLUSTER\_REGREPI              CLUSTER\_SET                   CLUSTER\_STATS
CLUSTER\_U                    CLUSTER\_U2                    CLUSTER\_U2DEFAULT
CLUSTER\_U2LDEFAULT           CLUSTER\_U2LOCAL               CLUSTER\_UDEFAULT
CLUSTER\_UEVAL                CLUSTER\_ULDEFAULT             CLUSTER\_ULOCAL
CLUSTER\_VATTRCNT             CLUSTER\_WIPE                  CLUSTER\_XGET
COLUMNS                      COSH                          COUNTSPECIAL
CRC32                        DELEXTRACT                    DESTROY
EDITANSI                     EE                            ERROR
EXP                          FBETWEEN                      FBOUND
GARBLE                       GLOBALROOM                    GUILD
HASDEPOWER                   HASQUOTA                      HASRXLEVEL
HASTOGGLE                    HASTXLEVEL                    INPROGRAM
INZONE                       ISALNUM                       ISALPHA
ISCLUSTER                    ISDIGIT                       ISHIDDEN
ISLOWER                      ISPUNCT                       ISSPACE
ISUPPER                      ISXDIGIT                      KEEPFLAGS
KEEPTYPE                     LAND                          LAVG
LCMDS                        LDEPOWERS                     LISTMATCH
LISTNEWSGROUPS               LISTRLEVELS                   LISTTOGGLES
LLOC                         LMAX                          LMIN
LMUL                         LNOR                          LOCALFUNC
LOCKDECODE                   LOCKENCODE                    LOGSTATUS
LOGTOFILE                    LOR                           LRAND
LROOMS                       LTOGGLES                      LXNOR
LXOR                         MONEYNAME                     MOON
MOVE                         NAMEQ                         NOSTR
NOTCHR                       NSLOOKUP                      ORCHR
PARENMATCH                   PFIND                         PGREP
POWER10                      PRIVATIZE                     PROGRAMMER
PUSHREGS                     RACE                          RANDMATCH
RANDPOS                      REGEDITALLILIT                REGEDITALLLIT
REGEDITILIT                  REGEDITLIT                    REGNUMMATCH
REGNUMMATCHI                 REMFLAGS                      REMTYPE
ROMAN                        ROTL                          ROTR
RSET                         RXLEVEL                       SAFEBUFF
SEES                         SETQMATCH                     SHIFT
SINH                         SORTLISAT                     STR
STRDISTANCE                  STREQ                         STREVAL
STRFUNC                      STRIP                         STRLENRAW
STRLENVIS                    STRMATH                       SUBNETMATCH
TANH                         TOGGLE                        TOTCMDS
TRACE                        TXLEVEL                       UEVAL
WHILE                        WILDMATCH                     WRAPCOLUMNS
WRITABLE                     XNOR                          XORCHR
XORFLAG                      ZFUNDEFAULT                   ZFUNEVAL
ZFUNLDEFAULT                 ZFUNLOCAL


\subsection{What may need to be modified to get softcode from PennMUSH, TinyMUSH2, TinyMUSH3, or MUX2 to work on Rhost}
\label{\detokenize{differences:what-may-need-to-be-modified-to-get-softcode-from-pennmush-tinymush2-tinymush3-or-mux2-to-work-on-rhost}}
\sphinxAtStartPar
RhostMUSH, for the most part, will work out of the box with most softcode gotten
from other codebases.  There are, however, exceptions.  Most of these exceptions
will be minor code differences between how ANSI is processed, the variences
of arguments or switches to commands or functions, or required flags.

\sphinxAtStartPar
Most changes will revolve around the ones listed in this document.


\subsubsection{Problematic functions between codebases}
\label{\detokenize{differences:problematic-functions-between-codebases}}
\sphinxAtStartPar
lsearch()/search(), align()/printf(), *attrval()


\subsubsection{Problematic features between codebases}
\label{\detokenize{differences:problematic-features-between-codebases}}
\sphinxAtStartPar
named variables for regexp patterns in \$commands are not supported.
@aliases on non\sphinxhyphen{}players are not supported.  Frankly I find them redundant.


\subsubsection{Problematic commands}
\label{\detokenize{differences:problematic-commands}}
\sphinxAtStartPar
@mapsql, hardcoded required comssytem commands (some are redundant)


\subsubsection{SIDEFX flag}
\label{\detokenize{differences:sidefx-flag}}\begin{quote}

\sphinxAtStartPar
Anything that uses sideeffects \textendash{}DIRECTLY\textendash{} requires this flag.
Sideeffects are like set(), pemit(), and so forth.  list(), while a
side\sphinxhyphen{}effect, does not require this flag as it is considered passive and safe.
\end{quote}


\subsubsection{Variable exits}
\label{\detokenize{differences:variable-exits}}\begin{quote}

\sphinxAtStartPar
Rhost handles them slightly different.  You do not link
exits to \#\sphinxhyphen{}4.  That’s an invalid destination.  I always found it, frankly,
stupid to save any data in the database that was literally invalid.  So,
you link the exit as you normally would, then @toggle the exit variable.
At that point you use @exitto like you would any other codebase.
\end{quote}


\subsubsection{Zones}
\label{\detokenize{differences:zones}}\begin{quote}

\sphinxAtStartPar
Zones actually can work near exactly as you would expect them to
work on TinyMUSH, MUX, or Penn.  Either at once or at different times.
We recognize multiple zones, zone masters, zone inheritance, zone
parenting, zone command processing, and the ability to bypass zones
entirely.  There’s a ton of flexbility with this.  However, the syntax
for adding/removing zones is different so the commands will have to be
ported to Rhost.
\end{quote}


\subsubsection{@powers}
\label{\detokenize{differences:powers}}\begin{quote}

\sphinxAtStartPar
Powers work a bit differently in Rhost and they’re named
differently, which should not be that big a surprise as they’re different
between all the codebases anyway.  The big difference is our powers are
tiered, meaning the can be limited or grown to a given bitlevel and are
not just toggle powers like the other codebases.  We also have @depower
that is the anti\sphinxhyphen{}thesis of @power
\end{quote}


\subsubsection{Attribute length}
\label{\detokenize{differences:attribute-length}}\begin{quote}

\sphinxAtStartPar
While we have 64 character attribute capabilities like
most other codebases, PennMUSH allows 1024 attribute length attributes.
Why you need one that long boggles the mind, but if you do use attribs
that long you need to make sure they are cut down to the proper length.
\end{quote}


\subsubsection{Attribute contents}
\label{\detokenize{differences:attribute-contents}}\begin{quote}

\sphinxAtStartPar
You’ll be happy to know that Rhost allows upwards
to 64,000 bytes of data to be assigned an LBUF.  We strongly recommand
to cap at 32,000 however as the various TCP socket protocols play nicer
with that value.
\end{quote}


\subsubsection{256 color}
\label{\detokenize{differences:color}}
\sphinxAtStartPar
Yup!  We got it.


\subsubsection{Unicode/UTF8}
\label{\detokenize{differences:unicode-utf8}}\begin{quote}

\sphinxAtStartPar
Yup!  We got this too.  Not quiet yet in the main branch,
but download Kage’s branch, you won’t be dissapointed.  We will have
UTF8 in Rhost 4.0 when released.
\end{quote}


\subsubsection{Attributes per object}
\label{\detokenize{differences:attributes-per-object}}\begin{quote}

\sphinxAtStartPar
This is configurable with the VLIMIT @admin
command, however, it is absolutely hard\sphinxhyphen{}limited at 10000 maximum.
This is to avoid any DoS type situation and because frankly there
should never be a reason to exceed that.  If you need more, use
@clusters.
\end{quote}


\subsubsection{Destroying}
\label{\detokenize{differences:destroying}}\begin{quote}

\sphinxAtStartPar
@nuke only works on players.  @destroy works on non\sphinxhyphen{}players.
Never the two will meet.  We also have a built in recycle bin meaning
anything destroyed will not be automatically recycled.  If you want it
recycled, you have to @purge it.  Yes, if you use  Myrddin’s CRON, it
has a built in entry to auto\sphinxhyphen{}purge anything older than 30 days.  This
also means you can on\sphinxhyphen{}line recover anything destroyed before that 30
days.  Groovy, eh?
\end{quote}


\subsubsection{object id’s}
\label{\detokenize{differences:object-id-s}}
\sphinxAtStartPar
Yup, we got them.  Even in searches, and, well, everything.


\subsubsection{lsearch() and search()}
\label{\detokenize{differences:lsearch-and-search}}\begin{quote}

\sphinxAtStartPar
lsearch() in Penn is not syntacically similar to non\sphinxhyphen{}Penn search().
This will have to be altered.  In addition, search() in non\sphinxhyphen{}penn games
have to have special consideration for escaping out the evaled args.
\end{quote}


\subsubsection{@locks can be different}
\label{\detokenize{differences:locks-can-be-different}}\begin{quote}

\sphinxAtStartPar
We have many more lock capabilities and options
so this should be a non\sphinxhyphen{}issue.
\end{quote}


\subsubsection{Customer user\sphinxhyphen{}locks}
\label{\detokenize{differences:customer-user-locks}}\begin{quote}

\sphinxAtStartPar
We do not have custom user\sphinxhyphen{}locks like Penn.  We do, however, have the way
to set encapsulated lock data into an attribute to fetch and compare
against which I find more useful and far more flexible.
See: lockencode(), lockdecode(), and lockcheck()
\end{quote}


\subsubsection{Attribute trees}
\label{\detokenize{differences:attribute-trees}}\begin{quote}

\sphinxAtStartPar
Unlike Penn, we don’t really have attribute trees.  We do support the
basic capabilities of it for compatibility if you load in softcode that
uses it, but it doesn’t have the advanced features of attribute trees.
Please see ‘help attribute tree’ for more information.
\end{quote}


\subsubsection{Prefix permission locking}
\label{\detokenize{differences:prefix-permission-locking}}\begin{quote}

\sphinxAtStartPar
We do allow prefix permission locking, and some very advanced abilities
of it.  Please see wizhelp on @aflags for more information.
\sphinxhyphen{} wizhelp @aflags
\sphinxhyphen{} wizhelp atrperms\_max
\sphinxhyphen{} wizhelp atrlock
\sphinxhyphen{} wizhelp atrperms
\end{quote}


\subsubsection{align() and printf()}
\label{\detokenize{differences:align-and-printf}}\begin{quote}

\sphinxAtStartPar
We do not have align().  Most of the code that uses align() will have to
be converted to our printf() (which is compatible but has different syntax)
\end{quote}


\subsubsection{MySQL}
\label{\detokenize{differences:mysql}}\begin{quote}

\sphinxAtStartPar
While we support MySQL, we do not have an async method like MUX2.  This
is just not possible, sorry.
\end{quote}


\subsubsection{Mail System}
\label{\detokenize{differences:mail-system}}\begin{quote}

\sphinxAtStartPar
There are mail wrappers to mimic MUX/TM3 and Penn mail systems.
\end{quote}


\subsubsection{Comsystem}
\label{\detokenize{differences:comsystem}}\begin{quote}

\sphinxAtStartPar
The softcoded comsystem mimics MUX/TM3 and Penn’s comsystem.
\end{quote}


\subsubsection{Various Functions}
\label{\detokenize{differences:various-functions}}\begin{quote}

\sphinxAtStartPar
There is a ‘softcode.minmax’ in the Mushcode directory that loads up a slew
of @function wrappers that will emulate various functions that MUX, Penn, or
TM3 has.  We have the functionality for nearly all of them, but either our
functions have different syntax, or we have different named functions that
duplicate the functionality.  It would be far better to recode it to use
the native functions, but the @function wrappers are there for lazyness :)
\end{quote}


\subsubsection{Empty Attributes}
\label{\detokenize{differences:empty-attributes}}\begin{quote}

\sphinxAtStartPar
Penn allows you to have empty attributes.  Non\sphinxhyphen{}penn codebases do not.
Thus, hasattrval and the like are not needed and should likely just point
to hasattr instead.
\end{quote}


\subsubsection{Player Stats}
\label{\detokenize{differences:player-stats}}\begin{quote}

\sphinxAtStartPar
MUX has some built in ways for player stats.  We do as well but they’re
either done via functions or attribute contents.  Code that requires this
will have to be recoded.
\end{quote}


\subsubsection{Percent Substitutions}
\label{\detokenize{differences:percent-substitutions}}\begin{quote}

\sphinxAtStartPar
Some percent substitutions may differ between codebases.  Luckily, Rhost
allows the ability to remap or creaete new ones if this is a problem.
\end{quote}


\subsubsection{Switches}
\label{\detokenize{differences:switches}}\begin{quote}

\sphinxAtStartPar
Some switches may not exist in Rhost that do in other codebases, in such
a case, Rhost does allow the ability to @hook a command to define your own
softcoded switch to a hardcoded command and work around the limitation.
\end{quote}


\subsubsection{Flags}
\label{\detokenize{differences:id1}}\begin{quote}

\sphinxAtStartPar
Some flags may be missing.  If it’s a dummy flag, feel free to use the
marker flags MARKER0 to MARKER9 to set them.  If it’s an existing flag
that does similar features, feel free to flag\_alias it or just flag\_name
it to the other name if you want.
\end{quote}


\subsubsection{Aliases}
\label{\detokenize{differences:aliases}}\begin{quote}

\sphinxAtStartPar
Multiple aliases are supported via @protect.
\end{quote}


\subsection{Things other mushes can do that Rhost can not and how to emulate it}
\label{\detokenize{differences:things-other-mushes-can-do-that-rhost-can-not-and-how-to-emulate-it}}

\subsubsection{PennMUSH}
\label{\detokenize{differences:id2}}\begin{itemize}
\item {} 
\sphinxAtStartPar
Attribute trees.  These are emulated as a base set and can be duplicated enough to at least port code.

\item {} 
\sphinxAtStartPar
lsearch() will have to be recoded to search()

\item {} 
\sphinxAtStartPar
align() will have to be recoded to printf()

\item {} 
\sphinxAtStartPar
Penn allows empty attributes.  Rhost does not.  Work will have to be done to take this into consideration.

\item {} 
\sphinxAtStartPar
All *val() functions in Penn that are used will have to be remapped to a non\sphinxhyphen{}*val() function.  It should be as simple as function\_alias to the non\sphinxhyphen{}*val.  Example:

\begin{sphinxVerbatim}[commandchars=\\\{\}]
\PYG{n+nd}{@admin} \PYG{n}{function\PYGZus{}alias}\PYG{o}{=}\PYG{n}{hasattrval} \PYG{n+nb}{hasattr}
\end{sphinxVerbatim}

\item {} 
\sphinxAtStartPar
Penn’s hardcoded comsystem is emulated with the softcode comsys

\item {} 
\sphinxAtStartPar
Penn’s @mail system is workable with mail wrappers

\item {} 
\sphinxAtStartPar
Pueblo is not supported.

\item {} 
\sphinxAtStartPar
json is not supported.

\item {} 
\sphinxAtStartPar
ssl is not natively supported (yet).

\item {} 
\sphinxAtStartPar
This uses @shutdown/restart, Rhost uses @reboot

\end{itemize}


\subsubsection{MUX}
\label{\detokenize{differences:id3}}\begin{itemize}
\item {} 
\sphinxAtStartPar
Mux has an async mysql database engine.  This is not possible with Rhost.  You’ll have to use the sync method instead.

\item {} 
\sphinxAtStartPar
UTF8 is supported but internally passed differently.

\item {} 
\sphinxAtStartPar
Mux’s hardcoded comsystem is emulated with the softcode comsys

\item {} 
\sphinxAtStartPar
MUX’s @mail system is workable with mail wrappers

\item {} 
\sphinxAtStartPar
Pueblo is not supported.

\item {} 
\sphinxAtStartPar
This uses @restart, Rhost uses @reboot

\end{itemize}


\section{Installation}
\label{\detokenize{installation:installation}}\label{\detokenize{installation::doc}}

\subsection{Compiling the code}
\label{\detokenize{installation:compiling-the-code}}
\sphinxAtStartPar
make confsource

\sphinxAtStartPar
Yes, that’s all you have to do.

\sphinxAtStartPar
You may also issue ‘make source’ if the Makefile is already defined how
you want it to be.  Please remember to ‘make clean’ before ‘make source’
whenever you alter the code or import new source code.


\subsubsection{Note about Compiling}
\label{\detokenize{installation:note-about-compiling}}
\sphinxAtStartPar
To install, type:  make confsource

\sphinxAtStartPar
If your binaries do not work or you get an error type:  ./bin/script\_setup.sh
Then type: make confsource

\sphinxAtStartPar
If you are importing a MUX2 flatfile, make ABSOLUTELY SURE that you select
mux passwords as a compatibility option, or you will NOT BE ABLE to log in
to players as the password will not be recognizeable.

\sphinxAtStartPar
Make sure to keep QDBM selected as it’s a much more stable database engine
that does not have attribute limit restrictions like GDBM does.

\sphinxAtStartPar
If you are converting from a Penn, TinyMUSH, or MUX database, make sure you
drill down into the LBUF section and select, at minimum, 8K lbufs.  You likely
want that anyway as it gives you far more room for attribute content storage.

\sphinxAtStartPar
You can go up to 32K safely.  While 64k is safe and does work, there are issues
with networking and older routers that use a 32K TCP buffer size that can
at times cut off the data as overflow resulting in output to the end\sphinxhyphen{}point
players not receiving their data.  So it is strongly recommended not to go
above 32K in lbuffer size.

\sphinxAtStartPar
Go ahead and select 64 char attributes.  It allows you to have 64 characters
for attribute names.  It’s handy to have.

\sphinxAtStartPar
If you wish at this point to set up mysql and/or sqlite, you  may do so.
Yes, you can use them in parallel without any issue.


\subsubsection{Note about Recompiling}
\label{\detokenize{installation:note-about-recompiling}}
\sphinxAtStartPar
If you plan to use ‘make confsource’ to recompile your source, you should first
issue a ‘make clean’ before re\sphinxhyphen{}issuing a ‘make confsource’.  ‘make confsource’
remembers the last options you used.

\sphinxAtStartPar
A failure to issue ‘make clean’ prior to re\sphinxhyphen{}compiling with ‘make confsource’ or
re\sphinxhyphen{}compiling with ‘make source’ can potentially leave stale object files which
may cause unforseen issues when running code, including but not limited to
random crashes.  Generally whenever recompiling it’s good to always make clean
first.


\subsubsection{Note about Patching}
\label{\detokenize{installation:note-about-patching}}
\sphinxAtStartPar
There’s two ways you can look to patch the source.  If you plan to run the
RhostMUSH source from a git repository, then please use the git repo to
constantly update your code.  If you knew enough to want to set up a git repo
then we expect knowledge on how to keep source trees updated in the git repo
to be used the same as any other source distribution.

\sphinxAtStartPar
If, however, you have no idea what a git repo even is, or have no inclination
of using git to manage your RhostMUSH source, or just don’t care one way
or another, then you can use the included patch.sh routine (from under the
Server directory) to patch your source at any time.

\sphinxAtStartPar
From the server directory just type: ./patch.sh

\sphinxAtStartPar
That will auto\sphinxhyphen{}compile your source, auto make all your header files and
essentially keep everything up to date to the latest source.
Once that’s done, all you do from within the game is two commands:
\begin{enumerate}
\sphinxsetlistlabels{\arabic}{enumi}{enumii}{}{.}%
\item {} 
\sphinxAtStartPar
@reboot (or @reboot/silent)  \textendash{} This will load in the new binary

\item {} 
\sphinxAtStartPar
@readcache  \textendash{} This will read in all the .txt file changes

\end{enumerate}


\subsection{Configuring the game}
\label{\detokenize{installation:configuring-the-game}}
\sphinxAtStartPar
When setting up a mush for the first time, make sure you
have all the files configured correctly.  This is with using
the following file for configuration:

\begin{sphinxVerbatim}[commandchars=\\\{\}]
\PYG{o}{\PYGZhy{}} \PYG{n}{netrhost}\PYG{o}{.}\PYG{n}{conf}
\end{sphinxVerbatim}


\subsection{Starting the game}
\label{\detokenize{installation:starting-the-game}}
\sphinxAtStartPar
Once done, you start up the system with the following command:

\begin{sphinxVerbatim}[commandchars=\\\{\}]
 \PYG{o}{\PYGZhy{}} \PYG{p}{[}\PYG{n}{sh}\PYG{o}{/}\PYG{n}{csh}\PYG{p}{]} \PYG{o}{.}\PYG{o}{/}\PYG{n}{Startmush}

\PYG{n}{It} \PYG{n}{will} \PYG{n}{prompt} \PYG{n}{you} \PYG{n}{to} \PYG{n}{start} \PYG{n}{a} \PYG{n}{new} \PYG{n}{db} \PYG{k}{if} \PYG{n}{it} \PYG{n}{doesn}\PYG{l+s+s1}{\PYGZsq{}}\PYG{l+s+s1}{t find one.}

\PYG{n}{You} \PYG{n}{may} \PYG{n}{also} \PYG{n}{do} \PYG{n}{the} \PYG{n}{commands} \PYG{n}{individually}\PYG{p}{:}\PYG{p}{:}

   \PYG{p}{[}\PYG{n}{csh}\PYG{p}{]} \PYG{n}{netrhost} \PYG{o}{\PYGZhy{}}\PYG{n}{s} \PYG{n}{netrhost}\PYG{o}{.}\PYG{n}{conf} \PYG{o}{\PYGZgt{}}\PYG{o}{\PYGZam{}} \PYG{n}{netrhost}\PYG{o}{.}\PYG{n}{log} \PYG{o}{\PYGZam{}}
   \PYG{p}{[}\PYG{n}{sh}\PYG{p}{]}  \PYG{n}{netrhost} \PYG{o}{\PYGZhy{}}\PYG{n}{s} \PYG{n}{netrhost}\PYG{o}{.}\PYG{n}{conf} \PYG{o}{\PYGZgt{}} \PYG{n}{netrhost}\PYG{o}{.}\PYG{n}{log} \PYG{l+m+mi}{2}\PYG{o}{\PYGZgt{}}\PYG{o}{\PYGZam{}}\PYG{l+m+mi}{1} \PYG{o}{\PYGZam{}}
\end{sphinxVerbatim}


\subsection{First login to the game}
\label{\detokenize{installation:first-login-to-the-game}}
\sphinxAtStartPar
Once started, log in the \#1 character (Wizard) with it’s appropiate
password (no, not ‘potrzebie’, but ‘Nyctasia’).  There were private
reasons for the password change.

\sphinxAtStartPar
Once in, do a @shutdown to save the database.  Then you can run Startup
normally.   You may make a backup of your database at anytime on\sphinxhyphen{}line by
utilizing the @dump/flat option.  A script comes with this distribution
that allows the ability of auto\sphinxhyphen{}archiving your database for a configurable
number of backups.


\subsection{Reporting bugs or getting help}
\label{\detokenize{installation:reporting-bugs-or-getting-help}}
\sphinxAtStartPar
If you find any bugs or problems, notify one of the developers of RhostMUSH and
a patch or workaround will be made available as soon as possible.  Current
developers are:  Seawolf, Thorin, Ashen\sphinxhyphen{}Shugar, Lensman, Kale, Mac, Zenty,
Ambrosia, Amos, and Morgan.  They can be found around the net.


\subsection{Installing using an ansible playbook}
\label{\detokenize{installation:installing-using-an-ansible-playbook}}
\sphinxAtStartPar
To begin, you will run the following command in a directory that will house your game:

\begin{sphinxVerbatim}[commandchars=\\\{\}]
\PYG{n}{git} \PYG{n}{clone} \PYG{n}{https}\PYG{p}{:}\PYG{o}{/}\PYG{o}{/}\PYG{n}{github}\PYG{o}{.}\PYG{n}{com}\PYG{o}{/}\PYG{n}{RhostMUSH}\PYG{o}{/}\PYG{n}{trunk} \PYG{n}{Rhost}
\end{sphinxVerbatim}

\sphinxAtStartPar
You may also just run the yml file and ansible (ansible\sphinxhyphen{}playbook) to install your RhostMUSH engine:

\begin{sphinxVerbatim}[commandchars=\\\{\}]
\PYG{n}{wget} \PYG{n}{https}\PYG{p}{:}\PYG{o}{/}\PYG{o}{/}\PYG{n}{raw}\PYG{o}{.}\PYG{n}{githubusercontent}\PYG{o}{.}\PYG{n}{com}\PYG{o}{/}\PYG{n}{RhostMUSH}\PYG{o}{/}\PYG{n}{trunk}\PYG{o}{/}\PYG{n}{master}\PYG{o}{/}\PYG{n}{rhostinstall}\PYG{o}{.}\PYG{n}{yml}
\PYG{n}{ansible}\PYG{o}{\PYGZhy{}}\PYG{n}{playbook} \PYG{n}{rhostinstall}\PYG{o}{.}\PYG{n}{yml}
\end{sphinxVerbatim}

\sphinxAtStartPar
This downloads the latest stable version of the code, bringing with it all patches and scripts, documentation and support tools that you will need.


\subsection{Quickinstall guide to RhostMUSH}
\label{\detokenize{installation:quickinstall-guide-to-rhostmush}}

\subsubsection{Compiling RhostMUSH}
\label{\detokenize{installation:compiling-rhostmush}}
\sphinxAtStartPar
Once ready to compile type:

\begin{sphinxVerbatim}[commandchars=\\\{\}]
\PYG{n}{make} \PYG{n}{confsource}
\end{sphinxVerbatim}

\sphinxAtStartPar
This will bring up a menu where you can selection options.


\subsubsection{Important before you actually start building}
\label{\detokenize{installation:important-before-you-actually-start-building}}
\sphinxAtStartPar
The main parts of making your RhostMUSH, easy pleasy:
\#.  The stunnel directory contains TLS/SSL connectivity.  This has to be linked to another port and will tunnel to the mush port.  The README file explains how to set up and configure your TLS/SSL connection.
\#.  ./patch.sh \textendash{} This makes sure you have the latest code.  If you got this by git clone \sphinxurl{https://github.com/RhostMUSH/trunk} then you can ignore patching.  You can use ./patch.sh at any time to update your code.  It ignores local.c incase you make your own modules.
\#.  make confsource.  Yup, it’s menu driven, nifty eh?
\begin{enumerate}
\sphinxsetlistlabels{\arabic}{enumi}{enumii}{}{.}%
\item {} 
\sphinxAtStartPar
Options you may want to select (other than the defaults):

\item {} 
\sphinxAtStartPar
5  (\%c is selected by default, but choose \%x as well for MUX/TM3 compat)

\item {} 
\sphinxAtStartPar
9  (if you want \$commands to require the COMMAND flag)

\item {} 
\sphinxAtStartPar
16 (if you want a wider WHO listing like older versions of MUX)

\item {} 
\sphinxAtStartPar
22 (if you’re converting a TinyMUSH3 or TinyMUX/MUX2 flatfile)

\item {} 
\sphinxAtStartPar
24 (if you have issues with \sphinxhyphen{}lssl not being found)

\item {} 
\sphinxAtStartPar
B3 (for 64 character attribute names)

\item {} 
\sphinxAtStartPar
B6 (select 8K for Penn/MUX2/TM3 default, up to 32K.  64K is network intensive)

\item {} 
\sphinxAtStartPar
B5 (will be autoselected if you choose 8K or more.  Pick this anyway)

\item {} 
\sphinxAtStartPar
B4 (if you have sqlite libraries and wish to use this)

\end{enumerate}
\begin{enumerate}
\sphinxsetlistlabels{\arabic}{enumi}{enumii}{}{.}%
\item {} 
\sphinxAtStartPar
‘r’ to compile with the settings you selected.

\item {} 
\sphinxAtStartPar
Modify your netrhost.conf file as specified.  Make sure to align your port and debug\_id as shown in the netrhost.conf file.

\item {} 
\sphinxAtStartPar
If you wish to port in an old flatfile, please refer to the readme directory on how to port your flatfile in (README.DBLOADING).

\end{enumerate}


\subsubsection{Using the prebuilt flatfile}
\label{\detokenize{installation:using-the-prebuilt-flatfile}}
\sphinxAtStartPar
There are pre\sphinxhyphen{}loaded flatfile databases you can use at this point.  The netrhost.db.flat
and corrisponding netrhost.conf file will be located in the minimal\sphinxhyphen{}DBs/minimal\_db directory.

\sphinxAtStartPar
You may auto\sphinxhyphen{}load the minimal db and corresponding netrhost.conf file with the command:

\begin{sphinxVerbatim}[commandchars=\\\{\}]
\PYG{o}{.}\PYG{o}{/}\PYG{n}{minimal}\PYG{o}{.}\PYG{n}{sh}
\end{sphinxVerbatim}

\sphinxAtStartPar
This is ran from within the ‘game’ directory.  Once this is ran, you will need
to customize the netrhost.conf file for your purposes.  The port and debug\_id must
be changed at the very least.  Keep the debug\_id coordinated to the port as described.


\paragraph{To load a prebuilt flatfile}
\label{\detokenize{installation:to-load-a-prebuilt-flatfile}}
\sphinxAtStartPar
To use these follow these steps:
1.  Make a backup of your existing netrhost.conf file:

\begin{sphinxVerbatim}[commandchars=\\\{\}]
\PYG{n}{cp} \PYG{n}{game}\PYG{o}{/}\PYG{n}{netrhost}\PYG{o}{.}\PYG{n}{conf} \PYG{n}{game}\PYG{o}{/}\PYG{n}{netrhost}\PYG{o}{.}\PYG{n}{conf}\PYG{o}{.}\PYG{n}{backup}
\end{sphinxVerbatim}
\begin{enumerate}
\sphinxsetlistlabels{\arabic}{enumi}{enumii}{}{.}%
\setcounter{enumi}{1}
\item {} 
\sphinxAtStartPar
Copy the netrhost.conf file into your game directory:

\begin{sphinxVerbatim}[commandchars=\\\{\}]
\PYG{n}{cp} \PYG{o}{\PYGZhy{}}\PYG{n}{f} \PYG{o}{.}\PYG{o}{/}\PYG{n}{minimal}\PYG{o}{\PYGZhy{}}\PYG{n}{DBs}\PYG{o}{/}\PYG{n}{minimal\PYGZus{}db}\PYG{o}{/}\PYG{n}{netrhost}\PYG{o}{.}\PYG{n}{conf} \PYG{o}{.}\PYG{o}{/}\PYG{n}{game}\PYG{o}{/}\PYG{n}{netrhost}\PYG{o}{.}\PYG{n}{conf}
\end{sphinxVerbatim}

\item {} 
\sphinxAtStartPar
At this point you can modify your netrhost.conf file settings in your game directory.
Using an editor modify the ‘port’ and ‘debug\_id’ respectively in your netrhost.conf as state.
The ‘port’ will be the port the mush listens on, the debug\_id is for the debug\sphinxhyphen{}stack and is
your port with a ‘5’ at the end.  So if your port is 4444, the debug\_id is 44445

\item {} 
\sphinxAtStartPar
Load in the flatfile into the mush (You could do this in the Startmush as well)
Manually:

\begin{sphinxVerbatim}[commandchars=\\\{\}]
\PYG{n}{cd} \PYG{n}{game}
\end{sphinxVerbatim}

\sphinxAtStartPar
./db\_load data/netrhost.gdbm ../minimal\sphinxhyphen{}DBs/minimal\_db/netrhost.db.flat data/netrhost.db.new dwF

\sphinxAtStartPar
Start your mush:

\begin{sphinxVerbatim}[commandchars=\\\{\}]
\PYG{o}{.}\PYG{o}{/}\PYG{n}{Startmush}
\end{sphinxVerbatim}

\sphinxAtStartPar
This will load the db that you loaded.

\sphinxAtStartPar
—————OR——\sphinxhyphen{}

\sphinxAtStartPar
From Startmush:

\begin{sphinxVerbatim}[commandchars=\\\{\}]
\PYG{o}{\PYGZhy{}}\PYG{o}{\PYGZhy{}}\PYG{o}{\PYGZgt{}} \PYG{o}{.}\PYG{o}{/}\PYG{n}{Startmush}
\end{sphinxVerbatim}

\end{enumerate}

\sphinxAtStartPar
when prompted, hit \textless{}RETURN\textgreater{} for searching then select the number of the netrhost.db.flat that is listed as \textasciitilde{}/minimal\sphinxhyphen{}DBs/minimal\_db/netrhost.db.flat


\subsubsection{Starting from scratch with a brand new database}
\label{\detokenize{installation:starting-from-scratch-with-a-brand-new-database}}\begin{enumerate}
\sphinxsetlistlabels{\arabic}{enumi}{enumii}{}{.}%
\item {} 
\sphinxAtStartPar
You can modify your netrhost.conf file settings in your game directory.
Using an editor modify the ‘port’ and ‘debug\_id’ respectively in your netrhost.conf as stated.
The ‘port’ will be the port the mush listens on, the debug\_id is for the debug\sphinxhyphen{}stack and is
your port with a ‘5’ at the end.  So if your port is 4444, the debug\_id is 44445

\item {} 
\sphinxAtStartPar
Start your mush:

\begin{sphinxVerbatim}[commandchars=\\\{\}]
\PYG{o}{\PYGZhy{}}\PYG{o}{\PYGZhy{}}\PYG{o}{\PYGZgt{}} \PYG{o}{.}\PYG{o}{/}\PYG{n}{Startmush}
\end{sphinxVerbatim}

\end{enumerate}

\sphinxAtStartPar
You can use the ‘vi’ editor or ‘nano’ if you like a more menu driven DOS like experience.
You can of course use any other editor you’re familar with.

\sphinxAtStartPar
For a more thorough understanding of how to set things up, keep reading!

\sphinxAtStartPar
If you have syntax issues running ‘make config’, ‘make confsource’
or ‘make bugreport’ please run the script: ./bin/script\_setup.sh

\sphinxAtStartPar
Now… things you may need to do on errors.


\subsubsection{Instructions for starting a new RhostMUSH}
\label{\detokenize{installation:instructions-for-starting-a-new-rhostmush}}

\paragraph{Setup directory permissions}
\label{\detokenize{installation:setup-directory-permissions}}\begin{quote}

\sphinxAtStartPar
run ./dirsetup.sh

\sphinxAtStartPar
This is a simple script that will change file permissions
and directory permissions to properly protect RhostMUSH.
These settings generally work fine out of the box so
you likely won’t even have to set this up if you don’t want to.
\end{quote}


\paragraph{Compile the source code}
\label{\detokenize{installation:compile-the-source-code}}
\sphinxAtStartPar
Make and run the RhostMUSH source:

\begin{sphinxVerbatim}[commandchars=\\\{\}]
\PYG{n}{Type}\PYG{p}{:}  \PYG{n}{make} \PYG{n}{confsource}
\end{sphinxVerbatim}

\sphinxAtStartPar
If you get an error running the script itself:

\begin{sphinxVerbatim}[commandchars=\\\{\}]
\PYG{n+nb}{type}\PYG{p}{:} \PYG{o}{.}\PYG{o}{/}\PYG{n+nb}{bin}\PYG{o}{/}\PYG{n}{script\PYGZus{}setup}\PYG{o}{.}\PYG{n}{sh}

\PYG{n}{Then} \PYG{n+nb}{type}\PYG{p}{:} \PYG{n}{make} \PYG{n}{confsource}

\PYG{n}{After} \PYG{n}{the} \PYG{n+nb}{compile} \PYG{n}{process} \PYG{o+ow}{is} \PYG{n}{done}\PYG{p}{,} \PYG{n}{you} \PYG{n}{should} \PYG{n}{be} \PYG{n}{good} \PYG{n}{to} \PYG{n}{go}\PYG{o}{.}
\PYG{n}{If} \PYG{n}{it} \PYG{n}{complains} \PYG{n}{about} \PYG{n}{missing} \PYG{n}{binaries} \PYG{n}{then} \PYG{n+nb}{type} \PYG{l+s+s1}{\PYGZsq{}}\PYG{l+s+s1}{make links}\PYG{l+s+s1}{\PYGZsq{}}
\end{sphinxVerbatim}


\subparagraph{Manual configuration of source code}
\label{\detokenize{installation:manual-configuration-of-source-code}}
\sphinxAtStartPar
To do manual configuration (skip if the previous step worked for you) And yes, this is a bit of a pain in the bottom, hopefully you will not need this.

\sphinxAtStartPar
You need the following definitions defined to make this work:
\begin{enumerate}
\sphinxsetlistlabels{\arabic}{enumi}{enumii}{}{.}%
\item {} 
\sphinxAtStartPar
TINY\_U, USE\_SIDEEFFECTS, MUX\_INCDEC, ATTR\_HACK

\item {} 
\sphinxAtStartPar
(u()/u2() switched)

\item {} 
\sphinxAtStartPar
(sideeffects)

\item {} 
\sphinxAtStartPar
(inc()/xinc() switched)

\item {} 
\sphinxAtStartPar
(support for \_/\textasciitilde{} attribs)

\end{enumerate}

\sphinxAtStartPar
You only need to do this if you received the RhostMUSH src.  If you received a binary, continue on to the next part.

\sphinxAtStartPar
To compile the code, just type ‘make confsource’.  It will prompt you with settings on what you need to do.  If you just want to quickly hand edit the Makefile, it is in the directory src (full path src/Makefile).  Then you may just run ‘make source’, if you so choose to hand\sphinxhyphen{}edit the Makefile.

\sphinxAtStartPar
After the compile process is done, type ‘make links’!


\subsubsection{Loading a database for your MUSH}
\label{\detokenize{installation:loading-a-database-for-your-mush}}
\sphinxAtStartPar
You now have a choice of optionally starting at a provided database or starting from scratch.


\paragraph{Option: Only perform these steps if using a provided database}
\label{\detokenize{installation:option-only-perform-these-steps-if-using-a-provided-database}}\begin{quote}
\begin{description}
\item[{Copy an existing flatfile and corresponding netrhost.conf file}] \leavevmode
\sphinxAtStartPar
Default provied example:

\begin{sphinxVerbatim}[commandchars=\\\{\}]
\PYG{l+m+mf}{1.}  \PYG{n}{cp} \PYG{n}{game}\PYG{o}{/}\PYG{n}{netrhost}\PYG{o}{.}\PYG{n}{conf} \PYG{n}{game}\PYG{o}{/}\PYG{n}{netrhost}\PYG{o}{.}\PYG{n}{conf}\PYG{o}{.}\PYG{n}{backup}
\PYG{l+m+mf}{2.}  \PYG{n}{cp} \PYG{o}{\PYGZhy{}}\PYG{n}{f} \PYG{n}{minimal}\PYG{o}{\PYGZhy{}}\PYG{n}{DBs}\PYG{o}{/}\PYG{n}{minimal\PYGZus{}db}\PYG{o}{/}\PYG{n}{netrhost}\PYG{o}{.}\PYG{n}{conf} \PYG{n}{game}\PYG{o}{/}\PYG{n}{netrhost}\PYG{o}{.}\PYG{n}{conf}
\PYG{l+m+mf}{3.}  \PYG{n}{cd} \PYG{n}{game}
\PYG{l+m+mf}{4.}  \PYG{o}{.}\PYG{o}{/}\PYG{n}{db\PYGZus{}load} \PYG{n}{data}\PYG{o}{/}\PYG{n}{netrhost}\PYG{o}{.}\PYG{n}{gdbm} \PYG{o}{.}\PYG{o}{.}\PYG{o}{/}\PYG{n}{minimal}\PYG{o}{\PYGZhy{}}\PYG{n}{DBs}\PYG{o}{/}\PYG{n}{minimal\PYGZus{}db}\PYG{o}{/}\PYG{n}{netrhost}\PYG{o}{.}\PYG{n}{db}\PYG{o}{.}\PYG{n}{flat} \PYG{n}{data}\PYG{o}{/}\PYG{n}{netrhost}\PYG{o}{.}\PYG{n}{db}\PYG{o}{.}\PYG{n}{new}
\end{sphinxVerbatim}

\end{description}
\end{quote}


\subsubsection{Configure the netrhost.conf file for your MUSH}
\label{\detokenize{installation:configure-the-netrhost-conf-file-for-your-mush}}\begin{quote}

\sphinxAtStartPar
Go into the game directory and modify the netrhost.conf file
The next step is configuring the mush to your config standards.
There is a file in the game subdirectory called ‘netrhost.conf’.
You hand\sphinxhyphen{}edit this file and just follow what it says each
one does.  It’s very well documented and should give you
great details on what to edit.  For most things, you can
feel comfortable to stick with the defaults unless you wish
to change them.  The port and debug\_id need to be changed.
\end{quote}


\subsubsection{Start the MUSH and login}
\label{\detokenize{installation:start-the-mush-and-login}}
\sphinxAtStartPar
From the game diretory issue:

\begin{sphinxVerbatim}[commandchars=\\\{\}]
\PYG{o}{.}\PYG{o}{/}\PYG{n}{Startmush}
\end{sphinxVerbatim}

\sphinxAtStartPar
To login:

\begin{sphinxVerbatim}[commandchars=\\\{\}]
\PYG{n}{co} \PYG{n}{Wizard} \PYG{n}{Nyctasia}
\end{sphinxVerbatim}


\subsubsection{Option: Things to do once you have connected if you did NOT use a provided database}
\label{\detokenize{installation:option-things-to-do-once-you-have-connected-if-you-did-not-use-a-provided-database}}\begin{enumerate}
\sphinxsetlistlabels{\arabic}{enumi}{enumii}{}{.}%
\item {} 
\sphinxAtStartPar
@dig your master room and in your netrhost.conf file define master\_room to this dbref (without the \#.  So like master\_room 2)

\item {} 
\sphinxAtStartPar
Create an immortal holder charater (@pcreate then @set immortal) Feel free to set up holder characters for all the bittypes which are: GUILDMASTER, ARCHITECT, COUNCILOR, WIZARD, IMMORTAL

\item {} 
\sphinxAtStartPar
@chown/preserve the master room and \#0 to the immortal holder character.

\item {} 
\sphinxAtStartPar
Log into the immortal character

\item {} 
\sphinxAtStartPar
@pcreate all your guest characters and set them up properly.  My suggestion:

\begin{sphinxVerbatim}[commandchars=\\\{\}]
\PYG{n+nd}{@dolist} \PYG{n}{lnum}\PYG{p}{(}\PYG{l+m+mi}{1}\PYG{p}{,}\PYG{l+m+mi}{10}\PYG{p}{)}\PYG{o}{=}\PYG{p}{\PYGZob{}}\PYG{n+nd}{@pcreate} \PYG{n}{Guest}\PYG{c+c1}{\PYGZsh{}\PYGZsh{}=guest;@set *Guest\PYGZsh{}\PYGZsh{}=guest;@desc *Guest\PYGZsh{}\PYGZsh{}=A guest player.;@adisconnect *Guest\PYGZsh{}\PYGZsh{}=home;@lock *Guest\PYGZsh{}\PYGZsh{}=*Guest\PYGZsh{}\PYGZsh{}\PYGZcb{}}

\PYG{n+nd}{@list} \PYG{n}{guest} \PYG{n}{will} \PYG{n}{show} \PYG{n}{your} \PYG{n}{guest} \PYG{n}{characters} \PYG{o+ow}{and} \PYG{k}{if} \PYG{n}{they}\PYG{l+s+s1}{\PYGZsq{}}\PYG{l+s+s1}{re set up properly.}
\end{sphinxVerbatim}

\item {} 
\sphinxAtStartPar
Any master room code you load in from your immholder character (or @chown/preserve to it) The readme directory has softfunctions.minmax that has MUX/Penn compatability functions and comsys.  All other softcode (like mail wrappers) can be found on \sphinxurl{https://github.com/RhostMUSH/trunk} in Mushcode.

\end{enumerate}


\subsubsection{Setup new character, staff, and take tasks that can only be accomplished by \#1}
\label{\detokenize{installation:setup-new-character-staff-and-take-tasks-that-can-only-be-accomplished-by-1}}
\sphinxAtStartPar
Set up any other characters you want.  Anyone immortal can issue @function, @admin, or anything \#1 can do.


\subsubsection{Setup daily backups for your game}
\label{\detokenize{installation:setup-daily-backups-for-your-game}}
\sphinxAtStartPar
Make SURE YOU RUN DAILY Backups.  Rhost is very stable, but things outside the mush can damage the game. paranoia is fine, especially when they really are out to get you.  TO make the backups, do the following:

\begin{sphinxVerbatim}[commandchars=\\\{\}]
\PYG{n+nd}{@dump}\PYG{o}{/}\PYG{n}{flat}      \PYG{o}{\PYGZhy{}}\PYG{o}{\PYGZhy{}} \PYG{n}{This} \PYG{n}{makes} \PYG{n}{a} \PYG{n}{flatfile} \PYG{n}{dump} \PYG{n}{of} \PYG{n}{the} \PYG{n}{main} \PYG{n}{database}\PYG{o}{.}  \PYG{n}{You} \PYG{n}{want} \PYG{n}{to} \PYG{n}{run} \PYG{n}{this} \PYG{n}{daily}\PYG{o}{.}
\PYG{n}{wmail}\PYG{o}{/}\PYG{n}{unload}    \PYG{o}{\PYGZhy{}}\PYG{o}{\PYGZhy{}} \PYG{n}{This} \PYG{n}{makes} \PYG{n}{a} \PYG{n}{flatfile} \PYG{n}{dump} \PYG{n}{of} \PYG{n}{the} \PYG{n}{mail} \PYG{n}{database}\PYG{o}{.}  \PYG{n}{You} \PYG{n}{want} \PYG{n}{to} \PYG{n}{run} \PYG{n}{this} \PYG{n}{daily}\PYG{o}{.}
\PYG{n+nd}{@areg}\PYG{o}{/}\PYG{n}{unload}    \PYG{o}{\PYGZhy{}}\PYG{o}{\PYGZhy{}} \PYG{n}{Only} \PYG{n}{worry} \PYG{n}{about} \PYG{n}{this} \PYG{k}{if} \PYG{n}{you} \PYG{n}{are} \PYG{n}{using} \PYG{n}{auto}\PYG{o}{\PYGZhy{}}\PYG{n}{registration} \PYG{n}{emailing}\PYG{o}{.}  \PYG{n}{Few} \PYG{n}{do}\PYG{o}{.}
\PYG{n}{newsdb}\PYG{o}{/}\PYG{n}{unload}   \PYG{o}{\PYGZhy{}}\PYG{o}{\PYGZhy{}} \PYG{n}{Only} \PYG{n}{worry} \PYG{k}{if} \PYG{n}{you} \PYG{n}{use} \PYG{n}{the} \PYG{n}{hardcoded} \PYG{n}{bbs} \PYG{n}{system}\PYG{o}{.}  \PYG{n}{Most} \PYG{n}{don}\PYG{l+s+s1}{\PYGZsq{}}\PYG{l+s+s1}{t use it.}
\end{sphinxVerbatim}

\sphinxAtStartPar
The backup\_flat.sh script (that launches automatically with Startmush) will archive all the above files if they exist.  It moves these flatfiles into the ‘prevflat’ directory, then tarballs those and dumps consecutive backups in the ‘oldflat’ directory.  By default it keeps 7 consecutive backups.  You may alter this in the backup\_flat.sh script itself.


\subsubsection{Customtize the textfiles for your game}
\label{\detokenize{installation:customtize-the-textfiles-for-your-game}}\begin{quote}

\sphinxAtStartPar
All connect.txt and customized files can be found in the \textasciitilde{}/Server/game/txt directory.  There is a
README file there that explains their purposes in more detail.  You can see more information on
all files and how they inter\sphinxhyphen{}relate with ‘wizhelp file’.
\end{quote}


\subsubsection{Make sure to read up further}
\label{\detokenize{installation:make-sure-to-read-up-further}}\begin{quote}

\sphinxAtStartPar
The wiz bits can be confusing, so ‘wizhelp control’ is very helpful to give a high overview
of what each bit does and their inter\sphinxhyphen{}relationship is.
\end{quote}


\subsection{What to type to configure and get your RhostMUSH up and running}
\label{\detokenize{installation:what-to-type-to-configure-and-get-your-rhostmush-up-and-running}}
\sphinxAtStartPar
You may configure Rhost three ways.


\subsubsection{Creating a new game with a blank database}
\label{\detokenize{installation:creating-a-new-game-with-a-blank-database}}\begin{quote}

\sphinxAtStartPar
Modify your ./game/netrhost.conf file or what settings you want.
Don’t feel overwhelmed, it’s all very well documented.
\end{quote}


\subsubsection{Creating a new game with Ambrosia’s default database}
\label{\detokenize{installation:creating-a-new-game-with-ambrosia-s-default-database}}
\sphinxAtStartPar
Follow minimal\sphinxhyphen{}DBs/Amb\sphinxhyphen{}MinimalRhost/IMPORTANT\_README

\sphinxAtStartPar
The netrhost.conf file you will copy is in minimal\sphinxhyphen{}DBs/Amb\sphinxhyphen{}MinimalRhost/game
Copy this netrhost.conf file into your ‘game’ directory.

\sphinxAtStartPar
You will want the custom txt files under Amb\sphinxhyphen{}MinimalRhost/txt in your game/txt directory and to mkindx all the txt files.  You can run ./Startmush \sphinxhyphen{}i to index.

\sphinxAtStartPar
When ./Startmush prompts you to load a flatfile, say ‘yes’ and hit \textless{}RETURN\textgreater{} to have it search for flatfiles, then select netrhost.db.flat from under the minimal\sphinxhyphen{}DBs/Amb\sphinxhyphen{}MinimalRhost directory.

\sphinxAtStartPar
The main steps to make sure you do for \textasciitilde{}/Server/minimal\sphinxhyphen{}DBs/Amb\sphinxhyphen{}MinimalRhost/netrhost.db.flat \textendash{} Ambrosia’s secure and featured minimal db
\begin{enumerate}
\sphinxsetlistlabels{\arabic}{enumi}{enumii}{}{.}%
\item {} 
\sphinxAtStartPar
Use the matching netrhost.conf file under the Amb\sphinxhyphen{}MinimalRhost/game directory

\item {} 
\sphinxAtStartPar
Load in the settings specified in the Amb\sphinxhyphen{}MinimalRhost/bin directory.
\begin{enumerate}
\sphinxsetlistlabels{\arabic}{enumii}{enumiii}{}{.}%
\item {} 
\sphinxAtStartPar
Copy this file into your \textasciitilde{}/Server/bin directory

\item {} 
\sphinxAtStartPar
From ‘Server’ directory type: make clean

\item {} 
\sphinxAtStartPar
From ‘Server’ directory type: make confsource and ‘l’oad option 0

\item {} 
\sphinxAtStartPar
Specify any \sphinxhyphen{}additional\sphinxhyphen{} options you want at this point.

\item {} 
\sphinxAtStartPar
Recompile your code

\end{enumerate}

\item {} 
\sphinxAtStartPar
Copy the files in Amb\sphinxhyphen{}MinimalRhost/game/txt into your \textasciitilde{}/Server/game/txt directory

\item {} 
\sphinxAtStartPar
from your \textasciitilde{}/Server/game txt file run on each of the txt files:

\begin{sphinxVerbatim}[commandchars=\\\{\}]
\PYG{o}{.}\PYG{o}{.}\PYG{o}{/}\PYG{n}{mkindx} \PYG{o}{\PYGZlt{}}\PYG{n}{txtfile}\PYG{o}{\PYGZgt{}}\PYG{o}{.}\PYG{n}{txt} \PYG{o}{\PYGZlt{}}\PYG{n}{txtfile}\PYG{o}{\PYGZgt{}}\PYG{o}{.}\PYG{n}{indx}
\PYG{n}{Where} \PYG{o}{\PYGZlt{}}\PYG{n}{txtfile}\PYG{o}{\PYGZgt{}} \PYG{o+ow}{is} \PYG{n}{the} \PYG{n}{name} \PYG{n}{of} \PYG{n}{the} \PYG{n}{file} \PYG{p}{(}\PYG{n}{minus} \PYG{n}{the} \PYG{o}{.}\PYG{n}{txt} \PYG{n}{extension}\PYG{p}{)}
\end{sphinxVerbatim}

\item {} 
\sphinxAtStartPar
If running, @reboot your game.

\end{enumerate}


\subsubsection{Creating a new game with the generic default database}
\label{\detokenize{installation:creating-a-new-game-with-the-generic-default-database}}\begin{quote}

\sphinxAtStartPar
Copy the netrhost.conf from minimal\sphinxhyphen{}DBs/minimal\_db to your game directory.

\sphinxAtStartPar
When ./Startmush prompts you to load a flatfile, say ‘yes’ and hit \textless{}RETURN\textgreater{}
to have it search for flatfiles, then select netrhost.db.flat from under
the minimal\sphinxhyphen{}DBs/minimal\_db directory.
\end{quote}


\subsubsection{Starting your MUSH}
\label{\detokenize{installation:starting-your-mush}}
\sphinxAtStartPar
Once you have used one of these three methods to obtaina database, you can start your mush up.
At this point type from the game directory:

\begin{sphinxVerbatim}[commandchars=\\\{\}]
\PYG{o}{.}\PYG{o}{/}\PYG{n}{Startmush}
\end{sphinxVerbatim}


\subsubsection{Backups for RhostMUSH}
\label{\detokenize{installation:backups-for-rhostmush}}\begin{quote}

\sphinxAtStartPar
Backups are already handled and integrated with a script ‘backup\_flat.sh’.
If you wish to customize this, feel free.  Again, it is well documented and
just require changing settings at the top of this script.

\sphinxAtStartPar
By default, it does 7 contiguous backups.  You may increase or decrease
this value to any value you want.

\sphinxAtStartPar
It will, by default, backup all your txt/\sphinxstyleemphasis{.txt files, your netrhost.conf
file, your netrhost.db.flat (mush db) file, your RhostMUSH.dump.}
(mail db) files, your RhostMUSH.news.* (internal news/bbs db \textendash{} if used),
your RhostMUSH.areg.* (the autoregistration db \textendash{} if used), and any sqlite
database you currently may be using which are OPTIONALLY backed up if you
remove the ‘\#’ from before it.

\sphinxAtStartPar
The backup script also will optionally rcp/scp, ftp, or mail any backups
you want to a remote destination.  Be forewarned, the backup files can
potentially get rather large for larger games, even compressed.  The
average size for these files will be 1\sphinxhyphen{}5MB.  It could potentially get
over 10\sphinxhyphen{}20MB in size for excessively large games, so plan accordingly.

\sphinxAtStartPar
Be aware that the backup system will NOT make successful backups if you
run out of disk space.  This includes actually running out of disk space
or running out of disk quota.  There is a mechanism inside the backup
script to specify an email address that you wish to get alerts from
in these instances.  I recommend using it.

\sphinxAtStartPar
If you make changes to your backup\_flat.sh script with an already
active and running mush and wish to just restart the backup procedure
just issue: ./backup\_restart.sh
\end{quote}


\subsubsection{Troubleshooting issues with starting up}
\label{\detokenize{installation:troubleshooting-issues-with-starting-up}}

\paragraph{Problem: If it says the shared ID is already in use}
\label{\detokenize{installation:problem-if-it-says-the-shared-id-is-already-in-use}}
\sphinxAtStartPar
A1: please verify that it is the right shared debug\_id in your netrhost.conf file

\sphinxAtStartPar
A2: Force a start by ./Startmush \sphinxhyphen{}f


\paragraph{Problem: Your log file is massive and your mush is running}
\label{\detokenize{installation:problem-your-log-file-is-massive-and-your-mush-is-running}}
\sphinxAtStartPar
A1: To rotate this use the @logrotate command. See wizhelp on @logrotate


\paragraph{Problem: The database flatfile you’re loading can’t load because a db is already defined}
\label{\detokenize{installation:problem-the-database-flatfile-you-re-loading-can-t-load-because-a-db-is-already-defined}}
\sphinxAtStartPar
A1: remove netrhost.db* and netrhost.gdbm* from your data directory


\paragraph{Problem: The mail database won’t load and mail shows ‘offline’}
\label{\detokenize{installation:problem-the-mail-database-won-t-load-and-mail-shows-offline}}
\sphinxAtStartPar
A1: wmail/load


\subsection{Windows}
\label{\detokenize{installation:windows}}

\subsubsection{Installing on Windows 10 with BASH}
\label{\detokenize{installation:installing-on-windows-10-with-bash}}
\sphinxAtStartPar
Rhost can be compiled and run under the new Bash on Ubuntu on Windows.
This has been tested with the Preview build 14342.

\sphinxAtStartPar
1. After installing Bash you will need to install the following packages:
git
make
gcc
openssl (optional)
libpcre3 (optional)
libpcre3\sphinxhyphen{}dev (optional)

\sphinxAtStartPar
2. When configuring rhost (using confsource) select the Disable Debugmon
option.
\begin{enumerate}
\sphinxsetlistlabels{\arabic}{enumi}{enumii}{}{.}%
\setcounter{enumi}{2}
\item {} 
\sphinxAtStartPar
When you issue Startmush, you must pass the \sphinxhyphen{}cyg option.

\end{enumerate}


\subsubsection{Installing on Windows with Cygwin}
\label{\detokenize{installation:installing-on-windows-with-cygwin}}
\sphinxAtStartPar
Rhost does work under windows using the cygwin package.

\sphinxAtStartPar
1.  When you do install cygwin, the following packages must be added:
make
git
gcc
crypt
openssl (optional)
gdbm
bash
\begin{enumerate}
\sphinxsetlistlabels{\arabic}{enumi}{enumii}{}{.}%
\setcounter{enumi}{1}
\item {} 
\sphinxAtStartPar
The src/Makefile has to manually have the CYGWIN line uncommented.

\item {} 
\sphinxAtStartPar
When you issue Startmush, you must pass it the \sphinxhyphen{}cyg option.

\end{enumerate}


\section{Database}
\label{\detokenize{database:database}}\label{\detokenize{database::doc}}

\subsection{Loading an existing Database}
\label{\detokenize{database:loading-an-existing-database}}
\sphinxAtStartPar
To load in a previous database, you run the db\_load script.

\sphinxAtStartPar
From the game directory you would type:

\sphinxAtStartPar
./db\_load data/netrhost.gdbm yourflatfilehere data/netrhost.db.new

\begin{sphinxadmonition}{note}{Note:}
\sphinxAtStartPar
You may also do: ./Startmush
Then you just follow the prompts to load in your flatfile there.
\end{sphinxadmonition}

\sphinxAtStartPar
If you wish to have \#1’s password reset to ‘Nyctasia’ please add this
to the bottom of your netrhost.conf file:

\sphinxAtStartPar
newpass\_god 777

\sphinxAtStartPar
The caveat is that you must not have any netrhost.db* or netrhost.gdbm* files
in your data directory prior to loading it in.  It’ll error out if previous
files exist.  So you will need to move all files that start with netrhost.db*
and all files that start with netrhost.gdbm* to another directory.

\sphinxAtStartPar
Your flatfile tends to be named ‘netrhost.db.flat’ which is in your data
directory.  You can, however, name your flatfile anything you want and have
it in any directory you want.

\sphinxAtStartPar
To make a flatfile in game, you just issue @dump/flat.  You can specify
a filename after it, otherwise it assumes the name ‘netrhost.db.flat’.


\subsection{Converting a flatfile database for use in RhostMUSH}
\label{\detokenize{database:converting-a-flatfile-database-for-use-in-rhostmush}}
\sphinxAtStartPar
In the \textasciitilde{}/Server/convert directory there is a script called ‘doconvert.sh’

\sphinxAtStartPar
This script will convert most flatfiles from existing mush engines to
RhostMUSH.  The exception is PennMUSH 1.8.0 and later.  For this there is a
BETA converter penn18x\_converter.tgz.  This is proven to work, most of the time,
with codebases between 1.8.0 and 1.8.2.  It has not been fully vetted with
the latest PennMUSH databases.  Our apologies.

\sphinxAtStartPar
To convert a non\sphinxhyphen{}pennmush game (or a pennmush 1.7.4 or earlier), you first
need a valid flatfile of the game you’re wanting to convert.  Please refer
to the documentatation of the mush engine in question (MUX, Penn 1.7, TM2/3)
on how to do this.  Once you have it type:

\sphinxAtStartPar
./doconvert.sh FLATFILETOCONVERT FLATFILEOUTPUT

\sphinxAtStartPar
In this instance, FLATFILETOCONVERT will be the filename (with full path) to
the flatfile you are wishing to convert.

\sphinxAtStartPar
The FLATFILEOUTPUT is anyfilename you wish to name the RhostMUSH converted
flatfile.  I suggest netrhost\_converted.db.flat so you know by the name
what it is.

\sphinxAtStartPar
Follow what it asks for and just let it do its thing.


\section{Gettin Started}
\label{\detokenize{gettingstarted:gettin-started}}\label{\detokenize{gettingstarted::doc}}

\subsection{What to ype to get the basics running if you did not choose a pre\sphinxhyphen{}existing flatfile}
\label{\detokenize{gettingstarted:what-to-ype-to-get-the-basics-running-if-you-did-not-choose-a-pre-existing-flatfile}}
\sphinxAtStartPar
If you decided to get a bare\sphinxhyphen{}bone configuration, you will find your mush has just two things.  The \#1 (God) player and the starting room \#0.  That’s it.


\subsubsection{Login to \#1 from the connect screen}
\label{\detokenize{gettingstarted:login-to-1-from-the-connect-screen}}
\sphinxAtStartPar
Nyctasia is the default password:

\begin{sphinxVerbatim}[commandchars=\\\{\}]
\PYG{n}{co} \PYG{c+c1}{\PYGZsh{}1 Nyctasia}
\end{sphinxVerbatim}


\subsubsection{Change \#1’s password to something you’ll remember but is hard to guess}
\label{\detokenize{gettingstarted:change-1-s-password-to-something-you-ll-remember-but-is-hard-to-guess}}
\sphinxAtStartPar
Note: yourpasswordgoeshere can be any password you choose.  Choose well:

\begin{sphinxVerbatim}[commandchars=\\\{\}]
\PYG{n+nd}{@password} \PYG{n}{Nyctasia}\PYG{o}{=}\PYG{n}{YOURPASSWORDGOESHERE}
\end{sphinxVerbatim}


\subsubsection{Master Room}
\label{\detokenize{gettingstarted:master-room}}
\sphinxAtStartPar
At this point you should create your master room:

\begin{sphinxVerbatim}[commandchars=\\\{\}]
\PYG{n+nd}{@dig} \PYG{n}{Master} \PYG{n}{Room}
\end{sphinxVerbatim}

\begin{sphinxadmonition}{note}{Note:}
\sphinxAtStartPar
Reason: You need a master room to contain global \$commands for players
\textasciicircum{}listens are not global for intentional reasons.  It’s far too much overhead for far too minimal gains that few people need or use.
\end{sphinxadmonition}


\subsubsection{Flag and protect Master Room}
\label{\detokenize{gettingstarted:flag-and-protect-master-room}}
\sphinxAtStartPar
It will return a dbref\#, it should be \#2 if you’ve not created anything else:

\begin{sphinxVerbatim}[commandchars=\\\{\}]
\PYG{n+nd}{@set} \PYG{c+c1}{\PYGZsh{}2=safe ind halt float}
\end{sphinxVerbatim}


\subsubsection{Player Holder Characters}
\label{\detokenize{gettingstarted:player-holder-characters}}
\sphinxAtStartPar
Feel free to change the password to what you want

\begin{sphinxadmonition}{note}{Note:}
\sphinxAtStartPar
Reason: You will want to chown global room or global areas to a given bitlevel and a method to keep organized.
Note: wizhelp control will give you a complete breakdown of what each bit can do.
\end{sphinxadmonition}


\paragraph{Immortal Holder}
\label{\detokenize{gettingstarted:immortal-holder}}
\begin{sphinxVerbatim}[commandchars=\\\{\}]
@pcreate ImmHolder=abc123
@set *Immholder=no\PYGZus{}connect !wanderer immortal
@badsite *immholder=*
\end{sphinxVerbatim}


\paragraph{Royalty/Wizard Holder}
\label{\detokenize{gettingstarted:royalty-wizard-holder}}
\begin{sphinxVerbatim}[commandchars=\\\{\}]
@pcreate WizHolder=abc123
@set *wizholder=no\PYGZus{}connect !wanderer royalty
@badsite *wizholder=*
\end{sphinxVerbatim}


\paragraph{Councilor/Admin Holder}
\label{\detokenize{gettingstarted:councilor-admin-holder}}
\begin{sphinxVerbatim}[commandchars=\\\{\}]
@pcreate AdminHolder=abc123
@set *adminholder=no\PYGZus{}connect !wanderer councilor
@badsite *adminholder=*
\end{sphinxVerbatim}


\paragraph{Architect/Staff Holder}
\label{\detokenize{gettingstarted:architect-staff-holder}}
\begin{sphinxVerbatim}[commandchars=\\\{\}]
@pcreate StaffHolder=abc123
@set *staffholder=no\PYGZus{}connect !wanderer architect
@badsite *staffholder=*
\end{sphinxVerbatim}


\paragraph{Guildmaster/Lead Holder}
\label{\detokenize{gettingstarted:guildmaster-lead-holder}}
\begin{sphinxVerbatim}[commandchars=\\\{\}]
@pcreate GuildHolder=abc123
@set *guildholder=no\PYGZus{}connect !wanderer guildmaster
@badsite *guildholder=*
\end{sphinxVerbatim}


\subsubsection{Chown \#0 (The starting room) and \#2 (The Master room) to immholder}
\label{\detokenize{gettingstarted:chown-0-the-starting-room-and-2-the-master-room-to-immholder}}
\begin{sphinxadmonition}{note}{Note:}
\sphinxAtStartPar
\#0 you can chown to a different bitlevel if you want, but the master room should always be owned by an immortal
\end{sphinxadmonition}

\begin{sphinxVerbatim}[commandchars=\\\{\}]
\PYG{n+nd}{@chown}\PYG{o}{/}\PYG{n}{preserve} \PYG{c+c1}{\PYGZsh{}0=*immholder}
\PYG{n+nd}{@chown}\PYG{o}{/}\PYG{n}{preserve} \PYG{c+c1}{\PYGZsh{}2=*immholder}
\end{sphinxVerbatim}


\subsubsection{Create yourself your own immortal player then log off \#1 and into this immortal player}
\label{\detokenize{gettingstarted:create-yourself-your-own-immortal-player-then-log-off-1-and-into-this-immortal-player}}
\begin{sphinxadmonition}{note}{Note:}
\sphinxAtStartPar
Pick what you want for playername and playerpassword
\end{sphinxadmonition}

\begin{sphinxVerbatim}[commandchars=\\\{\}]
@pcreate PLAYERNAME=PLAYERPASSWORD
@set *playername=!wanderer immortal
\end{sphinxVerbatim}


\subsubsection{Log out of \#1 and log into your immortal player}
\label{\detokenize{gettingstarted:log-out-of-1-and-log-into-your-immortal-player}}
\begin{sphinxadmonition}{note}{Note:}
\sphinxAtStartPar
Use the playername and password you created in the step before
\end{sphinxadmonition}

\begin{sphinxVerbatim}[commandchars=\\\{\}]
\PYG{n}{LOGOUT}
\PYG{n}{co} \PYG{n}{PLAYERNAME} \PYG{n}{PLAYERPASSWORD}
\end{sphinxVerbatim}


\subsubsection{Create your guest characters}
\label{\detokenize{gettingstarted:create-your-guest-characters}}
\begin{sphinxadmonition}{note}{Note:}
\sphinxAtStartPar
Feel free to change the description if you want
\end{sphinxadmonition}

\begin{sphinxVerbatim}[commandchars=\\\{\}]
\PYG{n+nd}{@dolist} \PYG{n}{lnum}\PYG{p}{(}\PYG{l+m+mi}{1}\PYG{p}{,}\PYG{l+m+mi}{10}\PYG{p}{)}\PYG{o}{=}\PYG{p}{\PYGZob{}}\PYG{n+nd}{@pcreate} \PYG{n}{Guest}\PYG{c+c1}{\PYGZsh{}\PYGZsh{}=guest;@set *Guest\PYGZsh{}\PYGZsh{}=guest;@adisconnect *Guest\PYGZsh{}\PYGZsh{}=home;@lock *Guest\PYGZsh{}\PYGZsh{}=*Guest\PYGZsh{}\PYGZsh{};@desc *Guest\PYGZsh{}\PYGZsh{}=A Stranger in a strange land.\PYGZcb{}}
\end{sphinxVerbatim}


\subsubsection{Dig a closet to store important objects but non\sphinxhyphen{}master room}
\label{\detokenize{gettingstarted:dig-a-closet-to-store-important-objects-but-non-master-room}}
\begin{sphinxadmonition}{note}{Note:}
\sphinxAtStartPar
name it anything you want, just remember it.
\end{sphinxadmonition}

\begin{sphinxVerbatim}[commandchars=\\\{\}]
\PYG{n+nd}{@dig} \PYG{n}{Closet}
\end{sphinxVerbatim}


\subsubsection{Set the flags on the closet and ownership of it}
\label{\detokenize{gettingstarted:set-the-flags-on-the-closet-and-ownership-of-it}}
\begin{sphinxadmonition}{note}{Note:}
\sphinxAtStartPar
Use the dbref\# that it returned when digging the closet and not \#123
\end{sphinxadmonition}

\begin{sphinxVerbatim}[commandchars=\\\{\}]
\PYG{n+nd}{@set} \PYG{c+c1}{\PYGZsh{}123=inh safe ind float}
\PYG{n+nd}{@chown}\PYG{o}{/}\PYG{n}{pres} \PYG{c+c1}{\PYGZsh{}123=*immholder}
\end{sphinxVerbatim}


\subsubsection{Create an Admin object for future ease of customization}
\label{\detokenize{gettingstarted:create-an-admin-object-for-future-ease-of-customization}}
\begin{sphinxVerbatim}[commandchars=\\\{\}]
\PYG{n+nd}{@create} \PYG{n}{AdminObject}
\end{sphinxVerbatim}


\subsubsection{Set the flags on the admin object and ownership and location}
\label{\detokenize{gettingstarted:set-the-flags-on-the-admin-object-and-ownership-and-location}}
\begin{sphinxadmonition}{note}{Note:}
\sphinxAtStartPar
this object must be immortal owned.  Use the dbref\# returned previously instead of \#123
\end{sphinxadmonition}

\begin{sphinxadmonition}{note}{Note:}
\sphinxAtStartPar
Use the closet dbref\# instead of \#234
\end{sphinxadmonition}

\begin{sphinxVerbatim}[commandchars=\\\{\}]
\PYG{n+nd}{@set} \PYG{n}{AdminObject}\PYG{o}{=}\PYG{n}{halt} \PYG{n}{safe} \PYG{n}{ind}
\PYG{n+nd}{@chown}\PYG{o}{/}\PYG{n}{pres} \PYG{c+c1}{\PYGZsh{}123=*immholder}
\PYG{n+nd}{@tel} \PYG{n}{adminobject}\PYG{o}{=}\PYG{c+c1}{\PYGZsh{}234}
\end{sphinxVerbatim}


\subsubsection{Add admin object to configuration}
\label{\detokenize{gettingstarted:add-admin-object-to-configuration}}
\sphinxAtStartPar
Modify the netrhost.conf file with the following line at the bottom after the line ‘\# define local alises here’ where you replace 123 with the dbref\# of the admin object that you made:

\begin{sphinxVerbatim}[commandchars=\\\{\}]
\PYG{n}{admin\PYGZus{}object} \PYG{l+m+mi}{123}
\end{sphinxVerbatim}


\subsubsection{Reboot your mush to load up the change for the admin object}
\label{\detokenize{gettingstarted:reboot-your-mush-to-load-up-the-change-for-the-admin-object}}
\begin{sphinxVerbatim}[commandchars=\\\{\}]
\PYG{n+nd}{@reboot}
\end{sphinxVerbatim}


\subsubsection{Do @admin/list to see if it shows the admin object}
\label{\detokenize{gettingstarted:do-admin-list-to-see-if-it-shows-the-admin-object}}
\begin{sphinxadmonition}{note}{Note:}
\sphinxAtStartPar
do wizhelp @admin for more information on how to use this
\end{sphinxadmonition}

\begin{sphinxVerbatim}[commandchars=\\\{\}]
\PYG{n+nd}{@admin}\PYG{o}{/}\PYG{n+nb}{list}
\end{sphinxVerbatim}


\subsubsection{Load in all the various softcode that you want}
\label{\detokenize{gettingstarted:load-in-all-the-various-softcode-that-you-want}}
\sphinxAtStartPar
This is client dependant based on whatever method it uses to load softcode.


\paragraph{Myrddin MushCron}
\label{\detokenize{gettingstarted:myrddin-mushcron}}
\sphinxAtStartPar
Load in the Myrddin Mush Cron.
It’s a very handy piece of software and strongly suggested to do so.  You can find this in the ‘Mushcode’ directory off the main Rhost directory.
Filename:

\begin{sphinxVerbatim}[commandchars=\\\{\}]
\PYG{o}{\PYGZti{}}\PYG{o}{/}\PYG{n}{Rhost}\PYG{o}{/}\PYG{n}{Mushcode}\PYG{o}{/}\PYG{n}{MyrddinCRON}
\end{sphinxVerbatim}

\begin{sphinxadmonition}{note}{Note:}
\sphinxAtStartPar
The globalroom() function returns the dbref\# of the master room.  Handy!
\end{sphinxadmonition}

\begin{sphinxVerbatim}[commandchars=\\\{\}]
\PYG{n+nd}{@chown}\PYG{o}{/}\PYG{n}{preserve} \PYG{n}{the} \PYG{n}{myrddin} \PYG{n}{mush} \PYG{n}{cron} \PYG{n}{to} \PYG{n}{immholder}\PYG{p}{,} \PYG{n}{then} \PYG{n}{move} \PYG{n}{to} \PYG{n}{maste} \PYG{n}{room}\PYG{o}{.}
\PYG{n+nd}{@chown}\PYG{o}{/}\PYG{n}{pres} \PYG{n}{Myrddin}\PYG{o}{=}\PYG{o}{*}\PYG{n}{Immholder}
\PYG{n+nd}{@tel} \PYG{n}{Myrddin}\PYG{o}{=}\PYG{c+c1}{\PYGZsh{}234 (where \PYGZsh{}234 is the dbref\PYGZsh{} of your code closet)}
\end{sphinxVerbatim}


\paragraph{AshCom}
\label{\detokenize{gettingstarted:ashcom}}
\sphinxAtStartPar
Load in default softcoded comsystem.  This is PennMUSH and MUX/TM3 compatible.
Filename:

\begin{sphinxVerbatim}[commandchars=\\\{\}]
\PYG{o}{\PYGZti{}}\PYG{o}{/}\PYG{n}{Rhost}\PYG{o}{/}\PYG{n}{Mushcode}\PYG{o}{/}\PYG{n}{comsys}
\end{sphinxVerbatim}

\sphinxAtStartPar
Chown the Comsystem and everything inside it to immholder:

\begin{sphinxVerbatim}[commandchars=\\\{\}]
\PYG{n+nd}{@chown}\PYG{o}{/}\PYG{n}{pres} \PYG{n}{ChanSys}\PYG{o}{=}\PYG{o}{*}\PYG{n}{Immholder}
\PYG{n+nd}{@dolist} \PYG{n}{lcon}\PYG{p}{(}\PYG{n}{chansys}\PYG{p}{)}\PYG{o}{=}\PYG{n+nd}{@chown}\PYG{o}{/}\PYG{n}{pres} \PYG{c+c1}{\PYGZsh{}\PYGZsh{}=*immholder}
\PYG{n+nd}{@tel} \PYG{n}{Chansys}\PYG{o}{=}\PYG{n}{globalroom}\PYG{p}{(}\PYG{p}{)}
\end{sphinxVerbatim}


\paragraph{Mail Wrappers}
\label{\detokenize{gettingstarted:mail-wrappers}}
\sphinxAtStartPar
Load in mail wrappers if you want MUX/TM3 and/or Penn mail wrapping.
Filename (MUX/TM3):

\begin{sphinxVerbatim}[commandchars=\\\{\}]
\PYG{o}{\PYGZti{}}\PYG{o}{/}\PYG{n}{Rhost}\PYG{o}{/}\PYG{n}{Mushcode}\PYG{o}{/}\PYG{n}{mailwrappers}\PYG{o}{/}\PYG{n}{muxmail}\PYG{o}{.}\PYG{n}{wrap}
\end{sphinxVerbatim}

\sphinxAtStartPar
Filename (Penn):

\begin{sphinxVerbatim}[commandchars=\\\{\}]
\PYG{o}{\PYGZti{}}\PYG{o}{/}\PYG{n}{Rhost}\PYG{o}{/}\PYG{n}{Mushcode}\PYG{o}{/}\PYG{n}{mailwrappers}\PYG{o}{/}\PYG{n}{pennmail}\PYG{o}{.}\PYG{n}{wrap}
\end{sphinxVerbatim}

\sphinxAtStartPar
Chown to immholder:

\begin{sphinxVerbatim}[commandchars=\\\{\}]
\PYG{n+nd}{@chown}\PYG{o}{/}\PYG{n}{pres} \PYG{n}{MUX}\PYG{o}{=}\PYG{o}{*}\PYG{n}{Immholder}
\PYG{n+nd}{@chown}\PYG{o}{/}\PYG{n}{pres} \PYG{n}{Penn}\PYG{o}{=}\PYG{o}{*}\PYG{n}{Immholder}
\PYG{n+nd}{@tel}\PYG{o}{/}\PYG{n+nb}{list} \PYG{n}{mux} \PYG{n}{penn}\PYG{o}{=}\PYG{n}{globalroom}\PYG{p}{(}\PYG{p}{)}
\end{sphinxVerbatim}


\paragraph{Myrddin BBS}
\label{\detokenize{gettingstarted:myrddin-bbs}}
\sphinxAtStartPar
Load in Myrddin’s BBS
Filename:

\begin{sphinxVerbatim}[commandchars=\\\{\}]
\PYG{o}{\PYGZti{}}\PYG{o}{/}\PYG{n}{Rhost}\PYG{o}{/}\PYG{n}{Mushcode}\PYG{o}{/}\PYG{n}{MyrddinBBS}
\end{sphinxVerbatim}

\sphinxAtStartPar
Chown to immholder and the contents of it as well:

\begin{sphinxVerbatim}[commandchars=\\\{\}]
\PYG{n+nd}{@chown}\PYG{o}{/}\PYG{n}{pres} \PYG{n}{Myrddin}\PYG{o}{=}\PYG{o}{*}\PYG{n}{Immholder}
\PYG{n+nd}{@dolist} \PYG{n}{lcon}\PYG{p}{(}\PYG{n}{myrddin}\PYG{p}{)}\PYG{o}{=}\PYG{n+nd}{@chown}\PYG{o}{/}\PYG{n}{pres} \PYG{c+c1}{\PYGZsh{}\PYGZsh{}=*immholder}
\PYG{n+nd}{@tel} \PYG{n}{myrddin}\PYG{o}{=}\PYG{n}{globalroom}\PYG{p}{(}\PYG{p}{)}
\end{sphinxVerbatim}


\paragraph{Other Mushcode}
\label{\detokenize{gettingstarted:other-mushcode}}
\sphinxAtStartPar
There’s other code in the Mushcode directory that you are welcome to install.  You would follow similar procedures
for loading it in, chowning it, and moving to the master room as you did above.

\sphinxAtStartPar
Likewise, any softcode you find on the internet or on other mushes should be portable to RhostMUSH with little to
no changes depending on the complexity of the code in question.


\subsection{Minimal DB instructions}
\label{\detokenize{gettingstarted:minimal-db-instructions}}
\sphinxAtStartPar
The ‘retired’ directory has older image files if you’re interested

\sphinxAtStartPar
Please use the netrhost.conf file with the database as they’re linked.

\sphinxAtStartPar
The flatfile must be loaded in as a new db

\sphinxAtStartPar
This is a minimal db with basic ‘features’ built in.

\sphinxAtStartPar
Copy the txt files into the Rhost’s txt directory off game:

\begin{sphinxVerbatim}[commandchars=\\\{\}]
\PYG{n}{cp} \PYG{n}{txt}\PYG{o}{/}\PYG{o}{*} \PYG{o}{\PYGZti{}}\PYG{o}{/}\PYG{n}{Rhost}\PYG{o}{/}\PYG{n}{Server}\PYG{o}{/}\PYG{n}{game}\PYG{o}{/}\PYG{n}{txt}
\end{sphinxVerbatim}

\sphinxAtStartPar
mkindx the files (substitute FILENAME with the filename):

\begin{sphinxVerbatim}[commandchars=\\\{\}]
\PYG{n}{cd} \PYG{o}{\PYGZti{}}\PYG{o}{/}\PYG{n}{Rhost}\PYG{o}{/}\PYG{n}{Server}\PYG{o}{/}\PYG{n}{game}\PYG{o}{/}\PYG{n}{txt}
\PYG{o}{.}\PYG{o}{.}\PYG{o}{/}\PYG{n}{mkindx} \PYG{n}{FILENAME}\PYG{o}{.}\PYG{n}{txt} \PYG{n}{FILENAME}\PYG{o}{.}\PYG{n}{indx}
\end{sphinxVerbatim}


\subsubsection{Startup Steps}
\label{\detokenize{gettingstarted:startup-steps}}\begin{enumerate}
\sphinxsetlistlabels{\arabic}{enumi}{enumii}{}{.}%
\item {} 
\sphinxAtStartPar
Using the Startmush utility for the first time, select the load db method

\end{enumerate}

\sphinxAtStartPar
— or —
\begin{enumerate}
\sphinxsetlistlabels{\arabic}{enumi}{enumii}{}{.}%
\item {} 
\sphinxAtStartPar
copy the netrhost.conf file into the games directory

\item {} 
\sphinxAtStartPar
make any relevant changes you wish

\item {} 
\sphinxAtStartPar
db\_load the flatfile
\begin{enumerate}
\sphinxsetlistlabels{\arabic}{enumii}{enumiii}{}{.}%
\item {} 
\sphinxAtStartPar
go into the game directory

\item {} 
\sphinxAtStartPar
type:

\begin{sphinxVerbatim}[commandchars=\\\{\}]
\PYG{o}{.}\PYG{o}{/}\PYG{n}{db\PYGZus{}load} \PYG{n}{data}\PYG{o}{/}\PYG{n}{netrhost}\PYG{o}{.}\PYG{n}{gdbm} \PYG{o}{.}\PYG{o}{.}\PYG{o}{/}\PYG{n}{minimal}\PYG{o}{\PYGZhy{}}\PYG{n}{DBs}\PYG{o}{/}\PYG{n}{minimal\PYGZus{}db}\PYG{o}{/}\PYG{n}{netrhost}\PYG{o}{.}\PYG{n}{db}\PYG{o}{.}\PYG{n}{flat} \PYG{n}{data}\PYG{o}{/}\PYG{n}{netrhost}\PYG{o}{.}\PYG{n}{db}\PYG{o}{.}\PYG{n}{new}
\end{sphinxVerbatim}

\end{enumerate}

\item {} 
\sphinxAtStartPar
Startmush as expected

\end{enumerate}


\subsection{Ambrosia’s Minimal Rhost DB}
\label{\detokenize{gettingstarted:ambrosia-s-minimal-rhost-db}}

\subsubsection{Version: 1.0.5          2020\sphinxhyphen{}01\sphinxhyphen{}31}
\label{\detokenize{gettingstarted:version-1-0-5-2020-01-31}}\begin{description}
\item[{Version history:}] \leavevmode\begin{description}
\item[{1.0.0}] \leavevmode\begin{itemize}
\item {} 
\sphinxAtStartPar
Initial database setup.

\end{itemize}

\item[{1.0.1}] \leavevmode\begin{itemize}
\item {} 
\sphinxAtStartPar
Small fixes of objid(), isstaff() and bccheck() permissions and handling.

\item {} 
\sphinxAtStartPar
bittype() access lowered to Architect level

\item {} 
\sphinxAtStartPar
NO\_CODE flag made visual to Architect

\end{itemize}

\item[{1.0.2}] \leavevmode\begin{itemize}
\item {} 
\sphinxAtStartPar
Several convenience changes and fixes: \_ Attributes moved to @aflags
system, allowing Architects to set, Guildmasters to see them.

\item {} 
\sphinxAtStartPar
@flagdef lowered to Royalty level. @quota/max, @quota/unlock and @convert
moved to Architect level.

\item {} 
\sphinxAtStartPar
NO\_CODE flag made settable/unsettable by Architects.

\item {} 
\sphinxAtStartPar
Fixed typo in conf file: ifselse \sphinxhyphen{}\textgreater{} ifelse

\item {} 
\sphinxAtStartPar
Switched \_Attributes to use the @aflags system
See: Guildmaster
Set: Architect

\end{itemize}

\item[{1.0.3}] \leavevmode\begin{itemize}
\item {} 
\sphinxAtStartPar
Removed @flagdefs from in\sphinxhyphen{}game softcode, converted to flag\_access\_*
config options

\item {} 
\sphinxAtStartPar
Lowered mailstatus() access to architect.

\end{itemize}

\item[{1.0.4}] \leavevmode\begin{itemize}
\item {} 
\sphinxAtStartPar
Changed softcoded objid() to tag(), due to Rhost’s new hardcoded
objid() which does perform a different functionality.

\item {} 
\sphinxAtStartPar
Added more staff recommendations to this file.

\item {} 
\sphinxAtStartPar
Added Reality TXLevel ‘Admin’ to all objects in the db except \#1

\end{itemize}

\item[{1.0.5}] \leavevmode\begin{itemize}
\item {} 
\sphinxAtStartPar
Replaced softcoded tag() system with Rhost’s new hardcoded @tag/tag()
system. All previous tags are set on the database. The Tag Object
was removed.

\item {} 
\sphinxAtStartPar
@function startup on BC\sphinxhyphen{}Admin\sphinxhyphen{}Royalty fixed \sphinxhyphen{} @wait 1 workaround for
Tags in place.

\item {} 
\sphinxAtStartPar
Places System @startup integrated into BC\sphinxhyphen{}Admin\sphinxhyphen{}Royalty’s @startup

\item {} 
\sphinxAtStartPar
Made @dump and @dump/flat available to Councilors in netrhost.conf

\end{itemize}

\item[{1.0.6}] \leavevmode\begin{itemize}
\item {} 
\sphinxAtStartPar
A small typo fix in netrhost.conf. float\_preciiosn \sphinxhyphen{}\textgreater{} precision and
functions\_max \sphinxhyphen{}\textgreater{} function\_max. Thanks to \sphinxhref{mailto:Bobbi@COH}{Bobbi@COH}

\end{itemize}

\end{description}

\end{description}


\subsubsection{Introduction}
\label{\detokenize{gettingstarted:introduction}}
\begin{sphinxadmonition}{note}{Note:}
\sphinxAtStartPar
READ THIS DOCUMENT CAREFULLY!
\end{sphinxadmonition}

\sphinxAtStartPar
Greetings,

\sphinxAtStartPar
This minimal Rhost DB was made with a secure setup, and as a good base to start
a new game off in mind.


\subsubsection{Features}
\label{\detokenize{gettingstarted:features}}

\paragraph{Configuration}
\label{\detokenize{gettingstarted:configuration}}\begin{itemize}
\item {} 
\sphinxAtStartPar
Limbo, Master Room and Auxiliary room.

\item {} 
\sphinxAtStartPar
BC\sphinxhyphen{}Admin\sphinxhyphen{}\textless{}bitlevel\textgreater{} characters set up for each bitlevel to own global and
data objects, and inherit to.

\item {} 
\sphinxAtStartPar
BC\sphinxhyphen{}Admin\sphinxhyphen{}Mortal is @powered EXAMINE\_ALL(Guildmaster), NOFORCE(Architect) and
LONG\_FINGERS.

\item {} 
\sphinxAtStartPar
@startup on BC\sphinxhyphen{}Admin\sphinxhyphen{}Immortal lowers DARK flag access to Councilor level, and
NO\_CODE visual access to Architect level.

\item {} 
\sphinxAtStartPar
Global Command objects inheriting from each bitlevel, with a separate staff\sphinxhyphen{}only object for each level.

\item {} 
\sphinxAtStartPar
Global Function objects inheriting from each bitlevel.

\item {} 
\sphinxAtStartPar
local Function objects inheriting from each bitlevel.

\item {} 
\sphinxAtStartPar
@function and @hook access lowered to Royalty level to remove immediate need
for Immortals or actual Immortal code.

\item {} 
\sphinxAtStartPar
@rxlevel, @txlevel, bittype() access lowered to Architect level to remove
immediate need for Royalty in many cases.

\item {} 
\sphinxAtStartPar
@startup on BC\sphinxhyphen{}Admin\sphinxhyphen{}Royalty to automatically load @hooks and @functions from
the Global Function objects, based on attribute naming.

\item {} 
\sphinxAtStartPar
Misc Data object to hold general data, like Staff lists etc.

\item {} 
\sphinxAtStartPar
Reality levels ‘Real’ and ‘Admin’.

\sphinxAtStartPar
All created items and players by default are in Recieve\sphinxhyphen{}Level ‘Real’ and
Transmit\sphinxhyphen{}Levels ‘Real’ and ‘Admin’.

\item {} 
\sphinxAtStartPar
All globals, Master Room, BCs\sphinxhyphen{}*, and other staff/code\sphinxhyphen{}related objects
currently have only ‘Admin’ as their Transmit\sphinxhyphen{}Level. This does not prevent
them fromi working properly. The only exception is \#1, who has empty reality
levels.

\item {} 
\sphinxAtStartPar
The supplied netrhost.conf offers a secure setup of options, allows Royalty
to use @hook and @function, and also sets the function\_access of several
functions to !no\_code, which allows NO\_CODE players to use the comsys
properly.

\sphinxAtStartPar
IT IS HIGHLY RECOMMENDED to use this .conf as a base for this DB.
The ‘Port’ configuration parameter is XXXX’d out. Set it first before starting
your game.

\item {} 
\sphinxAtStartPar
All existing objects have been @set SAFE and INDESTRUCTABLE.

\item {} 
\sphinxAtStartPar
All existing objects have a paranoid series of @locks set on themselves.

\item {} 
\sphinxAtStartPar
The +supersafe command is provided on \#1 as an example of what was used to
easily set this on objects.

\item {} 
\sphinxAtStartPar
Players are @set NO\_CODE and WANDERER by default. They get 40 credits on
creation, and a 1\sphinxhyphen{}credit\sphinxhyphen{}per\sphinxhyphen{}day paycheck.

\item {} 
\sphinxAtStartPar
All *mit sideeffects, as well as set(), create() and list() are enabled. The
latter three are necessary for the Comsystem. The rest of sideeffects are
disabled completely.

\item {} 
\sphinxAtStartPar
Flashing ansi is disabled.

\item {} 
\sphinxAtStartPar
\_Attributes are settable by Architects, and seeable by Guildmasters. Read:
Still invisible and unsettable by mortals.

\item {} 
\sphinxAtStartPar
Architects can set up, unlock, and change alternate @quota on players.

\item {} 
\sphinxAtStartPar
Architects can set/unset the NO\_CODE flag.

\item {} 
\sphinxAtStartPar
Guildmasters can see \_Attributes

\item {} 
\sphinxAtStartPar
Architects can set \_Attributes

\end{itemize}


\paragraph{Softcode}
\label{\detokenize{gettingstarted:softcode}}\begin{itemize}
\item {} 
\sphinxAtStartPar
Set\sphinxhyphen{}up compatibility SoftFunctions and @hook object.

\item {} 
\sphinxAtStartPar
Set up Comsys 1.0.9b at Architect level. (+bbhelp command)

\item {} 
\sphinxAtStartPar
Set up Myrddin +bboard at Architect level.

\item {} 
\sphinxAtStartPar
Set up Myrddon Cron at Architect level.

\item {} 
\sphinxAtStartPar
Anomaly Jobs (+jhelp)

\item {} 
\sphinxAtStartPar
SGP Places \& Mutter

\item {} 
\sphinxAtStartPar
Set up Penn\sphinxhyphen{}style follow.

\item {} 
\sphinxAtStartPar
Set up @scan (Up to architect\sphinxhyphen{}level items).

\item {} 
\sphinxAtStartPar
Set up Penn\sphinxhyphen{} and Mux compatibility Mailwrappers. (phelp and mhelp commands)

\item {} 
\sphinxAtStartPar
help .txt files and .indx files for the above.

\item {} 
\sphinxAtStartPar
@dynhelp access lowered to architect to call above helpfiles.

\end{itemize}


\paragraph{Functions}
\label{\detokenize{gettingstarted:functions}}\begin{itemize}
\item {} 
\sphinxAtStartPar
isstaff() \sphinxhyphen{} Softcoded function that returns ‘1’ if its argument matches
a \#dbref in the ‘isstaff’ attribute on the Misc Data object.

\item {} 
\sphinxAtStartPar
bccheck() \sphinxhyphen{} Softcoded function that returns ‘1’ if its argument matches
a \#dbref in the ‘bcs’ attribute on the Misc Data object.

\item {} 
\sphinxAtStartPar
width() \sphinxhyphen{} Softcoded function that returns 78 for now. For cross\sphinxhyphen{}MU*
compatibility.

\item {} 
\sphinxAtStartPar
pass() \sphinxhyphen{} Softcoded function that takes a number as its parameter, and return
a random string of that length. Perfect for setting random passwords.

\item {} 
\sphinxAtStartPar
cmdmessage() \sphinxhyphen{} Softcoded function that takes two strings as its parameters.
Returns ‘\textless{}\textless{} STRING1 \textgreater{}\textgreater{} String2’. The \textless{}\textless{}\textgreater{}\textgreater{} part is highlighted red. Good for
all kinds of messages sent by game commands.

\item {} 
\sphinxAtStartPar
header() \sphinxhyphen{} Highly versatile, and a buffer\sphinxhyphen{}saving alternative
to using printf() for centering with ansi borders. HIGHLY recommended to use
instead of printf() for such things.

\end{itemize}

\begin{sphinxVerbatim}[commandchars=\\\{\}]
\PYG{n}{header}\PYG{p}{(}\PYG{n}{text}\PYG{p}{,}\PYG{n}{width}\PYG{p}{,}\PYG{n}{filler}\PYG{p}{,}\PYG{n}{fillercolor}\PYG{p}{,}\PYG{n}{outerpadding}\PYG{p}{,}\PYG{n}{paddingcolor}\PYG{p}{,}
       \PYG{n}{leftinnerpadding}\PYG{p}{,}\PYG{n}{leftinnercolor}\PYG{p}{,}\PYG{n}{rightinnerpadding}\PYG{p}{,}\PYG{n}{rightinnercolor}\PYG{p}{)}
  \PYG{n}{text} \PYG{o}{\PYGZhy{}} \PYG{n}{Text} \PYG{n}{to} \PYG{n}{center}
  \PYG{n}{width} \PYG{o}{\PYGZhy{}} \PYG{n}{Width} \PYG{n}{of} \PYG{n}{the} \PYG{n}{header}\PYG{p}{,} \PYG{n}{defaults} \PYG{n}{to} \PYG{n}{width}\PYG{p}{(}\PYG{p}{)}
  \PYG{n}{filler} \PYG{o}{\PYGZhy{}} \PYG{n}{Character}\PYG{p}{(}\PYG{n}{s}\PYG{p}{)} \PYG{n}{to} \PYG{n}{draw} \PYG{n}{the} \PYG{n}{line} \PYG{k}{with}\PYG{o}{.} \PYG{n}{Default}\PYG{p}{:} \PYG{o}{=}
  \PYG{n}{fillercolor} \PYG{o}{\PYGZhy{}} \PYG{n}{ansicode} \PYG{n}{to} \PYG{n}{color} \PYG{n}{the} \PYG{n}{line} \PYG{k}{with}
  \PYG{n}{outerpadding} \PYG{o}{\PYGZhy{}} \PYG{n}{Characters} \PYG{n}{to} \PYG{n}{frame} \PYG{n}{the} \PYG{n}{outer} \PYG{n}{ends} \PYG{n}{of} \PYG{n}{the} \PYG{n}{line} \PYG{k}{with}\PYG{o}{.}
  \PYG{n}{paddingcolor} \PYG{o}{\PYGZhy{}} \PYG{n}{ansicode} \PYG{n}{to} \PYG{n}{color} \PYG{n}{the} \PYG{n}{outer} \PYG{n}{characters} \PYG{k}{with}
  \PYG{n}{leftinnerpadding} \PYG{o}{\PYGZhy{}} \PYG{n}{characters} \PYG{n}{to} \PYG{n}{put} \PYG{n}{at} \PYG{n}{the} \PYG{n}{left} \PYG{n}{side} \PYG{n}{of} \PYG{o}{\PYGZlt{}}\PYG{n}{text}\PYG{o}{\PYGZgt{}}
  \PYG{n}{leftinnercolor} \PYG{o}{\PYGZhy{}} \PYG{n}{ansicode} \PYG{n}{to} \PYG{n}{color} \PYG{n}{the} \PYG{n}{leftside} \PYG{n}{characters} \PYG{k}{with}
  \PYG{n}{rightinnerpadding} \PYG{o}{\PYGZhy{}} \PYG{n}{characters} \PYG{n}{to} \PYG{n}{put} \PYG{n}{at} \PYG{n}{the} \PYG{n}{right} \PYG{n}{side} \PYG{n}{of} \PYG{o}{\PYGZlt{}}\PYG{n}{text}\PYG{o}{\PYGZgt{}}
  \PYG{n}{rightinnercolor} \PYG{o}{\PYGZhy{}} \PYG{n}{ansicode} \PYG{n}{to} \PYG{n}{color} \PYG{n}{the} \PYG{n}{rightside} \PYG{n}{characters} \PYG{k}{with}
\end{sphinxVerbatim}

\begin{sphinxadmonition}{note}{Note:}
\sphinxAtStartPar
ALL of header()’s parameters are optional. By default, header() simply draws
a 78\sphinxhyphen{}char wide line of =’s. Simply leave parameters empty if you want to set
one of the latter parameters.
\end{sphinxadmonition}


\subsubsection{Bitlevels}
\label{\detokenize{gettingstarted:bitlevels}}
\sphinxAtStartPar
The whole DB is highly geared for a low\sphinxhyphen{}bitlevel setup.
I am a strong believer in least\sphinxhyphen{}privileges\sphinxhyphen{}needed to do the job. Bittypes and
powers are tools to do that job, not badges of friendship or trust that get
tossed about.

\begin{sphinxadmonition}{note}{Note:}
\sphinxAtStartPar
Here is my suggested list of powers and bittypes for staffers.
\end{sphinxadmonition}


\paragraph{Storytellers}
\label{\detokenize{gettingstarted:storytellers}}
\sphinxAtStartPar
@powered TEL\_ANYWHERE, JOIN\_PLAYER and GRAB\_PLAYER on Guildmaster level.


\paragraph{Builder\sphinxhyphen{}BCs}
\label{\detokenize{gettingstarted:builder-bcs}}
\sphinxAtStartPar
Mortal, with @quota and money for their job. There should be one
shared BC for each area of the game, like BC\sphinxhyphen{}Houston. Special
Rooms, items or exits that require privilegs to examine or @tel
a player should belong to a BC\sphinxhyphen{}Houston\sphinxhyphen{}Powered that is @powered
EXAMINE\_ALL, LONG\_FINGERS And TEL\_ANYTHING on Guildmaster level.
If the object actually needs to modify a player directly, have
it use a restricted staff Global, or if you absolutely must,
make a BC\sphinxhyphen{}Houston\sphinxhyphen{}Admin and @set it Architect. Do not give
normal builders access to it, only @chown things to it and @set
them inherit after review.

\sphinxAtStartPar
Both the \sphinxhyphen{}powered and \sphinxhyphen{}admin BCs can have random passwords and
be @set NO\_CONNECT.


\paragraph{Building Head}
\label{\detokenize{gettingstarted:building-head}}
\sphinxAtStartPar
@set Guildmaster, powered TEL\_ANYWHERE, TEL\_ANYTHING and
optionally CHOWN\_OTHER on Guildmaster level. Mind that the
latter technically allows them to @chown anything guildmaster\sphinxhyphen{}
and lower\sphinxhyphen{}owned in the master and auxiliary rooms. However,
it allows the Building Head to @chown items between BCs\sphinxhyphen{} and
to the BC\sphinxhyphen{}\textless{}location\textgreater{}\sphinxhyphen{}powered.


\paragraph{Enforcers}
\label{\detokenize{gettingstarted:enforcers}}
\sphinxAtStartPar
As Storyteller above, plus being @powered Security at
Guildmaster level, in order to handle problem players.

\sphinxAtStartPar
Optionally always given to Storytellers.


\paragraph{Coders}
\label{\detokenize{gettingstarted:coders}}
\sphinxAtStartPar
@set Architect


\paragraph{Head Coder}
\label{\detokenize{gettingstarted:head-coder}}
\sphinxAtStartPar
Always trust your head coder.
@set Architect for the everyday bit. Give access to a
maintenance Councilor bit for special code projects. Finished
code along those lines should get @chowned to the
bc\sphinxhyphen{}admin\sphinxhyphen{}councilor.

\sphinxAtStartPar
If you as the MU* Head(s) don’t know Softcode well, or want to
leave anything Code to your Head Coder, also give them optional
access to a maintenance Royalty bit in order to properly code
banning/blacklisting +commands and other rare code that requires
Royalty powers. Again, chown finished code to bc\sphinxhyphen{}admin\sphinxhyphen{}royalty.


\paragraph{MU* Head(s)}
\label{\detokenize{gettingstarted:mu-head-s}}
\sphinxAtStartPar
You’re the boss(es). But please use an Architect bit for your
everyday things. Keep Immortal to yourself. Keep \#1 to yourself.
And seriously avoid using either of them except for creating
a Royalty bit or doing intricate DB maintenance.


\paragraph{Site Admins}
\label{\detokenize{gettingstarted:site-admins}}
\sphinxAtStartPar
They already have more powers than any in\sphinxhyphen{}game bit can ever
have ;)

\sphinxAtStartPar
Depending on actual involvement with your game, their abilities
in\sphinxhyphen{}game can range from merely being @powered free\_wall for
notifying players of downtimes and/or being @powered shutdown in
order to shut down the game for maintenance, up to being the
only person with actual access to \#1.


\paragraph{Globals}
\label{\detokenize{gettingstarted:globals}}
\sphinxAtStartPar
Handle necessary functionality for adminning through the admin\sphinxhyphen{}only globals and
softcode.

\sphinxAtStartPar
The setup I personally suggest is to have ALL staffers be AT MAX Architect\sphinxhyphen{}level
for everyday work and communication, with special coders that \sphinxhyphen{}really\sphinxhyphen{} require
it having Councilor\sphinxhyphen{}characters available to log into for maintenance or special
code setup. Technically if everything is done right, the Coder(s) of the game do
not require higher privileges than Architect for the vast majority of things.
Royalty\sphinxhyphen{}level code should be a rare exception, if at all necessary. The MU*
Head(s) or site\sphinxhyphen{}admin should be the only one having access to \#1, Immortals or
perhaps even Royalty. The BC’s, Global Function objects and Function objects at
level Royalty and higher have simply been provided as a if\sphinxhyphen{}necessary convenience.

\sphinxAtStartPar
Current objects are only @chowned to certain bitlevels if it is really required
for them to function. Whenever possible, they have been @chowned to the mortal
BC\sphinxhyphen{}Admin\sphinxhyphen{}Mortal. All custom global functions listed above are on the semi\sphinxhyphen{}
\sphinxhyphen{}mortal Global Functions object. The Master Room and Auxiliary Room have been
@chowned to BC\sphinxhyphen{}Admin\sphinxhyphen{}Architect.

\sphinxAtStartPar
The Comsystem and +bboard are owned by BC\sphinxhyphen{}Admin\sphinxhyphen{}Architect, which means that
higher bitlevels might not be able to use those systems if they hide and set
themselves DARK. This is intentional. The Architect bitlevel is enough to freely
set attributes on players, so these systems did not need anything higher, and
it prioritizes Councilor+ as mere mainenance\sphinxhyphen{}duty bitlevels. Even the MU* Head
should log on as an Architect for everyday things.

\sphinxAtStartPar
The Comsystem and BBOARD have been modified to be configurable by Architect and
higher. Both systems have a CANUSE attribute with the according code on them.
Note that if you want both systems to be only configurable by Councilor+, that
instead of @chowning them to a Councilor after changing those attribute for
Councilor or higher, I suggest to simply @set the bboard and comsystem core
objects NO\_MODIFY instead, keeping them at Architect\sphinxhyphen{}power but making them
unmodifyable by Architects.


\subsubsection{Quota}
\label{\detokenize{gettingstarted:quota}}
\sphinxAtStartPar
I highly recommend the use of the alternative @quota system. BC\sphinxhyphen{}Admin\sphinxhyphen{}Mortal
and BC\sphinxhyphen{}Admin\sphinxhyphen{}Guildmaster both have this @quota system set up on themselves. Both
of them have a high amount of money for everyday operations. You should not give
them free quota or money.


\subsubsection{Functions}
\label{\detokenize{gettingstarted:id1}}
\sphinxAtStartPar
I also recommend to setup most global functions with /Privileged even if they
are mortal\sphinxhyphen{}powered, to make them work even when players are set NO\_CODE and
WANDERER by default.

\sphinxAtStartPar
Enjoy!


\subsubsection{Compiling}
\label{\detokenize{gettingstarted:compiling}}
\sphinxAtStartPar
P.S. the ‘bin/asksource.save0’ file has been supplied for loading in the
‘make config’ or ‘make confsource’ step of Rhost installation. It provides the
settings I have used when creating this database. Some settings, like the ANSI
substitution, are used in the DB.

\sphinxAtStartPar
\sphinxhref{mailto:--Ambrosia@RhostMUSH}{\textendash{}Ambrosia@RhostMUSH}


\section{What FLAGS, TOGGLES, POWERS, DEPOWERS, and BITLEVELS mean in RhostMUSH}
\label{\detokenize{toggles:what-flags-toggles-powers-depowers-and-bitlevels-mean-in-rhostmush}}\label{\detokenize{toggles::doc}}

\subsection{Flags}
\label{\detokenize{toggles:flags}}
\sphinxAtStartPar
Flags are pretty much exactly the same as any other mush.  It’s a flag
that you set or unset on a target which then enables/disables or
alters something that target can do.  There’s help on all the flags
in help and wizhelp.


\subsection{Toggles}
\label{\detokenize{toggles:toggles}}
\sphinxAtStartPar
Toggles were designed as a single point flag that immediately enables
or disables a set ability or condition, thus a ‘toggle’.  It works
exactly like a flag and was originally designed for two reasons.  To
distinguish from the multi\sphinxhyphen{}meaning of a ‘flag’, and because frankly
we ran out of flag space :)


\subsection{@power}
\label{\detokenize{toggles:power}}
\sphinxAtStartPar
A power is similar to a power on other mushes, but unlike them, our
powers are multi\sphinxhyphen{}tier.  This means that they can be customized to
empower something at a given bitlevel.  You may empower something
from guildmaster up to councilor level.  There are some powers
with a power level of N/A meaning they are a toggle power granting
an absolute power level as specified in the help for that power.
This requires the INHERIT flag for non\sphinxhyphen{}players to inherit powers,
however, a specific object can be granted a power as well.


\subsection{@depower}
\label{\detokenize{toggles:depower}}
\sphinxAtStartPar
This is the anti\sphinxhyphen{}thesis of @power.  Also, depowers do not require
inheritance.  They also have priority over flags, toggles, and
powers.  You can use depower to remove or lower abilities and
control from a target, even a full wizard (royalty) can be
depowered.


\subsection{Multi\sphinxhyphen{}tiered bitlevel systems}
\label{\detokenize{toggles:multi-tiered-bitlevel-systems}}
\sphinxAtStartPar
RhostMUSH offers a multi\sphinxhyphen{}tier bitlevel system.  They go in order of presidence
You do not have to use all these bits, only use what you want.


\subsubsection{ghod (\#1) \textless{}bitlevel 7\textgreater{}}
\label{\detokenize{toggles:ghod-1-bitlevel-7}}
\sphinxAtStartPar
This bitlevel can do everything.  Only those who you trust with absolute power should have this.  Period.


\subsubsection{Immortal(i) \textless{}Level 6\textgreater{} \sphinxhyphen{} Basically \#1}
\label{\detokenize{toggles:immortal-i-level-6-basically-1}}
\sphinxAtStartPar
The only thing this bitlevel can not do is directly effect \#1,
set/unset some internal flags/attributes, and set/unset the
immortal flag.  These players can do EVERYTHING else.  Treat
this bit as you would treat \#1.  Only give it to those you know
without a doubt you can trust.
\begin{itemize}
\item {} 
\sphinxAtStartPar
Can do everything except set some internal flags, effect \#1, and set/remove the immortal flag.

\end{itemize}


\subsubsection{Royalty(W) \textless{}Level 5\textgreater{} \sphinxhyphen{} FULL wizbit level}
\label{\detokenize{toggles:royalty-w-level-5-full-wizbit-level}}
\sphinxAtStartPar
This is your standard wizard.  They can do everything you’re
used to on other mushes that wizards can do.  In addition, they
also override all locks by default (this can be disabled), and
they have an enhanced wizcloaking ability (which also can be
disabled).  They can also set all the lower bitlevels.
\begin{itemize}
\item {} 
\sphinxAtStartPar
All things of Level 4 and lower

\item {} 
\sphinxAtStartPar
Ability to set more flags: STOP, NOSTOP, FUBAR

\item {} 
\sphinxAtStartPar
Ability to @attribute,

\item {} 
\sphinxAtStartPar
Ability to WIZCLOAK

\end{itemize}


\subsubsection{Councilor(a) \textless{}Level 4\textgreater{} \sphinxhyphen{} High wizbit level}
\label{\detokenize{toggles:councilor-a-level-4-high-wizbit-level}}
\sphinxAtStartPar
This is your almost\sphinxhyphen{}but\sphinxhyphen{}not\sphinxhyphen{}quite wizard.  They have access to
about 80\% of the wizard commands.  This includes @nuke, @toad,
@newpassword and the like.  The only things they can’t do that
wizards can is cloak, override locks, and use some of the
database manipulation tools in wizhelp.
\begin{itemize}
\item {} 
\sphinxAtStartPar
All things of Level 3 and lower

\item {} 
\sphinxAtStartPar
Ability to set more flags: NOCONNECT, WANDERER, FREE

\item {} 
\sphinxAtStartPar
Ability to @nuke, @toad, @boot, @chownall, @dbck, @poor, @newpassword, @pcreate, slay

\end{itemize}


\subsubsection{Architect(B) \textless{}Level 3\textgreater{} \sphinxhyphen{} Middle wizbit level}
\label{\detokenize{toggles:architect-b-level-3-middle-wizbit-level}}
\sphinxAtStartPar
This is your sub\sphinxhyphen{}wizard.  They still have the ability to control
anything their bitlevel and lower (including @chown, @destroy, etc)
but do not have any control of other players (like @nuke, @toad, etc)
but they can set the slave flag.  Otherwise, all things their level
and lower they can treat as if they owned it.
\begin{itemize}
\item {} 
\sphinxAtStartPar
All things of Level 2

\item {} 
\sphinxAtStartPar
Ability to fully control and modify anything their level and lower (including @cloning, @destroying, etc)

\item {} 
\sphinxAtStartPar
Ability to use @tel on anything their level and lower.

\item {} 
\sphinxAtStartPar
Ability to bypass jump\_ok rooms on anything their level \& lower.

\item {} 
\sphinxAtStartPar
Ability to set some restricted flags: SLAVE, NO\_YELL

\item {} 
\sphinxAtStartPar
Has infinite quota and money

\item {} 
\sphinxAtStartPar
Able to give negative money (Steal)

\item {} 
\sphinxAtStartPar
Able to @toggle the MONITOR

\end{itemize}


\subsubsection{Guildmaster(g) \textless{}Level 2\textgreater{} \sphinxhyphen{} Lowest wizbit level}
\label{\detokenize{toggles:guildmaster-g-level-2-lowest-wizbit-level}}
\sphinxAtStartPar
This is the lowest wiz bit.  They only have moderate abilities.
They can examine/decompile anything their level and lower, they can
@guild/@quota people, and they have a few other minor abilities.
They don’t have free money however.
\begin{itemize}
\item {} 
\sphinxAtStartPar
Ability to access things remotely (long\_fingers)

\item {} 
\sphinxAtStartPar
Things are FREE for them in the queue.

\item {} 
\sphinxAtStartPar
Can see dbref\#’s of things their level and lower

\item {} 
\sphinxAtStartPar
Can examine/decompile things their level and lower.

\item {} 
\sphinxAtStartPar
Can set @quota/@guild on their level and lower.

\end{itemize}


\subsubsection{Wanderer \textless{}bitlevel 0\textgreater{}}
\label{\detokenize{toggles:wanderer-bitlevel-0}}
\sphinxAtStartPar
This is a hinderance flag.  This flag is automatically set on new
players that are created (which can be disabled).  This flag stops
the player from creating/destroying any database information.   In
effect it stops them from any type of building type commands.  They
still are allowed to set/unset locks/attributes/etc though without
hinderance.


\subsubsection{Guest \textless{}bitlevel 0 as well\textgreater{}}
\label{\detokenize{toggles:guest-bitlevel-0-as-well}}
\sphinxAtStartPar
This is a bigger hinderance flag.  By default all guests should be
set this.  This flag stops the player from ANY database manipulation
along with @teleporting, and many other advanced commands.  It’s
extreamly dehibilatating.


\subsection{Altering bitlevels}
\label{\detokenize{toggles:altering-bitlevels}}
\sphinxAtStartPar
Please keep in mind each of these bitlevels can be tweeked with the @admin
parameters and with the @powers (accessable by royalty) or @depowers (only
by immortal and higher).


\section{Security}
\label{\detokenize{security:security}}\label{\detokenize{security::doc}}

\subsection{Considerations to locking down restrictions in RhostMUSH}
\label{\detokenize{security:considerations-to-locking-down-restrictions-in-rhostmush}}
\sphinxAtStartPar
Sometimes, you want to have things run at various privilage levels and do not
want to have things with too much access.  Weither that is online objects or
players you want to block from connecting to your mush.  Here’s things you can
do.

\sphinxAtStartPar
One thing to keep in mind is that RhostMUSH, unlike PennMUSH is not flag
dependant on permission level, it’s ownership based.  While setting a wizard
flag on an object would work, it’s not recommended and it is instead recommended
to chown the object in question to a wizard (like your wizard holder character)
Then the object must be set inherit to actually inherit the wizard.
\begin{description}
\item[{Note: inherit is required to inherit anything from the player.  Flags, powers,}] \leavevmode
\sphinxAtStartPar
toggles.  The only thing that is inherited automatically is depowers.

\end{description}


\subsubsection{Online: Blocking object abilities}
\label{\detokenize{security:online-blocking-object-abilities}}
\sphinxAtStartPar
We have various flag levels.  It is strongly recommanded you check wizhelp
on ‘control’ for a detailed overview of what each bitlevel can or can not do
prior to giving the ownership to the object.  Things useful for tweaking control
on players and objects:


\paragraph{FLAGS (access with @set)}
\label{\detokenize{security:flags-access-with-set}}\begin{quote}

\sphinxAtStartPar
IMMORTAL, ROYALTY, COUNCILOR, ARCHITECT, GUILDMASTER,
FUBAR, SLAVE, SIDEFX, NO\_CONNECT, WANDERER, SAFE,
AUDITORIUM, BACKSTAGE, NOBACKSTAGE, INDESTRUCTIBLE,
INHERIT, JUMP\_OK, NO\_TEL, NO\_WALL, NO\_EXAMINE,
NO\_MODIFY, NO\_CONNECT, NO\_POSSESS, NO\_PESTER,
NO\_OVERRIDE, NO\_USELOCK, NO\_MOVE, NO\_YELL, CLOAK,
SCLOAK, DARK, UNFINDABLE, SEE\_OEMIT, TELOK, SUSPECT,
SPAMMONITOR
\end{quote}


\paragraph{TOGGLES (@toggle)}
\label{\detokenize{security:toggles-toggle}}\begin{quote}

\sphinxAtStartPar
BRANDY\_MAIL, PENN\_MAIL, MUXPAGE, VPAGE, NOISY,
MONITOR* (all monitor toggles), MORTALREALITY,
NODEFAULT, NO\_FORMAT, PAGELOCK, SNUFFDARK, VARIABLE
\end{quote}


\paragraph{@powers, @depowers, and @locks}
\label{\detokenize{security:powers-depowers-and-locks}}
\sphinxAtStartPar
Please review help (and wizhelp) for each of these items on how it can affect
a player, thing, exit, or room.  The help is quite verbose.


\subsubsection{Offline: Blocking twinks from being abusive on your game}
\label{\detokenize{security:offline-blocking-twinks-from-being-abusive-on-your-game}}
\sphinxAtStartPar
FLAGS                   : FUBAR, SLAVE, NO\_CONNECT
Commands:               : @boot, @nuke, @toad, @turtle
Sitelocks: (@admin)     : forbid\_host, forbid\_site, register\_host, register\_site, noguest\_host, noguest\_site
Sitelock by player      : @badsite, @goodsite, NO\_CONNECT (flag)
Monitoring player       : SUSPECT (flag), @snoop
TOR/Proxy blocking:     : @blacklist (see shell’s tor\_pull.sh), @admin proxy\_checker (see wizhelp), @tor (see wizhelp)


\subsection{Extended lockdown of the mush and considerations}
\label{\detokenize{security:extended-lockdown-of-the-mush-and-considerations}}
\sphinxAtStartPar
These are flags, powers, toggles, and various conditions for consideration
when you decide to use some of the advanced features of RhostMUSH.
These are not all that is availble, but tend to be the juicier ones to consider.


\subsubsection{Attribute Restriction}
\label{\detokenize{security:attribute-restriction}}\begin{quote}

\sphinxAtStartPar
@attribute \textendash{} Used for user\sphinxhyphen{}defined attributes
@admin attr\_access \textendash{} used for built in attributes (like desc)
@aflags \textendash{} Used to set up lovely delicious attribute permission masks
\end{quote}


\subsubsection{Command Restriction}
\label{\detokenize{security:command-restriction}}\begin{quote}

\sphinxAtStartPar
@icmd    \sphinxhyphen{} Very useful.   Please see wizhelp on it.  It disallows commands from executing including overriding them with softcode alternatives
@admin access \sphinxhyphen{} Changes permissions, disables, or sets to be overridden a command.  Useful when you plan to override commands with softcode.
\end{quote}


\subsubsection{Flag/Toggle Restriction}
\label{\detokenize{security:flag-toggle-restriction}}\begin{quote}

\sphinxAtStartPar
@flagdef \sphinxhyphen{} Again, see wizhelp on this.  There are also netrhost.conf options so you can have them loaded at start.  This allows tweaking flags and toggles to who can set/unset/see as well as what type can use it or wha type it can be used on.
\end{quote}


\subsubsection{Config restrictions}
\label{\detokenize{security:config-restrictions}}\begin{quote}

\sphinxAtStartPar
@admin config\_access \sphinxhyphen{} Changes permission of who can set a config param
\end{quote}


\subsubsection{Function Restrictions}
\label{\detokenize{security:function-restrictions}}\begin{quote}

\sphinxAtStartPar
@function/@lfunction \textendash{} Allows softcoded functions that you can optionally lock down at your leasure
@admin function\_access \textendash{} You can use this even on softcoded functions if you so desired.
\end{quote}


\subsubsection{Flags}
\label{\detokenize{security:flags}}\begin{quote}

\sphinxAtStartPar
GUEST    \sphinxhyphen{} This is your guest flag, it should only be set on guests
WANDERER \sphinxhyphen{} the WANDERER flag is default on new players.  This flag disables all building abilitites of the player.
NO\_COMMAND \sphinxhyphen{} You can use this to stop a player from being able to connect without worrying about changing their password
FUBAR      \sphinxhyphen{} As the flag states, it f*’s them up beind all recognition.  It essentially stops them from doing absolutely anything in the mush but pose and say.  Yes, it even disables the quit command.
SLAVE      \sphinxhyphen{} Funny enough, slave allows anything but say and pose.  To ruin a troll’s life, set both FUBAR and SLAVE and sit back and smile.
NO\_TEL     \sphinxhyphen{} The target can’t teleport or be teleported
NO\_MOVE    \sphinxhyphen{} The target is locked at their location unable to move at all
NO\_WALL    \sphinxhyphen{} They do not see any @wall except a wizard @wall/no\_prefix.  This has the bonus of snuffing db save messages.
NO\_POSSESS \sphinxhyphen{} Sometimes it’s useful to grant a builder character to multiple players.  The NO\_POSSESS flag makes it so that player can not be logged in more than 2 times.
NO\_MODIFY  \sphinxhyphen{} The target can not be modified (except by immortal/\#1)
NO\_EXAMINE \sphinxhyphen{} The target can not be examined/decompiled (except by immortal/\#1)
STOP       \sphinxhyphen{} Once a matching \$command is found on an object set STOP, it ‘stops’ trying to find other \$command matches.
NOSTOP     \sphinxhyphen{} If a target that is set STOP is also set NOSTOP, it will check the master room for a command and execute that as well if found.
NO\_PESTER  \sphinxhyphen{} Stops target from @pemit or whisper.  You may use @icmd as well.
NO\_OVERRIDE \sphinxhyphen{} Useful for immortals.  By default they override all locks, including attribute locks.  This makes it so an immortal’s passing of locks will behave like a mortals NO\_USELOCK  \sphinxhyphen{} This is like NO\_OVERRIDE but only effects uselocks.  You likely want to set this on your immortal and wizard.
NO\_ANSINAME \sphinxhyphen{} stops a target from having an ansified name
NO\_CODE     \sphinxhyphen{} lock down advanced coding from a target
SPAMMONITOR \sphinxhyphen{} stop a target from issuing more than 60 commands a minute.
FREE        \sphinxhyphen{} Stop costing money for day to day processing of commands/building
\end{quote}


\subsubsection{Toggles}
\label{\detokenize{security:toggles}}\begin{quote}

\sphinxAtStartPar
MONITOR            \sphinxhyphen{} Enables site monitoring.  This is the main toggle
MONITOR\_SITE       \sphinxhyphen{} Adds site information to site monitoring
MONITOR\_USERID     \sphinxhyphen{} Adds the userid to site monitoring
MONITOR\_STATS      \sphinxhyphen{} Adds connection stats to site monitoring
MONITOR\_FAIL       \sphinxhyphen{} Adds showing failed connections to site monitoring
MONITOR\_CONN       \sphinxhyphen{} Adds connection monitoring to site monitoring
MONITOR\_DISREASON  \sphinxhyphen{} Adds disconnect reasons to site monitoring
MONITOR\_TIME       \sphinxhyphen{} Adds time stamps to site monitoring
MONITOR\_BAD        \sphinxhyphen{} Shows all bad creation attempts to site monitoring
MONITOR\_VLIMIT     \sphinxhyphen{} Shows attempts to bypass MAX ATTRIBUTES
MONITOR\_AREG       \sphinxhyphen{} Shows all auto registration attempts
MONITOR\_CPU        \sphinxhyphen{} Shows all CPU warnings and/or alerts on the mush
NO\_FORMAT          \sphinxhyphen{} Bypasses @conformat, @exitformat, and other formats
SEE\_SUSPECT        \sphinxhyphen{} Allows you to see suspect info in the WHO/DOING
FORCEHALTED        \sphinxhyphen{} Allows you to @force/@sudo a HALTED target
NOSHPROG           \sphinxhyphen{} Disallows using ‘|’ to execute commands outside @program
PROG               \sphinxhyphen{} Allows the target to use @program
IMMPROG            \sphinxhyphen{} Disables the ability to use @quitprogram from a @program
PROG\_ON\_CONNECT    \sphinxhyphen{} Allows a @program to resume if someone reconnects
IGNOREZONE         \sphinxhyphen{} Enables a zone to process @icmd’s
PAGELOCK           \sphinxhyphen{} Enforces target to require passing pagelocks
MAIL\_LOCKDOWN      \sphinxhyphen{} Blocks the ability of a wizard to check another player’s mail
ATRUSE             \sphinxhyphen{} Enables the attribute to use attribute content locking
NOGLOBPARENT       \sphinxhyphen{} Disables the target from inheriting global parenting
LOGROOM            \sphinxhyphen{} Enables system logs on the room
EXFULLWIZATTR      \sphinxhyphen{} Allows target to examine all wizard attributes
NODEFAULT          \sphinxhyphen{} Disables attribute formatting/handling on the target
CHKREALITY         \sphinxhyphen{} Enables the use of reality locks on the target
HIDEIDLE           \sphinxhyphen{} Disables deidling when you execute any command
MORTALREALITY      \sphinxhyphen{} Enforces a wizard to pass realities as a mortal
SNUFFDARK          \sphinxhyphen{} Hides dark exits from a wizard
\end{quote}


\subsubsection{@powers}
\label{\detokenize{security:powers}}\begin{quote}

\sphinxAtStartPar
WIZ\_WHO            \sphinxhyphen{} Allows target to see sites ala wizard who
NOFORCE            \sphinxhyphen{} target an not be forced (except by immortal/\#1)
FREE\_QUOTA         \sphinxhyphen{} Allow target to have unlimited quota
JOIN\_PLAYER        \sphinxhyphen{} Allow to ‘join’ a player’s location
NO\_BOOT            \sphinxhyphen{} Player can not be booted except by immortal/\#1
STEAL              \sphinxhyphen{} Player can give negative money
TEL\_ANYWHERE       \sphinxhyphen{} Player can teleport anywhere
STAT\_ANY           \sphinxhyphen{} Player can @search, @stat, or @find things
HALT\_QUEUE\_ALL     \sphinxhyphen{} Player can halt the queue
SEARCH\_ANY         \sphinxhyphen{} Player can search for anything
WHO\_UNFIND         \sphinxhyphen{} Player can see hidden player on WHO
SHUTDOWN           \sphinxhyphen{} Player can @shutdown the mush
PURGE              \sphinxhyphen{} Player can use /purge to @destroy and @nuke
EXAMINE\_FULL       \sphinxhyphen{} Player can examine anything (not set NO\_EXAMINE or cloaked)
FORMATTING         \sphinxhyphen{} @*formats allow passing what a person sees as \%0, \%1, etc
CHOWN\_ANYWHERE     \sphinxhyphen{} Chown anything anywhere to yourself
CHOWN\_OTHER        \sphinxhyphen{} Chown something you don’t own to yourself
EXAMINE\_ALL        \sphinxhyphen{} Examine other things (tiered)
SEE\_QUEUE\_ALL      \sphinxhyphen{} Player can see the full queue
GRAB\_PLAYER        \sphinxhyphen{} Player can grab a remote player and pull them to location
LONG\_FINGERS       \sphinxhyphen{} Player is granted remote control of things they own
BOOT               \sphinxhyphen{} Player can @boot
SEE\_QUEUE          \sphinxhyphen{} Player can see advanced queue features
TEL\_ANYTHING       \sphinxhyphen{} Player can @teleport anything
PCREATE            \sphinxhyphen{} Player can @pcreate players
NOWHO              \sphinxhyphen{} Allows the use of @hide
HALT\_QUEUE         \sphinxhyphen{} Allows halting queue by specified bitlevel
SECURITY           \sphinxhyphen{} Allows setting  NOMOVE    NO\_TEL   SLAVE   NO\_YELL
WRAITH             \sphinxhyphen{} Allows bypassing exit locks
HIDEBIT            \sphinxhyphen{} Hides your admin level from lower levels
FULLTEL            \sphinxhyphen{} Allows full immortal level teleportation
EXECSCRIPT         \sphinxhyphen{} Allows executing external scripts in \textasciitilde{}/game/scripts
\end{quote}


\subsubsection{@depowers}
\label{\detokenize{security:depowers}}\begin{quote}

\sphinxAtStartPar
WALL               \sphinxhyphen{} Disables the ability to @wall
LONG\_FINGERS       \sphinxhyphen{} Disables remote access to things
STEAL              \sphinxhyphen{} Can not steal money
CREATE             \sphinxhyphen{} Can not create anything
WIZ\_WHO            \sphinxhyphen{} Can not access wizard who
CLOAK              \sphinxhyphen{} Can not wizard cloak
BOOT               \sphinxhyphen{} Can not boot
PAGE               \sphinxhyphen{} Can not page
FORCE              \sphinxhyphen{} Can not @force/@sudo
LOCKS              \sphinxhyphen{} Can not pass locks
COMMAND            \sphinxhyphen{} Can not execute any \$command (including master room)
MASTER             \sphinxhyphen{} Can not use any master room \$command
EXAMINE            \sphinxhyphen{} Lowers or disables the ability to examine/decompile
NUKE               \sphinxhyphen{} Can not nuke, toad, or turtle
FREE               \sphinxhyphen{} No longer has free money for anything
OVERRIDE           \sphinxhyphen{} No longer can override anything
TEL\_ANYWHERE       \sphinxhyphen{} No longer can teleport anywhere
TEL\_ANYTHING       \sphinxhyphen{} No longer can teleport anything other than themselves
POWER              \sphinxhyphen{} Can no longer use @power
MODIFY             \sphinxhyphen{} Can not modify things
CHOWN\_ME           \sphinxhyphen{} Can not chown anything to themselves
CHOWN\_OTHER        \sphinxhyphen{} Can not chown anything to others
ABUSE              \sphinxhyphen{} Can not use \$commands on anything they do not own
UNL\_QUOTA          \sphinxhyphen{} No longer has infinite quota
SEARCH\_ANY         \sphinxhyphen{} Disables the ability to @search/@find things
GIVE               \sphinxhyphen{} Disables ability to give things/money
RECEIVE            \sphinxhyphen{} Disables the ability to recieve things/money
NOGOLD             \sphinxhyphen{} Limits (or disables) how much gold someone can give
NOSTEAL            \sphinxhyphen{} Can not give negative gold
PASSWORD           \sphinxhyphen{} Can not change password
MORTAL\_EXAMINE     \sphinxhyphen{} drops examine and all fetching to mortal only
PERSONAL\_COMMAND   \sphinxhyphen{} Disables all \$commands on anything they own
\end{quote}


\subsubsection{Site Restrictions}
\label{\detokenize{security:site-restrictions}}\begin{quote}

\sphinxAtStartPar
These are accessable via the @admin command, and the following options are
allowable.

\sphinxAtStartPar
You may see all site information at any time with: @list sites
\end{quote}


\paragraph{IP based restrictions}
\label{\detokenize{security:ip-based-restrictions}}\begin{quote}

\sphinxAtStartPar
You may use CIDR notation such as /32 instead of 255.255.255.255.
Config file:  (see section on forbid\_site as it describes and gives examples)
Online Syntax: MASK:

\begin{sphinxVerbatim}[commandchars=\\\{\}]
                   \PYG{n+nd}{@admin} \PYG{n}{forbid\PYGZus{}site}\PYG{o}{=}\PYG{l+m+mf}{123.123}\PYG{o}{.}\PYG{l+m+mf}{123.0} \PYG{l+m+mf}{255.255}\PYG{o}{.}\PYG{l+m+mf}{255.0}
                   \PYG{n+nd}{@admin} \PYG{n}{forbid\PYGZus{}site}\PYG{o}{=}\PYG{l+m+mf}{123.123}\PYG{o}{.}\PYG{l+m+mf}{123.123} \PYG{l+m+mf}{255.255}\PYG{o}{.}\PYG{l+m+mf}{255.255}

   \PYG{n}{CIDR}\PYG{p}{:}\PYG{p}{:}

                   \PYG{n+nd}{@admin} \PYG{n}{forbid\PYGZus{}site}\PYG{o}{=}\PYG{l+m+mf}{123.123}\PYG{o}{.}\PYG{l+m+mf}{123.0} \PYG{o}{/}\PYG{l+m+mi}{24}
                   \PYG{n+nd}{@admin} \PYG{n}{forbid\PYGZus{}site}\PYG{o}{=}\PYG{l+m+mf}{123.123}\PYG{o}{.}\PYG{l+m+mf}{123.123} \PYG{o}{/}\PYG{l+m+mi}{32}

   \PYG{n}{REMOVING}\PYG{p}{:} \PYG{n}{MASK}\PYG{p}{:}\PYG{p}{:}

                   \PYG{n+nd}{@site}\PYG{o}{/}\PYG{n+nb}{all} \PYG{l+m+mf}{123.123}\PYG{o}{.}\PYG{l+m+mf}{123.123}\PYG{o}{=}\PYG{l+m+mf}{255.255}\PYG{o}{.}\PYG{l+m+mf}{255.255}
                   \PYG{n+nd}{@site}\PYG{o}{/}\PYG{n}{forbid} \PYG{l+m+mf}{123.123}\PYG{o}{.}\PYG{l+m+mf}{123.0}\PYG{o}{=}\PYG{l+m+mf}{254.255}\PYG{o}{.}\PYG{l+m+mf}{255.0}

             \PYG{n}{CIDR}\PYG{p}{:}\PYG{p}{:}

                   \PYG{n+nd}{@site}\PYG{o}{/}\PYG{n+nb}{all} \PYG{l+m+mf}{123.123}\PYG{o}{.}\PYG{l+m+mf}{123.123}\PYG{o}{=}\PYG{o}{/}\PYG{l+m+mi}{32}
                   \PYG{n+nd}{@site}\PYG{o}{/}\PYG{n}{forbid} \PYG{l+m+mf}{123.123}\PYG{o}{.}\PYG{l+m+mf}{123.0}\PYG{o}{=}\PYG{o}{/}\PYG{l+m+mi}{24}

\PYG{n}{forbid\PYGZus{}site}      \PYG{o}{\PYGZhy{}} \PYG{n}{Set} \PYG{n}{the} \PYG{n}{specified} \PYG{n}{site} \PYG{n}{forbid} \PYG{n}{only}
\PYG{n}{register\PYGZus{}site}    \PYG{o}{\PYGZhy{}} \PYG{n}{Set} \PYG{n}{the} \PYG{n}{specified} \PYG{n}{site} \PYG{n}{register} \PYG{n}{only}
\PYG{n}{noguest\PYGZus{}site}     \PYG{o}{\PYGZhy{}} \PYG{n}{Set} \PYG{n}{the} \PYG{n}{specified} \PYG{n}{site} \PYG{n}{unable} \PYG{n}{to} \PYG{n}{connect} \PYG{n}{to} \PYG{n}{guests}
\PYG{n}{suspect\PYGZus{}site}     \PYG{o}{\PYGZhy{}} \PYG{n}{Set} \PYG{n}{the} \PYG{n}{specified} \PYG{n}{site} \PYG{n}{suspect} \PYG{n}{on} \PYG{n}{connect}
\PYG{n}{noautoreg\PYGZus{}site}   \PYG{o}{\PYGZhy{}} \PYG{n}{Set} \PYG{n}{the} \PYG{n}{specified} \PYG{n}{site} \PYG{n}{to} \PYG{o+ow}{not} \PYG{n}{allow} \PYG{n}{autoregistration}
\PYG{n}{trust\PYGZus{}site}       \PYG{o}{\PYGZhy{}} \PYG{n}{Allow} \PYG{n}{site} \PYG{n}{to} \PYG{n}{bypass} \PYG{n}{suspect} \PYG{n}{site} \PYG{n}{restrictions}
\PYG{n}{permit\PYGZus{}site}      \PYG{o}{\PYGZhy{}} \PYG{n}{Allow} \PYG{n}{site} \PYG{n}{to} \PYG{n}{bypass} \PYG{n}{sitelock} \PYG{n}{restrictions}
\PYG{n}{nodns\PYGZus{}site}       \PYG{o}{\PYGZhy{}} \PYG{n}{Site} \PYG{n}{will} \PYG{n}{no} \PYG{n}{longer} \PYG{n}{do} \PYG{n}{reverse} \PYG{n}{DNS} \PYG{n}{lookups}
\PYG{n}{noauth\PYGZus{}site}      \PYG{o}{\PYGZhy{}} \PYG{n}{Site} \PYG{n}{will} \PYG{n}{no} \PYG{n}{longer} \PYG{n}{do} \PYG{n}{AUTH} \PYG{n}{ident} \PYG{n}{lookups}
\end{sphinxVerbatim}
\end{quote}


\paragraph{DNS based restrictions}
\label{\detokenize{security:dns-based-restrictions}}\begin{quote}

\sphinxAtStartPar
These allow globbing wildcard matches.
The advanced feature is you can specify filtering on
when the condition is matched, such as allowing 2 players from a site to
be connected before disallowing anyone else to connect.
Config File: (see section on forbid\_host as it describes and gives examples)
Online Syntax:

\begin{sphinxVerbatim}[commandchars=\\\{\}]
        ADD:      @admin forbid\PYGZus{}host=*.dsl*.comcast.net *.aol.com *another.site
        DEL:      @admin forbid\PYGZus{}host=!*.aol.com
        ADVANCED: @admin forbid\PYGZus{}host=mudconnect.com|3 (allow 3 at once only)

forbid\PYGZus{}host     \PYGZhy{} Set the specified site(s) forbid only
register\PYGZus{}host   \PYGZhy{} Set the specified site(s) register only
noguest\PYGZus{}host    \PYGZhy{} Set the specified site(s) unable to connect to guests
suspect\PYGZus{}host    \PYGZhy{} Set the specified site(s) suspect on connect
noautoreg\PYGZus{}host  \PYGZhy{} Set the specified site(s) to not allow autoregistration
validate\PYGZus{}host   \PYGZhy{} Do not allow any autoregistration from emails matching site
goodmail\PYGZus{}host   \PYGZhy{} Always allow autoregistration from emails matching site
nobroadcast\PYGZus{}host \PYGZhy{} Snuff online site broadcasts via MONITOR for specified site
\end{sphinxVerbatim}
\end{quote}


\subsection{Methods to block anonymous connections and the pros and cons of doing so}
\label{\detokenize{security:methods-to-block-anonymous-connections-and-the-pros-and-cons-of-doing-so}}\begin{quote}

\sphinxAtStartPar
Now let’s assume you have some troll attempting to use proxies to connect.
There’s multiple ways to stop this.
\end{quote}


\subsubsection{Blacklisting through external tor\_pull.sh script}
\label{\detokenize{security:blacklisting-through-external-tor-pull-sh-script}}\begin{quote}

\sphinxAtStartPar
In \textasciitilde{}/Rhost/Server/game you will see a script called tor\_pull.sh
Execute this by running (from the game directory) ./tor\_pull.sh
This populates the blacklist with registered proxies from various sites
on the internet.  If you want specified ip’s added, feel free to add
them at the end of this file.
\end{quote}


\subsubsection{Blacklisting through internal @blacklist command}
\label{\detokenize{security:blacklisting-through-internal-blacklist-command}}\begin{quote}

\sphinxAtStartPar
On the mush, have as part of your startup @blacklist/load
This will load in the generated blacklist file for automatic forbid
sites based on the ip.
\end{quote}


\subsubsection{Blacklisting through internal @tor command}
\label{\detokenize{security:blacklisting-through-internal-tor-command}}\begin{quote}

\sphinxAtStartPar
@tor.  Please see ‘wizhelp tor’ on how to set this up.  It in effect
will actively block all known exit nodes for TOR’s annonymous proxy
service.  It self\sphinxhyphen{}updates and will actively block TOR connections.
\end{quote}


\subsubsection{Blacklisting through internal @admin command}
\label{\detokenize{security:blacklisting-through-internal-admin-command}}\begin{quote}

\sphinxAtStartPar
@admin proxy\_checker (please see wizhelp)
This little doodad uses MTU checking against packet size which will
detect most methods of proxies.  Sadly, this also has false positives
because some situations require a differentating MTU value such as
briged network connect with things like cloud services, docker, or
similar encapsulated network services.  However, this option has
several settings from just monitoring/alerting of possible proxies
to downright forbidding them.  If you’re being actively attacked,
it may be worth considering adding this to add additional protection.
\end{quote}


\subsection{Setting up an SSL tunnel for secure connection options}
\label{\detokenize{security:setting-up-an-ssl-tunnel-for-secure-connection-options}}

\subsubsection{Quickstart for SSL setup with stunnel}
\label{\detokenize{security:quickstart-for-ssl-setup-with-stunnel}}\begin{enumerate}
\sphinxsetlistlabels{\arabic}{enumi}{enumii}{}{.}%
\item {} 
\sphinxAtStartPar
Modify your netrhost.conf file and add/change the following parameters:
\begin{enumerate}
\sphinxsetlistlabels{\arabic}{enumii}{enumiii}{}{.}%
\item {} 
\sphinxAtStartPar
stunnel\_reip 1

\item {} 
\sphinxAtStartPar
stunnel\_cmd SECRET\sphinxhyphen{}MAGIC\sphinxhyphen{}COOKIE
\begin{enumerate}
\sphinxsetlistlabels{\arabic}{enumiii}{enumiv}{}{.}%
\item {} 
\sphinxAtStartPar
SECRET\sphinxhyphen{}MAGIC\sphinxhyphen{}COOKIE is a case sensitive single word phrase. Any printable character other than the ‘\#’ character is allowable.  You may use up to 30 characters.

\item {} 
\sphinxAtStartPar
Make sure the secret is a hard to guess phrase.  This is used by stunnel to forward on the originating IP address.

\end{enumerate}

\item {} 
\sphinxAtStartPar
stunnel\_host localhost 127.0.0.1 othersite.goes.here
\begin{enumerate}
\sphinxsetlistlabels{\arabic}{enumiii}{enumiv}{}{.}%
\item {} 
\sphinxAtStartPar
This is optional.

\item {} 
\sphinxAtStartPar
If you do not specify it it defaults to ‘localhost 127.0.0.1’.  If your domain has a unique name like ‘localhost.localdomain’ like some ubuntu distributions, then you should customize your stunnel\_host.

\end{enumerate}

\end{enumerate}

\item {} 
\sphinxAtStartPar
go into the stunnel directory

\item {} 
\sphinxAtStartPar
./stunnel\_setup.sh
\begin{enumerate}
\sphinxsetlistlabels{\arabic}{enumii}{enumiii}{}{.}%
\item {} 
\sphinxAtStartPar
Choose the defaults or alter them based on preferences

\item {} 
\sphinxAtStartPar
Make sure to choose the warpbubble conf file

\end{enumerate}

\item {} 
\sphinxAtStartPar
./stunnel\_start.sh

\item {} 
\sphinxAtStartPar
Use ./stunnel\_stop.sh to stop the SSL tunnel at any time

\end{enumerate}

\sphinxAtStartPar
You do not need to shutdown the ssl handler if you shutdown the mush.  They
are entirely separate processes.


\subsubsection{Detailed SSL setup with stunnel}
\label{\detokenize{security:detailed-ssl-setup-with-stunnel}}
\sphinxAtStartPar
To setup SSL connectivity, we use the STUNNEL application to tunnel SSL to
the mush.  This acts a bit like a man in the middle but remains controlled
by the game owner which would have access to the end point anyway.

\sphinxAtStartPar
Note: it is assumed you will have already initially set up your netrhost.conf.


\paragraph{stunnel directory}
\label{\detokenize{security:stunnel-directory}}
\sphinxAtStartPar
You would set up the stunnel from the ‘stunnel’ directory.  There the following
files are of relevance:

\sphinxAtStartPar
README                     \textendash{} a readme explaining the points of stunnel
stunnel.conf.example       \textendash{} The example stunnel.conf file.  If you wish to create this manually you’re welcome to.  Just make sure the end file is stunnel.conf
stunnel\_setup.sh           \textendash{} the script to build a stunnel.conf file for you which will be dropped at your specified location.
stunnel\_kill.sh            \textendash{} Stop/terminate the stunnel process.
stunnel\_start.sh           \textendash{} Start the stunnel process.
warpbubble.pl              \textendash{} the perl script that handles stunnel to mush connectivity.
stunnel\_src                \textendash{} If you do not have stunnel, this directory will allow you to download, compile and locally install.


\paragraph{Modifying netrhost.conf}
\label{\detokenize{security:modifying-netrhost-conf}}
\sphinxAtStartPar
To be able to utilize SSL, you first must set your netrhost.conf file with
the relevant information to enable SSL connectiions.  These three config
options must be set to be able to use SSL, however, sconnect\_host if
not set will default to ‘localhost 127.0.0.1’.

\sphinxAtStartPar
sconnect\_reip 1         \textendash{} This enables the SSL tunnel layer handler within rhost.
sconnect\_cmd XYZZY      \textendash{} this will set the secret SSL command handshake command to XYZZY.  This is case sensitive and can be up to 31 characters.  Please make sure to only use printable non\sphinxhyphen{}whitespace characters.  Ergo: one word
sconnect\_host wildcards \textendash{} This allows wildcarded sites (one or more) to allow to access the sconnect/stunnel handler.  This defaults to ‘localhost’ and ‘127.0.0.1’ so if you have ‘localhost.localdomain’ instead then you must set this to whatever is seen as ‘localhost’ to you.  You can verify this by checking your /etc/hosts file.

\sphinxAtStartPar
Note: the sconnect\_host is optional.  If you do not specify it, it will default to two values:  ‘localhost’ and ‘127.0.0.1’.


\paragraph{Running the stunnel setup program}
\label{\detokenize{security:running-the-stunnel-setup-program}}
\sphinxAtStartPar
At this point you’re ready to run the stunnel setup program.  So at this point type the following:

\begin{sphinxVerbatim}[commandchars=\\\{\}]
\PYG{o}{.}\PYG{o}{/}\PYG{n}{stunnel\PYGZus{}setup}\PYG{o}{.}\PYG{n}{sh}
\end{sphinxVerbatim}

\sphinxAtStartPar
This will prompt you through settings.  Most you can select the defaults to.
The SSL port you may need to change based on your administrative requirements.
It will prompt you with whatever you set for your mush name.  If you have not
selected a mush name at this point, you can select the defaults.

\sphinxAtStartPar
You will want to use the config file for warpbubble as this hides the secret.

\sphinxAtStartPar
Be aware that if you have DNS host lookups disabled on your mush, you
MUST have 127.0.0.1 as an entry for your sconnect\_host file.


\paragraph{Starting the stunnel proxy}
\label{\detokenize{security:starting-the-stunnel-proxy}}
\sphinxAtStartPar
When you have your stunnel.conf file to the way you want, you then
issue the following command to run your SSL layer:

\begin{sphinxVerbatim}[commandchars=\\\{\}]
\PYG{o}{.}\PYG{o}{/}\PYG{n}{stunnel\PYGZus{}start}\PYG{o}{.}\PYG{n}{sh}
\end{sphinxVerbatim}


\paragraph{Shutting down the stunnel proxy}
\label{\detokenize{security:shutting-down-the-stunnel-proxy}}
\sphinxAtStartPar
If ever you need to bring down the SSL layer, you may kill it with the command:

\begin{sphinxVerbatim}[commandchars=\\\{\}]
\PYG{o}{.}\PYG{o}{/}\PYG{n}{stunnel\PYGZus{}stop}\PYG{o}{.}\PYG{n}{sh}
\end{sphinxVerbatim}


\paragraph{Configuring firewall on the host}
\label{\detokenize{security:configuring-firewall-on-the-host}}
\sphinxAtStartPar
Please be aware that the port that the SSL layer is on must be opened
in any firewall rule you specified to allow the connectivity.  This also must
not be the port the mush is running on and requires a separate port.


\section{Maintenance}
\label{\detokenize{maintenance:maintenance}}\label{\detokenize{maintenance::doc}}

\subsection{Signals and why you need them for control}
\label{\detokenize{maintenance:signals-and-why-you-need-them-for-control}}
\sphinxAtStartPar
Rhost by default allows signals at the shell to handle various processes in\sphinxhyphen{}game.

\sphinxAtStartPar
The following signals are useful.


\subsubsection{TERM (kill \sphinxhyphen{}TERM or kill \sphinxhyphen{}15)}
\label{\detokenize{maintenance:term-kill-term-or-kill-15}}\begin{itemize}
\item {} 
\sphinxAtStartPar
This will immediately terminate the mush, dumping a special flatfile called
netrhost.db.TERM and scramming the db in question by force\sphinxhyphen{}closing it
without any writes.  A TERM is the effort for the mush to shut down the
mush as fast as possible to avoid any db corruption if possible since
a TERM signal is common during a server shutdown, so time is paramount.

\end{itemize}


\subsubsection{USR1 (kill \sphinxhyphen{}USR1)}
\label{\detokenize{maintenance:usr1-kill-usr1}}\begin{itemize}
\item {} 
\sphinxAtStartPar
This by default issues a reboot on the server.  This is a special parameter
because this can actually be changed in\sphinxhyphen{}game to do any number of other
things.  Please refer to the RhostMUSH running in question if this is
the default behavior or if the method for USR1 is doing something else.

\end{itemize}


\subsubsection{USR2 (kill \sphinxhyphen{}USR2)}
\label{\detokenize{maintenance:usr2-kill-usr2}}\begin{itemize}
\item {} 
\sphinxAtStartPar
This will shutdown (cleanly) the mush.  This behaves as if you issued
a @shutdown from within the game, and follows all proper procedure
in bringing the game down cleanly and safely.  This shoudl be used
when doing maintenance on the game or if you need to bring it down
from the shell.

\end{itemize}


\subsubsection{KILL (kill \sphinxhyphen{}KILL or kill \sphinxhyphen{}9)}
\label{\detokenize{maintenance:kill-kill-kill-or-kill-9}}\begin{itemize}
\item {} 
\sphinxAtStartPar
This signal can not be caught and will immediately terminate the game
without any safty to the database at all.  Short of something horribly
wrong going on, this should never be used to bring down your mush.
Doing so will almost certainly corrupt your databases (ALL OF THEM)
that are open, including but not limited to your main database, your
mail database, your autoregistration database, and so forth.  So if
you do this, plan to do some database recovery from your flat files.
Also, when you bring down a mush in this manner, you need to issue
Startmush \sphinxhyphen{}f to bring it back up.

\end{itemize}


\subsection{Shutting down gracefully}
\label{\detokenize{maintenance:shutting-down-gracefully}}

\subsubsection{Rhostmush has many ways to shut down the game cleanly}
\label{\detokenize{maintenance:rhostmush-has-many-ways-to-shut-down-the-game-cleanly}}\begin{enumerate}
\sphinxsetlistlabels{\arabic}{enumi}{enumii}{}{.}%
\item {} 
\sphinxAtStartPar
Log into the mush and issue @shutdown

\item {} 
\sphinxAtStartPar
Issue a kill \sphinxhyphen{}USR2 to the mush which issues an emergency @shutdown

\item {} 
\sphinxAtStartPar
Issue a kill \sphinxhyphen{}TERM to the mush which issues an emergency abort and clean shutdown.

\end{enumerate}


\subsubsection{WARNING: Never kill \sphinxhyphen{}9 Rhost}
\label{\detokenize{maintenance:warning-never-kill-9-rhost}}
\sphinxAtStartPar
Under NO CIRCUMSTANCES should you kill \sphinxhyphen{}9 your mush unless you don’t care for the
database.  The reason is if the mush happens to be saving, in any method, to the
database, especially a QDBM database, you will likely have just corrupted your
database, so pull out a flatfile to recover.

\sphinxAtStartPar
Sadly, this also may occur if the server hosting you takes a nose\sphinxhyphen{}dive in the middle
of a db write.  Rhost can recover corruption in\sphinxhyphen{}game while up, but if it bombs
in the middle of a write, all bets are off. :)


\subsection{Autoshutdown script}
\label{\detokenize{maintenance:autoshutdown-script}}
\sphinxAtStartPar
The makefile will ‘make’ the program that will STOP the mush.
Please tweek ‘autolog.h’ with the proper parameters.

\sphinxAtStartPar
The ‘startup.sh’ will START the mush.

\sphinxAtStartPar
Both of these are intended to be used for automations (automated processes)
like your rc.local file and/or startup scripts when you bring your server up.


\subsection{Network Port redirector}
\label{\detokenize{maintenance:network-port-redirector}}
\sphinxAtStartPar
This is a port redirector incase you decide to move your mush
to a new site/port.  To use, first, compile the code.  To do
this you would type the following:
\begin{quote}

\sphinxAtStartPar
cc portmsg.c \sphinxhyphen{}o portmsg
\end{quote}

\sphinxAtStartPar
if ‘cc’ is not defined, try the following:
\begin{quote}

\sphinxAtStartPar
gcc portmsg.c \sphinxhyphen{}o portmsg
\end{quote}

\sphinxAtStartPar
Once compiled, you would then modify the file ‘file’ to describe
the mush, what was done, where it’s moved to, then specify the
IP address and the PORT where specified.

\sphinxAtStartPar
To launch the application, you would then type:
\begin{quote}

\sphinxAtStartPar
./portmsg file \textless{}port\textgreater{}
\end{quote}

\sphinxAtStartPar
Where \textless{}port\textgreater{} is the port where the mush used to run at.


\subsection{Using the built\sphinxhyphen{}in cron system for periodically running commands}
\label{\detokenize{maintenance:using-the-built-in-cron-system-for-periodically-running-commands}}

\subsubsection{Syntax for rhost.cron}
\label{\detokenize{maintenance:syntax-for-rhost-cron}}
\sphinxAtStartPar
The rhost.cron file will be in the syntax as follows:

\sphinxAtStartPar
username (or dbref\#)
command1;command2;command3;…;commandN
command
command
command1;command2;command3;…;commandN

\sphinxAtStartPar
You can have commands strung together with a semicolon
on the same line.  This counts as a single line of input.
You can have at most 20 lines of commands after the target
you wish to execute the commands as.  The target may
be a player name OR a dbref\# of any valid dbref\# within
the game.  Invalid targets will abort the cron process.
Non\sphinxhyphen{}printable characters in the cron file will abort
the process.  Any aborts or warnings will be logged
in the netrhost.gamelog.

\sphinxAtStartPar
Here is a working example of the code cron file.
This example will perform dumps of the mush.


\subsubsection{Example syntaxt for rhost.cron}
\label{\detokenize{maintenance:example-syntaxt-for-rhost-cron}}
\sphinxAtStartPar
\#1
@dump/flat; @@ dump the main game database to flatfile
wmail/unload; @@ dump the mail database to flatfile
@areg/unload; @@ dump the registration database to flatfile
newsdb/unload; @@ dump the news bbs database to flatfile


\subsection{The following scripts are used in the game directory}
\label{\detokenize{maintenance:the-following-scripts-are-used-in-the-game-directory}}
\sphinxAtStartPar
Startmush               \textendash{} used to Start up the mush
backup\_flat.sh          \textendash{} Used to run backups with @dump/flat within the game (Started with Startmush automatically)
backup\_restart.sh       \textendash{} Restart the backup\_flat.sh if changes are done
compress\_logs.sh        \textendash{} Compress logs in ‘oldlogs’.  Ran with Startmush
findit.sh               \textendash{} Internal script used to check for flatfile validity
mailhide.sh             \textendash{} Wrapper to hide from address using the ‘mail’ progam
minimal.sh              \textendash{} Auto\sphinxhyphen{}load the minimal db into the mush
proxysnarf.sha          \textendash{} Internal script for the tor\_pull.sh tor proxy blacklist
tor\_pullit.sh           \textendash{} Internal script for the tor\_pull.sh for proxy blacklist
recovery.sh             \textendash{} If your db is corrupt, run this to auto\sphinxhyphen{}revert to an earlier flatfile
tor\_pull.sh             \textendash{} Create a blacklist.txt file that can be loaded via the internal @blacklist command for proxy handling


\subsection{Textfiles for RhostMUSH}
\label{\detokenize{maintenance:textfiles-for-rhostmush}}
\sphinxAtStartPar
areghost.txt           \sphinxhyphen{} file player gets when autoregistration on registered host.
autoreg.txt            \sphinxhyphen{} file player gets when autoregistration on non\sphinxhyphen{}registered host.
autoreg\_include.txt    \sphinxhyphen{} file player receives in email when they autoregister attached to login/passwd
badsite.txt            \sphinxhyphen{} file player gets when site is not allowed.
connect.txt            \sphinxhyphen{} file player gets when connect
create\_reg.txt         \sphinxhyphen{} file player gets when their site is register and they can’t create.
doorconf.txt           \sphinxhyphen{} file that is searched for information regarding @door.

\begin{sphinxadmonition}{note}{Note:}
\sphinxAtStartPar
Need to mkindx doorconf.txt doorconf.indx for this file
\end{sphinxadmonition}

\sphinxAtStartPar
down.txt               \sphinxhyphen{} file player gets when the mush has logins disabled (@disable login)
error.txt              \sphinxhyphen{} the ‘Huh? (type help for help)’ messages.

\begin{sphinxadmonition}{note}{Note:}
\sphinxAtStartPar
Need to mkindx error.txt error.indx for this file
\end{sphinxadmonition}

\sphinxAtStartPar
full.txt               \sphinxhyphen{} file player gets when the mush can’t have any more players.
guest.txt              \sphinxhyphen{} file player gets when they connect as a guest.
help.txt               \sphinxhyphen{} your help file

\begin{sphinxadmonition}{note}{Note:}
\sphinxAtStartPar
Need to mkindx help.txt help.indx for this file
\end{sphinxadmonition}

\sphinxAtStartPar
motd.txt               \sphinxhyphen{} your motd file
news.txt               \sphinxhyphen{} your news file

\begin{sphinxadmonition}{note}{Note:}
\sphinxAtStartPar
Need to mkindx news.txt news.indx for this file
\end{sphinxadmonition}

\sphinxAtStartPar
newuser.txt            \sphinxhyphen{} file newly created players get when they connect for the first time.
noguest.txt            \sphinxhyphen{} file player gets when they are not allowed to connect to a guest.
plushelp.txt           \sphinxhyphen{} optional +help file. (needs compile time option)

\begin{sphinxadmonition}{note}{Note:}
\sphinxAtStartPar
Need to mkindx plushelp.txt plushelp.indx for this file
\end{sphinxadmonition}

\sphinxAtStartPar
quit.txt               \sphinxhyphen{} file player gets when they disconnect.
register.txt           \sphinxhyphen{} file player gets when the site is locked down for registration.
wizhelp.txt            \sphinxhyphen{} your wizhelp file

\begin{sphinxadmonition}{note}{Note:}
\sphinxAtStartPar
Need to mkindx wizhelp.txt wizhelp.indx for this file
\end{sphinxadmonition}

\sphinxAtStartPar
wizmotd.txt            \sphinxhyphen{} your wiz motd file


\subsubsection{Textfile Frequently Asked Questions}
\label{\detokenize{maintenance:textfile-frequently-asked-questions}}
\sphinxAtStartPar
Q:  How do I put color in these files?
A1: Look at ansi.h and you need to put the literal ASCII codes.  They will look like: \textasciicircum{}{[}{[}0m (for ANSI\_NORMAL).  That’s \textless{}ESC\textgreater{}{[}
A2: You can enable ansi\_txtfiles then use \%c (or \%x/\%m) encoding for ansi, however you compiled your Rhost.

\sphinxAtStartPar
Q:  I want to design my own txt files to read in the mush.
A:  Easy.  Design them like help.txt would be set up, mkindx the file, then you can access it via @dynhelp online.

\sphinxAtStartPar
Q:  Do I have to mkindx these files whenever I make changes?
A:  Only the ones that have ‘\& ‘ index. (help.txt, wizhelp.txt, news.txt, etc)

\sphinxAtStartPar
Q:  Do I have to @readcache in the game whenever I make a change?
A:  Only when you modify any of the files listed in README.TXTFILES.  Not the ones you use with @dynhelp.

\sphinxAtStartPar
Q:  Can’t I just make code in the mush that then is used for these silly txt files?
A:  Absolutely.  Check ‘wizhelp file\_object’.


\section{Troubleshooting}
\label{\detokenize{troubleshooting:troubleshooting}}\label{\detokenize{troubleshooting::doc}}

\subsection{Stack limit and debugging}
\label{\detokenize{troubleshooting:stack-limit-and-debugging}}
\sphinxAtStartPar
Rhost uses a stack limit in the debug monitor.

\sphinxAtStartPar
This stack limit is set to a reasonable amount of 1000.
This is defined in the debug.h file in the hdrs directory.

\sphinxAtStartPar
This will directly impact the function\_recursion\_limit from being
set above 100.  If, for whatever reason, you really must have
a ridiculously high recursion limit, then it is a suggestion to
manually modify the stack limit in debug.h to a higher number.

\sphinxAtStartPar
We have reasonably set it to 10000 without too much issue, but keep
in mind, the overhead is higher for every stack you throw on the
process table.  Higher stack means more memory used.

\sphinxAtStartPar
Also be aware that your shell stack limit directly is affected
to this value.

\sphinxAtStartPar
Type: ulimit \sphinxhyphen{}a

\sphinxAtStartPar
This will show you your shell stack limits.  Do NOT set the
STACKMAX value higher than your shell’s stack value.

\sphinxAtStartPar
The value in \textasciitilde{}/Rhost/Server/hdrs/debug.h is currently set as:

\sphinxAtStartPar
\#define STACKMAX 1000

\sphinxAtStartPar
Feel free to change this to a higher value if you wish.

\sphinxAtStartPar
The caveat.  This effects the debug stack daemon.  Meaning,
the only way for this to be updated is through @shutdown and
then a fresh ./Startmush.

\sphinxAtStartPar
A @reboot WILL NOT LOAD IN A NEW DEBUG MONITOR!!!!

\sphinxAtStartPar
You can issue @list stack to see the current stack ceiling ingame.


\subsection{How to reset the password for \#1}
\label{\detokenize{troubleshooting:how-to-reset-the-password-for-1}}
\sphinxAtStartPar
You can only use one of these options at a time. Make sure to change back your nerhost.conf after making the changes.


\subsubsection{Method 1}
\label{\detokenize{troubleshooting:method-1}}
\sphinxAtStartPar
in your netrhost.conf file add:
newpass\_god 777

\sphinxAtStartPar
This will reset \#1’s password to the default ‘Nyctasia’.


\subsubsection{Method 2}
\label{\detokenize{troubleshooting:method-2}}
\sphinxAtStartPar
in your netrhost.conf file add:
newpass\_god 1

\sphinxAtStartPar
This will allow IMMORTAL players to @newpassword \#1 upon reboot.


\subsection{Troubleshooting difficulties compiling RhostMUSH}
\label{\detokenize{troubleshooting:troubleshooting-difficulties-compiling-rhostmush}}

\subsubsection{Changes to conf for high\sphinxhyphen{}bit CPUs}
\label{\detokenize{troubleshooting:changes-to-conf-for-high-bit-cpus}}
\sphinxAtStartPar
RhostMUSH automatically detects 64\sphinxhyphen{}bit platforms, and should compile
cleanly on these.

\sphinxAtStartPar
In case you are trying to compile Rhost on some other crazy\sphinxhyphen{}wide CPUs
such as the PS2, PS3 or other 128 or 256 bit CPUs, you can easily do
so by changing a few lines of code in conf.c.

\sphinxAtStartPar
change:
typedef unsigned int    pmath1;
typedef int             pmath2;
\#define ALLIGN1 4

\sphinxAtStartPar
to:
typedef unsigned long   pmath1;
typedef long            pmath2;
\#define ALLIGN1 8

\sphinxAtStartPar
, replacing 8 with the size of your CPU’s long integer. (4 for 32\sphinxhyphen{}bit,
8 for 64\sphinxhyphen{}bit, 16 for 128\sphinxhyphen{}bit, etc etc)

\sphinxAtStartPar
RhostMUSH has only been tested to work on the AMD64, but there is no
reason to believe the same will not hold true for IA64.


\subsubsection{Changes to autconf for certain systems}
\label{\detokenize{troubleshooting:changes-to-autconf-for-certain-systems}}
\sphinxAtStartPar
You should not have to worry about this, but incase something really
weird occurs, you may need to look into these changes…

\sphinxAtStartPar
The autoconfig.h file needs to have modifications to it by hand.

\sphinxAtStartPar
There are three manual entries:

\sphinxAtStartPar
This one sets how it defines the int to character pointer.  It’s safe
to keep it as ‘unsigned int’ for 32 bit platforms.  For non 32\sphinxhyphen{}bit,
define it to  how an int is defined on that system.
\begin{quote}

\sphinxAtStartPar
typedef unsigned int    pmath1;
\end{quote}

\sphinxAtStartPar
This one sets how it defines the signed int to character pointer.  Same
restrictions apply as above for unsigned int.
\begin{quote}

\sphinxAtStartPar
typedef int     pmath2;
\end{quote}

\sphinxAtStartPar
This sets the allignment for the given platform.  4 represents a 32 bit
platform.  8 would represent a 64 bit platform, etc.  Change accordingly.
\begin{quote}

\sphinxAtStartPar
\#define ALLIGN1 4
\end{quote}

\sphinxAtStartPar
Make sure these three entries are defined in your autoconf.h file else
the mush will not compile.


\subsection{Dealing with DB Corruption}
\label{\detokenize{troubleshooting:dealing-with-db-corruption}}
\sphinxAtStartPar
Ok.  Your database won’t come up.

\sphinxAtStartPar
If you are reading this, then likely the scenerio is one of the following:
\begin{enumerate}
\sphinxsetlistlabels{\arabic}{enumi}{enumii}{}{.}%
\item {} 
\sphinxAtStartPar
The mush says it can’t find your database files.

\item {} 
\sphinxAtStartPar
The mush says it can’t read or load your database files.

\item {} 
\sphinxAtStartPar
The mush seems to load fine but I can’t log in anyone and when I do
all the names and attributes of things seem to be gone!

\item {} 
\sphinxAtStartPar
Bringing up your mail database

\end{enumerate}

\sphinxAtStartPar
First thing is first.  Don’t have a panic attack.


\subsubsection{If the mush says it can’t find your database files}
\label{\detokenize{troubleshooting:if-the-mush-says-it-can-t-find-your-database-files}}

\paragraph{Check the names of the database files in your ‘data’ directory}
\label{\detokenize{troubleshooting:check-the-names-of-the-database-files-in-your-data-directory}}
\sphinxAtStartPar
They should be named something like:
netrhost.db
netrhost.db.old
netrhost.db.old.prev
netrhost.gdbm.dir
netrhost.gdbm.pag

\sphinxAtStartPar
And you may see a netrhost.db.flat


\paragraph{Check your netrhost.conf file}
\label{\detokenize{troubleshooting:check-your-netrhost-conf-file}}
\sphinxAtStartPar
If you never touched the *database or muddb\_name params, you should be good.

\sphinxAtStartPar
Verify your *database params (and muddb\_name) are still set to ‘netrhost’ as
part of the name.  Ergo, the default values and you didn’t change them.
These should match up with the filenames in your data directory.

\sphinxAtStartPar
If these names do not match up, it can’t find the database files to load.
So you shouldn’t have to change these names, ever. :)


\paragraph{Check your mush.config file}
\label{\detokenize{troubleshooting:check-your-mush-config-file}}
\sphinxAtStartPar
If you never modified this file, you should be good.

\sphinxAtStartPar
The gamename should be ‘netrhost’ for this file.  This does NOT control
the name of your game.  This controls the name of all the files
as associated to the mush.  So changing this means the netrhost.conf
file, all your database files, and so forth.  Please don’t change this :)


\subsubsection{If the mush says it can’t read or load your database files}
\label{\detokenize{troubleshooting:if-the-mush-says-it-can-t-read-or-load-your-database-files}}
\sphinxAtStartPar
Double check everything for the previous issue. Make sure everything is named properly.


\paragraph{Verify you have enough disk space. (quota)}
\label{\detokenize{troubleshooting:verify-you-have-enough-disk-space-quota}}
\sphinxAtStartPar
Some account have a limited quota to run in.  If you reached or exceed
your disk quota, you can have a corrupted database.  So always keep
your eye on the size.  quota \sphinxhyphen{}s to see a human readable format to see
how much quota you have left.  You want to make sure current in use is
below the ‘grace’ and soft/hard limits shown.  If not, you’re out of
space.

\sphinxAtStartPar
You will need to remove some files before you repair and bring up your
mush again.  Try to keep your quota at least 200 megs free to allow
plenty of wonderful growth space for the mush.


\paragraph{Verify you have enough disk space.  (system)}
\label{\detokenize{troubleshooting:verify-you-have-enough-disk-space-system}}
\sphinxAtStartPar
The second way you can run out of disk space is by the filesystem itself.
do a df \sphinxhyphen{}h . in your ‘game’ directory’.  That is df \sphinxhyphen{}h \textless{}period\textgreater{}.
This will return how much disk space is being used and how much remains.
If it shows 100\% used, you’re out of disk space and the db is corrupt.

\sphinxAtStartPar
At this point, you’re pretty screwed.  You can see if anything exists
in your system to free up some space, but if the filesystem itself
is filled, reach out to the owner of the server and let them know.
It’s a much bigger deal than just your mush if that’s the case.

\sphinxAtStartPar
Until this issue is resolved, you can not repair and bring up your mush.
No disk to run the game.


\subsubsection{If the mush seems to load fine but I can’t log in anyone and when I do all the names and attributes of things seem to be gone!}
\label{\detokenize{troubleshooting:if-the-mush-seems-to-load-fine-but-i-can-t-log-in-anyone-and-when-i-do-all-the-names-and-attributes-of-things-seem-to-be-gone}}
\sphinxAtStartPar
Ok, at this point you likely had your mush up when the physical server
went down hard.  Weither through an emergency shutdown or a physical
power outage, your db likely was brought down hard during a write,
so it left it in a corrupt state.  These things happen.  This is
why we always strongly request you make daily flatfile dumps.

\sphinxAtStartPar
So, to recover your database.


\paragraph{The bad news}
\label{\detokenize{troubleshooting:the-bad-news}}
\sphinxAtStartPar
If you have no flatfile backup or never bothered with backups?
I’m sorry, at this point you’re SOA.  There’s no easy way to
recover a corrupted binary database.  If you absolutely need
data out of it we may be able to help you to piece meal things
out of it, but otherwise it’s a lost cause.  You’ll have to start
over from scratch.  I’m sorry.


\paragraph{The good news}
\label{\detokenize{troubleshooting:the-good-news}}
\sphinxAtStartPar
If you made backups, or if the server had a normal shutdown, you
likely have a flatfile backup.  You will see a netrhost.db.flat
in either the ‘data’ directory or ‘prevflat’ directory.  That
is your manual flatfile backup.

\sphinxAtStartPar
If the server had a normal shutdown, you will see a file called
netrhost.db.TERMFLAT.  This is a scram\sphinxhyphen{}emergency db flatfile.
It attempts to write this at the time of server shutdown to
hopefully keep a clean backup in the case of issues since
it identifies the server is coming down hard.  Make sure
if you plan to use the TERMFLAT as your recovery flatfile
that the very last line shows something like ** END OF DUMP **.
That shows you had a successful backup.


\paragraph{Now, to restore your database?}
\label{\detokenize{troubleshooting:now-to-restore-your-database}}
\sphinxAtStartPar
Please refer to the file ‘README.DBLOADING’.


\subsubsection{Bringing up your mail database}
\label{\detokenize{troubleshooting:bringing-up-your-mail-database}}
\sphinxAtStartPar
Your mail db may or may not come up at this point.


\paragraph{If after restoring main database your mail database works}
\label{\detokenize{troubleshooting:if-after-restoring-main-database-your-mail-database-works}}\begin{description}
\item[{If your mail database came up and does not show}] \leavevmode
\sphinxAtStartPar
‘Mail: mail is currently off’ then you should be good to go.

\end{description}

\sphinxAtStartPar
Please issue:
wmail/fix
wmail/lfix

\sphinxAtStartPar
This will put your mail system in sync with your current database and
fix up any errors that may exist.  wmail/fix fixes the mail, wmail/lfix
loads in the fixes.


\paragraph{If after restoring main database your mail database does not work}
\label{\detokenize{troubleshooting:if-after-restoring-main-database-your-mail-database-does-not-work}}\begin{quote}
\begin{description}
\item[{If your mail database is not up and shows}] \leavevmode
\sphinxAtStartPar
‘Mail: mail is currently off’ then your mail db is currupt.

\end{description}
\end{quote}

\sphinxAtStartPar
To fix your mail db please refer to file ‘README.MAILCANNOTLOAD’


\subsection{Dealing with a corrupt mail database}
\label{\detokenize{troubleshooting:dealing-with-a-corrupt-mail-database}}
\sphinxAtStartPar
It says when you try to access mail that mail is disabled and/or off.

\sphinxAtStartPar
Nothing you do can bring it on line.  Well, this is bad, but not horrible.

\sphinxAtStartPar
The mail db is totally separate from the main game database.  This means
that it in no way damaged or corrupted your main mush database.

\sphinxAtStartPar
The bad news?  Yes.  Your mail database is corrupt.  To bring it back,
is it hopes that you read ahead of time about how to backup your mush,
which would include the mail database.


\subsubsection{Backing up your mail database}
\label{\detokenize{troubleshooting:backing-up-your-mail-database}}
\sphinxAtStartPar
wmail/unload \textendash{} this flatfile dumps your mail db.  You should run it daily.

\sphinxAtStartPar
To recover your mail, it assumes you have a mail flatfile in either the
\textasciitilde{}/Server/game/data directory or the \textasciitilde{}/Server/game/prevflat directory.  The
latter directory is used in junction to the backup\_flat.sh and will always
house the latest flatfile if not one recently dumped in your data directory.


\subsubsection{Automatically recovering your mail database}
\label{\detokenize{troubleshooting:automatically-recovering-your-mail-database}}
\sphinxAtStartPar
wmail/load

\sphinxAtStartPar
Yup, that’s it.  It’ll take care of everything else.  Isn’t automation grand?

\sphinxAtStartPar
Doesn’t even require a reboot :)
\begin{description}
\item[{NOTE:  You may at this point wish to run the following:}] \leavevmode
\sphinxAtStartPar
wmail/fix  \textendash{} this fixes the mail database and sync’s it to the mush db.
wmail/lfix \textendash{} this loads in the fixed mail database

\end{description}

\sphinxAtStartPar
If you have a very old mail database, this is likely going to be required
to sync against nuked players and other changes to the game since the flatfile.

\sphinxAtStartPar
If this is a new db that you have, you can skip the fixing.


\subsubsection{Manually recovering your mail database}
\label{\detokenize{troubleshooting:manually-recovering-your-mail-database}}
\sphinxAtStartPar
To recover your mail manually, you need to delete your mail databases,
reboot, then reload your mail flatfiles.  If you have no mail flatfiles,
well, you’re going to have to start over with the mail database.  Sorry.

\sphinxAtStartPar
First, go into the ‘game’ subdirectory.  Inside that directory is a ‘data’
directory.

\sphinxAtStartPar
You will be deleting all the files with the following names:

\sphinxAtStartPar
RhostMUSH.mail.*                (like RhostMUSH.mail.dir/RhostMUSH.mail.pag)
RhostMUSH.folder.*              (like RhostMUSH.folder.dir/RhostMUSH.folder.pag)

\sphinxAtStartPar
DO NOT DELETE OTHER NAMED FILES!!!

\sphinxAtStartPar
Once these files are deleted, you may issue a @reboot to restart the mush.
This will unlock the mail system and load in a fresh db.

\sphinxAtStartPar
Now, if you have flatfiles of the old mail database, you will see in either
the ‘data’ directory or the ‘prevflat’ directory files that are called:

\sphinxAtStartPar
RhostMUSH.dump.folder
RhostMUSH.dump.mail

\sphinxAtStartPar
Make sure these two files are in the ‘data’ subdirectory.  Copy them in
if they exist in your ‘prevflat’ directory.

\sphinxAtStartPar
Once they are in the ‘data’ directory, within the mush type: wmail/load

\sphinxAtStartPar
This loads in the flatfile and recover the mail database.

\sphinxAtStartPar
Now, at this point the mail db may not be 100\% in\sphinxhyphen{}sync with the game db.

\sphinxAtStartPar
So let’s fix it.

\sphinxAtStartPar
wmail/fix   \textendash{} this will run a fix on the mail db and repair any issues.

\sphinxAtStartPar
wmail/lfix  \textendash{} this will load the fixed flatfile back into the mush.

\sphinxAtStartPar
At this point you should be good to go.


\section{Upgrading}
\label{\detokenize{upgrade:upgrading}}\label{\detokenize{upgrade::doc}}

\subsection{Converting database betwen GDBM and QDBM}
\label{\detokenize{upgrade:converting-database-betwen-gdbm-and-qdbm}}
\sphinxAtStartPar
Ok, if you plan to recompile your game that is using GDBM to QDBM, or visa versa
some bad news.

\sphinxAtStartPar
The databases are NOT COMPATIBLE to each other, at least in the binary form.


\subsubsection{Downgrading QDBM to GDBM}
\label{\detokenize{upgrade:downgrading-qdbm-to-gdbm}}
\begin{sphinxadmonition}{warning}{Warning:}
\sphinxAtStartPar
I would NEVER change from QDBM back to GDBM, but if you’re set on it these steps:

\sphinxAtStartPar
You would use the same steps if you plan to move QDBM to GDBM.  I however would
not do this.  Moving from QDBM to GDBM is a huge step backwards.  Seriously,
don’t do it unless you have absolutely no other recourse.

\sphinxAtStartPar
IF you plan (for whatever reason) to move from QDBM to GDBM, you should verify
the following
\end{sphinxadmonition}
\begin{enumerate}
\sphinxsetlistlabels{\arabic}{enumi}{enumii}{}{.}%
\item {} 
\sphinxAtStartPar
You have on a 64 bit system, no object that has more than 400 attributes on it.

\item {} 
\sphinxAtStartPar
You have on a 32 bit system, no object that has more than 750 attributes on it.

\item {} 
\sphinxAtStartPar
Any CONTENT of any attribute must be below 4000 characters in length.

\item {} 
\sphinxAtStartPar
Once you have that done, you may follow the procedures below on converting (upgrade) from GDBM to QDBM.  This works the same as converting (downgrading) QDBM back down to GDBM

\end{enumerate}


\subsubsection{Upgradging GDBM to QDBM}
\label{\detokenize{upgrade:upgradging-gdbm-to-qdbm}}
\sphinxAtStartPar
Now, if you’ve kept reading and plan to convert your GDBM database to QDBM great news!
It’s more stable, it’s faster, and lets you have far more flexibility.

\sphinxAtStartPar
So, BEFORE YOU RECOMPILE YOUR CODE.  This is what you have to do.


\paragraph{While logged in to your mush, issue the following commands}
\label{\detokenize{upgrade:while-logged-in-to-your-mush-issue-the-following-commands}}\begin{enumerate}
\sphinxsetlistlabels{\Alph}{enumi}{enumii}{}{.}%
\item {} 
\sphinxAtStartPar
@dump/flat    \textendash{} This will make a flatfile dump of your MUSH database

\item {} 
\sphinxAtStartPar
wmail/unload  \textendash{} This will make a flatfile dump of your MAIL database

\item {} 
\sphinxAtStartPar
@areg/unload  \textendash{} If you use the AutoRegistration engine, this dumps it

\item {} 
\sphinxAtStartPar
newsdb/unload \textendash{} If you use the hardcoded news/bbs engine.  This dumps it

\end{enumerate}


\paragraph{Verify the files exist}
\label{\detokenize{upgrade:verify-the-files-exist}}\begin{enumerate}
\sphinxsetlistlabels{\Alph}{enumi}{enumii}{}{.}%
\item {} 
\sphinxAtStartPar
Server/game/data/netrhost.db.flat

\item {} 
\sphinxAtStartPar
Server/game/data/RhostMUSH.dump.folder
Server/game/data/RhostMUSH.dump.mail

\item {} 
\sphinxAtStartPar
(Optional) Server/game/data/RhostMUSH.areg.dump

\item {} 
\sphinxAtStartPar
(Optional) Server/game/data/RhostMUSH.news.flat

\end{enumerate}


\paragraph{Shutdown the MUSH}
\label{\detokenize{upgrade:shutdown-the-mush}}
\sphinxAtStartPar
@shutdown your mush


\paragraph{From the Server directory}
\label{\detokenize{upgrade:from-the-server-directory}}\begin{enumerate}
\sphinxsetlistlabels{\Alph}{enumi}{enumii}{}{.}%
\item {} 
\sphinxAtStartPar
make clean

\item {} 
\sphinxAtStartPar
make confsource
1.  Select QDBM and if you wish at this time increase your LBUF size
2.  Select any other options you may want

\item {} 
\sphinxAtStartPar
(r)un and let it compile.

\item {} 
\sphinxAtStartPar
Main DB: Delete (rm) the following files (from Rhost/Server/game/data)
netrhost.gdbm*
netrhost.db
netrhost.db.new
netrhost.db.new.prev

\item {} 
\sphinxAtStartPar
Mail DB: Delete (rm) the following files (from Rhost/Server/game/data)
RhostMUSH.folder.dir
RhostMUSH.folder.pag
RhostMUSH.mail.dir
RhostMUSH.mail.pag

\item {} 
\sphinxAtStartPar
(Optional) AutoReg DB: Delete (rm) the following files (from Rhost/Server/game/data)
RhostMUSH.areg.dir
RhostMUSH.areg.pag

\item {} 
\sphinxAtStartPar
(Optional) News/BBS DB: Delete (rm) the following files (from Rhost/Server/game/data)
RhostMUSH.news.dir
RhostMUSH.news.pag

\end{enumerate}


\paragraph{From the Server/game directory}
\label{\detokenize{upgrade:from-the-server-game-directory}}\begin{enumerate}
\sphinxsetlistlabels{\Alph}{enumi}{enumii}{}{.}%
\item {} 
\sphinxAtStartPar
./db\_load data/netrhost.gdbm data/netrhost.db.flat data/netrhost.db.new

\item {} 
\sphinxAtStartPar
./Startmush

\end{enumerate}


\paragraph{While logged into the mush issue the following commands}
\label{\detokenize{upgrade:while-logged-into-the-mush-issue-the-following-commands}}\begin{enumerate}
\sphinxsetlistlabels{\Alph}{enumi}{enumii}{}{.}%
\item {} 
\sphinxAtStartPar
Load in the mail database
wmail/load

\item {} 
\sphinxAtStartPar
(optional) Load in the autoreg database
@areg/load

\item {} 
\sphinxAtStartPar
(optional) Load in the news/bbs database
newsdb/load

\end{enumerate}


\paragraph{Verify that you have QDBM running and your valid values}
\label{\detokenize{upgrade:verify-that-you-have-qdbm-running-and-your-valid-values}}\begin{enumerate}
\sphinxsetlistlabels{\Alph}{enumi}{enumii}{}{.}%
\item {} 
\sphinxAtStartPar
@list options system

\item {} 
\sphinxAtStartPar
@list options (spammy)

\end{enumerate}


\subsection{Updating RhostMUSH prior to 3.9.5p2}
\label{\detokenize{upgrade:updating-rhostmush-prior-to-3-9-5p2}}
\sphinxAtStartPar
Ok.

\sphinxAtStartPar
So you’re running an old RhostMUSH.

\sphinxAtStartPar
One prior to 3.9.5p2 and want to take advantage of the new
format of the Makefile and the automated mysql stuff and
all the other goodies that isn’t really (easilly) done
with just patch.sh.

\sphinxAtStartPar
Well, you’re in luck.  It is actually fairly easy to do.

\sphinxAtStartPar
This is what you have to do.

\sphinxAtStartPar
First thing’s first.
\begin{enumerate}
\sphinxsetlistlabels{\arabic}{enumi}{enumii}{}{.}%
\item {} 
\sphinxAtStartPar
Log into your existing mush.  Let’s make current backups
of all your flatfiles.  Issue:
A. @dump/flat
B. wmail/unload
C. @areg/unload
D. newsdb/unload

\item {} 
\sphinxAtStartPar
Shutdown your game (@shutdown)

\item {} 
\sphinxAtStartPar
Make an image of all your current backed up files.  From The Server/game directory you would type:

\begin{sphinxVerbatim}[commandchars=\\\{\}]
\PYG{o}{.}\PYG{o}{/}\PYG{n}{backup\PYGZus{}flat}\PYG{o}{.}\PYG{n}{sh} \PYG{o}{\PYGZhy{}}\PYG{n}{s}
\end{sphinxVerbatim}

\sphinxAtStartPar
Please remember the ‘\sphinxhyphen{}s’ to the ./backup\_flat.sh.

\item {} 
\sphinxAtStartPar
Make note of the most recently created file in the directory Server/game/oldflat.  It’s usually named something like:

\begin{sphinxVerbatim}[commandchars=\\\{\}]
\PYG{n}{RhostMUSH}\PYG{o}{.}\PYG{n}{dbflat1}\PYG{o}{.}\PYG{n}{tar}\PYG{o}{.}\PYG{n}{gz}
\end{sphinxVerbatim}

\sphinxAtStartPar
You will need this file later.

\item {} 
\sphinxAtStartPar
Rename your ‘Rhost’ directory to something else.  This is the directory that you have containing the ‘Server’ directory.  Name it anything you want other than ‘Rhost’.  For those not used to unix you would type:

\begin{sphinxVerbatim}[commandchars=\\\{\}]
\PYG{n}{mv} \PYG{n}{Rhost} \PYG{n}{Rhost\PYGZus{}old}
\end{sphinxVerbatim}

\item {} 
\sphinxAtStartPar
Pull in the latest Rhost.  You would type:

\begin{sphinxVerbatim}[commandchars=\\\{\}]
\PYG{n}{git} \PYG{n}{clone} \PYG{n}{https}\PYG{p}{:}\PYG{o}{/}\PYG{o}{/}\PYG{n}{github}\PYG{o}{.}\PYG{n}{com}\PYG{o}{/}\PYG{n}{RhostMUSH}\PYG{o}{/}\PYG{n}{trunk} \PYG{n}{Rhost}
\end{sphinxVerbatim}

\sphinxAtStartPar
You would type this in the same directory you have renamed your old ‘Rhost’ directory

\item {} 
\sphinxAtStartPar
go into the Rhost/Server directory.   Type:

\begin{sphinxVerbatim}[commandchars=\\\{\}]
\PYG{n}{make} \PYG{n}{confsource}
\end{sphinxVerbatim}

\sphinxAtStartPar
Select what options you want (including the mysql and other goodies) then compile it (also within the menu, it’s the ‘r’ option).

\end{enumerate}

\sphinxAtStartPar
8   Once your game is compiled and ready to go you need to copy in the data from your old game.  Copy the RhostMUSH.dbflat1.tar.gz we mentioned in step \#4 to the Server/game directory of your NEW GAME DIRECTORY.  From within the ‘game’ directory of your current game you should be able to issue (if you named the old one Rhost\_old). Again this needs to be done FROM YOUR Server/game directory!!!
\begin{quote}
\begin{enumerate}
\sphinxsetlistlabels{\Alph}{enumi}{enumii}{}{.}%
\item {} 
\sphinxAtStartPar
cp netrhost.conf netrhost.conf.orig

\item {} 
\sphinxAtStartPar
cp ../../Rhost\_old/Server/game/RhostMUSH.dbflat1.tar.gz .

\item {} 
\sphinxAtStartPar
tar \sphinxhyphen{}zxvf RhostMUSH.dbflat1.tar.gz

\item {} 
\sphinxAtStartPar
Compare your current netrhost.conf to the default one that came with the source (that you renamed to netrhost.conf.orig).  Likely the only sections you have to add to your current netrhost.conf (that came with your RhostMUSH.dbflat1.tar.gz archive), will be toward the end, with the include rhost\_ingame.conf and rhost\_mysql.conf.  Depending on how old your game is coming from you may need to add more options.  Any config option that is the same between the netrhost.conf files do not have to be copied over, and you want to keep your custom settings (like don’t port or other stuff you have already customized).

\item {} 
\sphinxAtStartPar
Load in your flatfile information:

\begin{sphinxVerbatim}[commandchars=\\\{\}]
\PYG{o}{.}\PYG{o}{/}\PYG{n}{db\PYGZus{}load} \PYG{n}{data}\PYG{o}{/}\PYG{n}{netrhost}\PYG{o}{.}\PYG{n}{gdbm} \PYG{n}{data}\PYG{o}{/}\PYG{n}{netrhost}\PYG{o}{.}\PYG{n}{db}\PYG{o}{.}\PYG{n}{flat} \PYG{n}{data}\PYG{o}{/}\PYG{n}{netrhost}\PYG{o}{.}\PYG{n}{db}\PYG{o}{.}\PYG{n}{new}
\end{sphinxVerbatim}

\item {} 
\sphinxAtStartPar
Your ./Startmush should re\sphinxhyphen{}index all your txt files you originally made:

\begin{sphinxVerbatim}[commandchars=\\\{\}]
\PYG{o}{.}\PYG{o}{/}\PYG{n}{Startmush}
\end{sphinxVerbatim}

\item {} 
\sphinxAtStartPar
In your game type the following as an immortal or as \#1.

\end{enumerate}
\begin{enumerate}
\sphinxsetlistlabels{\arabic}{enumi}{enumii}{}{.}%
\item {} 
\sphinxAtStartPar
Load in your mail flatfile:

\begin{sphinxVerbatim}[commandchars=\\\{\}]
\PYG{n}{wmail}\PYG{o}{/}\PYG{n}{load}
\end{sphinxVerbatim}

\item {} 
\sphinxAtStartPar
Load in your autoregistration flatfile (if available):

\begin{sphinxVerbatim}[commandchars=\\\{\}]
\PYG{n+nd}{@areg}\PYG{o}{/}\PYG{n}{load}
\end{sphinxVerbatim}

\item {} 
\sphinxAtStartPar
Load in your hardcoded bbs flatfile (if available):

\begin{sphinxVerbatim}[commandchars=\\\{\}]
\PYG{n}{newsdb}\PYG{o}{/}\PYG{n}{load}
\end{sphinxVerbatim}

\end{enumerate}
\end{quote}
\begin{enumerate}
\sphinxsetlistlabels{\arabic}{enumi}{enumii}{}{.}%
\setcounter{enumi}{8}
\item {} 
\sphinxAtStartPar
You should be good to go on a current directory structure for Rhost.  Enjoy!

\end{enumerate}


\subsection{Adding MySQL to RhostMUSH older than 3.9.5p2}
\label{\detokenize{upgrade:adding-mysql-to-rhostmush-older-than-3-9-5p2}}
\sphinxAtStartPar
MySQL is now native in RhostMUSH as of 3.9.5p2.

\begin{sphinxadmonition}{warning}{Warning:}
\sphinxAtStartPar
To autodetect it, YOU MUST HAVE mysql\_config installed and running on your server.  Without this, even if you have mysql db installed it won’t be able to recognize the parameters you will need for it and will thus fail.  Please check your linux distribution to see what packages are needed to install mysql\_config.
\end{sphinxadmonition}

\sphinxAtStartPar
Download the git repository to a seperate directory so that you can
copy over the files that it requires you to.

\sphinxAtStartPar
Suggestion:  git clone \sphinxurl{https://github.com/RhostMUSH/trunk} \textasciitilde{}/tmprho

\sphinxAtStartPar
If you are patching UP from an older version, you need to update
the following files:
\begin{enumerate}
\sphinxsetlistlabels{\arabic}{enumi}{enumii}{}{.}%
\item {} 
\sphinxAtStartPar
update your src/Makefile to the one in the 3.9.5p2+ repo
( cp \textasciitilde{}/tmprho/Server/src/Makefile \textasciitilde{}/Rhost/Server/src/Makefile )

\item {} 
\sphinxAtStartPar
update your bin/asksource.* files to the one in the 3.9.5p2+ repo
( cp \textasciitilde{}/tmprho/Server/bin/asksource.* \textasciitilde{}/Rhost/Server/bin/ )

\item {} 
\sphinxAtStartPar
append ‘include rhost\_mysql.conf’ BEFORE the rhost\_ingame.conf file
and before the section that says ‘define local aliases’ toward the end of
your netrhost.conf file.
( edit your \textasciitilde{}/Rhost/Server/game/netrhost.conf file )

\item {} 
\sphinxAtStartPar
copy the game/rhost\_mysql.conf file from the 3.9.5p2+ repo
( cp \textasciitilde{}/tmprho/Server/game/rhost\_mysql.conf \textasciitilde{}/Rhost/Server/game/ )

\item {} 
\sphinxAtStartPar
The following lines have to be REPLACED/CHANGED in local.c ( toward the top ):
( you may edit this or copy the one from the other distro )
( do either:  edit \textasciitilde{}/Rhost/Server/src/local.c )
(        or:  cp \textasciitilde{}/tmprho/Server/src/local.c \textasciitilde{}/Rhost/Server/src/local.c )

\end{enumerate}

\begin{sphinxadmonition}{note}{Note:}
\sphinxAtStartPar
IF REPLACING/CHANGING local.c COPY BELOW
\end{sphinxadmonition}

\begin{sphinxVerbatim}[commandchars=\\\{\}]
\PYG{c+cm}{/* Called when the mush starts up, immediatly prior to the main game}
\PYG{c+cm}{ * loop being entered. By this point all databases are loaded and}
\PYG{c+cm}{ * all variables configured.}
\PYG{c+cm}{ */}
\PYG{c+cp}{\PYGZsh{}}\PYG{c+cp}{ifdef MYSQL\PYGZus{}VERSION}
   \PYG{k}{extern} \PYG{k+kt}{void} \PYG{n}{local\PYGZus{}mysql\PYGZus{}init}\PYG{p}{(}\PYG{k+kt}{void}\PYG{p}{)}\PYG{p}{;}
   \PYG{k}{extern} \PYG{k+kt}{int} \PYG{n}{sql\PYGZus{}shutdown}\PYG{p}{(}\PYG{n}{dbref} \PYG{n}{player}\PYG{p}{)}\PYG{p}{;}
\PYG{c+cp}{\PYGZsh{}}\PYG{c+cp}{endif}

\PYG{c+cp}{\PYGZsh{}}\PYG{c+cp}{ifdef SQLITE}
   \PYG{k}{extern} \PYG{k+kt}{void} \PYG{n}{local\PYGZus{}sqlite\PYGZus{}init}\PYG{p}{(}\PYG{k+kt}{void}\PYG{p}{)}\PYG{p}{;}
\PYG{c+cp}{\PYGZsh{}}\PYG{c+cp}{endif }\PYG{c+cm}{/* SQLITE */}

\PYG{k+kt}{void} \PYG{n+nf}{local\PYGZus{}startup}\PYG{p}{(}\PYG{k+kt}{void}\PYG{p}{)} \PYG{p}{\PYGZob{}}
\PYG{c+cp}{\PYGZsh{}}\PYG{c+cp}{ifdef SQLITE}
   \PYG{n}{local\PYGZus{}sqlite\PYGZus{}init}\PYG{p}{(}\PYG{p}{)}\PYG{p}{;}
\PYG{c+cp}{\PYGZsh{}}\PYG{c+cp}{endif }\PYG{c+cm}{/* SQLITE */}
\PYG{c+cp}{\PYGZsh{}}\PYG{c+cp}{ifdef MYSQL\PYGZus{}VERSION}
   \PYG{n}{local\PYGZus{}mysql\PYGZus{}init}\PYG{p}{(}\PYG{p}{)}\PYG{p}{;}
\PYG{c+cp}{\PYGZsh{}}\PYG{c+cp}{endif}
   \PYG{n}{load\PYGZus{}regexp\PYGZus{}functions}\PYG{p}{(}\PYG{p}{)}\PYG{p}{;}
\PYG{p}{\PYGZcb{}}

\PYG{c+cm}{/* Called immediatly after the main game loop exits. At this point}
\PYG{c+cm}{ * all databases and variables are still configured}
\PYG{c+cm}{ */}
\PYG{k+kt}{void} \PYG{n+nf}{local\PYGZus{}shutdown}\PYG{p}{(}\PYG{k+kt}{void}\PYG{p}{)} \PYG{p}{\PYGZob{}}
\PYG{c+cp}{\PYGZsh{}}\PYG{c+cp}{ifdef MYSQL\PYGZus{}VERSION}
   \PYG{n}{sql\PYGZus{}shutdown}\PYG{p}{(}\PYG{l+m+mi}{\PYGZhy{}1}\PYG{p}{)}\PYG{p}{;}
\PYG{c+cp}{\PYGZsh{}}\PYG{c+cp}{endif}
\PYG{p}{\PYGZcb{}}
\end{sphinxVerbatim}
\begin{enumerate}
\sphinxsetlistlabels{\arabic}{enumi}{enumii}{}{.}%
\setcounter{enumi}{5}
\item {} 
\sphinxAtStartPar
Issue ‘make clean’ then make confsource to rebuild using the latest
builder script to build in the mysql changes.

\end{enumerate}


\section{Advanced Features}
\label{\detokenize{advanced:advanced-features}}\label{\detokenize{advanced::doc}}

\subsection{Adding hardcoded modules}
\label{\detokenize{advanced:adding-hardcoded-modules}}
\sphinxAtStartPar
RhostMUSH does support module writing.


\subsubsection{Modifying sourcode to add a module}
\label{\detokenize{advanced:modifying-sourcode-to-add-a-module}}
\sphinxAtStartPar
This requires hooking your changes into local.c, then modifying the Makefile (in the src directory)
for any new source files that you wish to add.

\sphinxAtStartPar
Something to be aware of is that all localized data is ran after the system cache subroutine.


\subsubsection{Adding an @startup to make use of modules}
\label{\detokenize{advanced:adding-an-startup-to-make-use-of-modules}}
\sphinxAtStartPar
This means that if your code is depending on @startups, you need to put a delay in the @startup
so that your local code can be loaded in as modules prior to the @startup execution.

\sphinxAtStartPar
Something that will not work:

\begin{sphinxVerbatim}[commandchars=\\\{\}]
\PYG{n+nd}{@startup} \PYG{n}{me}\PYG{o}{=}\PYG{n+nd}{@superhappyfuncommand} \PYG{n}{loadmeup}\PYG{o}{=}\PYG{n}{now}
\end{sphinxVerbatim}

\sphinxAtStartPar
A small alteration that will likely make this work fine:

\begin{sphinxVerbatim}[commandchars=\\\{\}]
\PYG{n+nd}{@startup} \PYG{n}{me}\PYG{o}{=}\PYG{n+nd}{@wait} \PYG{l+m+mi}{1}\PYG{o}{=}\PYG{n+nd}{@superhappyfuncommand} \PYG{n}{loadmeup}\PYG{o}{=}\PYG{n}{now}
\end{sphinxVerbatim}

\sphinxAtStartPar
That 1 second delay for the queue will give the game engine time to load in your module code.


\subsubsection{Contributing your module back to Rhost}
\label{\detokenize{advanced:contributing-your-module-back-to-rhost}}
\sphinxAtStartPar
If you wish your modules to be part of the main Rhost distribution you have two options:
\begin{enumerate}
\sphinxsetlistlabels{\arabic}{enumi}{enumii}{}{.}%
\item {} 
\sphinxAtStartPar
Attempt to hack the bin/asksource.sh and bin/asksource.blank files.

\item {} 
\sphinxAtStartPar
Ask one of the Rhost devs to do it for you :)

\end{enumerate}


\subsection{Shutting down gracefully}
\label{\detokenize{advanced:shutting-down-gracefully}}

\subsubsection{Rhostmush has many ways to shut down the game cleanly}
\label{\detokenize{advanced:rhostmush-has-many-ways-to-shut-down-the-game-cleanly}}\begin{enumerate}
\sphinxsetlistlabels{\arabic}{enumi}{enumii}{}{.}%
\item {} 
\sphinxAtStartPar
Log into the mush and issue @shutdown

\item {} 
\sphinxAtStartPar
Issue a kill \sphinxhyphen{}USR2 to the mush which issues an emergency @shutdown

\item {} 
\sphinxAtStartPar
Issue a kill \sphinxhyphen{}TERM to the mush which issues an emergency abort and clean shutdown.

\end{enumerate}


\subsubsection{WARNING: Never kill \sphinxhyphen{}9 Rhost}
\label{\detokenize{advanced:warning-never-kill-9-rhost}}
\sphinxAtStartPar
Under NO CIRCUMSTANCES should you kill \sphinxhyphen{}9 your mush unless you don’t care for the
database.  The reason is if the mush happens to be saving, in any method, to the
database, especially a QDBM database, you will likely have just corrupted your
database, so pull out a flatfile to recover.

\sphinxAtStartPar
Sadly, this also may occur if the server hosting you takes a nose\sphinxhyphen{}dive in the middle
of a db write.  Rhost can recover corruption in\sphinxhyphen{}game while up, but if it bombs
in the middle of a write, all bets are off. :)


\subsection{Reality levels}
\label{\detokenize{advanced:reality-levels}}
\sphinxAtStartPar
Reality levels are a means to forbid (or allow) interaction between objects
in the same location.


\subsubsection{Visibility}
\label{\detokenize{advanced:visibility}}
\sphinxAtStartPar
Each object (player, room, exit, thing) has two lists of reality levels.
An Rx list, which describe what it can see and a Tx list, which describe
where it can be seen. Those are bitfields. In order for X to see Y a bitwise
‘and’ is performed between X’s RxLevel and Y’s TxLevel. If the result is not
0, then X sees Y. If the result is 0, as far as X is concerned, Y doesn’t
exist. This affects contents lists, exit lists, look, say, pose, @emit,
@verb, connect/disconnect, has arrived/has left messages, exit and object
matching. ‘here’ and ‘me’ match always.

\sphinxAtStartPar
It doesn’t affect @remit, @pemit, page, WHO or channels.
By default, all new objects are created with an RxLevel of 1 and TxLevel of
1. Rooms are an exception, created with an RxLevel of 1 and a TxLevel of
0xFFFFFFFF. Those default levels can be changed with configuration
parameters.
An object is always visible to itself, even if its Rx and Tx levels don’t
match. (See examples below)


\subsubsection{Descriptions}
\label{\detokenize{advanced:descriptions}}
\sphinxAtStartPar
For every reality level defined, you can define an attribute that serves as
description. If you look at something and match more than one of its
TxLevels, you’ll see all the corresponding descriptions on the target
object. If the object doesn’t have any descriptions for the matching levels,
you’ll see the regular @desc.

\sphinxAtStartPar
The @adesc attribute on the target is only triggered if the target can see
the looker in turn. It’s only triggered once, no matter how many descs the
looker sees. The @odesc is shown only to those people that see /both/ the
looker and the target.

\sphinxAtStartPar
Through extension, @afail/@ofail and similar pairs (@adrop/@odrop,
@asucc/@osucc etc) work in the same way. @verb commands are similary
affected.

\sphinxAtStartPar
Softcoded commands are only matched on the objects that can see the player.
The player doesn’t have to see the object. This includes commands in the
Master Room.

\sphinxAtStartPar
Exits are treated specially. In order to be able to use an exit name (or to
use the ‘move \textless{}exit\textgreater{}’ command) the exit must be visible to the enactor. In
order to pass through the exit, the exit must see the enactor in turn. There
are reasons for this, which will become evident in the examples below.


\subsubsection{Configuration parameters}
\label{\detokenize{advanced:configuration-parameters}}
\sphinxAtStartPar
A few configuration parameters have been introduced to deal with the reality
levels:

\begin{sphinxVerbatim}[commandchars=\\\{\}]
\PYG{n}{reality\PYGZus{}level} \PYG{o}{\PYGZlt{}}\PYG{n}{name}\PYG{o}{\PYGZgt{}} \PYG{o}{\PYGZlt{}}\PYG{n}{value}\PYG{o}{\PYGZgt{}} \PYG{p}{[}\PYG{o}{\PYGZlt{}}\PYG{n}{desc} \PYG{n}{attribute} \PYG{n}{name}\PYG{o}{\PYGZgt{}}\PYG{p}{]}
\end{sphinxVerbatim}

\sphinxAtStartPar
This directive can only be used in the config file (not with the @admin
command) and should be repeated for each reality level you want to define.
It defines a new level named \textless{}name\textgreater{} with a bitvalue of \textless{}value\textgreater{} and an
optional desc attribute. There is a limit of 8 characters on \textless{}name\textgreater{}, a
32\sphinxhyphen{}bit value on \textless{}value\textgreater{} (basically an unsigned long) and 32 characters on
the attribute name. A maximum of 32 reality levels can be defined:

\begin{sphinxVerbatim}[commandchars=\\\{\}]
\PYG{n}{def\PYGZus{}exit\PYGZus{}tx} \PYG{o}{\PYGZlt{}}\PYG{n}{value}\PYG{o}{\PYGZgt{}}
\PYG{n}{def\PYGZus{}exit\PYGZus{}rx} \PYG{o}{\PYGZlt{}}\PYG{n}{value}\PYG{o}{\PYGZgt{}}
\PYG{n}{def\PYGZus{}room\PYGZus{}tx} \PYG{o}{\PYGZlt{}}\PYG{n}{value}\PYG{o}{\PYGZgt{}}
\PYG{n}{def\PYGZus{}room\PYGZus{}rx} \PYG{o}{\PYGZlt{}}\PYG{n}{value}\PYG{o}{\PYGZgt{}}
\PYG{n}{def\PYGZus{}player\PYGZus{}rx} \PYG{o}{\PYGZlt{}}\PYG{n}{value}\PYG{o}{\PYGZgt{}}
\PYG{n}{def\PYGZus{}player\PYGZus{}tx} \PYG{o}{\PYGZlt{}}\PYG{n}{value}\PYG{o}{\PYGZgt{}}
\PYG{n}{def\PYGZus{}thing\PYGZus{}rx} \PYG{o}{\PYGZlt{}}\PYG{n}{value}\PYG{o}{\PYGZgt{}}
\PYG{n}{def\PYGZus{}thing\PYGZus{}tx} \PYG{o}{\PYGZlt{}}\PYG{n}{value}\PYG{o}{\PYGZgt{}}
\end{sphinxVerbatim}

\sphinxAtStartPar
These 8 directives define the default reality levels of newly created
objects. They can be set in the config file or with the @admin command.
Like above, \textless{}value\textgreater{} must be a decimal number:

\begin{sphinxVerbatim}[commandchars=\\\{\}]
\PYG{n}{wiz\PYGZus{}always\PYGZus{}real} \PYG{o}{\PYGZlt{}}\PYG{l+m+mi}{0}\PYG{o}{|}\PYG{l+m+mi}{1}\PYG{o}{\PYGZgt{}}
\end{sphinxVerbatim}

\sphinxAtStartPar
If this parameter is set to 1 then wizards (and immortals on Rhost) will see
everything and will be visible to everyone. Their effective Rx and Tx levels
will always be 0xFFFFFFFF. Also settable in the config file and with the
@admin command.

\sphinxAtStartPar
Compile with \sphinxhyphen{}DREALITY\_LEVELS compile time option to enable ‘Real’ needs to be ‘1’
This is an example file only to be added to the mush.conf file Format:

\begin{sphinxVerbatim}[commandchars=\\\{\}]
\PYG{n}{reality\PYGZus{}level} \PYG{o}{\PYGZlt{}}\PYG{l+m+mi}{8} \PYG{n}{char} \PYG{n}{name}\PYG{o}{\PYGZgt{}} \PYG{o}{\PYGZlt{}}\PYG{n+nb}{hex}\PYG{o}{\PYGZhy{}}\PYG{n}{byte}\PYG{o}{\PYGZhy{}}\PYG{n}{mask}\PYG{o}{\PYGZgt{}} \PYG{o}{\PYGZlt{}}\PYG{n}{optional}\PYG{o}{\PYGZhy{}}\PYG{n}{desc}\PYG{p}{:} \PYG{n}{DESC} \PYG{n}{default}\PYG{o}{\PYGZgt{}}
\end{sphinxVerbatim}


\subsubsection{Example mush.conf}
\label{\detokenize{advanced:example-mush-conf}}
\sphinxAtStartPar
reality\_level Real 1
reality\_level Obf1 2
reality\_level Obf2 4
reality\_level Obf3 8 OBFDESC
reality\_level Obf4 16 OBFDESC
reality\_level Obf5 32 OBFDESC
reality\_level Obf6 64 OBFDESC
reality\_level Obf7 128 OBFDESC
reality\_level Obf8 256 OBFDESC
reality\_level Obf9 512 OBFDESC
reality\_level Obf10 1024 OBFDESC
reality\_level Umbra 2048 UMBRADESC
reality\_level Fae 4096 FAEDESC
reality\_level Shadow 8192 SHADOWDESC
reality\_level Spy 16384
reality\_level Death 32768 DEATHDESC
reality\_level All 4294967295


\subsubsection{Commands}
\label{\detokenize{advanced:commands}}
\sphinxAtStartPar
Two wiz\sphinxhyphen{}only commands are used to set the reality levels of an object:

\begin{sphinxVerbatim}[commandchars=\\\{\}]
\PYG{n+nd}{@rxlevel} \PYG{o}{\PYGZlt{}}\PYG{n+nb}{object}\PYG{o}{\PYGZgt{}}\PYG{o}{=}\PYG{o}{\PYGZlt{}}\PYG{n+nb}{list}\PYG{o}{\PYGZgt{}}
\PYG{n+nd}{@txlevel} \PYG{o}{\PYGZlt{}}\PYG{n+nb}{object}\PYG{o}{\PYGZgt{}}\PYG{o}{=}\PYG{o}{\PYGZlt{}}\PYG{n+nb}{list}\PYG{o}{\PYGZgt{}}
\end{sphinxVerbatim}

\sphinxAtStartPar
\textless{}list\textgreater{} is a space\sphinxhyphen{}separated list of level names that have to be set on the
object. If a level name is prefixed with an exclamation mark (!) that level
will be cleared from the object.

\begin{sphinxadmonition}{warning}{Warning:}
\sphinxAtStartPar
Changing the Tx levels of an object might make it invisible to you.
In this case, you can still manipulate it by using his \#dbref (or *player
for players).
\end{sphinxadmonition}


\subsubsection{Functions}
\label{\detokenize{advanced:functions}}
\sphinxAtStartPar
There are five functions that deal with reality levels:

\begin{sphinxVerbatim}[commandchars=\\\{\}]
\PYG{n}{hasrxlevel}\PYG{p}{(}\PYG{o}{\PYGZlt{}}\PYG{n+nb}{object}\PYG{o}{\PYGZgt{}}\PYG{p}{,}\PYG{o}{\PYGZlt{}}\PYG{n}{level}\PYG{o}{\PYGZgt{}}\PYG{p}{)}
\PYG{n}{hastxlevel}\PYG{p}{(}\PYG{o}{\PYGZlt{}}\PYG{n+nb}{object}\PYG{o}{\PYGZgt{}}\PYG{p}{,}\PYG{o}{\PYGZlt{}}\PYG{n}{level}\PYG{o}{\PYGZgt{}}\PYG{p}{)}
\end{sphinxVerbatim}

\sphinxAtStartPar
These two functions check if an object has the specified Rx or Tx level.
You must control \textless{}object\textgreater{}. They return 0 or 1 and \#\sphinxhyphen{}1 in case the object
does not exist or you don’t have permissions:

\begin{sphinxVerbatim}[commandchars=\\\{\}]
\PYG{n}{rxlevel}\PYG{p}{(}\PYG{o}{\PYGZlt{}}\PYG{n+nb}{object}\PYG{o}{\PYGZgt{}}\PYG{p}{)}
\PYG{n}{txlevel}\PYG{p}{(}\PYG{o}{\PYGZlt{}}\PYG{n+nb}{object}\PYG{o}{\PYGZgt{}}\PYG{p}{)}
\end{sphinxVerbatim}

\sphinxAtStartPar
These two functions return a space\sphinxhyphen{}separated list of the object’s Rx or Tx
levels. Again, you must control the object:

\begin{sphinxVerbatim}[commandchars=\\\{\}]
\PYG{n}{cansee}\PYG{p}{(}\PYG{o}{\PYGZlt{}}\PYG{n}{obj1}\PYG{o}{\PYGZgt{}}\PYG{p}{,}\PYG{o}{\PYGZlt{}}\PYG{n}{obj2}\PYG{o}{\PYGZgt{}}\PYG{p}{)}
\end{sphinxVerbatim}

\sphinxAtStartPar
A wiz\sphinxhyphen{}only function, returns 1 of \textless{}obj1\textgreater{} can see \textless{}obj2\textgreater{} from a reality
levels point of view. It doesn’t check if the objects are in the same
location, the DARK/CLOAKED flags and so on. Just \textless{}obj1\textgreater{}’s Rx level against
\textless{}obj2\textgreater{}’s Tx level.

\begin{sphinxadmonition}{warning}{Warning:}
\sphinxAtStartPar
If you are using it on MUX2.0 with /both/ reality levels and Wod
Realms enabled, the function will perform both checks and the Wod Realms
version checks against the DARK flag.
\end{sphinxadmonition}


\subsubsection{Example 1: A simplified Witchcraft setup}
\label{\detokenize{advanced:example-1-a-simplified-witchcraft-setup}}
\sphinxAtStartPar
In Witchcraft, besides the various Gifted classes, characters can be spirits
There are spirit realms to which the mundane can not travel. Therefore we
will use 2 reality levels: Real and Ghost. Since some spirits can become
solid for a limited period of time, we will also use an additional desc for
the Ghost level, called GHOSTDESC. Therefore in the config file we will
have:

\begin{sphinxVerbatim}[commandchars=\\\{\}]
\PYG{n}{reality\PYGZus{}level} \PYG{n}{Real} \PYG{l+m+mi}{1}
\PYG{n}{reality\PYGZus{}level} \PYG{n}{Ghost} \PYG{l+m+mi}{2} \PYG{n}{GHOSTDESC}
\end{sphinxVerbatim}

\sphinxAtStartPar
Ghosts can pass through most mundane locks, so the exists should allows the
ghosts to pass:

\begin{sphinxVerbatim}[commandchars=\\\{\}]
\PYG{n}{def\PYGZus{}exit\PYGZus{}rx} \PYG{l+m+mi}{3}
\end{sphinxVerbatim}

\sphinxAtStartPar
Note that def\_exit\_tx isn’t set. Why? Because ghosts see the mundane world
anyway, so a spirit character will have:

\begin{sphinxVerbatim}[commandchars=\\\{\}]
@txlevel \PYGZlt{}player\PYGZgt{}=!Real Ghost
@rxlevel \PYGZlt{}player\PYGZgt{}=Real Ghost
\end{sphinxVerbatim}

\sphinxAtStartPar
Let’s assume 3 players:
John is a Mundane. He won’t see spirits.
John’s Rx: Real
John’s Tx: Real
John’s @desc: This is John.
John’s \&GHOSTDESC: (Not important, since it’s never visible)
Johh’s @adesc: \%N has looked at you.
John’s @odesc: has looked at John.

\sphinxAtStartPar
Jack is a Gifted. He will sense spirits, but is still made from flesh
and blood so visible to mundanes.
Jack’s Rx: Real Ghost
Jack’s Tx: Real
Jack’s @desc: This is Jack.
Jack’s \&GHOSTDESC: (Not important, since it’s never visible)
Jack’s @adesc: \%N has looked at you.
Jack’s @odesc: has looked at Jack.

\sphinxAtStartPar
Frank is a ghost. He will see other spirits as well as mundanes, but won’t
be visible to mundanes. He can also become visible to everybody.
Frank’s Rx: Real Ghost
Frank’s Tx: Ghost
Frank’s @desc: This is Frank, looking human.
Frank’s \&GHOSTDESC: This is Frank’s ghostly shape.
Frank’s @adesc: \%N has looked at you.
Frank’s @odesc: has looked at Frank.

\sphinxAtStartPar
Following are commands that each of the players enter and what they see.
I’ll assume the +materialize command is defined like:
\&CMD\_MATERIALIZE \textless{}cmdobject\textgreater{}=\$+materialize:@txlevel \%\#=Real; @pemit \%\#=You
are now material.
\begin{quote}
\begin{description}
\item[{John            |         Jack          |         Frank}] \leavevmode
\begin{DUlineblock}{0em}
\item[] {\color{red}\bfseries{}|}
\end{DUlineblock}

\end{description}
\end{quote}

\sphinxAtStartPar
\textgreater{} l                     |                       |
A room                  |                       |
This is a bare room.    |                       |
Contents:               |                       |
Jack                    |                       |
Obvious exits:          |                       |
Out \textless{}O\textgreater{}                 |                       |
\begin{quote}

\sphinxAtStartPar
{\color{red}\bfseries{}|}\textgreater{} l                    |
{\color{red}\bfseries{}|}A room                 |
{\color{red}\bfseries{}|}This is a bare room.   |
{\color{red}\bfseries{}|}Contents:              |
{\color{red}\bfseries{}|}John Frank             |
{\color{red}\bfseries{}|}Obvious exits:         |
{\color{red}\bfseries{}|}Out \textless{}O\textgreater{}                |
|                       {\color{red}\bfseries{}|}\textgreater{} l
|                       {\color{red}\bfseries{}|}A room
|                       {\color{red}\bfseries{}|}This is a bare room.
|                       {\color{red}\bfseries{}|}Contents:
|                       {\color{red}\bfseries{}|}John Jack
|                       {\color{red}\bfseries{}|}Obvious exits:
|                       {\color{red}\bfseries{}|}Out \textless{}O\textgreater{}
\end{quote}

\sphinxAtStartPar
\textgreater{}l Jack                 |                       |
Jack                    {\color{red}\bfseries{}|}John has looked at you.|John has looked at Jack.
This is Jack.           |                       |
\textgreater{}l Frank                |                       |
I don’t see that here.  |                       |
\begin{quote}

\sphinxAtStartPar
{\color{red}\bfseries{}|\textgreater{}l Frank               |
|Frank                  |Jack has looked at you.
|This is Frank\textquotesingle{}s ghostly|}
{\color{red}\bfseries{}|}shape.                 |
|                       {\color{red}\bfseries{}|}\textgreater{}l John
{\color{red}\bfseries{}|}Frank has looked at    {\color{red}\bfseries{}|}John
{\color{red}\bfseries{}|}John.                  {\color{red}\bfseries{}|}This is John.
|                       {\color{red}\bfseries{}|}\textgreater{}+materialize
|                       {\color{red}\bfseries{}|}You are now material.
\end{quote}

\sphinxAtStartPar
\textgreater{}l Frank                |                       |
Frank                   {\color{red}\bfseries{}|}John has looked at     {\color{red}\bfseries{}|}Frank has looked at you.
This is Frank, looking  {\color{red}\bfseries{}|}Frank.                 |
human.                  |                       |
\begin{quote}

\sphinxAtStartPar
{\color{red}\bfseries{}|}\textgreater{}l Frank               |
\end{quote}

\sphinxAtStartPar
Jack has looked at      {\color{red}\bfseries{}|}Frank                  {\color{red}\bfseries{}|}John has looked at you.
Frank.                  {\color{red}\bfseries{}|}This is Frank, looking |
\begin{quote}

\sphinxAtStartPar
{\color{red}\bfseries{}|human.                 |
|This is Frank\textquotesingle{}s ghostly|}
{\color{red}\bfseries{}|}shape.                 |
\end{quote}


\subsubsection{Example 2: A WoD setup}
\label{\detokenize{advanced:example-2-a-wod-setup}}
\sphinxAtStartPar
The reality levels will be defined like this:
reality\_level           Real 1
reality\_level           Obf1 2
reality\_level           Obf2 4
reality\_level           Obf3 8 OBFDESC
reality\_level           Obf4 16 OBFDESC
reality\_level           Obf5 32 OBFDESC
reality\_level           Umbra 64 UMBRADESC
reality\_level           Fae 128 FAEDESC
reality\_level           Shadow 256 SHADOWDESC
reality\_level           All 511

\sphinxAtStartPar
5 levels of Obfuscation, Umbra, Dreaming, Wraiths. ‘All’ is a handy
replacement for all levels, useful for wizards and wizobjects that should
be visible on all levels. Also useful when you want to set an object’s
levels to something without knowing what he had before.

\sphinxAtStartPar
@rxlevel \#276=!All Real

\sphinxAtStartPar
!All will clear all levels, then the object will gain the Real level.
There is more than one Obfuscation level because of the relation between
Auspex and Obfuscation.

\sphinxAtStartPar
A vampire with Obfuscate 2, should not be visible by one with Auspex 1.
However one with Auspex 3 should see another vampire with Obfuscate 1, 2
/or/ 3.

\sphinxAtStartPar
Obfuscated players can move if they have Obf \textgreater{} 1. Umbral and Shadow players
should also be able to see most of the exits. So the exits at creation
should have default levels of Real + Obf2 + Obf3 + Obf4 + Obf5 + Umbra +
Shadow = 1 + 4 + 8 + 16 + 32 + 64 + 256 = 381:

\begin{sphinxVerbatim}[commandchars=\\\{\}]
\PYG{n}{def\PYGZus{}exit\PYGZus{}rx} \PYG{l+m+mi}{381}
\PYG{n}{def\PYGZus{}exit\PYGZus{}tx} \PYG{l+m+mi}{381}
\end{sphinxVerbatim}

\sphinxAtStartPar
Obf1 is not included since an Obfuscated vampire should not be able to move
if it only has Obf1. Therefore they won’t see the exits. If you want them
to be able to see the exits, but not to use them, change the default Tx of
the exits:

\begin{sphinxVerbatim}[commandchars=\\\{\}]
\PYG{n}{def\PYGZus{}exit\PYGZus{}rx} \PYG{l+m+mi}{381}
\PYG{n}{def\PYGZus{}exit\PYGZus{}tx} \PYG{l+m+mi}{383}
\end{sphinxVerbatim}

\sphinxAtStartPar
Joe the Mortal will have an RxLevel: Real and a TxLevel: Real
Jack the Malk, who likes to walk around Obfuscated and has Obfuscate 2 will
have an RxLevel: Real (he sees what the mortals see) but a TxLevel: Obf2
Jimmy the Nossie, who is using the Mask and has Obfuscate 4, but doesn’t
try to make himself invisible will have an RxLevel: Real (as Jack)
and a TxLevel: Real Obf4. He will also set his @desc to what the mortals see and
\&OBFDESC to his real slimy desc. Simply put, he will be visible to mortals,
but not with his real desc.

\sphinxAtStartPar
Aldrin the Gangrel, has Auspex 4 and activates it. Therefore, his TxLevel
will still be Real, but RxLevel: Real Obf1 Obf2 Obf3 Obf4 (all of them). So
he can see Joe, Jack and Jimmy’s both descs.
Joe, on the other hand, won’t see Jack at all. He will still see Jimmy, but
only Jimmy’s @desc, not the OBFDESC

\sphinxAtStartPar
Frida the Fae… will have RxLevel: Real Fae and TxLevel: Real Fae. @desc
set to the mundane desc, \&FAEDESC set to the Chimerical desc.
Emily the Enchanted will have an RxLevel: Real Fae, but the TxLevel: Real.
No \&FAEDESC on her, although she’ll be able to see it the one on Frida.
Gil the Garou, while travelling through the Umbra, will have RxLevel: Umbra
and TxLevel: Umbra. \&UMBRADESC is his friend. He won’t see mortals or other
characters who are not in the Umbra.

\sphinxAtStartPar
Barbie the Bastet, who’s only peeking in the Umbra, without going there,
will have RxLevel: Umbra, TxLevel: Real. Dangerous position since she
can’t see the things that see her.

\sphinxAtStartPar
Deanna the Drake, who activates her spirit vision, will have
RxLevel: Real Umbra and TxLevel: Real. She will see characters in Umbra and
real world at the same time and perceive the desc appropiate to the realm
the ohter character is in.

\sphinxAtStartPar
Wanda the Wraith: RxLevel: Real Shadow, TxLevel: Shadow. Her @desc
would be empty, but the \&SHADOWDESC should be set.
Marie the Mortal+ Medium: RxLevel: Real Shadow, TxLevel: Real

\sphinxAtStartPar
Global code objects that all characters should be able to use:
RxLevel: All, TxLevel: All


\subsubsection{Example 3: Softcode}
\label{\detokenize{advanced:example-3-softcode}}
\sphinxAtStartPar
Considering the config directives from example 2 and assuming a function
getstat(\textless{}dbref\textgreater{},\textless{}stat\textgreater{}) that will return the value of a player’s stat from
the sheet here are softcode examples that implement some of the WoD powers.
In a real game you would have to use some more checks, of course.

\sphinxAtStartPar
@create Reality Levels Commands (RLS)
\&CMD\_OBFON rls=\$+obf/on:@switch {[}setr(0, getstat(\%\#,Obfuscate)){]}=0, @pemit
\%\#=You don’t have Obfuscate!, \{@txlevel \%\#=!All Obf\%q0; @pemit \%\#=You are
now invisible.\}
\&CMD\_OBFOFF rls=\$+obf/off:@txlevel \%\#=Real; @pemit \%\#=You are now visible.
@@ Note: +obf/on clears all TxLevels before setting the appropiate Obf
@@ This is necesary, because a character might advance from Obf2 to
@@ Obf3 and he should be visible /only/ on the Obf3 level.
@@ +obf/off simply sets the Real Tx level, without clearing the Obf. The
@@ reason is the Mask. Players with Obf3 or higher who use the Mask should
@@ +obf/on, then +obf/off after approval and everything is set.
\&CMD\_AUSPEXON rls=\$+auspex/on:@switch {[}setr(0, getstat(\%\#, Auspex)){]}=0,
@pemit \%\#=You don’t have Auspex!, \{@rxlevel \%\#={[}iter(lnum(1, \%q0), Obf\#\#){]};
@pemit \%\#=Auspex enabled.\}
\&CMD\_AUSPEXOFF rls=\$+auspex/off:@switch {[}hasrxlevel(\%\#, Obf1){]}=0, @pemit \%\#=
You don’t have Auspex enabled!, \{@rxlevel \%\#={[}iter(lnum(1, 5), !Obf\#\#){]};
@pemit \%\#=Auspex disabled.\}
\&CMD\_UMBRAENTER rls=\$+umbra/enter:@rxlevel \%\#=!Real Umbra; @txlevel \%\#=
!Real Umbra; @pemit \%\#=You are now in the Umbra.
\&CMD\_UMBRALEAVE rls=\$+umbra/leave:@rxlevel \%\#=Real !Umbra; @txlevel \%\#=
Real !Umbra; @pemit \%\#=You left the Umbra.
\&CMD\_PEEKON rls=\$+peek/on:@switch hastxlevel(\%\#,Umbra)=1, \{@rxlevel \%\#=Real
!Umbra; @pemit \%\#=You are now peeking in the real world\}, \{@rxlevel \%\#=!Real
Umbra; @pemit \%\#=You are now peeking into the Umbra\}
\&CMD\_PEEKOFF rls=\$+peek/off:@rxlevel \%\#=!Real !Umbra {[}setinter(Real Umbra,
txlevel(\%\#)){]}; @pemit \%\#=You are no longer peeking.


\subsection{Execscript and external programs and scripts}
\label{\detokenize{advanced:execscript-and-external-programs-and-scripts}}

\subsubsection{Mush variables}
\label{\detokenize{advanced:mush-variables}}

\paragraph{Built in variables}
\label{\detokenize{advanced:built-in-variables}}
\sphinxAtStartPar
Variable                   Description
MUSH\_PLAYER                player dbref\#
MUSH\_CAUSE                 cause dbref\#
MUSH\_CALLER                caller dbref\#
MUSH\_OWNER                 owner of player dbref\#
MUSH\_FLAGS                 space delimited list of flags on player
MUSH\_TOGGLES               space delimited list of toggles on player
MUSH\_OFLAGS                space delimited list of flags of player owner
MUSH\_OTOGGLES              space delimited list of toggles of player owner
MUSHL\_VARS                 space delimited list of MUSH attributes from player
\begin{quote}

\sphinxAtStartPar
This is passed from the mush’s EXECSCRIPT\_VARS attr
\end{quote}


\paragraph{Dynamic variables}
\label{\detokenize{advanced:dynamic-variables}}\begin{description}
\item[{MUSHV\_\textless{}arg\textgreater{}                \textless{}arg\textgreater{} variable passed from MUSHL\_VARS}] \leavevmode
\sphinxAtStartPar
These are the attributes from EXECSCRIPT\_VARS

\end{description}


\paragraph{Register variables}
\label{\detokenize{advanced:register-variables}}
\sphinxAtStartPar
MUSHQ\_\textless{}arg\textgreater{}                setq registers 0\sphinxhyphen{}9 and a\sphinxhyphen{}z
MUSHQN\_\textless{}arg\textgreater{}               labels that are assigned the setq vars
MUSHN\_\textless{}arg\textgreater{}                The labels that were defined by any register
\begin{quote}

\sphinxAtStartPar
Note: they must be ASCII\sphinxhyphen{}7 clean and contain no white spaces
\end{quote}


\subsubsection{set object}
\label{\detokenize{advanced:set-object}}
\sphinxAtStartPar
The script executed with execscript() will read in a file with the same name
as the script ending in ‘.set’.  This is a loader object that will set attributes
and registers back into the mush that you wish to pass from the script. The
fields are SPACE SEPARATED.  The values are NOT evaluated.


\paragraph{The format of the fields are}
\label{\detokenize{advanced:the-format-of-the-fields-are}}

\subparagraph{Dynamic variables}
\label{\detokenize{advanced:id65}}
\sphinxAtStartPar
VARNAME        OWNER        CONTENTS (or leave null to clear)


\subparagraph{Examples}
\label{\detokenize{advanced:examples}}
\sphinxAtStartPar
SEX \#123 Male
DESC \#123 \%r\%tThis is a willow tree of unique description\%r\%rIt sways in the wind.
RED \#123 This is the color \%ch\%crred\%cn.
WIPETHISATTR \#123
MULTILINE \#123 This is a line
that continues on
because of the line feed (a control\sphinxhyphen{}M) on each line
on the lines above


\subparagraph{Register variables}
\label{\detokenize{advanced:id66}}
\sphinxAtStartPar
REGISTER       Q            CONTENTS (or leave null to clear)


\subparagraph{Examples (The last example clears register 0)}
\label{\detokenize{advanced:examples-the-last-example-clears-register-0}}
\sphinxAtStartPar
W Q This is stored in register W
1 Q This is stored in register 1
0 Q
foo QN this sets register with label ‘foo’


\subsubsection{Example bash script}
\label{\detokenize{advanced:example-bash-script}}
\sphinxAtStartPar
\#!/bin/bash
echo “This was called by player \$\{MUSH\_PLAYER\} that is owned by \$\{MUSH\_OWNER\}”
echo “Displaying Registers:”
regs=”0 1 2 3 4 5 6 7 8 9 A B C D E F G H I J K L M N O P Q R S T U V W X Y Z”
for list in \$\{regs\}
do
\begin{quote}

\sphinxAtStartPar
eval echo “Register \$\{list\}: \$\{MUSHQ\_\$\{list\}\}”
\end{quote}

\sphinxAtStartPar
done
echo “Displaying variables:”
for vars in \$\{MUSHL\_VARS\}
do
\begin{quote}

\sphinxAtStartPar
eval echo “Variable \$\{vars\}: \$\{MUSHV\_\$\{vars\}\}”
\end{quote}

\sphinxAtStartPar
done


\subsubsection{Notes and warnings}
\label{\detokenize{advanced:notes-and-warnings}}
\sphinxAtStartPar
While MUSHL\_VARS are sanitized on what is allowable as a mush variable, this
is not necessarilly sanitized on what the calling script can fetch as a valid
variable.  Of note, you can not set environment variables that contain an
equals sign.  Be aware of this limitation.

\sphinxAtStartPar
Remember, MUSHL\_VARS is the environment variable seen by the script.
This is EXECSCRIPT\_VARS on the mush itself, that is the attribute set
on the target that contains the execscript() that is being executed.


\subsubsection{Scripts to be used with execscript}
\label{\detokenize{advanced:scripts-to-be-used-with-execscript}}
\sphinxAtStartPar
account/                       \textendash{} Directory for execscripts relating to account creation
compile39.sh                   \textendash{} Script for patching and compiling RhostMUSH 3.9
compile.sh                     \textendash{} Script for patching and compiling RhostMUSH
config.sh                      \textendash{} Script for setting compile time options for RhostMUSH
debug.sh                       \textendash{} Script for debugging RhostMUSH
dict.sh                        \textendash{} Script for querying a dictionary
diff.sh                        \textendash{} Script for querying differences between two arguments
fortune.sh                     \textendash{} Script for querying fortune program
fullweather.sh                 \textendash{} Script for querying a graphical weather forecast (alternative)
git.sh                         \textendash{} Script for querying git version of RhostMUSH
hello.sh                       \textendash{} Script for teaching execscript for ‘Hello World’
iostat.sh                      \textendash{} Script for querying server stats of RhostMUSH
jsonvalidate.sh                \textendash{} Python Script for validating JSON
logsearch.sh                   \textendash{} Script for searching throgh logfiles for RhostMUSH
math\_example.sh                \textendash{} Examples of math operations to be used with math.sh
math.sh                        \textendash{} Script for mathematical operations
memory.sh                      \textendash{} Script for querying memory usage of RhostMUSH
mkindx.sh                      \textendash{} Script for indexing RhostMSH helpfiles
pastebinread.sh                \textendash{} Script for reading data from a pastebin URL
pastebinwrite.sh               \textendash{} Script for writing data to a pastebin
qspell.sh                      \textendash{} Script for checking spelling (alternative)
quota.sh                       \textendash{} Script for checking disk quote and usage
random.sh                      \textendash{} Script for getting a random number
roomlog.sh                     \textendash{} Script for viewing logs in roomlog directory
spell.sh                       \textendash{} Script for checking spelling
stats.sh                       \textendash{} Script for querying server and process stats for RhostMUSH
thes.sh                        \textendash{} Script for adding a word to the dictionary for spell scripts
tinyurl.sh                     \textendash{} Script for shortening a URL
weather.sh                     \textendash{} Script for querying a graphical weather forecast
web.sh                         \textendash{} Script for querying an arbitary website


\subsection{Using printf() for advanced text output}
\label{\detokenize{advanced:using-printf-for-advanced-text-output}}
\sphinxAtStartPar
The function printf() in Rhost can be used to greatly reduce coding in efforts for outputs,
screens and data display.  It can automatically center, justify and wrap the text parameters given to it.


\subsubsection{Example one}
\label{\detokenize{advanced:example-one}}
\begin{sphinxVerbatim}[commandchars=\\\{\}]
@emit printf(|\PYGZdl{}\PYGZhy{}12s|\PYGZdl{}12s|\PYGZdl{}\PYGZca{}12s\PYGZdl{}\PYGZam{}14s\PYGZdl{}\PYGZus{}12s|,a b c, d e f, g h i, wrap(lnum(20),12, l, |, |), j k l)

|a b c       |       d e f|   g h i    |0 1 2 3 4 5 |j     k    l|
                                       |6 7 8 9 10  |
                                       |11 12 13 14 |
                                       |15 16 17 18 |
                                       |19          |
\end{sphinxVerbatim}


\subsubsection{Example two}
\label{\detokenize{advanced:example-two}}
\begin{sphinxVerbatim}[commandchars=\\\{\}]
  @emit printf(\PYGZdl{}14\PYGZam{}s \PYGZdl{}\PYGZca{}4\PYGZam{}s \PYGZdl{}\PYGZhy{}3\PYGZam{}s \PYGZdl{}15\PYGZam{}s,
  iter(Bruised|Hurt|Injured|Wounded|Mauled|Crippled|Incapacitated,\PYGZsh{}\PYGZsh{},|,\PYGZpc{}R),
  iter(|\PYGZhy{}1|\PYGZhy{}1|\PYGZhy{}2|\PYGZhy{}2|\PYGZhy{}5|,\PYGZsh{}\PYGZsh{},|,\PYGZpc{}r),iter(lnum(1,7),\PYGZpc{}[[if(gte(get(\PYGZpc{}\PYGZsh{}/damage),\PYGZsh{}\PYGZsh{}),X,\PYGZpc{}b)]\PYGZpc{}],,\PYGZpc{}r),
  * Aggravated\PYGZpc{}RX Lethal\PYGZpc{}R/ Bashing)

      Bruised      [ ]    * Aggravated
         Hurt  \PYGZhy{}1  [ ]        X Lethal
      Injured  \PYGZhy{}1  [ ]       / Bashing
      Wounded  \PYGZhy{}2  [ ]
       Mauled  \PYGZhy{}2  [ ]
     Crippled  \PYGZhy{}5  [ ]
Incapacitated      [ ]
\end{sphinxVerbatim}


\subsubsection{Example three}
\label{\detokenize{advanced:example-three}}
\begin{sphinxVerbatim}[commandchars=\\\{\}]
@emit [printf(\PYGZdl{}\PYGZhy{}10|\PYGZdq{}\PYGZsq{}s\PYGZdl{}\PYGZhy{}60|\PYGZdq{}s,a b c d e f g h i j k l m n o p q r s t u v w x y z,
this is a test a groovy test blah blah blah [repeat(blah\PYGZpc{}b,100)])]END!

a b c d e this is a test a groovy test blah blah blah blah blah blah
f g h i j blah blah blah blah blah blah blah blah blah blah blah blah
k l m n o blah blah blah blah blah blah blah blah blah blah blah blah
p q r s t blah blah blah blah blah blah blah blah blah blah blah blah
u v w x y blah blah blah blah blah blah blah blah blah blah blah blah
z         blah blah blah blah blah blah blah blah blah blah blah blah
blah blah blah blah blah blah blah blah blah blah blah blah blah blah
blah blah blah blah blah blah blah blah blah blah blah blah blah blah
blah blah blah blah blah blah blah                                    END!
\end{sphinxVerbatim}


\section{Format for image files}
\label{\detokenize{advanced:format-for-image-files}}
\sphinxAtStartPar
The image format goes like this:

\sphinxAtStartPar
Data Type  Example  Description
———  ——\sphinxhyphen{}  ———————————————————\sphinxhyphen{}
INT        3        TYPE: room 0, thing 1, exit 2, player 3, zone 4, garbage 5
STRING     Wizard   NAME: of the target.  Verbatum, no quotes surround it
{\color{red}\bfseries{}*}INT       123      LOCATION: dbref\# of the target.  No prepending ‘\#’ used.
{\color{red}\bfseries{}*}INT       234      CONTENTS: The first content in a linked list content table (\sphinxhyphen{}1 if none)
{\color{red}\bfseries{}*}INT       345      EXITS: The first exit in a linked list exit table (\sphinxhyphen{}1 if none)
{\color{red}\bfseries{}*}INT       0        LINK: This is the ‘home’ of the object or what it’s linked to (\sphinxhyphen{}1 for none)
{\color{red}\bfseries{}*}INT       123      NEXT: The next thing after this item for a content holder
STRING     \#123     LOCK: The boolean string lock if it exists.  (empty if no lock)
{\color{red}\bfseries{}*}INT       1        OWNER: The dbref\# owner of the target.  For players same dbref as player.
INT        789      PARENT: The parent of the target.  (\sphinxhyphen{}1 if none)
{\color{red}\bfseries{}*}INT       99999    MONEY: The int value of the money the players has.
INT        194592   FLAGS1: The first word of flags (@set flags) on a player      (see FLAGS)
INT        194222   FLAGS2: The second word of flags (@set flags) on a player     (see FLAGS)
INT        199999   FLAGS3: The third word of flags (@set flags {[}{]}) on a player   (see FLAGS)
INT        1582958  FLAGS4: The forth word of flags (@set flags {[}{]}) on a player   (see FLAGS)
INT        159955   TOGGLES1: The first word of toggles (@toggle) on a player    (see TOGGLES)
INT        159958   TOGGLES2: The second word of toggles (@toggle) on a player   (see TOGGLES)
INT        159958   POWER1: The first word of powers (@power) on a player         (see POWERS)
INT        159958   POWER2: The second word of powers (@power) on a player        (see POWERS)
INT        159958   POWER3: The third word of powers (@power) on a player         (see POWERS)
INT        159958   DEPOWER1: The first word of depowers (@depower) on a player  (see DEPOWERS)
INT        159958   DEPOWER2: The second word of depowers (@depower) on a player (see DEPOWERS)
INT        159958   DEPOWER3: The third word of depowers (@depower) on a player  (see DEPOWERS)
INT        \sphinxhyphen{}1       ZONE(S): The list of zones starting here and ending with ‘\sphinxhyphen{}1’. (see ZONES)
\textgreater{}STRING    \textgreater{}VA      ATTRIBUTENAME: Attribute name to store, starts with \textgreater{} identifier
STRING     Wheee    ATTRIBUTECONTENTS: Contents of attribute.  Multi\sphinxhyphen{}lines seperate with \textasciicircum{}M (control\sphinxhyphen{}M)
\textgreater{}STRING    \textgreater{}Desc    ATTRIBUTENAME: Another attribute to chain in
STRING     Ugly     ATTRIBUTECONTENTS: Contents of the next attribute
\textgreater{}STRING    {\color{red}\bfseries{}*}Password PASSWORDATTRIB: Special password attribute.  Attribute name is ‘{\color{red}\bfseries{}*}Password’
STRING     \$6\$xy\$xy PASSWORDCONTENTS: The SHA512 password (if glibc 2.7+ supported on system) (see PASS)
\textless{}          \textless{}        This is the marker to specify the end of the attribute contents.  This is always the last line

\begin{sphinxadmonition}{note}{Note:}
\sphinxAtStartPar
Any Data type starting with ‘*’ is ignored when @snapshot/loading.
\end{sphinxadmonition}

\sphinxAtStartPar
The structure above with the examples would look like this in the file:

\sphinxAtStartPar
3
Wizard
123
234
345
0
123
\#123
1
789
99999
194592
194222
199999
1582958
159955
159958
159958
159958
159958
159958
159958
159958
\sphinxhyphen{}1
\textgreater{}VA
Wheee
\textgreater{}Desc
Ugly
\textgreater{}*Password
\$6\$xy\$xy
\textless{}


\subsection{HELP key indexes for the values:}
\label{\detokenize{advanced:help-key-indexes-for-the-values}}\begin{description}
\item[{FLAGS: The following flags are to be used.  They are BITWISE masks that you}] \leavevmode
\sphinxAtStartPar
need to add together for the values tghat you apply

\end{description}

\sphinxAtStartPar
/* First word of flags \sphinxstyleemphasis{/
\#define SEETHRU         0x00000008      /} Can see through to the other side \sphinxstyleemphasis{/
\#define WIZARD          0x00000010      /} gets automatic control \sphinxstyleemphasis{/
\#define LINK\_OK         0x00000020      /} anybody can link to this room \sphinxstyleemphasis{/
\#define DARK            0x00000040      /} Don’t show contents or presence \sphinxstyleemphasis{/
\#define JUMP\_OK         0x00000080      /} Others may @tel here \sphinxstyleemphasis{/
\#define STICKY          0x00000100      /} Object goes home when dropped \sphinxstyleemphasis{/
\#define DESTROY\_OK      0x00000200      /} Others may @destroy \sphinxstyleemphasis{/
\#define HAVEN           0x00000400      /} No killing here, or no pages \sphinxstyleemphasis{/
\#define QUIET           0x00000800      /} Prevent ‘feelgood’ messages \sphinxstyleemphasis{/
\#define HALT            0x00001000      /} object cannot perform actions \sphinxstyleemphasis{/
\#define TRACE           0x00002000      /} Generate evaluation trace output \sphinxstyleemphasis{/
\#define GOING           0x00004000      /} object is available for recycling \sphinxstyleemphasis{/
\#define MONITOR         0x00008000      /} Process \textasciicircum{}x:action listens on obj? \sphinxstyleemphasis{/
\#define MYOPIC          0x00010000      /} See things as nonowner/nonwizard \sphinxstyleemphasis{/
\#define PUPPET          0x00020000      /} Relays ALL messages to owner \sphinxstyleemphasis{/
\#define CHOWN\_OK        0x00040000      /} Object may be @chowned freely \sphinxstyleemphasis{/
\#define ENTER\_OK        0x00080000      /} Object may be ENTERed \sphinxstyleemphasis{/
\#define VISUAL          0x00100000      /} Everyone can see properties \sphinxstyleemphasis{/
\#define IMMORTAL        0x00200000      /} Object can’t be killed \sphinxstyleemphasis{/
\#define HAS\_STARTUP     0x00400000      /} Load some attrs at startup \sphinxstyleemphasis{/
\#define OPAQUE          0x00800000      /} Can’t see inside \sphinxstyleemphasis{/
\#define VERBOSE         0x01000000      /} Tells owner everything it does. \sphinxstyleemphasis{/
\#define INHERIT         0x02000000      /} Gets owner’s privs. (i.e. Wiz) \sphinxstyleemphasis{/
\#define NOSPOOF         0x04000000      /} Report originator of all actions. \sphinxstyleemphasis{/
\#define ROBOT           0x08000000      /} Player is a ROBOT \sphinxstyleemphasis{/
\#define SAFE            0x10000000      /} Need /override to @destroy \sphinxstyleemphasis{/
\#define CONTROL\_OK      0x20000000      /} ControlLk specifies who ctrls me \sphinxstyleemphasis{/
\#define HEARTHRU        0x40000000      /} Can hear out of this obj or exit \sphinxstyleemphasis{/
\#define TERSE           0x80000000      /} Only show room name on look {\color{red}\bfseries{}*}/

\sphinxAtStartPar
/* Second word of flags \sphinxstyleemphasis{/
\#define KEY             0x00000001      /} No puppets \sphinxstyleemphasis{/
\#define ABODE           0x00000002      /} May @set home here \sphinxstyleemphasis{/
\#define FLOATING        0x00000004      /} Inhibit Floating room.. msgs \sphinxstyleemphasis{/
\#define UNFINDABLE      0x00000008      /} Cant loc() from afar \sphinxstyleemphasis{/
\#define PARENT\_OK       0x00000010      /} Others may @parent to me \sphinxstyleemphasis{/
\#define LIGHT           0x00000020      /} Visible in dark places \sphinxstyleemphasis{/
\#define HAS\_LISTEN      0x00000040      /} Internal: LISTEN attr set \sphinxstyleemphasis{/
\#define HAS\_FWDLIST     0x00000080      /} Internal: FORWARDLIST attr set \sphinxstyleemphasis{/
\#define ADMIN           0x00000100      /} Player has admin privs \sphinxstyleemphasis{/
\#define GUILDOBJ        0x00000200
\#define GUILDMASTER     0x00000400      /} Player has gm privs \sphinxstyleemphasis{/
\#define NO\_WALLS        0x00000800      /} So to stop normal walls \sphinxstyleemphasis{/
\#define REQUIRE\_TREES   0x00001000      /} Trees are required on this target for attrib sets \sphinxstyleemphasis{/
/} —\sphinxhyphen{}FREE—\sphinxhyphen{}         0x00002000 \sphinxstyleemphasis{/   /} \#define OLD\_NOROBOT  0x00002000 \sphinxstyleemphasis{/
\#define SCLOAK          0x00004000
\#define CLOAK           0x00008000
\#define FUBAR           0x00010000
\#define INDESTRUCTABLE  0x00020000      /} object can’t be nuked \sphinxstyleemphasis{/
\#define NO\_YELL         0x00040000      /} player can’t @wall \sphinxstyleemphasis{/
\#define NO\_TEL          0x00080000      /} player can’t @tel or be @tel’d \sphinxstyleemphasis{/
\#define FREE            0x00100000      /} object/player has unlim money \sphinxstyleemphasis{/
\#define GUEST\_FLAG      0x00200000
\#define RECOVER         0x00400000
\#define BYEROOM         0x00800000
\#define WANDERER        0x01000000
\#define ANSI            0x02000000
\#define ANSICOLOR       0x04000000
\#define NOFLASH         0x08000000
\#define SUSPECT         0x10000000      /} Report some activities to wizards \sphinxstyleemphasis{/
\#define BUILDER         0x20000000      /} Player has architect privs \sphinxstyleemphasis{/
\#define CONNECTED       0x40000000      /} Player is connected \sphinxstyleemphasis{/
\#define SLAVE           0x80000000      /} Disallow most commands {\color{red}\bfseries{}*}/

\sphinxAtStartPar
/* Third word of flags \sphinxhyphen{} Thorin 3/95 \sphinxstyleemphasis{/
\#define NOCONNECT       0x00000001
\#define DPSHIFT         0x00000002
\#define NOPOSSESS       0x00000004
\#define COMBAT          0x00000008
\#define IC              0x00000010
\#define ZONEMASTER      0x00000020
\#define ALTQUOTA        0x00000040
\#define NOEXAMINE       0x00000080
\#define NOMODIFY        0x00000100
\#define CMDCHECK        0x00000200
\#define DOORRED         0x00000400
\#define PRIVATE         0x00000800      /} For exits only \sphinxstyleemphasis{/
\#define NOMOVE          0x00001000
\#define STOP            0x00002000
\#define NOSTOP          0x00004000
\#define NOCOMMAND       0x00008000
\#define AUDIT           0x00010000
\#define SEE\_OEMIT       0x00020000
\#define NO\_GOBJ         0x00040000
\#define NO\_PESTER       0x00080000
\#define LRFLAG          0x00100000
\#define TELOK           0x00200000
\#define NO\_OVERRIDE     0x00400000
\#define NO\_USELOCK      0x00800000
\#define DR\_PURGE        0x01000000      /} For rooms only…internal \sphinxstyleemphasis{/
\#define NO\_ANSINAME     0x02000000      /} Remove the ability to set @ansiname \sphinxstyleemphasis{/
\#define SPOOF           0x04000000
\#define SIDEFX          0x08000000      /} Allow enactor to use side\sphinxhyphen{}effects \sphinxstyleemphasis{/
\#define ZONECONTENTS    0x10000000      /} Search contents of zonemaster for \$commands \sphinxstyleemphasis{/
\#define NOWHO           0x20000000      /} Player in WHO doesn’t show up \sphinxhyphen{} use with @hide \sphinxstyleemphasis{/
\#define ANONYMOUS       0x40000000      /} Player set shows up as ‘Someone’ when talking \sphinxstyleemphasis{/
\#define BACKSTAGE       0x80000000      /} Immortal toggle for items on control {\color{red}\bfseries{}*}/

\sphinxAtStartPar
/* Forth word of flags \sphinxhyphen{} Thorin 3/95 \sphinxstyleemphasis{/
\#define NOBACKSTAGE     0x00000001      /} Immortal toggle to control no\sphinxhyphen{}backstage \sphinxstyleemphasis{/
\#define LOGIN           0x00000002      /} Enable player to login past @disable logins \sphinxstyleemphasis{/
\#define INPROGRAM       0x00000004      /} Player is inside a program \sphinxstyleemphasis{/
\#define COMMANDS        0x00000008      /} Optional define for \$commands \sphinxstyleemphasis{/
\#define MARKER0         0x00000010      /} TM 3.0 marker flags \sphinxstyleemphasis{/
\#define MARKER1         0x00000020
\#define MARKER2         0x00000040
\#define MARKER3         0x00000080
\#define MARKER4         0x00000100
\#define MARKER5         0x00000200
\#define MARKER6         0x00000400
\#define MARKER7         0x00000800
\#define MARKER8         0x00001000
\#define MARKER9         0x00002000
\#define BOUNCE          0x00004000      /} That lovly TM 3.0 Bouncey thingy \sphinxstyleemphasis{/
\#define SHOWFAILCMD     0x00008000      /} Show failed \$commands defauilt error \sphinxstyleemphasis{/
\#define NOUNDERLINE     0x00010000      /} Strip UNDERLINE character from ANSI \sphinxstyleemphasis{/
\#define NONAME          0x00020000      /} Target does not display name with look \sphinxstyleemphasis{/
\#define ZONEPARENT      0x00040000      /} Target zone allows inheritance of attribs \sphinxstyleemphasis{/
\#define SPAMMONITOR     0x00080000      /} Monitor the target for spam \sphinxstyleemphasis{/
\#define BLIND           0x00100000      /} Exits and locations snuff arrived/left \sphinxstyleemphasis{/
\#define NOCODE          0x00200000      /} Players may not code \sphinxstyleemphasis{/
\#define HAS\_PROTECT     0x00400000      /} Player target has protect name data \sphinxstyleemphasis{/
\#define XTERMCOLOR      0x00800000      /} Extended AnSI Xterm colors \sphinxstyleemphasis{/
\#define HAS\_ATTRPIPE    0x01000000      /} Attribute piping via @pipe \sphinxstyleemphasis{/
/} 0x02000000 free \sphinxstyleemphasis{/
/} 0x04000000 free \sphinxstyleemphasis{/
/} 0x08000000 free \sphinxstyleemphasis{/
/} 0x10000000 free \sphinxstyleemphasis{/
/} 0x20000000 free \sphinxstyleemphasis{/
/} 0x40000000 free \sphinxstyleemphasis{/
/} 0x80000000 free {\color{red}\bfseries{}*}/


\bigskip\hrule\bigskip

\begin{description}
\item[{TOGGLES: Toggles are BITWISE masks taht need to be applied for each word like}] \leavevmode
\sphinxAtStartPar
the flags above.  They are added together for each word type

\end{description}

\sphinxAtStartPar
/* First word of toggles \sphinxhyphen{} Thorin 3/95 \sphinxstyleemphasis{/
\#define TOG\_MONITOR             0x00000001      /} Active monitor on player \sphinxstyleemphasis{/
\#define TOG\_MONITOR\_USERID      0x00000002      /} show userid \sphinxstyleemphasis{/
\#define TOG\_MONITOR\_SITE        0x00000004      /} show site \sphinxstyleemphasis{/
\#define TOG\_MONITOR\_STATS       0x00000008      /} show stats \sphinxstyleemphasis{/
\#define TOG\_MONITOR\_FAIL        0x00000010      /} show fails \sphinxstyleemphasis{/
\#define TOG\_MONITOR\_CONN        0x00000020
\#define TOG\_VANILLA\_ERRORS      0x00000040      /} show normal error msg \sphinxstyleemphasis{/
\#define TOG\_NO\_ANSI\_EX          0x00000080      /} supress ansi stuff in ex \sphinxstyleemphasis{/
\#define TOG\_CPUTIME             0x00000100      /} show cpu time for cmds \sphinxstyleemphasis{/
\#define TOG\_MONITOR\_DISREASON   0x00000200
\#define TOG\_MONITOR\_VLIMIT      0x00000400
\#define TOG\_NOTIFY\_LINK         0x00000800
\#define TOG\_MONITOR\_AREG        0x00001000
\#define TOG\_MONITOR\_TIME        0x00002000
\#define TOG\_CLUSTER             0x00004000      /} Object is part of a cluster \sphinxstyleemphasis{/
\#define TOG\_SNUFFDARK           0x00008000      /} Snuff Dark Exit Viewing \sphinxstyleemphasis{/
\#define TOG\_NOANSI\_PLAYER       0x00010000      /} Do not show ansi player names \sphinxstyleemphasis{/
\#define TOG\_NOANSI\_THING        0x00020000      /} … things \sphinxstyleemphasis{/
\#define TOG\_NOANSI\_ROOM         0x00040000      /} … rooms \sphinxstyleemphasis{/
\#define TOG\_NOANSI\_EXIT         0x00080000      /} … exits \sphinxstyleemphasis{/
\#define TOG\_NO\_TIMESTAMP        0x00100000      /} Do not modify timestamps on target \sphinxstyleemphasis{/
\#define TOG\_NO\_FORMAT           0x00200000      /} Override @conformat/@exitformat \sphinxstyleemphasis{/
\#define TOG\_ZONE\_AUTOADD        0x00400000      /} Automatically add FIRST zone in list \sphinxstyleemphasis{/
\#define TOG\_ZONE\_AUTOADDALL     0x00800000      /} Automatically add ALL zones in list \sphinxstyleemphasis{/
\#define TOG\_WIELDABLE           0x01000000      /} Marker to specify if object is wieldable \sphinxstyleemphasis{/
\#define TOG\_WEARABLE            0x02000000      /} Marker to specify if object is wearable \sphinxstyleemphasis{/
\#define TOG\_SEE\_SUSPECT         0x04000000      /} Specify who sees suspect in WHO/MONITOR \sphinxstyleemphasis{/
\#define TOG\_MONITOR\_CPU         0x08000000      /} Specify who sees CPU overflow allerts \sphinxstyleemphasis{/
\#define TOG\_BRANDY\_MAIL         0x10000000      /} Define brandy like mail interface \sphinxstyleemphasis{/
\#define TOG\_FORCEHALTED         0x20000000      /} The item toggled can @force halted things \sphinxstyleemphasis{/
\#define TOG\_PROG                0x40000000      /} Can use @program on other people/things \sphinxstyleemphasis{/
\#define TOG\_NOSHELLPROG         0x80000000      /} Target can not issue commands inside a prog {\color{red}\bfseries{}*}/

\sphinxAtStartPar
/* Second word of toggles \textendash{} Ash \sphinxstyleemphasis{/
\#define TOG\_EXTANSI             0x00000001      /} Specify if target can used extended ansi naming \sphinxstyleemphasis{/
\#define TOG\_IMMPROG             0x00000002      /} Only an immortal can @quitprogram them \sphinxstyleemphasis{/
\#define TOG\_MONITOR\_BFAIL       0x00000004      /} Monitor if a failed connect on bad character \sphinxstyleemphasis{/
\#define TOG\_PROG\_ON\_CONNECT     0x00000008      /} Reverse logic of program on connect \sphinxstyleemphasis{/
\#define TOG\_MAIL\_STRIPRETURN    0x00000010      /} Strip carrage return in mail combining \sphinxstyleemphasis{/
\#define TOG\_PENN\_MAIL           0x00000020      /} Use PENN style syntax \sphinxstyleemphasis{/
\#define TOG\_SILENTEFFECTS       0x00000040      /} Silents did\_it() functionality on target \sphinxstyleemphasis{/
\#define TOG\_IGNOREZONE          0x00000080      /} Target is set to @icmd zones \sphinxstyleemphasis{/
\#define TOG\_VPAGE               0x00000100      /} Target sees alias in pages \sphinxstyleemphasis{/
\#define TOG\_PAGELOCK            0x00000200      /} Target issues pagelocks as normal \sphinxstyleemphasis{/
\#define TOG\_MAIL\_NOPARSE        0x00000400      /} Don’t parse \%t/\%b/\%r in mail \sphinxstyleemphasis{/
\#define TOG\_MAIL\_LOCKDOWN       0x00000800      /} Mortal\sphinxhyphen{}accessed mail/number and mail/check \sphinxstyleemphasis{/
\#define TOG\_MUXPAGE             0x00001000      /} Have ‘page’ work like MUX \sphinxstyleemphasis{/
\#define TOG\_NOZONEPARENT        0x00002000      /} Zone Child does NOT inherit parent attribs \sphinxstyleemphasis{/
\#define TOG\_ATRUSE              0x00004000      /} Enactor can use Attribute based USELOCKS \sphinxstyleemphasis{/
\#define TOG\_VARIABLE            0x00008000      /} Set exit to be variable \sphinxstyleemphasis{/
\#define TOG\_KEEPALIVE           0x00010000      /} Send ‘keepalives’ to the target player \sphinxstyleemphasis{/
\#define TOG\_CHKREALITY          0x00020000      /} Target checks @lock/user for Reality passes \sphinxstyleemphasis{/
\#define TOG\_NOISY               0x00040000      /} Always do noisy sets \sphinxstyleemphasis{/
\#define TOG\_ZONECMDCHK          0x00080000      /} Zone commands checked on target like @parent \sphinxstyleemphasis{/
\#define TOG\_HIDEIDLE            0x00100000      /} Allow wizards/immortals to hide their idle time \sphinxstyleemphasis{/
\#define TOG\_MORTALREALITY       0x00200000      /} Override the wiz\_always\_real setting \sphinxstyleemphasis{/
\#define TOG\_ACCENTS             0x00400000      /} Accents being displayed \sphinxstyleemphasis{/
\#define TOG\_PREMAILVALIDATE     0x00800000      /} Pre\sphinxhyphen{}Validate the mail send list before sending mail \sphinxstyleemphasis{/
\#define TOG\_SAFELOG             0x01000000      /} Allow ‘clean logging’ by the player \sphinxstyleemphasis{/
\#define TOG\_UTF8                0x02000000      /} UTF8 being displayed \sphinxstyleemphasis{/
/} 0x04000000 free \sphinxstyleemphasis{/
\#define TOG\_NODEFAULT           0x08000000      /} Allow target to inherit default attribs \sphinxstyleemphasis{/
\#define TOG\_EXFULLWIZATTR       0x10000000      /} Examine Wiz attribs \sphinxstyleemphasis{/
\#ifdef ENH\_LOGROOM
\#define TOG\_LOGROOMENH          0x20000000      /} Enhanced Room Logging \sphinxstyleemphasis{/
\#endif
\#define TOG\_LOGROOM             0x40000000      /} Log Room’s location/contents \sphinxstyleemphasis{/
\#define TOG\_NOGLOBPARENT        0x80000000      /} Target does not inherit global attributes {\color{red}\bfseries{}*}/


\bigskip\hrule\bigskip

\begin{description}
\item[{POWERS:  Powers are handled a bit differently.  They’re used as BITWISE shift}] \leavevmode
\sphinxAtStartPar
markers that you would have to compute the shift then add it after
the fact.

\item[{/* First word of power positions.  Each position is 2 bits so the}] \leavevmode
\sphinxAtStartPar
number here is how far over to shift the 2 bit pattern         {\color{red}\bfseries{}*}/

\end{description}

\sphinxAtStartPar
\#define POWER\_CHANGE\_QUOTAS     0
\#define POWER\_CHOWN\_ME          2
\#define POWER\_CHOWN\_ANYWHERE    4
\#define POWER\_CHOWN\_OTHER       6
\#define POWER\_WIZ\_WHO           8
\#define POWER\_EX\_ALL            10
\#define POWER\_NOFORCE           12
\#define POWER\_SEE\_QUEUE\_ALL     14
\#define POWER\_FREE\_QUOTA        16
\#define POWER\_GRAB\_PLAYER       18
\#define POWER\_JOIN\_PLAYER       20
\#define POWER\_LONG\_FINGERS      22
\#define POWER\_NO\_BOOT           24
\#define POWER\_BOOT              26
\#define POWER\_STEAL             28
\#define POWER\_SEE\_QUEUE         30

\sphinxAtStartPar
/* Second word of power positions. {\color{red}\bfseries{}*}/
\#define POWER\_SHUTDOWN          0
\#define POWER\_TEL\_ANYWHERE      2
\#define POWER\_TEL\_ANYTHING      4
\#define POWER\_PCREATE           6
\#define POWER\_STAT\_ANY          8
\#define POWER\_FREE\_WALL         10
\#define POWER\_EXECSCRIPT        12
\#define POWER\_FREE\_PAGE         14
\#define POWER\_HALT\_QUEUE        16
\#define POWER\_HALT\_QUEUE\_ALL    18
\#define POWER\_FORMATTING        20
\#define POWER\_NOKILL            22
\#define POWER\_SEARCH\_ANY        24
\#define POWER\_SECURITY          26
\#define POWER\_WHO\_UNFIND        28

\sphinxAtStartPar
/* Third word of power positions. \sphinxstyleemphasis{/
\#define POWER\_OPURGE            0
\#define POWER\_HIDEBIT           2
\#define POWER\_NOWHO             4
\#define POWER\_FULLTEL\_ANYWHERE  6
\#define POWER\_EX\_FULL           8
\#define POWER\_API               10
\#define POWER\_MONITORAPI        12
\#define POWER\_WIZ\_IDLE          14
\#define POWER\_WIZ\_SPOOF         16
/} 18 free \sphinxstyleemphasis{/
/} 20 free \sphinxstyleemphasis{/
/} 22 free \sphinxstyleemphasis{/
/} 24 free \sphinxstyleemphasis{/
/} 26 free \sphinxstyleemphasis{/
/} 28 free \sphinxstyleemphasis{/
/} 30 free {\color{red}\bfseries{}*}/


\bigskip\hrule\bigskip

\begin{description}
\item[{DEPOWERS:  like @powers they are handled with a BITWISE offshift that you}] \leavevmode
\sphinxAtStartPar
will have to calculate then add

\end{description}

\sphinxAtStartPar
/* First word {\color{red}\bfseries{}*}/
\#define DP\_WALL                 0
\#define DP\_LONG\_FINGERS         2
\#define DP\_STEAL                4
\#define DP\_CREATE               6
\#define DP\_WIZ\_WHO              8
\#define DP\_CLOAK                10
\#define DP\_BOOT                 12
\#define DP\_PAGE                 14
\#define DP\_FORCE                16
\#define DP\_LOCKS                18
\#define DP\_COM                  20
\#define DP\_COMMAND              22
\#define DP\_MASTER               24
\#define DP\_EXAMINE              26
\#define DP\_NUKE                 28
\#define DP\_FREE                 30

\sphinxAtStartPar
/* Second word \sphinxstyleemphasis{/
\#define DP\_OVERRIDE             0
\#define DP\_TEL\_ANYWHERE         2
\#define DP\_TEL\_ANYTHING         4
\#define DP\_PCREATE              6
\#define DP\_POWER                8
\#define DP\_QUOTA                10
\#define DP\_MODIFY               12
\#define DP\_CHOWN\_ME             14
\#define DP\_CHOWN\_OTHER          16
\#define DP\_ABUSE                18
\#define DP\_UNL\_QUOTA            20
\#define DP\_SEARCH\_ANY           22
\#define DP\_GIVE                 24
\#define DP\_RECEIVE              26
\#define DP\_NOGOLD               28
\#define DP\_NOSTEAL              30
/} Third word…and there was much rejoicing \sphinxstyleemphasis{/
\#define DP\_PASSWORD             0
\#define DP\_MORTAL\_EXAMINE       2
\#define DP\_PERSONAL\_COMMANDS    4
/} 6  free \sphinxstyleemphasis{/
\#define DP\_DARK                 8
/} 10 free \sphinxstyleemphasis{/
/} 12 free \sphinxstyleemphasis{/
/} 14 free \sphinxstyleemphasis{/
/} 16 free \sphinxstyleemphasis{/
/} 18 free \sphinxstyleemphasis{/
/} 20 free \sphinxstyleemphasis{/
/} 22 free \sphinxstyleemphasis{/
/} 24 free \sphinxstyleemphasis{/
/} 26 free \sphinxstyleemphasis{/
/} 28 free \sphinxstyleemphasis{/
/} 30 free {\color{red}\bfseries{}*}/


\bigskip\hrule\bigskip


\begin{sphinxadmonition}{note}{Note:}
\sphinxAtStartPar
ZONES:  Zones are special.  If there are no zones, the value will be ‘\sphinxhyphen{}1’.
\end{sphinxadmonition}

\sphinxAtStartPar
So entering zones if there are no zones:
\sphinxhyphen{}1

\sphinxAtStartPar
Entering zones if it has three zones (\#123, \#456, and \#789)
123
456
789
\sphinxhyphen{}1

\sphinxAtStartPar
As you see, the last value of the zone \sphinxstyleemphasis{MUST} be \sphinxhyphen{}1.  This tells it
that there are no more zones to add.


\section{Comparison of Flags}
\label{\detokenize{flags:comparison-of-flags}}\label{\detokenize{flags::doc}}
\sphinxAtStartPar
ABODE           \sphinxhyphen{} ABODE
BLIND           \sphinxhyphen{} BLIND
CHOWN\_OK        \sphinxhyphen{} CHOWN\_OK
DARK            \sphinxhyphen{} DARK
FREE            \sphinxhyphen{} FREE
GOING           \sphinxhyphen{} GOING/BYEROOM
HAVEN           \sphinxhyphen{} HAVEN
INHERIT         \sphinxhyphen{} INHERIT
JUMP\_OK         \sphinxhyphen{} JUMP\_OK
KEY             \sphinxhyphen{} KEY
LINK\_OK         \sphinxhyphen{} LINK\_OK
MONITOR         \sphinxhyphen{} MONITOR
NOSPOOF         \sphinxhyphen{} NOSPOOF
OPAQUE          \sphinxhyphen{} OPAQUE
QUIET           \sphinxhyphen{} QUIET
STICKY          \sphinxhyphen{} STICKY
TRACE           \sphinxhyphen{} TRACE
UNFINDABLE      \sphinxhyphen{} UNFINDABLE
VISUAL          \sphinxhyphen{} VISUAL
WIZARD          \sphinxhyphen{} ROYALTY
ANSI            \sphinxhyphen{} ANSI/ANSICOLOR
PARENT\_OK       \sphinxhyphen{} PARENT\_OK
ROYALTY         \sphinxhyphen{} COUNCILOR/ARCHITECT
AUDIBLE         \sphinxhyphen{} AUDIBLE
BOUNCE          \sphinxhyphen{} BOUNCE
CONNECTED       \sphinxhyphen{} CONNECTED
DESTROY\_OK      \sphinxhyphen{} DESTROY\_OK
ENTER\_OK        \sphinxhyphen{} ENTER\_OK
FIXED           \sphinxhyphen{} NO\_TEL
UNINSPECTED     \sphinxhyphen{} Not Available \sphinxhyphen{} Just a marker flag
HALTED          \sphinxhyphen{} HALTED
IMMORTAL        \sphinxhyphen{} GUILDMASTER (You don’t want IMMORTAL)
GAGGED          \sphinxhyphen{} FUBAR
CONSTANT        \sphinxhyphen{} NO\_MODIFY
LIGHT           \sphinxhyphen{} LIGHT
MYOPIC          \sphinxhyphen{} MYOPIC
AUDITORIUM      \sphinxhyphen{} AUDITORIUM
ZONE            \sphinxhyphen{} Use @zone
PUPPET          \sphinxhyphen{} PUPPET
TERSE           \sphinxhyphen{} TERSE
ROBOT           \sphinxhyphen{} ROBOT
SAFE            \sphinxhyphen{} SAFE
TRANSPARENT     \sphinxhyphen{} TRANSPARENT
SUSPECT         \sphinxhyphen{} SUSPECT
VERBOSE         \sphinxhyphen{} VERBOSE
STAFF           \sphinxhyphen{} Not Available \sphinxhyphen{} Just a marker flag.
SLAVE           \sphinxhyphen{} SLAVE
ORPHAN          \sphinxhyphen{} Not Available \sphinxhyphen{} @lock/use the parent instead
CONTROL\_OK      \sphinxhyphen{} Not Available \sphinxhyphen{} Use @lock/ZoneWizLock
STOP            \sphinxhyphen{} STOP (See also NOSTOP)
COMMANDS        \sphinxhyphen{} COMMANDS
PRESENCE        \sphinxhyphen{} Not Available \sphinxhyphen{} See: Reality Levels
NOBLEED         \sphinxhyphen{} Not Needed.  Rhost doesn’t bleed ANSI.
VACATION        \sphinxhyphen{} Not Available \sphinxhyphen{} Just a marker flag.
HEAD            \sphinxhyphen{} Not Available \sphinxhyphen{} Just a marker flag.
WATCHER         \sphinxhyphen{} Not Available \sphinxhyphen{} @toggle MONITOR
HTML            \sphinxhyphen{} Not Available \sphinxhyphen{} Rhost doesn’t support Pueblo
REDIR\_OK        \sphinxhyphen{} Not Available \sphinxhyphen{} Rhost doesn’t support @redirect
SPEECHMOD       \sphinxhyphen{} Not Available \sphinxhyphen{} Rhost doesn’t support @speechmod \sphinxhyphen{} use @icmd
MARKER0\sphinxhyphen{}MARKER9 \sphinxhyphen{} MARKER0\sphinxhyphen{}MARKER9


\section{Comparison of powers}
\label{\detokenize{powers:comparison-of-powers}}\label{\detokenize{powers::doc}}
\sphinxAtStartPar
announce              Can use the @wall command.
Rhost Equiv: \sphinxhyphen{} FREE\_WALL (@power)

\sphinxAtStartPar
boot                  Can use the @boot command.
Rhost Equiv: BOOT (@power)

\sphinxAtStartPar
builder               Can build, if the builder power is enabled.
Rhost Equiv: ARCHITECT (flag)

\sphinxAtStartPar
chown\_anything        Can @chown anything to anyone.
Rhost Equiv: CHOWN\_OTHER (@power)

\sphinxAtStartPar
comm\_all              Like a wizard with respect to channels.
Rhost has no hardcoded comsystem.  You can tweek the softcode.

\sphinxAtStartPar
control\_all           Can modify any object in the database. (God\sphinxhyphen{}set only.)
Rhost Equiv: TwinkLock (@lock)

\sphinxAtStartPar
expanded\_who          Sees the wizard WHO, and SESSION commands.
Rhost Equiv: WIZ\_WHO (@power)

\sphinxAtStartPar
find\_unfindable       Can locate unfindable people.
see\_hidden            Can see hidden (DARK) players on WHO, etc.
Rhost Equiv: WHO\_UNFIND (@power)

\sphinxAtStartPar
free\_money            Unlimited money.
Rhost Equiv: FREE (flag)

\sphinxAtStartPar
free\_quota            Unlimited quota.
Rhost Equiv: FREE\_QUOTA (@power)

\sphinxAtStartPar
guest                 Is this a guest character?
Rhost Equiv: GUEST (flag)

\sphinxAtStartPar
halt                  Can @halt anything, and @halt/all.
Rhost Equiv: HALT\_QUEUE (@power) or HALT\_QUEUE\_ALL (@power)

\sphinxAtStartPar
hide                  Can set themselves DARK.
Rhost Equiv: NOWHO (@power)

\sphinxAtStartPar
idle                  No idle timeout.
Rhost Equiv: @timeout *player=\sphinxhyphen{}1

\sphinxAtStartPar
link\_variable         Can @link an exit to “variable”.
Rhost Equiv: Anyone can do this.  VARIABLE (@toggle)

\sphinxAtStartPar
link\_to\_anything      Can @link an exit to any (non\sphinxhyphen{}variable) destination.
Rhost Equiv: @lock/link (@lock)

\sphinxAtStartPar
long\_fingers          Can get, look, whisper, etc from a distance.
Rhost Equiv: LONG\_FINGERS (@power)

\sphinxAtStartPar
no\_destroy            Cannot be @toad’ed.
Rhost Equiv: INDESTRUCTABLE (flag)

\sphinxAtStartPar
open\_anywhere         Can @open an exit from any location.
Rhost Equiv: @lock/open (@lock)

\sphinxAtStartPar
poll                  Can set the @poll.
Rhost has nothing equivelant.  Just softcode a +poll, or @hook it for permissions.

\sphinxAtStartPar
prog                  Can use @program on players other than themself.
Rhost Equiv: PROG (@toggle)

\sphinxAtStartPar
search                Can @search anyone.
Rhost Equiv: SEARCH\_ANY (@power)

\sphinxAtStartPar
see\_all               Can examine and see attributes like a wizard.
Rhost Equiv: EXAMINE\_FULL (@power) (and EXFULLWIZATTR (@toggle) for wiz only attribs)

\sphinxAtStartPar
see\_queue             Can @ps/all or @ps any player.
Rhost Equiv: SEE\_QUEUE (@power) or SEE\_QUEUE\_ALL (@power)

\sphinxAtStartPar
stat\_any              Can @stat any player.
Rhost Equiv: STAT\_ANY (@power)

\sphinxAtStartPar
steal\_money           Can give negative money.
Rhost Equiv: STEAL (@power)

\sphinxAtStartPar
tel\_anywhere          Can teleport anywhere.
Rhost Equiv: TEL\_ANYWHERE (@power) or FULL\_TEL (@power)

\sphinxAtStartPar
tel\_anything          Can teleport anything (includes tel\_anywhere)
Rhost Equiv: TEL\_ANYTHING (@power)

\sphinxAtStartPar
unkillable            Cannot be killed with the ‘kill’ command.
Rhost Equiv: NOKILL (@power)

\sphinxAtStartPar
use\_sql               Can call the SQL() function. (God\sphinxhyphen{}set only.)
Rhost Equiv: SQL is a 3rd party patch.

\sphinxAtStartPar
watch\_logins          Can set or reset the WATCHER flag on themselves.
Rhost Equiv: MONITOR (@toggle)


\section{RhostMUSH Internal Help Files}
\label{\detokenize{helpfile:rhostmush-internal-help-files}}\label{\detokenize{helpfile::doc}}



\section{RhostMUSH Internal Wizhelp Files}
\label{\detokenize{wizhelpfile:rhostmush-internal-wizhelp-files}}\label{\detokenize{wizhelpfile::doc}}



\section{Changelog}
\label{\detokenize{changelog:changelog}}\label{\detokenize{changelog::doc}}

\subsection{RhostMUSH 4.0 Update}
\label{\detokenize{changelog:rhostmush-4-0-update}}\begin{description}
\item[{MUX passwords didn’t work properly because of a memcmp() bug.}] \leavevmode\begin{itemize}
\item {} 
\sphinxAtStartPar
Thanks Locke

\end{itemize}

\item[{@nuke didn’t properly wipe mail if issued from a non\sphinxhyphen{}player.}] \leavevmode\begin{itemize}
\item {} 
\sphinxAtStartPar
Thanks Benzaiten \& Rockpath

\end{itemize}

\item[{@protect set/unset the wrong flags.}] \leavevmode\begin{itemize}
\item {} 
\sphinxAtStartPar
Thanks Rockpath

\end{itemize}

\item[{Help file typos fixed}] \leavevmode\begin{itemize}
\item {} 
\sphinxAtStartPar
Thanks Rockpath

\end{itemize}

\item[{Bug in EVAL could overwrite static pointer to attribute fetches}] \leavevmode\begin{itemize}
\item {} 
\sphinxAtStartPar
Thanks Ixokai

\end{itemize}

\item[{Bug in printf that could cause a SIGSEGV}] \leavevmode\begin{itemize}
\item {} 
\sphinxAtStartPar
Thanks Ixokai

\end{itemize}

\item[{Bug in AUTH and API handling where AUTH lookups happened with internal checks}] \leavevmode\begin{itemize}
\item {} 
\sphinxAtStartPar
Thanks Neil “Polk” Stevens

\end{itemize}

\item[{@set and set() now allow optionally setting contents starting with ‘\_’.}] \leavevmode\begin{itemize}
\item {} 
\sphinxAtStartPar
Thanks \sphinxhref{mailto:Gallifrey@BrazilMUX}{Gallifrey@BrazilMUX}

\end{itemize}

\item[{Added better user error handling to the build script.}] \leavevmode\begin{itemize}
\item {} 
\sphinxAtStartPar
Thanks qa’toq

\end{itemize}

\item[{Due to an insanely old bug in singleuser mode with attribute caps, flatfiles would cut off attributes at 750 on dbloading.}] \leavevmode\begin{itemize}
\item {} 
\sphinxAtStartPar
Thanks \sphinxhref{mailto:Aqua@MuxNexus}{Aqua@MuxNexus}

\end{itemize}

\item[{@recover/detail to show attributes and details of recoverable item}] \leavevmode\begin{itemize}
\item {} 
\sphinxAtStartPar
Thanks Ixokai

\end{itemize}

\item[{Suggestions in help files inspired from PennMUSH.}] \leavevmode\begin{itemize}
\item {} 
\sphinxAtStartPar
Thanks PennMUSH (Raevnos)

\end{itemize}

\item[{@door/push could crash with non\sphinxhyphen{}players}] \leavevmode\begin{itemize}
\item {} 
\sphinxAtStartPar
Thanks Xperta/Paige

\end{itemize}

\item[{Startmush had a confusing message for Nyctasia as \#1’s password}] \leavevmode\begin{itemize}
\item {} 
\sphinxAtStartPar
Thanks Ixokai

\end{itemize}

\item[{@rxlevel/@txlevel handle bits as well as tabs}] \leavevmode\begin{itemize}
\item {} 
\sphinxAtStartPar
Thanks Myrddin

\end{itemize}

\item[{@fpose/nospace didn’t work because of wrong bitwise flags}] \leavevmode\begin{itemize}
\item {} 
\sphinxAtStartPar
Thanks Myrddin

\end{itemize}

\item[{Potential crash with freeing unitialized buffering in trace stack in eval.c}] \leavevmode\begin{itemize}
\item {} 
\sphinxAtStartPar
Thanks Myrddin

\end{itemize}

\item[{Crash with trace located as a 32 byte SBUF if compiled for that}] \leavevmode\begin{itemize}
\item {} 
\sphinxAtStartPar
Thanks Myrddin

\end{itemize}

\item[{idle\_timeout included in the netrhost.conf by default as highly utilized}] \leavevmode\begin{itemize}
\item {} 
\sphinxAtStartPar
Thanks skew

\end{itemize}

\item[{all l*() math functions allowed empty args for backward compatibility}] \leavevmode\begin{itemize}
\item {} 
\sphinxAtStartPar
Thanks skew

\end{itemize}

\item[{Found bug in lattr() with command  matching.}] \leavevmode\begin{itemize}
\item {} 
\sphinxAtStartPar
Thanks Damascus

\end{itemize}

\item[{Bug in final sorting for setunion/setdiff/setinter.}] \leavevmode\begin{itemize}
\item {} 
\sphinxAtStartPar
Thanks Aqua

\end{itemize}

\item[{Duplicate entries possible with help suggestions.}] \leavevmode\begin{itemize}
\item {} 
\sphinxAtStartPar
Thanks Alley

\end{itemize}

\item[{Global parents (ancestors) should inherit from its own parents.}] \leavevmode\begin{itemize}
\item {} 
\sphinxAtStartPar
Thanks Matrix

\end{itemize}

\item[{parsestr() was missing a prefix handler for the | option.}] \leavevmode\begin{itemize}
\item {} 
\sphinxAtStartPar
Thanks Aqua

\end{itemize}

\item[{conf files didn’t ignore spaces/blank lines}] \leavevmode\begin{itemize}
\item {} 
\sphinxAtStartPar
Thanks ELpH

\end{itemize}

\item[{Erraneous warning when vattr\_cmds not defined}] \leavevmode\begin{itemize}
\item {} 
\sphinxAtStartPar
Thanks ELpH

\end{itemize}

\item[{Added ability to specify subdirectories in a controlled method via @admin for execscript()}] \leavevmode\begin{itemize}
\item {} 
\sphinxAtStartPar
Thanks Kumakun

\end{itemize}

\item[{Added optional target sender for mailsend()}] \leavevmode\begin{itemize}
\item {} 
\sphinxAtStartPar
Thanks Rockpath

\end{itemize}

\item[{Mysql would abort on sub\sphinxhyphen{}results if one of the results were NULL}] \leavevmode\begin{itemize}
\item {} 
\sphinxAtStartPar
Thanks Myrddin

\end{itemize}

\item[{libxcrypt broke SHA512 passwords. (Ubuntu 20+)}] \leavevmode\begin{itemize}
\item {} 
\sphinxAtStartPar
Thanks Darren

\end{itemize}

\item[{Typo in speech.c with a variable.}] \leavevmode\begin{itemize}
\item {} 
\sphinxAtStartPar
Thanks Oleo

\end{itemize}

\item[{SIDEFX permissions were borked for normal players.}] \leavevmode\begin{itemize}
\item {} 
\sphinxAtStartPar
Thanks jan6

\end{itemize}

\item[{\#lambda wasn’t case insensitive}] \leavevmode\begin{itemize}
\item {} 
\sphinxAtStartPar
Thanks Alley

\end{itemize}

\end{description}


\subsection{RhostMUSH 3.9.5 Update}
\label{\detokenize{changelog:rhostmush-3-9-5-update}}\begin{description}
\item[{Softcode overrides for connect files}] \leavevmode\begin{itemize}
\item {} 
\sphinxAtStartPar
Thanks Matrix

\end{itemize}

\item[{Bug in the milisecond timers with regards to dumps}] \leavevmode\begin{itemize}
\item {} 
\sphinxAtStartPar
Thanks Matrix

\end{itemize}

\item[{Ansi auto\sphinxhyphen{}recognized in connect.txt (optionally)}] \leavevmode\begin{itemize}
\item {} 
\sphinxAtStartPar
Thanks Rook

\end{itemize}

\item[{floating point can dynamically be increased for precision.}] \leavevmode\begin{itemize}
\item {} 
\sphinxAtStartPar
Thanks Stephen

\end{itemize}

\item[{Ansi compression and optimization encoding}] \leavevmode\begin{itemize}
\item {} 
\sphinxAtStartPar
Thanks Exaurdon

\end{itemize}

\item[{Fix for timers with milisecond and alarms}] \leavevmode\begin{itemize}
\item {} 
\sphinxAtStartPar
Thanks Ol’Sarge

\end{itemize}

\item[{Fix for Mysql.c issue in a sigfault}] \leavevmode\begin{itemize}
\item {} 
\sphinxAtStartPar
Thanks Ol’Sarge

\end{itemize}

\item[{@depower didn’t work right for inheritance on power\_objects}] \leavevmode\begin{itemize}
\item {} 
\sphinxAtStartPar
Thanks Sunder

\end{itemize}

\item[{Buffer overrun in ansi\_txtfile because of non\sphinxhyphen{}null termination.}] \leavevmode\begin{itemize}
\item {} 
\sphinxAtStartPar
Thanks Rook

\end{itemize}

\item[{mail/mark/save would not alert you if you hit MAX SAVED messages.}] \leavevmode\begin{itemize}
\item {} 
\sphinxAtStartPar
Thanks Mercutio

\end{itemize}

\item[{old\_setq had an issue with ‘!’ parameter not reusing registers.}] \leavevmode\begin{itemize}
\item {} 
\sphinxAtStartPar
Thanks Ixokai

\end{itemize}

\item[{missing test case for scandir() function fixed.}] \leavevmode\begin{itemize}
\item {} 
\sphinxAtStartPar
Thanks Maighdlin

\end{itemize}

\item[{suggestion to make textfile() more useful in functions.}] \leavevmode\begin{itemize}
\item {} 
\sphinxAtStartPar
Thanks qa’toq

\end{itemize}

\item[{alteration of new Makefile build procedure rebuilt based on \sphinxhref{mailto:Ternary@Dark}{Ternary@Dark} Metal’s suggestions.}] \leavevmode\begin{itemize}
\item {} 
\sphinxAtStartPar
Thanks Ternary/Ol’Sarge

\end{itemize}

\item[{fix to the src/Makefile to redefine default SHELL since latest Debian horked it with dash.}] \leavevmode\begin{itemize}
\item {} 
\sphinxAtStartPar
Thanks Mercutio

\end{itemize}

\item[{Make ‘.’ as first char allowable in attribute names.}] \leavevmode\begin{itemize}
\item {} 
\sphinxAtStartPar
Thanks Ixokai

\end{itemize}

\item[{@limit with unlimited values for @destroy and vlimit was broke.}] \leavevmode\begin{itemize}
\item {} 
\sphinxAtStartPar
Thanks Ixokai

\end{itemize}

\item[{@decompile didn’t show @toggles}] \leavevmode\begin{itemize}
\item {} 
\sphinxAtStartPar
Thanks Tesagk

\end{itemize}

\item[{xinc() and xdec() didn’t properly handle labels for registers.}] \leavevmode\begin{itemize}
\item {} 
\sphinxAtStartPar
Thanks UnRegistered Guest (Wanted to be Anonymous)

\end{itemize}

\item[{columns() bug with left justification if fed null for field.}] \leavevmode\begin{itemize}
\item {} 
\sphinxAtStartPar
Thanks Merit

\end{itemize}

\item[{@dynhelp/noindex to snuff the hilight index in @dynhelp.}] \leavevmode\begin{itemize}
\item {} 
\sphinxAtStartPar
Thanks Darren

\end{itemize}

\item[{ltoggles()/hastoggle() didn’t mirror lflags()/hasflag() for permissions.}] \leavevmode\begin{itemize}
\item {} 
\sphinxAtStartPar
Thanks Polk

\end{itemize}

\item[{dark permission makes on\sphinxhyphen{}connect screen command ignored to be overridden.}] \leavevmode\begin{itemize}
\item {} 
\sphinxAtStartPar
Thanks Ixokai

\end{itemize}

\item[{crashbug on random error messages if error.txt was empty}] \leavevmode\begin{itemize}
\item {} 
\sphinxAtStartPar
Thanks Darren

\end{itemize}

\item[{compile issue when you do not have sideeffects enabled at comiletime}] \leavevmode\begin{itemize}
\item {} 
\sphinxAtStartPar
Thanks benzaiten

\end{itemize}

\item[{Idea for /quiet for @pipe}] \leavevmode\begin{itemize}
\item {} 
\sphinxAtStartPar
Thanks Ixokai

\end{itemize}

\item[{Missing showing PARIS mode enabled in WHO/DOING for @list options}] \leavevmode\begin{itemize}
\item {} 
\sphinxAtStartPar
Thanks Myrddin

\end{itemize}

\item[{El Capitan 10.11.5 would not compile cleanly out of the box due to sudden header file location changes.}] \leavevmode\begin{itemize}
\item {} 
\sphinxAtStartPar
Thanks Darren

\end{itemize}

\item[{Main Makefile no longer worked on latest FreeBSD because of their change from gmake.}] \leavevmode\begin{itemize}
\item {} 
\sphinxAtStartPar
Thanks Oleo

\end{itemize}

\item[{Added RPAD and LPAD softcode wrappers}] \leavevmode\begin{itemize}
\item {} 
\sphinxAtStartPar
Thanks Darren

\end{itemize}

\item[{Dark exits were broken with sees()}] \leavevmode\begin{itemize}
\item {} 
\sphinxAtStartPar
Thanks Mercutio

\end{itemize}

\item[{objid’s were based on localtime and not gmtime}] \leavevmode\begin{itemize}
\item {} 
\sphinxAtStartPar
Thanks Fantom

\end{itemize}

\item[{Bug in safer\_ufun with setting attributes on self and permissions with u().}] \leavevmode\begin{itemize}
\item {} 
\sphinxAtStartPar
Thanks Mike/Talvo

\end{itemize}

\item[{Improvement of help entries for @label and various documentation fixes.}] \leavevmode\begin{itemize}
\item {} 
\sphinxAtStartPar
Thanks Mike/Talvo

\end{itemize}

\item[{Added additional spacing to parenmatch() to help with pretty printing.}] \leavevmode\begin{itemize}
\item {} 
\sphinxAtStartPar
Suggestion from Thenomain (thanks!)

\end{itemize}

\item[{\&pageformat and \&outpageformat idea from PennMUSH}] \leavevmode\begin{itemize}
\item {} 
\sphinxAtStartPar
Thanks Mike and the PennMUSH folk :)

\end{itemize}

\item[{Added help entry for a suggestion on special characters.}] \leavevmode\begin{itemize}
\item {} 
\sphinxAtStartPar
Thanks Tesagk

\end{itemize}

\item[{Added ‘d’ and ‘D’ options for wildcarding on editansi.}] \leavevmode\begin{itemize}
\item {} 
\sphinxAtStartPar
Thanks Anixy

\end{itemize}

\item[{Removed attribute flags from @decompile/tf}] \leavevmode\begin{itemize}
\item {} 
\sphinxAtStartPar
Thanks Darren

\end{itemize}

\item[{Potential crash bug with flag\sphinxhyphen{}handling with NONAME in use.}] \leavevmode\begin{itemize}
\item {} 
\sphinxAtStartPar
Thanks Polk \& Aqua

\end{itemize}

\item[{Missing free on an sbuf in sqlite.c}] \leavevmode\begin{itemize}
\item {} 
\sphinxAtStartPar
Thanks Darren

\end{itemize}

\item[{Added \%\_\textless{}\sphinxhyphen{}\textgreater{} to pop last label used.}] \leavevmode\begin{itemize}
\item {} 
\sphinxAtStartPar
Suggestion \textendash{} Ixokai

\end{itemize}

\item[{Nested /notrace on @function/@lfunction was broke.}] \leavevmode\begin{itemize}
\item {} 
\sphinxAtStartPar
Thanks Ixokai

\end{itemize}

\item[{Idea and general code for NO\_CONNECT message \textendash{} Kage}] \leavevmode\begin{itemize}
\item {} 
\sphinxAtStartPar
Thanks Kage

\end{itemize}

\item[{Alternate date formats for convtime()}] \leavevmode\begin{itemize}
\item {} 
\sphinxAtStartPar
Thanks Ixokai

\end{itemize}

\item[{Typos in help.txt}] \leavevmode\begin{itemize}
\item {} 
\sphinxAtStartPar
Thanks Kilmoran

\end{itemize}

\item[{all localization didn’t save state for register names.}] \leavevmode\begin{itemize}
\item {} 
\sphinxAtStartPar
Thanks Myrddin

\end{itemize}

\item[{All locatization with CLEAR didn’t wipe and reset register names.}] \leavevmode\begin{itemize}
\item {} 
\sphinxAtStartPar
Thanks Myrddin

\end{itemize}

\item[{\&SPEECH\_PREFIX/\&SPEECH\_SUFFIX for say/pose pre and post processing.}] \leavevmode\begin{itemize}
\item {} 
\sphinxAtStartPar
Thanks MuSoapBox (Auspice, Sparks, Seamus, faraday, Thenomain, and others)

\end{itemize}

\item[{{]} for @hook/ignore and @hook/permit was broken.}] \leavevmode\begin{itemize}
\item {} 
\sphinxAtStartPar
Thanks Ixokai

\end{itemize}

\end{description}


\subsection{RhostMUSH 3.9.4 Update}
\label{\detokenize{changelog:rhostmush-3-9-4-update}}\begin{description}
\item[{REALITY\_LEVELS wouldn’t compile because of undeclared function.}] \leavevmode\begin{itemize}
\item {} 
\sphinxAtStartPar
Thanks Dahan

\end{itemize}

\item[{HELP spelling fixes}] \leavevmode\begin{itemize}
\item {} 
\sphinxAtStartPar
Thanks Sketch

\end{itemize}

\item[{compile time issue when enhanced ansi is deselected.}] \leavevmode\begin{itemize}
\item {} 
\sphinxAtStartPar
Thanks Wisdom

\end{itemize}

\item[{if you have an ssl library that conflicts with openssl dev libs, openssl support is confused}] \leavevmode\begin{itemize}
\item {} 
\sphinxAtStartPar
Thanks Wisdom

\end{itemize}

\item[{reality\_compare \textendash{} option to alter how descs are seen}] \leavevmode\begin{itemize}
\item {} 
\sphinxAtStartPar
Thanks Derek (from SVN site)

\end{itemize}

\item[{Crash bug in @blacklist/list fixed}] \leavevmode\begin{itemize}
\item {} 
\sphinxAtStartPar
Thanks Distraida

\end{itemize}

\item[{Bug in CPU alerting with new player creation on their first connect.}] \leavevmode\begin{itemize}
\item {} 
\sphinxAtStartPar
Thanks \sphinxhref{mailto:Darren@Nightlife}{Darren@Nightlife}

\end{itemize}

\item[{Bug with compiling without BANG support with undeclared variables.}] \leavevmode\begin{itemize}
\item {} 
\sphinxAtStartPar
Thanks \sphinxhref{mailto:Darren@Nightlife}{Darren@Nightlife}

\end{itemize}

\item[{You can now input extended ASCII right into the mush (and it converts to markup)}] \leavevmode\begin{itemize}
\item {} 
\sphinxAtStartPar
Thanks mindboosternoori

\end{itemize}

\item[{functions.c bombed with the clang compiler because of restrictive type\sphinxhyphen{}casting comparisons.}] \leavevmode\begin{itemize}
\item {} 
\sphinxAtStartPar
Thanks Fraibert

\end{itemize}

\item[{@include didn’t properly null out args if forcefully specified null.}] \leavevmode\begin{itemize}
\item {} 
\sphinxAtStartPar
Thanks Volund

\end{itemize}

\item[{@skip/@ifelse, @switch, and @sudo didn’t evaluate substitutions properly.}] \leavevmode\begin{itemize}
\item {} 
\sphinxAtStartPar
Thanks Volund

\end{itemize}

\item[{Compiletime bug in speech.c with REALITY\_LEVELS not defined.}] \leavevmode\begin{itemize}
\item {} 
\sphinxAtStartPar
Thanks psc943

\end{itemize}

\end{description}


\subsection{RhostMUSH 3.9.3 Update}
\label{\detokenize{changelog:rhostmush-3-9-3-update}}\begin{description}
\item[{@include \textendash{} Idea Copied from Penn}] \leavevmode\begin{itemize}
\item {} 
\sphinxAtStartPar
Thanks Jules (and PennMUSH)

\end{itemize}

\item[{Fix for \$Z in timefmt()}] \leavevmode\begin{itemize}
\item {} 
\sphinxAtStartPar
Thanks Chime/Loki (Haunted)

\end{itemize}

\end{description}


\subsection{RhostMUSH 3.9.2 Update}
\label{\detokenize{changelog:rhostmush-3-9-2-update}}\begin{description}
\item[{@assert/@break didn’t handle \{\} correctly.}] \leavevmode\begin{itemize}
\item {} 
\sphinxAtStartPar
Thanks Wyrd

\end{itemize}

\item[{singletime() handles (w)eeks, (M)onths, and (y)ears like MUX.}] \leavevmode\begin{itemize}
\item {} 
\sphinxAtStartPar
Thanks Chime \& MUX2

\end{itemize}

\item[{crash bug in command.c with regards to mail handler}] \leavevmode\begin{itemize}
\item {} 
\sphinxAtStartPar
Thanks Chime

\end{itemize}

\item[{aliased rjust to rj and ljust to lj and updated help files appropiately.}] \leavevmode\begin{itemize}
\item {} 
\sphinxAtStartPar
Thanks Montague

\end{itemize}

\end{description}

\sphinxAtStartPar
glibc for MAX\_INT/MIN\_INT had issues with math functions \sphinxhyphen{} wrapper fixes this.
\begin{description}
\item[{parser issue with regexp, parenthesis, and backslashes \textendash{} Fixed}] \leavevmode\begin{itemize}
\item {} 
\sphinxAtStartPar
Thanks Chime

\end{itemize}

\item[{Idea for @titlecaption}] \leavevmode\begin{itemize}
\item {} 
\sphinxAtStartPar
Thanks \sphinxhref{mailto:Zero@NewJediOrder}{Zero@NewJediOrder}

\end{itemize}

\item[{IDLE had a missmatch on r and n parsing \textendash{} Fixed}] \leavevmode\begin{itemize}
\item {} 
\sphinxAtStartPar
Thanks \sphinxhref{mailto:Mike@M*U*S*H}{Mike@M*U*S*H}

\end{itemize}

\item[{Discussion/Ideas for strdistance() function.}] \leavevmode\begin{itemize}
\item {} 
\sphinxAtStartPar
Thanks \sphinxhref{mailto:Sketch@M*U*S*H}{Sketch@M*U*S*H}

\end{itemize}

\end{description}


\subsection{RhostMUSH 3.9.1 Update}
\label{\detokenize{changelog:rhostmush-3-9-1-update}}\begin{description}
\item[{cluster\_hasattr() bug for \#\sphinxhyphen{}1 that should be 0 for non\sphinxhyphen{}existant attribs.}] \leavevmode\begin{itemize}
\item {} 
\sphinxAtStartPar
Thanks \sphinxhref{mailto:Ol'Sarge@Cajun}{Ol’Sarge@Cajun}

\end{itemize}

\item[{cluster\_flags()/cluster\_hasflag() added for cluster support}] \leavevmode\begin{itemize}
\item {} 
\sphinxAtStartPar
Thanks Cody

\end{itemize}

\item[{Fix for log.c C99 compile issues on old compilers.}] \leavevmode\begin{itemize}
\item {} 
\sphinxAtStartPar
Thanks Sombranox

\end{itemize}

\item[{sees() handles optional third argument for exits}] \leavevmode\begin{itemize}
\item {} 
\sphinxAtStartPar
Thanks Sombranox

\end{itemize}

\item[{\#lambda inherited from parent wrongly \sphinxhyphen{} Fixed}] \leavevmode\begin{itemize}
\item {} 
\sphinxAtStartPar
Thanks Xandar

\end{itemize}

\item[{writable() added for TM3 compatability.}] \leavevmode\begin{itemize}
\item {} 
\sphinxAtStartPar
Thanks Wyrd

\end{itemize}

\item[{@list user\_attrib now does flag based matching}] \leavevmode\begin{itemize}
\item {} 
\sphinxAtStartPar
Thanks sombranox

\end{itemize}

\item[{cluster\_set() didn’t evaluate attributes right}] \leavevmode\begin{itemize}
\item {} 
\sphinxAtStartPar
Thanks Cody

\end{itemize}

\item[{cluster\_regrep*() didn’t select the right proper target object}] \leavevmode\begin{itemize}
\item {} 
\sphinxAtStartPar
Thanks Cody

\end{itemize}

\item[{flag setting/notifying identifies if target was set/unset before}] \leavevmode\begin{itemize}
\item {} 
\sphinxAtStartPar
Thanks PennMUSH (Kimiko)

\end{itemize}

\item[{elist() didn’t evaluate properly \sphinxhyphen{} Introduced 3.9.1p2}] \leavevmode\begin{itemize}
\item {} 
\sphinxAtStartPar
Thanks Cary

\end{itemize}

\item[{page alerts you if you’re cloaked from people you page.}] \leavevmode\begin{itemize}
\item {} 
\sphinxAtStartPar
Thanks Wyrd

\end{itemize}

\item[{Added flatfile stale attribute cleaner and penn 1.8 flatfile converter}] \leavevmode\begin{itemize}
\item {} 
\sphinxAtStartPar
Thanks Wyrd

\end{itemize}

\item[{Added switch\_search @admin param for switching functionality.}] \leavevmode\begin{itemize}
\item {} 
\sphinxAtStartPar
Thanks Minion

\end{itemize}

\item[{Added @include to Rhost}] \leavevmode\begin{itemize}
\item {} 
\sphinxAtStartPar
Thanks PennMUSH (Javelin)

\end{itemize}

\item[{@trigger can handle \$command/\textasciicircum{}listen attributes}] \leavevmode\begin{itemize}
\item {} 
\sphinxAtStartPar
Thanks TinyMUSH3/PennMUSH

\end{itemize}

\item[{error.txt files can evaluate arguments}] \leavevmode\begin{itemize}
\item {} 
\sphinxAtStartPar
Thanks Camber/Orion

\end{itemize}

\item[{mux\_lcon\_compat added to allow lcon() to return empty string.}] \leavevmode\begin{itemize}
\item {} 
\sphinxAtStartPar
Thanks Minion

\end{itemize}

\item[{cluster\_wipe added for clusters}] \leavevmode\begin{itemize}
\item {} 
\sphinxAtStartPar
Thanks Cary

\end{itemize}

\item[{NO\_CODE broke global\_error\_obj as it’s executed by enactor.}] \leavevmode\begin{itemize}
\item {} 
\sphinxAtStartPar
Thanks Matrix

\end{itemize}

\item[{Help file typos for various cluster commands.}] \leavevmode\begin{itemize}
\item {} 
\sphinxAtStartPar
Thanks \sphinxhref{mailto:Mike@M*U*S*H}{Mike@M*U*S*H}

\end{itemize}

\end{description}


\subsection{RhostMUSH 3.9.0 p00 Update}
\label{\detokenize{changelog:rhostmush-3-9-0-p00-update}}
\begin{sphinxadmonition}{note}{Note:}
\sphinxAtStartPar
v3.9 is the development branch leading to RhostMUSH v4.0
\end{sphinxadmonition}


\subsubsection{Changes}
\label{\detokenize{changelog:changes}}
\sphinxAtStartPar
Fixed some mishandled signals which should not cause Rhost to panic shutdown.

\sphinxAtStartPar
Added a Makefile define IGNORE\_SIGNALS to turn off signal handling entirely.

\sphinxAtStartPar
Changed Makefile to use RFC date (date \sphinxhyphen{}R) instead of local datestring.

\sphinxAtStartPar
Updated version() to allow showing build number by supplying an argument.

\sphinxAtStartPar
Added signal handling of SIGUSR1, SIGTERM, SIGUSR2.

\sphinxAtStartPar
Added user configurable SIGUSR1 handling through signal\_object.
\begin{description}
\item[{Added zones\_like\_parents for zones to search like @parents.}] \leavevmode\begin{itemize}
\item {} 
\sphinxAtStartPar
this does require the ZONECMDCHK @toggle set on each item/player.

\end{itemize}

\end{description}

\sphinxAtStartPar
Modify snooplog to show site information.
\begin{description}
\item[{Added \%@ support to parser (caller)}] \leavevmode\begin{itemize}
\item {} 
\sphinxAtStartPar
Penn/MUX/TM3

\end{itemize}

\item[{Added \%+ support to parser (args sent to functions)}] \leavevmode\begin{itemize}
\item {} 
\sphinxAtStartPar
Penn

\end{itemize}

\item[{Added \%? support to parser (invocations)}] \leavevmode\begin{itemize}
\item {} 
\sphinxAtStartPar
Penn

\end{itemize}

\end{description}

\sphinxAtStartPar
Added NO\_CODE flag for easier control of coding tool avaiability

\sphinxAtStartPar
Added full REGEXP support (based on Penn’s PCRE implementation)

\sphinxAtStartPar
Added some extra bounds checking to internal varabiables (non\sphinxhyphen{}issue)
\begin{description}
\item[{{[}18\sphinxhyphen{}9{]} Added compatibility with @assert and @break for Penn Compat}] \leavevmode\begin{itemize}
\item {} 
\sphinxAtStartPar
PennMUSH

\end{itemize}

\item[{Added a QUIET switch to tel()}] \leavevmode\begin{itemize}
\item {} 
\sphinxAtStartPar
Leona @ Faetopia

\end{itemize}

\end{description}

\sphinxAtStartPar
{[}18\sphinxhyphen{}9{]} Modify snooplog to show site information.

\sphinxAtStartPar
{[}18\sphinxhyphen{}10{]} lattr() modification to handle lock() shows.

\sphinxAtStartPar
{[}18\sphinxhyphen{}11{]} lattr() modification to handle attribute flag checks.

\sphinxAtStartPar
{[}18\sphinxhyphen{}11{]} examine/parent now shows dbref\#’s of the targetted items

\sphinxAtStartPar
{[}18\sphinxhyphen{}16{]} 7th argument to wrap() to allow last line to ‘left justify’
\begin{description}
\item[{{[}18\sphinxhyphen{}18{]} Added \%@ support (for Penn, TM3, and MUX compatibility)}] \leavevmode\begin{itemize}
\item {} 
\sphinxAtStartPar
Lots and Lots and Lots of people.

\end{itemize}

\item[{Merged all 3.2.4 code to 3.9.0}] \leavevmode\begin{itemize}
\item {} 
\sphinxAtStartPar
Ashen\sphinxhyphen{}Shugar

\end{itemize}

\end{description}

\sphinxAtStartPar
Added MUX, TM3, Penn, Rhost default options to asksource.sh

\sphinxAtStartPar
Modified valid ‘good characters’ for starting attribs for MUX compat.
\begin{description}
\item[{Tprintf handler for buffering rewritten to be ‘safe’}] \leavevmode\begin{itemize}
\item {} 
\sphinxAtStartPar
Lensman

\end{itemize}

\end{description}

\sphinxAtStartPar
lnum/lnum2 added to allow ‘stepping’ argument.

\sphinxAtStartPar
Reality locktypes 2\&3 could be too restrictive, so some lax was given.

\sphinxAtStartPar
Added reality locktypes 4 \& 5 to duplicate 2 \& 3 but with ‘lock\sphinxhyphen{}pass\sphinxhyphen{}nonexit’

\sphinxAtStartPar
@log allows writing to subdirectories.  5 depth maximum.
\begin{description}
\item[{@edit has /check argument to run a ‘simulation’ of an @edit.}] \leavevmode\begin{itemize}
\item {} 
\sphinxAtStartPar
Penn

\end{itemize}

\end{description}

\sphinxAtStartPar
@search has a /nogarbage switch to check against GOING/RECOVER items.

\sphinxAtStartPar
moon() takes optional second argument (boolean) to show only percentage of phase

\sphinxAtStartPar
@grep takes /regexp switch for regexp matching.

\sphinxAtStartPar
Added {]} pre\sphinxhyphen{}parser command

\sphinxAtStartPar
Rewrite of permission handler to allow a second permission structure (32 more permissions)

\sphinxAtStartPar
@edit has /single argument to allow editing first match only.

\sphinxAtStartPar
edit() has 4th optional argument to specify an edit of single match only.

\sphinxAtStartPar
Change in WANDERER and GUEST to check owner as well as target.

\sphinxAtStartPar
Added Good\_chk() macro to include Good\_obj() \&\& !Going() \&\& !Recover()

\sphinxAtStartPar
Added eval/no\_eval @admin permissions for function evaluation/non\sphinxhyphen{}evaluation.

\sphinxAtStartPar
list() given 4th argument to handle optional header.

\sphinxAtStartPar
Added /toreality switch to @pemit. (works only with contents)

\sphinxAtStartPar
Vastly improved TinyMUSH3 converter.

\sphinxAtStartPar
@hook modified to show sub\sphinxhyphen{}overrides and sub\sphinxhyphen{}includes

\sphinxAtStartPar
Added the ability to pass delim and params to self\sphinxhyphen{}made \%\sphinxhyphen{}subs.
\begin{description}
\item[{Added a new @admin boolean parameter round\_kludge.}] \leavevmode\begin{itemize}
\item {} 
\sphinxAtStartPar
Loki

\end{itemize}

\end{description}

\sphinxAtStartPar
Added support of labels for setq(), setr(), r(). (SVN 84, fix in SVN 87)

\sphinxAtStartPar
Added extra protection to garble() on exec() call for CPU (SVN 84)

\sphinxAtStartPar
Added templates (ala SETQ\_TEMPLATE) for setq labels. (SVN 89)

\sphinxAtStartPar
Enhancement to idle timeout to take into account lagging servers (SVN 93)

\sphinxAtStartPar
Center allows NON\sphinxhyphen{}ANSI multi\sphinxhyphen{}character filler {[}SVN 94{]}

\sphinxAtStartPar
Lexits() and Lcon() take optional output seperators {[}SVN 95\sphinxhyphen{}\textgreater{}96{]}

\sphinxAtStartPar
Lexits and Lcon() take arguments to turn \#dbrefs into names {[}SVN 101\sphinxhyphen{}\textgreater{}102{]}

\sphinxAtStartPar
Added /notify switch to @switch {[}SVN 108{]}

\sphinxAtStartPar
Added additional arguments to dice() {[}SVN 110{]}

\sphinxAtStartPar
Made v() allow registers 11 through MAX\sphinxhyphen{}ARGS {[}SVN 111{]}

\sphinxAtStartPar
Made strfunc() aware of bypass() {[}SVN 112{]}

\sphinxAtStartPar
Mail now displays a ‘m’ by any message set for auto\sphinxhyphen{}deletion {[}SVN 117{]}

\sphinxAtStartPar
map() now allows ArgN arguments to be passed to it as \%1\sphinxhyphen{}\%?? {[}SVN 118{]}

\sphinxAtStartPar
Added time skew detection and protection to readjust mush time {[}SVN 123{]}

\sphinxAtStartPar
Added some addition protection for network flooding {[}SVN 123{]}

\sphinxAtStartPar
Added optional arguments to dig() and open() for location/returns {[}SVN 123{]}

\sphinxAtStartPar
Enhanced door code to be useable {[}SVN 125{]}

\sphinxAtStartPar
Modified asksource.sh script to handle all new options {[}SVN 125{]}

\sphinxAtStartPar
Added MUX2 password compatibility for reading converted databases {[}SVN 125{]}

\sphinxAtStartPar
Added beta option support for MySQL, Doors, and 64bit SBUF/attribs {[}SVN 125{]}

\sphinxAtStartPar
Added auto\sphinxhyphen{}detection of 64 bit platforms {[}SVN 125{]}

\sphinxAtStartPar
Added MAILFILTER attribute to change destination of incoming mail {[}SVN 127{]}
\begin{description}
\item[{Fixed some uninitialized variables in view\_atr {[}SVN 129{]}}] \leavevmode\begin{itemize}
\item {} 
\sphinxAtStartPar
Loki

\end{itemize}

\item[{Altered case() and caseall() to support \#\$ substitution like switch() {[}SVN 130{]}}] \leavevmode\begin{itemize}
\item {} 
\sphinxAtStartPar
Loki

\end{itemize}

\item[{Slight mistake in the case()/caseall() fix in 130. Fixed {[}SVN 136{]}}] \leavevmode\begin{itemize}
\item {} 
\sphinxAtStartPar
Loki

\end{itemize}

\item[{QDBM support added as an optional database layer {[}SVN 131\sphinxhyphen{}135{]}}] \leavevmode\begin{itemize}
\item {} 
\sphinxAtStartPar
Ambrosia

\end{itemize}

\item[{Error() accepts an optional player argument to base error messages on.}] \leavevmode\begin{itemize}
\item {} 
\sphinxAtStartPar
Thanks Ratio (and many others)

\end{itemize}

\item[{FIX\_AIX obsoleted by a small mail.c/mailfix.c rewrute}] \leavevmode\begin{itemize}
\item {} 
\sphinxAtStartPar
Ashen {[}SVN 162{]}

\end{itemize}

\end{description}

\sphinxAtStartPar
modified sub\_include so that the CHR\_\textless{}str\textgreater{} value if fed an integer specifies how many values to take.
\begin{description}
\item[{Copy/paste error lead to @break behaving like @assert. {[}SVN 168{]}}] \leavevmode\begin{itemize}
\item {} 
\sphinxAtStartPar
Loki

\end{itemize}

\item[{Removed minimal\_db from the distro. It can be found on the download site under Contrib. {[}SVN 171{]}}] \leavevmode\begin{itemize}
\item {} 
\sphinxAtStartPar
LOKI

\end{itemize}

\end{description}

\sphinxAtStartPar
timefmt(), convsecs(), convtime() and moon() now handle 64 bit time. {[}SVN 172{]}

\sphinxAtStartPar
list() has an optional target player for output (must control target) {[}SVN 177{]}

\sphinxAtStartPar
garble() has new optional type value to return character count instead of string {[}SVN 177{]}

\sphinxAtStartPar
remit() has optional key value to specify if it’s a to\sphinxhyphen{}reality output or normal remit() {[}SVN 177{]}

\sphinxAtStartPar
/DISPLAY switch for @function to display details on target function. {[}SVN 177{]}

\sphinxAtStartPar
/LIST switch for @function allows wildcarding. {[}SVN 177{]}

\sphinxAtStartPar
@list buffers no longer spams the living crap out of you. {[}SVN 177{]}
\begin{description}
\item[{Sideeffects: Tidied up the wizhelp entries and added example page Allowed for keyword negation (i.e. sideeffects = PENN !OPEN MOVE). {[}SVN 178\sphinxhyphen{}179{]}}] \leavevmode\begin{itemize}
\item {} 
\sphinxAtStartPar
Lensman

\end{itemize}

\end{description}

\sphinxAtStartPar
strip() takes optional 3rd argument to specify allowing only specified characters {[}SVN 180{]}

\sphinxAtStartPar
accents are now allowed as a markup language via the \%f substitution.  Requires ZENTY\_ANSI {[}SVN 180{]}

\sphinxAtStartPar
printf() modified with ‘\&’ identifier for carrage return alignment processing {[}SVN 181{]}

\sphinxAtStartPar
filter() can now take multiple arguments. {[}SVN 190{]}

\sphinxAtStartPar
/STOP and /CONT switches added to @halt.  {[}SVN 192{]}

\sphinxAtStartPar
@wait/pid modified to be much nicer on the queue {[}SVN 192{]}

\sphinxAtStartPar
chr() allows characters 160 to 255 to be done via markups (\%\textless{}3digit\textgreater{})

\sphinxAtStartPar
Cleaned up extension characters by integrating into the accent parser {[}SVN 201{]}
\begin{description}
\item[{On rooms @toggled LOGROOM, if there is a LOGNAME attribute on the room, it uses that instead {[}SVN 204{]}}] \leavevmode\begin{itemize}
\item {} 
\sphinxAtStartPar
Thanks Ixokai

\end{itemize}

\end{description}

\begin{sphinxadmonition}{note}{Note:}
\sphinxAtStartPar
Name must be alphanumeric and be under 70 characters in length.
\end{sphinxadmonition}

\sphinxAtStartPar
\%q\textless{}variable\textgreater{} added for TM3 compatability with named labels {[}SVN 205{]}

\sphinxAtStartPar
setq/setr family now allow ! to asign next available register or re\sphinxhyphen{}assign existing register {[}SVN 208{]}

\sphinxAtStartPar
nameq optionally displays register asigned to label {[}SVN 208{]}

\sphinxAtStartPar
the PID of the running process now stored to netrhost.pid {[}SVN 208{]}

\sphinxAtStartPar
Signal handling improved for USR1 and USR2 {[}SVN 212{]}

\sphinxAtStartPar
/oneeval switch to @pemit for single evaluation of a @pemit/list {[}SVN 213{]}

\sphinxAtStartPar
lexits() allow page listings {[}SVN 233{]}

\sphinxAtStartPar
lrooms() \sphinxhyphen{} fix for specific room identification {[}SVN 233{]}

\sphinxAtStartPar
@function/display now shows flags of functions {[}SVN 233{]}

\sphinxAtStartPar
rework lattr() and lattrp() as well as cluster\_lattr() to use centralized core functionality {[}SVN 233{]}

\sphinxAtStartPar
rework of ueval() to handle cluster or non\sphinxhyphen{}cluster {[}SVN 233{]}

\sphinxAtStartPar
lock() allows optional third argument to set/clear attribute locks {[}SVN 233{]}

\sphinxAtStartPar
merging of all cluster\_Defaults into single handler {[}SVN 233{]}

\sphinxAtStartPar
action lists attached to \textgreater{}, @cluster/set, and cluster\_set() {[}SVN 233{]}
\begin{description}
\item[{@wait with +/\sphinxhyphen{} values {[}SVN 233{]}}] \leavevmode\begin{itemize}
\item {} 
\sphinxAtStartPar
Thanks Raevnos

\end{itemize}

\end{description}

\sphinxAtStartPar
BOUNCEFORWARD attribute parser for BOUNCE flag {[}SVN 233{]}

\sphinxAtStartPar
VATTRCNT() has optional 2nd argument to reset the vlimit on the dbref\# {[}SVN 244{]}
\begin{description}
\item[{pgrep()/cluster\_grep() has additional arguments for outputting matches {[}SVN 244{]}}] \leavevmode\begin{itemize}
\item {} 
\sphinxAtStartPar
Thanks Cary

\end{itemize}

\end{description}

\sphinxAtStartPar
Updated help for printf() {[}SVN 253{]}
\begin{description}
\item[{Fix for OSX 10.5.2 with regards to more restrictive header include files {[}SVN 254{]}}] \leavevmode\begin{itemize}
\item {} 
\sphinxAtStartPar
Thanks Lyoc

\end{itemize}

\end{description}

\begin{sphinxadmonition}{note}{Note:}
\sphinxAtStartPar
TO BE DONE: add python interpreter
TO BE DONE: @plugin
\end{sphinxadmonition}


\subsubsection{Additions}
\label{\detokenize{changelog:additions}}

\paragraph{Functions}
\label{\detokenize{changelog:functions}}
\sphinxAtStartPar
{[}18\sphinxhyphen{}13{]} Add isint() for MUX/Penn compatability
{[}18\sphinxhyphen{}17{]} Added fmod() for Penn compatability.
regmatch()      \sphinxhyphen{} regexp (penn)
regmatchi()     \sphinxhyphen{} regexp (penn)
regedit()       \sphinxhyphen{} regexp (penn)
regediti()      \sphinxhyphen{} regexp (penn)
regeditall()    \sphinxhyphen{} regexp (penn)
regeditalli()   \sphinxhyphen{} regexp (penn)
reswitch()      \sphinxhyphen{} regexp (penn)
reswitchi()     \sphinxhyphen{} regexp (penn)
reswitchall()   \sphinxhyphen{} regexp (penn)
reswitchalli()  \sphinxhyphen{} regexp (penn)
regrab()        \sphinxhyphen{} regexp (penn)
regrabi()       \sphinxhyphen{} regexp (penn)
regraball()     \sphinxhyphen{} regexp (penn)
regraballi()    \sphinxhyphen{} regexp (penn)
regrep()        \sphinxhyphen{} regexp (penn)
regrepi()       \sphinxhyphen{} regexp (penn)
bypass()        \sphinxhyphen{} used in @functions to bypass hardcoded limits
logtofile()     \sphinxhyphen{} Function equiv of @log \sphinxhyphen{} limit 1/command.
searchng()      \sphinxhyphen{} Like search() but don’t return GOING/RECOVER.
keepflags()     \sphinxhyphen{} keep data type based on flags
remflags()      \sphinxhyphen{} remove data type based on flags
foldercurrent)  \sphinxhyphen{} list current folder or share folder
folderlist()    \sphinxhyphen{} list all folders of target
nameq()         \sphinxhyphen{} name/rename labels for setq registers (SVN 84)
spellnum()      \sphinxhyphen{} Penn compatable returns full name {[}SVN 125{]}
ibreak()        \sphinxhyphen{} for iter() and list() {[}SVN 109{]}
shift()         \sphinxhyphen{} and additional arguments to dice() {[}SVN 110{]}
fbound()        \sphinxhyphen{} like bound but for floating point {[}SVN 137{]}
fbetween()      \sphinxhyphen{} like between but for floating point {[}SVN 137{]}
strdistance()   \sphinxhyphen{} Levenshtein Distance between two strings {[}SVN 146{]}
tr()            \sphinxhyphen{} Transform string based on find/replace {[}SVN 156{]}
digest()        \sphinxhyphen{} SSL algorithm encryptions.  Yay is life. {[}SVN 158{]}
roman()         \sphinxhyphen{} Number to Roman Numeral converter {[}SVN 160{]}
printf()        \sphinxhyphen{} Printf() moved to the mush!  Yay is life. {[}SVN 177{]}
accent()        \sphinxhyphen{} Allow accent() markup characters specified by string {[}SVN 180{]}
stripaccents()  \sphinxhyphen{} Strip the accents on the target string {[}SVN 180{]}
pid()           \sphinxhyphen{} Show pid processes of anything you control {[}SVN 192{]}
pgrep()         \sphinxhyphen{} grep for parents {[}SVN 233{]}
cluster\_u()             \sphinxhyphen{} cluster u() {[}SVN 233{]}
cluster\_u2()            \sphinxhyphen{} cluster u2() {[}SVN 233{]}
cluster\_ulocal()        \sphinxhyphen{} cluster ulocal() {[}SVN 233{]}
cluster\_u2local()       \sphinxhyphen{} cluster u2local() {[}SVN 233{]}
cluster\_uldefault()     \sphinxhyphen{} cluster uldefault() {[}SVN 233{]}
cluster\_u2ldefault()    \sphinxhyphen{} cluster u2ldefault() {[}SVN 233{]}
cluster\_udefault()      \sphinxhyphen{} cluster udefault() {[}SVN 233{]}
cluster\_u2default()     \sphinxhyphen{} cluster u2default() {[}SVN 233{]}
cluster\_grep()          \sphinxhyphen{} cluster grep() {[}SVN 233{]}
cluster\_regrep()        \sphinxhyphen{} cluster regrep() {[}SVN 233{]}
cluster\_regrepi()       \sphinxhyphen{} cluster regrepi() {[}SVN 233{]}
cluster\_hasattr()       \sphinxhyphen{} cluster hasattr() {[}SVN 233{]}
cluster\_lattr()         \sphinxhyphen{} cluster lattr() {[}SVN 233{]}
cluster\_attrcnt()       \sphinxhyphen{} cluster attrcnt() {[}SVN 233{]}
cluster\_vattrcnt()      \sphinxhyphen{} cluster vattrcnt() {[}SVN 233{]}
cluster\_get()           \sphinxhyphen{} cluster get() {[}SVN 233{]}
cluster\_xget()          \sphinxhyphen{} cluster xget() {[}SVN 233{]}
cluster\_set()           \sphinxhyphen{} cluster set() {[}SVN 233{]}
cluster\_default         \sphinxhyphen{} cluster default() {[}SVN 233{]}
cluster\_edefault        \sphinxhyphen{} cluster edefault() {[}SVN 233{]}
cluster\_stats()         \sphinxhyphen{} specifies statistics on cluster {[}SVN 233{]}
cluster\_get\_eval()      \sphinxhyphen{} cluster get\_eval() {[}SVN 233{]}
cluster\_ueval()         \sphinxhyphen{} cluster ueval() {[}SVN 233{]}


\paragraph{Commands}
\label{\detokenize{changelog:commands}}
\sphinxAtStartPar
@assert \sphinxhyphen{} Works like @break but reverse logic {[}18\sphinxhyphen{}9{]}
idle \sphinxhyphen{} a ‘nothing’ command that won’t effect idle time.  Wizzes can optionally add command.
\textgreater{} for cluster sets (instead of \&) {[}SVN 233{]}
@cluster/new \textless{}dbref\textgreater{}                    \textendash{} won’t work on cluster object {[}SVN 233{]}
@cluster/add \textless{}dbref\textgreater{}=\textless{}dbref\textgreater{}            \textendash{} won’t add a pre\sphinxhyphen{}existing cluster object {[}SVN 233{]}
@cluster/del \textless{}dbref\textgreater{}                    \textendash{} won’t delete a non\sphinxhyphen{}existing cluster object {[}SVN 233{]}
@cluster/clear \textless{}dbref\textgreater{}                  \textendash{} purges the cluster list {[}SVN 233{]}
@cluster/list \textless{}dbref\textgreater{}                   \textendash{} lists objects in cluster, total attributes, threshold, and action list {[}SVN 233{]}
@cluster/threshhold \textless{}dbref\textgreater{}=\textless{}number\textgreater{}    \textendash{} sets a threshold on the cluster {[}SVN 233{]}
@cluster/action \textless{}dbref\textgreater{}=\textless{}string\textgreater{}        \textendash{} sets action when treshhold is met {[}SVN 233{]}
@cluster/edit \textless{}dbref\textgreater{}=\textless{}string\textgreater{},\textless{}string\textgreater{} \textendash{} edits the attribute(s) in a cluster {[}SVN 233{]}
@cluster/set                            \textendash{} @set for clusters {[}SVN 233{]}
@cluster/repair                         \textendash{} repairs a damaged cluster (or tries to) {[}SVN 233{]}
@cluster/grep                           \textendash{} grep for a string in a cluster {[}SVN 233{]}
@cluster/reaction                       \textendash{} edit for actions {[}SVN 233{]}
@cluster/cut                            \textendash{} physically cuts the item from a cluster \sphinxhyphen{} only use as last resort {[}SVN 233{]}
@cluster/trigger                        \textendash{} trigger action on cluster {[}SVN 233{]}


\paragraph{Flags}
\label{\detokenize{changelog:flags}}
\sphinxAtStartPar
NO\_CODE  \sphinxhyphen{} Controls usage of code\sphinxhyphen{}commandss/functions.  Uses new second permission table.


\paragraph{Toggles}
\label{\detokenize{changelog:toggles}}
\sphinxAtStartPar
HIDEIDLE \sphinxhyphen{} Stops the ‘idle’ from being updated as well as command count.  Wiz+ only.
MORTALREALITY \sphinxhyphen{} disables the inherit ability of wizards for wiz\_always\_real {[}SVN 126{]}
\begin{description}
\item[{MAILVALIDATE \sphinxhyphen{} validate the user list and abort mail without sending to anyone if invalid.}] \leavevmode\begin{itemize}
\item {} 
\sphinxAtStartPar
Thanks \sphinxhref{mailto:Charlotte@Cajun}{Charlotte@Cajun}

\end{itemize}

\end{description}

\sphinxAtStartPar
CLUSTER \sphinxhyphen{} (internal toggle to handle cluster objects)


\subsubsection{Admin Params}
\label{\detokenize{changelog:admin-params}}
\sphinxAtStartPar
{[}18\sphinxhyphen{}9{]} Added zones\_like\_parents for zones to search like @parents.  \sphinxhyphen{} this does require the ZONECMDCHK @toggle set on each item/player.
log\_maximum \sphinxhyphen{} specify the number of logtofile() calls allowed per command.
power\_objects \sphinxhyphen{} enable @power/@depower checks on non\sphinxhyphen{}players
rooms\_can\_open \sphinxhyphen{} allow rooms to be able to @open exits inside itself
sub\_override \sphinxhyphen{} specify what \%\sphinxhyphen{}subs can be overridden
\begin{description}
\item[{sub\_include \sphinxhyphen{} specify what \%\sphinxhyphen{}subs you want added}] \leavevmode\begin{itemize}
\item {} 
\sphinxAtStartPar
Ambrosia

\end{itemize}

\item[{signal\_object \sphinxhyphen{} Object containing signal handling attribute for SIGUSR1.}] \leavevmode\begin{itemize}
\item {} 
\sphinxAtStartPar
Odin

\end{itemize}

\end{description}

\sphinxAtStartPar
break\_compatibility \sphinxhyphen{} @break/@assert allowed/disallowed double\sphinxhyphen{}evaluation (disabled by default)
signal\_object\_type \sphinxhyphen{} Type of execution to do.  Default is ‘0’ (function only)’.  1 is a @trigger effect. {[}SVN 212{]}
log\_network\_errors      \sphinxhyphen{} enables (or disables) logging of network errors on sockets


\subsubsection{Bug Fixes}
\label{\detokenize{changelog:bug-fixes}}\begin{description}
\item[{SORTBY was mangling the enactor.}] \leavevmode\begin{itemize}
\item {} 
\sphinxAtStartPar
Melpomine @ Vieux Carre

\end{itemize}

\item[{{[}18\sphinxhyphen{}7{]} THIS IS AN UNOFFICIAL PATCH.  This fixes a Reality Level Lock issue.}] \leavevmode\begin{itemize}
\item {} 
\sphinxAtStartPar
Thanks Ixokai

\end{itemize}

\end{description}

\sphinxAtStartPar
{[}18\sphinxhyphen{}8{]} Fixes a vprintf allocation issue.
\begin{description}
\item[{{[}18\sphinxhyphen{}9{]} Fixed a logical error with unfindable and the connect flag.}] \leavevmode\begin{itemize}
\item {} 
\sphinxAtStartPar
Thanks Xandar

\end{itemize}

\end{description}

\sphinxAtStartPar
{[}18\sphinxhyphen{}12{]} Fix double eval to @break/@assert to mimic Penn
{[}18\sphinxhyphen{}13{]} Fixed the convert scripts to handle irrigular behavior in TM3 flatfile
{[}18\sphinxhyphen{}14{]} Fix for LBUF free in elist()
{[}18\sphinxhyphen{}15{]} Fix for permission issue with regards to no\_examine and attrib fetches
{[}18\sphinxhyphen{}15{]} Fix for possible array out\sphinxhyphen{}of\sphinxhyphen{}bounds with regards to backstage
{[}18\sphinxhyphen{}16{]} Fix in wrap() for wrap\_out code with possible overrun (non\sphinxhyphen{}crash)

\begin{sphinxadmonition}{note}{Note:}
\sphinxAtStartPar
this effected both wrap() and wrapcolumns()
\end{sphinxadmonition}
\begin{description}
\item[{{[}18\sphinxhyphen{}17{]} Fix for moon() on displaying waxing/waning exact matches.}] \leavevmode\begin{itemize}
\item {} 
\sphinxAtStartPar
Thanks Jeff

\end{itemize}

\end{description}

\sphinxAtStartPar
{[}18\sphinxhyphen{}17{]} asksource.sh (make config/make confsource) fixed with (l)oad issue.
\begin{description}
\item[{{[}18\sphinxhyphen{}17{]} Fix for news/verbose switch that broke previous workings \sphinxhyphen{} Fixed}] \leavevmode\begin{itemize}
\item {} 
\sphinxAtStartPar
Thanks Ambrosia

\end{itemize}

\item[{{[}18\sphinxhyphen{}18{]} Fix for soundex() with a buffering issue (non\sphinxhyphen{}crashing) \sphinxhyphen{} Fixed}] \leavevmode\begin{itemize}
\item {} 
\sphinxAtStartPar
Thanks Jeff

\end{itemize}

\end{description}

\sphinxAtStartPar
Bug in (d)elete option in asksource script \sphinxhyphen{} Fixed
\begin{description}
\item[{Bug in ‘home’ with reality level following \sphinxhyphen{} Fixed}] \leavevmode\begin{itemize}
\item {} 
\sphinxAtStartPar
Thanks Ol’Sarge @ Cajun

\end{itemize}

\item[{Bug in \sphinxhref{mailto:'@idesc}{‘@idesc}’ with reality level if ‘Real’ level not using \sphinxhref{mailto:'@desc}{‘@desc}’ \sphinxhyphen{} Fixed}] \leavevmode\begin{itemize}
\item {} 
\sphinxAtStartPar
Thanks Toby @ Cajun

\end{itemize}

\item[{Bug in \sphinxhref{mailto:'@rsrvdesc2}{‘@rsrvdesc2}’ if target desc is \sphinxhref{mailto:'@idesc}{‘@idesc}’. \sphinxhyphen{} Fixed}] \leavevmode\begin{itemize}
\item {} 
\sphinxAtStartPar
Thanks Toby @ Cajun

\end{itemize}

\item[{Bug in double\sphinxhyphen{}free if default globals enabled \sphinxhyphen{} Fixed}] \leavevmode\begin{itemize}
\item {} 
\sphinxAtStartPar
Thanks Ixokai

\end{itemize}

\item[{Make config/confsource for option 15 and option 9 didn’t escape the ‘\$’. \sphinxhyphen{} Fixed}] \leavevmode\begin{itemize}
\item {} 
\sphinxAtStartPar
Thanks Odin

\end{itemize}

\end{description}

\sphinxAtStartPar
Small issue in mail code where if max\sphinxhyphen{}index is reached mail won’t send \sphinxhyphen{} Fixed
\begin{description}
\item[{Bug in do\_dbck with db loading/startup/reboot \sphinxhyphen{} small chance of crash. \sphinxhyphen{}Fixed}] \leavevmode\begin{itemize}
\item {} 
\sphinxAtStartPar
Thanks Odin

\end{itemize}

\item[{Feature bug in read\_remote\_name.  Didn’t check examinable \sphinxhyphen{} Fixed}] \leavevmode\begin{itemize}
\item {} 
\sphinxAtStartPar
Ambrosia

\end{itemize}

\item[{Hook ‘fail’ didn’t work in hook\_cmd.  \sphinxhyphen{} Fixed}] \leavevmode\begin{itemize}
\item {} 
\sphinxAtStartPar
Rook

\end{itemize}

\item[{Bug in mail.c where it could effect @nuking players \sphinxhyphen{} fixed}] \leavevmode\begin{itemize}
\item {} 
\sphinxAtStartPar
Odin

\end{itemize}

\item[{Bug in news.c where it could effect @nuking players \sphinxhyphen{} fixed}] \leavevmode\begin{itemize}
\item {} 
\sphinxAtStartPar
Odin

\end{itemize}

\item[{Bug in @quota where buffer was mistakenly freed prior to display \sphinxhyphen{} Fixed}] \leavevmode\begin{itemize}
\item {} 
\sphinxAtStartPar
Ambrosia/Ashen

\end{itemize}

\item[{Issue with page/port and the MUXPAGE toggle. Resolved, page/port now cannot be used in combination with MUXPAGE}] \leavevmode\begin{itemize}
\item {} 
\sphinxAtStartPar
Lensman

\end{itemize}

\item[{Several missing helpfile entries added: isupper/islower, brace\_compatibility, format\_compatibility}] \leavevmode
\sphinxAtStartPar
\sphinxhyphen{}Ambrosia

\item[{Bug in speech.c when compiling on some flavors of BSD fixed. Also cleaned some warnings.}] \leavevmode\begin{itemize}
\item {} 
\sphinxAtStartPar
Odin

\end{itemize}

\item[{Fixed a bug in cque.c relating to queue accounting on exceeding player\_queue\_max.}] \leavevmode\begin{itemize}
\item {} 
\sphinxAtStartPar
Odin, fixed and found by Brazil.

\end{itemize}

\item[{Fixed an issue with certain locks with displaying attribute names in examine. NON\sphinxhyphen{}CRITICAL\sphinxhyphen{}ISSUE}] \leavevmode\begin{itemize}
\item {} 
\sphinxAtStartPar
Ixokai

\end{itemize}

\item[{Fixed a strip issue of \{\}’s with regards to ‘{]}’.  The {]} command shouldn’t strip anything.}] \leavevmode\begin{itemize}
\item {} 
\sphinxAtStartPar
Odin \& Ambrosia

\end{itemize}

\end{description}

\sphinxAtStartPar
Fixed a problem with 8\sphinxhyphen{}bit dates being passed to the compiler as build time. We now use date \sphinxhyphen{}R for RFC time.
Fixed a problem with examine on targets you didn’t control.
Fixed problem with @kick and the queue.
Fixed buffer clobber issue in lloc() (introduce 3.9)
Fixed logic error of command pathing with ignore on ‘goto’ command.
Fixed ‘N’ command from showing up. (for {]} alias)
Fixed display for percent subs with regards to @hook.
\begin{description}
\item[{Fixed wrapcolumns() where it could crash on unique strings.}] \leavevmode\begin{itemize}
\item {} 
\sphinxAtStartPar
Thanks melkir

\end{itemize}

\end{description}

\sphinxAtStartPar
Fixed a bug with flag\_name and possible duplicated entries
Fixed chomp() to handle \%r conditions cleaner.
\begin{description}
\item[{sub\_include had a possible unitialized condition for \%0\sphinxhyphen{}\%9 on rare occurances. (SVN\sphinxhyphen{}87)}] \leavevmode\begin{itemize}
\item {} 
\sphinxAtStartPar
Thanks Loki

\end{itemize}

\end{description}

\sphinxAtStartPar
crash bug in lcon/lexit additions \sphinxhyphen{} fixed (SVN\sphinxhyphen{}105)
Finally believe the output snuffing issue is fixed (SVN\sphinxhyphen{}105)
SIGSYS and some other signals were mistakenly handled as panic shutdown.
\begin{description}
\item[{setq()/setr() didn’t evaluate first argument.  Introduced SVN 87 {[}SVN 113{]}}] \leavevmode\begin{itemize}
\item {} 
\sphinxAtStartPar
Thanks Melkir

\end{itemize}

\end{description}

\sphinxAtStartPar
Somehow @list options values got broken by a typo.  Weird.  Fixed. {[}SVN 114{]}
Tprintf off by one issue fixed for buffer allocation {[}SVN 115 \sphinxhyphen{} SVN 116{]}
Off by one bug in map() from SVN 118.  Non crash, just argument list off {[}SVN 119{]}
Crash bug in ATRLOCK if attribute did not previously exist in hash lookup {[}SVN 120{]}
Realities had a logical bug.  Now previous (broke) behavior available as option {[}SVN 122\sphinxhyphen{}124{]}
Fixed a bug in Reality level handling and CHKREALITY Toggle {[}SVN 125{]}
Fixed uninitalized variable with regards to dig()/@dig.  Introduced SVN 123 {[}SVN 128{]}
Fix for QDBM compiles on aborting on the ‘rm’ section.  Introduced SVN 136 {[}SVN 137\sphinxhyphen{}138{]}
Critical fix for time skew on AMD’s.  Infinite loop.  Introduced SVN 123 {[}SVN 139{]}
Fix for sha1.c and mushcrypt.c for compiling issues.  MUX2 provided code.  {[}SVN 140{]}
Forgot to update this bloody file {[}SVN 141{]}
Fix for null entries deidling a player.  It was intended behavior, but this could be confusing so was fixed.  It still will de\sphinxhyphen{}cloak an idle\sphinxhyphen{}cloaked wiz. {[}SVN 142{]}
Fix for timeskew affecting internal timers, such as database dumping. {[}SVN 143{]}
Fixed a bad define for 64\sphinxhyphen{}bit autodetection. {[}SVN 143{]}
\begin{description}
\item[{Fix for timeskew affecting global timers. Introduced SVN 140. {[}SVN 144{]} (Rehash: SVN 151)}] \leavevmode\begin{itemize}
\item {} 
\sphinxAtStartPar
Thanks Ratio

\end{itemize}

\end{description}

\sphinxAtStartPar
Clean up code a bit with implicit declarations {[}SVN 145{]}
\begin{description}
\item[{Bug in signal handling.  If during a dump, it happened during an alarm() state, alarms would be ignored and as such the timer would be broke from that point on.  Fix causes signal restore to re\sphinxhyphen{}set the alarm state.}] \leavevmode\begin{itemize}
\item {} 
\sphinxAtStartPar
Thanks Ratio

\end{itemize}

\item[{spellnum() had a comparision bug with ‘tens’ digit.  {[}SVN 152{]}}] \leavevmode\begin{itemize}
\item {} 
\sphinxAtStartPar
Thanks Ratio

\end{itemize}

\item[{signal handling interferred with @reboots randomly (Introduced 3.9 SVN 100) {[}SVN 153{]}}] \leavevmode\begin{itemize}
\item {} 
\sphinxAtStartPar
Thanks Ratio

\end{itemize}

\end{description}

\sphinxAtStartPar
Crashbug in RANDMATCH() (Introduced 3.9) {[}SVN 157{]}
Missing free\_lbuf() in caseall() (Introduced SVN 136) {[}SVN 157{]}
Fix for QDBM and mail if the mail db can’t load correctly. {[}SVN 161{]}
\begin{description}
\item[{Fix for @break/@assert to stop double\sphinxhyphen{}eval of first argument {[}SVN 164{]}}] \leavevmode\begin{itemize}
\item {} 
\sphinxAtStartPar
Thanks Brazil

\end{itemize}

\item[{Fix for all of the *default() functions evaluating default behavior all the time.  It really probably shouldn’t have done that.  {[}SVN 166{]}}] \leavevmode\begin{itemize}
\item {} 
\sphinxAtStartPar
Thanks Orion

\end{itemize}

\end{description}

\sphinxAtStartPar
Wouldn’t you know it, the *default() fix broke some backward compatability (introduced SVN 166).  Fixed {[}SVN 167{]}
Yet another fix but this time to evaluate arguments to all the *u*default() functions (intro SVN 166) Fixed {[}SVN 168{]}
Fix for ptimefmt as the structure declaration was missing an argument.  Bad mojo. \sphinxhyphen{} Fixed {[}SVN 174{]}
Rhost’s hardcoded news system failed to free an lbuf on subscribe checks \sphinxhyphen{} Fixed {[}SVN 177{]}
creplace didn’t evaluate its second arguments and some syntax issues. \sphinxhyphen{} Fixed {[}SVN 183{]}
Tweeking of unsafe\_tprintf to safe\_printf in some locations where heavy usage is seen {[}SVN 190{]}
mail/reply with the the all option wouldn’t work if targets had spaces in names. {[}SVN 194{]}
vsprintf didn’t like the double \%\% for statically stating \% in certain libraries {[}SVN 195{]}
translate() didn’t take \%0\sphinxhyphen{}\%9 arguments {[}SVN 196{]}
citer() had a double lbuf free on replace\_tokens {[}SVN 198{]}
\begin{description}
\item[{@whereis/@whereall didn’t hide the location if user was hidden/dark correctly {[}SVN 202{]}}] \leavevmode\begin{itemize}
\item {} 
\sphinxAtStartPar
thanks Planet X

\end{itemize}

\end{description}

\sphinxAtStartPar
@mvattr/@cpattr should handle SBUF sized attribute names cleaner. \sphinxhyphen{} Fixed {[}SVN 207{]}
\begin{description}
\item[{Help topic for @assert reversered \sphinxhyphen{} Fixed {[}SVN 209{]}}] \leavevmode\begin{itemize}
\item {} 
\sphinxAtStartPar
Bug ID 0000004 (BlackRook)

\end{itemize}

\item[{\%q\textless{}label\textgreater{} broke with trace conditions.  Introduced SVN 108 \sphinxhyphen{} Fixed {[}SVN 209{]}}] \leavevmode\begin{itemize}
\item {} 
\sphinxAtStartPar
Thanks Melpomine

\end{itemize}

\item[{bang support didn’t work for overloaded functions.  Fixed {[}SVN 210{]}}] \leavevmode\begin{itemize}
\item {} 
\sphinxAtStartPar
Thanks Melpomine

\end{itemize}

\end{description}

\sphinxAtStartPar
SIGUSR1 would ignore the signal after use at times.  Fixed {[}SVN 212{]}
Possible SIGSEGV with argument preprocessing on certain commands.  Fixed {[}SVN 213{]}
\begin{description}
\item[{Fix for SIGALRM being ignored when IGNORE\_SIGNALS compile time option enabled. fixed {[}SVN 225{]}}] \leavevmode\begin{itemize}
\item {} 
\sphinxAtStartPar
Thanks grump

\end{itemize}

\item[{Fix for map()/filter() for argument list passed.  Fixed {[}SVN 227{]}}] \leavevmode\begin{itemize}
\item {} 
\sphinxAtStartPar
Thanks Ratio

\end{itemize}

\item[{Help not documented for fill character in ljc()/rjc() \sphinxhyphen{} Fixed {[}SVN 228{]}}] \leavevmode\begin{itemize}
\item {} 
\sphinxAtStartPar
Thanks Cary

\end{itemize}

\item[{OUTPUTPREFIX/OUTPUTSUFFIX stored through reboots \sphinxhyphen{} Fixed {[}SVN 233{]}}] \leavevmode\begin{itemize}
\item {} 
\sphinxAtStartPar
Thanks Adrick

\end{itemize}

\item[{Added missing topics for clustering \sphinxhyphen{} Fixed {[}SVN 244{]}}] \leavevmode\begin{itemize}
\item {} 
\sphinxAtStartPar
Thanks Cary

\end{itemize}

\end{description}

\sphinxAtStartPar
Logging network errors could cause heavy logging on really bad networks \sphinxhyphen{} Fixed {[}SVN 245{]}
\begin{description}
\item[{Connect honors @hide/unfindable/dark {[}SVN 320{]}}] \leavevmode\begin{itemize}
\item {} 
\sphinxAtStartPar
Thanks Planet X

\end{itemize}

\end{description}


\subsection{RhostMUSH 3.2.4 p18 Update}
\label{\detokenize{changelog:rhostmush-3-2-4-p18-update}}
\sphinxAtStartPar
{[}06/28/2004{]}


\subsubsection{Changes}
\label{\detokenize{changelog:id1}}
\sphinxAtStartPar
@aflags takes /full switch to give count on how many objects have the attribute.  useful incase you have concerns of attributes dissapearing.
\begin{description}
\item[{Added queue\_compatible config option to allow negative decremental of the semaphore attribute for elsemush compatibility}] \leavevmode\begin{itemize}
\item {} 
\sphinxAtStartPar
Lyle, Saffron, Dragon @ Paris

\end{itemize}

\item[{Added \sphinxhyphen{}DPARIS compile time option for older Penn/MUSH format compatibility}] \leavevmode\begin{itemize}
\item {} 
\sphinxAtStartPar
Dragon @ Paris

\end{itemize}

\item[{Added \sphinxhyphen{}OLD\_SETQ compile time option to switch setq/setq\_old compatibility}] \leavevmode\begin{itemize}
\item {} 
\sphinxAtStartPar
Thanks Lyle, Saffron, Dragon @ Paris

\end{itemize}

\item[{In all source, header, help, and readme files, ‘compatable’ renamed to ‘compatible’}] \leavevmode\begin{itemize}
\item {} 
\sphinxAtStartPar
Ambrosia

\end{itemize}

\item[{Enhanced mail/recall to show more stats (lots of people wanted them)}] \leavevmode\begin{itemize}
\item {} 
\sphinxAtStartPar
Lots of people

\end{itemize}

\end{description}

\sphinxAtStartPar
Added percent substitution ceiling for uniquely created DoS attacks.  Heh.
FUBAR flag no longer allows you to execute any functions.
\begin{description}
\item[{Added config param ‘lcon\_checks\_dark’ that will enforce dark/unfindable.}] \leavevmode\begin{itemize}
\item {} 
\sphinxAtStartPar
Thanks Xandar

\end{itemize}

\end{description}

\sphinxAtStartPar
Beefed up on\sphinxhyphen{}line help for reality levels.  Hopefully they make sense now :)
Mail/read recognizes ‘both’ to handle first new/unread message combination.
mail/anon optionally hides sender
KEEPALIVE @toggle added. \sphinxhyphen{} (AuroraMUX/Soruk)
chkreality() function to check if victom sees target’s reality.
CHKREALITY toggle that allows @Lock/user to become a reality lock
\begin{description}
\item[{@admin param ‘reality\_locks’ to globally enable reality level lock checks.}] \leavevmode\begin{itemize}
\item {} 
\sphinxAtStartPar
Thanks lots of people.

\end{itemize}

\item[{Added /fail switch to @Hook.}] \leavevmode\begin{itemize}
\item {} 
\sphinxAtStartPar
Originally Amborsia, modified. (MUX2 compat)

\end{itemize}

\end{description}

\sphinxAtStartPar
Added @admin param reality\_locktype that optionally chooses how reality locks are issued.  This should allow much more flexability.
\begin{description}
\item[{Sideeffects could double eval.  This was intended for some backward compatability with muse, but does allow a security risk.  Now have an optional compile time option of SECURE\_SIDEEFFECT that disables this.}] \leavevmode\begin{itemize}
\item {} 
\sphinxAtStartPar
Ambrosia

\end{itemize}

\end{description}

\sphinxAtStartPar
Added LOGGED attribute flag that will log anytime that attribute is changed, modified, set, or cleared.
\begin{description}
\item[{NOISY toggle to allow consistant noisy sets}] \leavevmode\begin{itemize}
\item {} 
\sphinxAtStartPar
Thanks Tam

\end{itemize}

\end{description}

\sphinxAtStartPar
backup\_flat.sh now accept a ‘\sphinxhyphen{}s’ option for single\sphinxhyphen{}mode for cron entry.


\subsubsection{Bug Fixes}
\label{\detokenize{changelog:id2}}\begin{description}
\item[{WHO/DOING on the connect screen had a broken conditional where if you were NOT unfindable you still wouldn’t show up on WHO/DOING on the connect screen.  \sphinxhyphen{} Fixed}] \leavevmode\begin{itemize}
\item {} 
\sphinxAtStartPar
Ambrosia \& \sphinxhref{mailto:Tam@Stargate}{Tam@Stargate}: Alpha Site

\end{itemize}

\item[{help files wouldn’t process escape characters right with ZENTY\_ANSI \sphinxhyphen{} Fixed}] \leavevmode\begin{itemize}
\item {} 
\sphinxAtStartPar
Thanks Zivilyn/Xandar

\end{itemize}

\end{description}

\sphinxAtStartPar
mail/recall couldn’t do lookups based by\sphinxhyphen{}player on multi\sphinxhyphen{}sent messages \sphinxhyphen{} Fixed
Wizard objects not inherit hit attr @limits \sphinxhyphen{} Fixed
\begin{description}
\item[{Possible crash bug with mail dynamic aliases via ‘\$’.  \sphinxhyphen{} Fixed}] \leavevmode\begin{itemize}
\item {} 
\sphinxAtStartPar
Thanks Xandar

\end{itemize}

\end{description}

\sphinxAtStartPar
@dynhelp/parse wouldn’t display ansi correctly (from previous fix) \sphinxhyphen{} Fixed

\sphinxAtStartPar
CPU Protection not as robust on certain cases \sphinxhyphen{} Fixed
\begin{description}
\item[{Month was off by one in mail/recall (introduced p18)}] \leavevmode\begin{itemize}
\item {} 
\sphinxAtStartPar
Thanks Mortalis @ Cajun

\end{itemize}

\end{description}

\sphinxAtStartPar
mail/recall had off\sphinxhyphen{}by\sphinxhyphen{}one for month (introduced p18)

\sphinxAtStartPar
logf() conflicted with built\sphinxhyphen{}in function in newest gcc compilers \sphinxhyphen{} Fixed

\sphinxAtStartPar
type cast warning in random function \sphinxhyphen{} Fixed

\sphinxAtStartPar
Unused variable cleanup in code. \sphinxhyphen{} Fixed

\sphinxAtStartPar
format\_name undocumented \sphinxhyphen{} Fixed

\sphinxAtStartPar
IGSWITCH @hook didn’t work with new format of commands in some instances \sphinxhyphen{} Fixed

\sphinxAtStartPar
Trace output issues with CPU protection \sphinxhyphen{} Fixed

\sphinxAtStartPar
LBUF failed to be free on examining in rare cases. (introduced p18) \sphinxhyphen{} Fixed

\sphinxAtStartPar
Cleanup of some initialization variables.  Not a problem.  Just a cleanup.

\sphinxAtStartPar
asksource broke on solaris systems. \sphinxhyphen{} Fixed

\sphinxAtStartPar
Lots of cleanup for Solaris 2.8/2.9 builds.  Solaris didn’t declare right. \sphinxhyphen{} Fixed

\sphinxAtStartPar
Security issue with descs clobbering setq regs \sphinxhyphen{} Fixed (optional @admin param)

\sphinxAtStartPar
new backup\_flat.sh script broke for remote archives \sphinxhyphen{} Fixed

\sphinxAtStartPar
player cache had possible dereferenced pointer call \sphinxhyphen{} Fixed

\sphinxAtStartPar
autoreg\_include.txt didn’t check txt subdirectory. \sphinxhyphen{} Fixed

\sphinxAtStartPar
reality locks overwrote match state data (introduced p18) \sphinxhyphen{} Fixed

\sphinxAtStartPar
{[}18\sphinxhyphen{}1{]} Fix for IGSWITCH @hook and multiple switches \sphinxhyphen{} Fixed
\begin{description}
\item[{{[}18\sphinxhyphen{}2{]} Fix for unfindable and loc() and other functions \sphinxhyphen{} Fixed}] \leavevmode\begin{itemize}
\item {} 
\sphinxAtStartPar
Thanks Spatterlight

\end{itemize}

\item[{{[}18\sphinxhyphen{}3{]} Fix for \%l in trace output (same issue as loc() had) \sphinxhyphen{} Fixed}] \leavevmode\begin{itemize}
\item {} 
\sphinxAtStartPar
Thanks Spatterlight

\end{itemize}

\end{description}

\sphinxAtStartPar
{[}18\sphinxhyphen{}4{]} Fix for global attribute formatting where ‘what’ should have been ‘owhat’
\begin{description}
\item[{{[}18\sphinxhyphen{}4{]} Modification to allow localized formatting with \&format\textless{}attr\textgreater{}}] \leavevmode\begin{itemize}
\item {} 
\sphinxAtStartPar
Thanks Kevin

\end{itemize}

\end{description}

\sphinxAtStartPar
{[}18\sphinxhyphen{}5{]} Added content searches for all forms of help.

\sphinxAtStartPar
{[}18\sphinxhyphen{}6{]} Fix LBUF free issue (introduced 18\sphinxhyphen{}4)
\begin{description}
\item[{Fix mail issue with marking and folders}] \leavevmode\begin{itemize}
\item {} 
\sphinxAtStartPar
Thanks Kevin

\end{itemize}

\end{description}

\sphinxAtStartPar
Enhanced @break to accept optional argument to ‘branch’


\subsection{RhostMUSH 3.2.4 p17 Update}
\label{\detokenize{changelog:rhostmush-3-2-4-p17-update}}
\sphinxAtStartPar
{[} 06/04/2004{]}


\subsubsection{Changes}
\label{\detokenize{changelog:id3}}
\sphinxAtStartPar
Rhost 3.2.4 is a locked version.  only bug fixes from this point on.  The next release will be 3.9 which will be alpha/beta leading to the 4.0 release.
Added sub\sphinxhyphen{}patchlevel versioning for pre\sphinxhyphen{}release patchlevel testing.
\begin{description}
\item[{Sanitizing of help files to correct missing/misdefined config parameters.}] \leavevmode\begin{itemize}
\item {} 
\sphinxAtStartPar
Ambrosia

\end{itemize}

\item[{An optional .conf file now has all available options to choose from.}] \leavevmode\begin{itemize}
\item {} 
\sphinxAtStartPar
Ambrosia

\end{itemize}

\item[{LOGROOM toggle now logs to subdirectory ‘roomlogs’.}] \leavevmode\begin{itemize}
\item {} 
\sphinxAtStartPar
Thanks Xandar

\end{itemize}

\end{description}

\sphinxAtStartPar
new admin param ‘roomlog\_path’ specifies path of above LOGROOM path.
attrcnt() modified to help rebuild wonked attribute tables
@aflags modified to help list attribute information


\subsubsection{Bug Fixes}
\label{\detokenize{changelog:id4}}
\sphinxAtStartPar
malloc.h wasn’t properly identified in the configure tool \sphinxhyphen{} Fixed
ZENTY\_ANSI didn’t handle ansi correctly if TINY\_SUB was also defined \sphinxhyphen{} Fixed
\begin{description}
\item[{@admin param garbage\_chunk wasn’t used \sphinxhyphen{} Removed}] \leavevmode\begin{itemize}
\item {} 
\sphinxAtStartPar
Ambrosia

\end{itemize}

\item[{@admin param precmd\_obj wasn’t used \sphinxhyphen{} Removed}] \leavevmode\begin{itemize}
\item {} 
\sphinxAtStartPar
Ambrosia

\end{itemize}

\end{description}

\sphinxAtStartPar
sortby() imported for backward compatibility.
\begin{description}
\item[{updated Rhost minimal\_db to handle new conf file parameters.}] \leavevmode\begin{itemize}
\item {} 
\sphinxAtStartPar
Thanks Cloud

\end{itemize}

\end{description}

\sphinxAtStartPar
You could alias over hardcoded functionality \sphinxhyphen{} Fixed
Improved attribute handling for corrupting attributes.
@attribute/delete now does sanity checking for attribute existing first
@aflag shows the numerical unique identifer of the attribute
\begin{description}
\item[{@pcreate/reg would crash if issued by a non\sphinxhyphen{}player \sphinxhyphen{} fixed}] \leavevmode\begin{itemize}
\item {} 
\sphinxAtStartPar
Thanks Xandar

\end{itemize}

\end{description}


\subsection{RhostMUSH 3.2.4 p16 Update}
\label{\detokenize{changelog:rhostmush-3-2-4-p16-update}}
\sphinxAtStartPar
{[}02/24/2004{]}


\subsubsection{Changes}
\label{\detokenize{changelog:id5}}\begin{description}
\item[{+proof in mail now displays text for forwarding/replied messages.}] \leavevmode\begin{itemize}
\item {} 
\sphinxAtStartPar
Many Many People

\end{itemize}

\end{description}

\sphinxAtStartPar
@cpattr, @mvattr, and @edit now take attribute content locking into effect.
inc() and dec() now take strings as ‘0’ values.
@door rewrite to handle mangled port openings/closings.
\begin{description}
\item[{@decompile now has /attribs, /flags, and /all switches}] \leavevmode\begin{itemize}
\item {} 
\sphinxAtStartPar
Thanks \sphinxhref{mailto:Tam@Stargate}{Tam@Stargate}: Alpha Site

\end{itemize}

\item[{@set with the /noisy switch notifies if attributes (cleared)}] \leavevmode\begin{itemize}
\item {} 
\sphinxAtStartPar
Thanks \sphinxhref{mailto:Tam@Stargate}{Tam@Stargate}: Alpha Site

\end{itemize}

\end{description}

\sphinxAtStartPar
@dynhelp (dynhelp()/textfile()) now does partial\sphinxhyphen{}matching.
\begin{description}
\item[{\sphinxhyphen{} for mail writing will now allow you to insert ‘=’ without escaping.}] \leavevmode\begin{itemize}
\item {} 
\sphinxAtStartPar
Thanks \sphinxhref{mailto:Alibi@Cajun}{Alibi@Cajun}

\end{itemize}

\item[{New bugreport script to handle bug report mailing.}] \leavevmode\begin{itemize}
\item {} 
\sphinxAtStartPar
Lensman \& Ashen\sphinxhyphen{}Shugar

\end{itemize}

\end{description}

\sphinxAtStartPar
New auto configurator scripts for make options.
ZoneWizLock now allows you to examine any area in the zone as well as modify it.  Royalty and higher ownership overrides this lock.
Help and Wizhelp have improved topics.
Third party work on PHP and SQL support \sphinxhyphen{} Still in pre\sphinxhyphen{}beta, not included.  contact \sphinxhref{mailto:lensman@rhostmush.org}{lensman@rhostmush.org} for SQL
@hook for security related commands shored up.
Modified credits for MUXPAGE toggle.  Penn had the feature first :)
Wizards now optionally checked for @limits
The non\sphinxhyphen{}command alias functionalty has been given a well\sphinxhyphen{}needed overhall
Better error handling was added to the functions in htab.c
Unified some of the Makefile defines for OsX, Cygwin, BSD.
Added funceval to @list
\begin{description}
\item[{@doing and @doing/header expanded in length to full\sphinxhyphen{}use.}] \leavevmode\begin{itemize}
\item {} 
\sphinxAtStartPar
Thanksla Falor

\end{itemize}

\end{description}

\sphinxAtStartPar
switch(), switchall(), and @switch now optionally take \#\$ substitutions.
GREATLY updated auto\sphinxhyphen{}makefile configurator to handle most common library checks.
\begin{description}
\item[{lattr() takes multiple page values}] \leavevmode\begin{itemize}
\item {} 
\sphinxAtStartPar
Thanks \sphinxhref{mailto:Sirona@Cajun}{Sirona@Cajun}

\end{itemize}

\end{description}


\subsubsection{Additions}
\label{\detokenize{changelog:id6}}

\paragraph{Functions}
\label{\detokenize{changelog:id7}}
\sphinxAtStartPar
aiindex() \sphinxhyphen{} works like iindex() except will append to a null list.
sortby() \sphinxhyphen{} Added for MUX2 compatibility


\subsubsection{Admin Params}
\label{\detokenize{changelog:id8}}
\sphinxAtStartPar
wizmax\_vattr\_limit   \sphinxhyphen{} Attribute limiter for wizards
wizmax\_dest\_limit    \sphinxhyphen{} Destroy limiter for wizards
vattr\_limit\_checkwiz \sphinxhyphen{} Enable/Disable wiz limiters
\begin{description}
\item[{guild\_attrname       \sphinxhyphen{} Make the GUILD column in WHO read a different attribute. (new players also have this attr set with the default value)}] \leavevmode\begin{itemize}
\item {} 
\sphinxAtStartPar
Suggested by \sphinxhref{mailto:Leona@Faetopia}{Leona@Faetopia}

\end{itemize}

\end{description}

\sphinxAtStartPar
exits\_connect\_rooms  \sphinxhyphen{} This parameter defines if rooms with exits will never be floating.  This parameter will disregard the need for exits to always be indirectly linked to the starting room.
switch\_substitutions \sphinxhyphen{} If enabled, allows \#\$ substitutions in switches.
examine\_restrictive  \sphinxhyphen{} Settable 1\sphinxhyphen{}5 (0 disables) to restrict examine based on ‘level’


\subsubsection{Bug Fixes}
\label{\detokenize{changelog:id9}}
\sphinxAtStartPar
Double free issue with caseall() \sphinxhyphen{} Fixed
Possible SIGSEGV condition with nobroadcast\_host (introduced p15)  \sphinxhyphen{} Fixed
If memory/system corrupted, could corrupt room check in command.c \sphinxhyphen{} Fixed
Possible SIGSEGV with +proofing mail (introduced p15) \sphinxhyphen{} Fixed
\begin{description}
\item[{Exit movement backward compatibility broke with unlinked exits (introduced p15) \sphinxhyphen{} Fixed}] \leavevmode\begin{itemize}
\item {} 
\sphinxAtStartPar
Thanks \sphinxhref{mailto:Ol'Sarge@Cajun}{Ol’Sarge@Cajun}

\end{itemize}

\item[{Possible SIGSEGV legacy bug with flag handling \sphinxhyphen{} Fixed}] \leavevmode\begin{itemize}
\item {} 
\sphinxAtStartPar
Thanks \sphinxhref{mailto:Tethys@Cajun}{Tethys@Cajun}

\end{itemize}

\end{description}

\sphinxAtStartPar
strmath() didn’t work like expected with ‘amt’ argument \sphinxhyphen{} Fixed
BOUNCE flag was not cleared on recoverable objects \sphinxhyphen{} Fixed
\begin{itemize}
\item {} 
\sphinxAtStartPar
Thanks Iuz

\end{itemize}
\begin{description}
\item[{debugstack extra free in alias table (introduced p15) \sphinxhyphen{} fixed}] \leavevmode\begin{itemize}
\item {} 
\sphinxAtStartPar
Thanks \sphinxhref{mailto:Jamie@M*U*S*H}{Jamie@M*U*S*H}

\end{itemize}

\end{description}

\sphinxAtStartPar
NDBM in make confsource assumed ndbm.h on server sane.  It’s usually not. \sphinxhyphen{} Fixed
Wizard limits wern’t verified on ownership if not set INHERIT \sphinxhyphen{} Fixed
\begin{description}
\item[{help/wizhelp didn’t show that you could do topic wildcard searches \sphinxhyphen{} Fixed}] \leavevmode\begin{itemize}
\item {} 
\sphinxAtStartPar
Thanks Falor

\end{itemize}

\item[{help syntax error with regards to null() and @@() \sphinxhyphen{} Fixed}] \leavevmode\begin{itemize}
\item {} 
\sphinxAtStartPar
Thanks Falor

\end{itemize}

\item[{Passing only one attr to @admin logout\_cmd\_alias caused a crash \sphinxhyphen{} Fixed}] \leavevmode\begin{itemize}
\item {} 
\sphinxAtStartPar
Thanks Ambrosia

\end{itemize}

\item[{BLIND wasn’t consistant for all commands \sphinxhyphen{} Fixed}] \leavevmode\begin{itemize}
\item {} 
\sphinxAtStartPar
Thanks Xandar

\end{itemize}

\end{description}

\sphinxAtStartPar
The hash tables could appear to loose information where aliases were used.  (introduced p15) \sphinxhyphen{} Fixed
The Hash tables are sized too small and don’t reflect ‘todays Rhost’.  \sphinxhyphen{} Fixed
@flag/remove wouldn’t return an error message if no alias existed \sphinxhyphen{} Fixed
hash with an off\sphinxhyphen{}by\sphinxhyphen{}one count. (Introduced p16 beta) \sphinxhyphen{} Fixed
\begin{description}
\item[{Typo with help.txt with regards to terse flag \sphinxhyphen{} Fixed}] \leavevmode\begin{itemize}
\item {} 
\sphinxAtStartPar
Thanks Sasaki \sphinxhref{mailto:Chie@Fantasy}{Chie@Fantasy} Moon

\end{itemize}

\end{description}


\subsection{RhostMUSH 3.2.4 p15 Update}
\label{\detokenize{changelog:rhostmush-3-2-4-p15-update}}
\sphinxAtStartPar
{[}09/09/2003{]}


\subsubsection{Changes}
\label{\detokenize{changelog:id10}}\begin{description}
\item[{MASSIVE internal rewrite of internal flag/command/function structure to prepare for loadable/unloadable module support.}] \leavevmode\begin{itemize}
\item {} 
\sphinxAtStartPar
Lensman

\end{itemize}

\item[{New Directory Structure for data, txt, src, hdrs.}] \leavevmode\begin{itemize}
\item {} 
\sphinxAtStartPar
Lensman

\end{itemize}

\item[{Script Rewrites for better resilliancy}] \leavevmode\begin{itemize}
\item {} 
\sphinxAtStartPar
Lensman, Ambrosia, Ashen\sphinxhyphen{}Shugar

\end{itemize}

\item[{local.c plugin support}] \leavevmode\begin{itemize}
\item {} 
\sphinxAtStartPar
Lensman

\end{itemize}

\item[{Alias system overhauled and rewritten to be more robust}] \leavevmode\begin{itemize}
\item {} 
\sphinxAtStartPar
Lensman

\end{itemize}

\item[{! and !! patch}] \leavevmode\begin{itemize}
\item {} 
\sphinxAtStartPar
Written and provided by Jeff

\end{itemize}

\item[{We now run on gdbm 1.8.3}] \leavevmode\begin{itemize}
\item {} 
\sphinxAtStartPar
Thanks Lensman

\end{itemize}

\item[{wildmatch fix}] \leavevmode\begin{itemize}
\item {} 
\sphinxAtStartPar
Lensman \& Ash

\end{itemize}

\end{description}

\sphinxAtStartPar
minimal database provided that includes SGP, myrddin bbs, global code Add method for global (master room) exits to be shown with ‘look’

\begin{sphinxadmonition}{note}{Note:}
\sphinxAtStartPar
PRIVATE and/or FLOATING flags remove this if set on exit.
\end{sphinxadmonition}
\begin{description}
\item[{Add anonymous mail via the /anon switch}] \leavevmode\begin{itemize}
\item {} 
\sphinxAtStartPar
Thanks Saffron \& Lyle @ Paris

\end{itemize}

\item[{mail/recall cleaned up for better display/information.}] \leavevmode\begin{itemize}
\item {} 
\sphinxAtStartPar
Thanks \sphinxhref{mailto:Ivory@Mednights}{Ivory@Mednights}

\end{itemize}

\item[{Timestamps added for original message in forward/reply}] \leavevmode\begin{itemize}
\item {} 
\sphinxAtStartPar
Thanks \sphinxhref{mailto:Ivory@Mednights}{Ivory@Mednights}

\end{itemize}

\end{description}

\sphinxAtStartPar
lit() can accept commas as part of it’s input string
\begin{description}
\item[{+bcc in mail/write now gives mail BLINDLY to get target. They won’t see the To: list(s).}] \leavevmode\begin{itemize}
\item {} 
\sphinxAtStartPar
Thanks Saffron \& Lyle @ Paris

\end{itemize}

\item[{modified hastype() to have same permissions as type()}] \leavevmode\begin{itemize}
\item {} 
\sphinxAtStartPar
Thanks Rynos

\end{itemize}

\item[{fixed mail issue with dynamic mail aliases and permissions.}] \leavevmode\begin{itemize}
\item {} 
\sphinxAtStartPar
Thanks Erik

\end{itemize}

\item[{Add way for @emit to parse ‘\#\#’ for target it receives.  /sub switch.}] \leavevmode\begin{itemize}
\item {} 
\sphinxAtStartPar
Thanks Jared Leisner @ Ennerseas

\end{itemize}

\item[{Add method to see ‘mail/status’ with mailstatus() function.}] \leavevmode\begin{itemize}
\item {} 
\sphinxAtStartPar
Thanks Rosalind @ Nevermore

\end{itemize}

\item[{config(sideeffects\_txt) return the string of sideeffects enabled.}] \leavevmode\begin{itemize}
\item {} 
\sphinxAtStartPar
Lensman

\end{itemize}

\item[{Penn conversion script included in distribution.}] \leavevmode\begin{itemize}
\item {} 
\sphinxAtStartPar
Thanks Mac

\end{itemize}

\item[{/preserve switch to @chown and @chownall.  Will keep all flags constant.}] \leavevmode\begin{itemize}
\item {} 
\sphinxAtStartPar
Lensman

\end{itemize}

\end{description}

\sphinxAtStartPar
@list stack now shows filenames instead of file pointers.
\begin{description}
\item[{/pid added to @notify to handle PID processes}] \leavevmode\begin{itemize}
\item {} 
\sphinxAtStartPar
Thanks Storm

\end{itemize}

\item[{/recpid added to @wait to record PID process to setq\sphinxhyphen{}registers specified.}] \leavevmode\begin{itemize}
\item {} 
\sphinxAtStartPar
Thanks Storm

\end{itemize}

\end{description}

\sphinxAtStartPar
@function takes /d{[}elete{]} switch to delete user\sphinxhyphen{}defined functions.
lzones() now take optional arguments for large zone lists.
Improved CPU protection (yes, even more).
@doors are no longer preserved between reboots.
\begin{description}
\item[{13th argument to columns() to allow padding of short columns}] \leavevmode\begin{itemize}
\item {} 
\sphinxAtStartPar
Thanks Patrick Bogen

\end{itemize}

\item[{inc()/dec() (the hardcoded xinc/xdec) now, like other servers, handle non\sphinxhyphen{}numericals for arguments.}] \leavevmode\begin{itemize}
\item {} 
\sphinxAtStartPar
Thanks Xandar

\end{itemize}

\item[{@pcreate now allows you the /register switch so wizards can email the pwds.}] \leavevmode\begin{itemize}
\item {} 
\sphinxAtStartPar
Thanks Xandar

\end{itemize}

\item[{@pemit (and all switches) now handle REALITY LEVELS with /reality switch.}] \leavevmode\begin{itemize}
\item {} 
\sphinxAtStartPar
Thanks \sphinxhref{mailto:Dervish@Cajun}{Dervish@Cajun}

\end{itemize}

\item[{@admin alias allows re\sphinxhyphen{}aliasing existing aliases or deleting aliases.}] \leavevmode\begin{itemize}
\item {} 
\sphinxAtStartPar
Lensman

\end{itemize}

\end{description}


\subsubsection{Additions}
\label{\detokenize{changelog:id11}}

\paragraph{Functions}
\label{\detokenize{changelog:id12}}\begin{description}
\item[{elementsmux() \sphinxhyphen{} for MUX elements() compatibility.}] \leavevmode\begin{itemize}
\item {} 
\sphinxAtStartPar
Thanks \sphinxhref{mailto:Rook@EnnerSeas}{Rook@EnnerSeas}

\end{itemize}

\end{description}

\sphinxAtStartPar
parsestr() \sphinxhyphen{}  helps with speech formatting.
\begin{description}
\item[{chomp() \sphinxhyphen{} used to strip returns before, after, or both sides of string.}] \leavevmode\begin{itemize}
\item {} 
\sphinxAtStartPar
Thanks Storm

\end{itemize}

\end{description}

\sphinxAtStartPar
escapex() \sphinxhyphen{} just like escape() but you can choose what chars to \_not\_ escape

\sphinxAtStartPar
securex() \sphinxhyphen{} just like escapex() but for secure()’s counterpart


\paragraph{Flags}
\label{\detokenize{changelog:id13}}\begin{description}
\item[{BLIND \sphinxhyphen{} flag for exits and rooms to make target arrival/leaving ‘snuffed’.}] \leavevmode\begin{itemize}
\item {} 
\sphinxAtStartPar
Idea from many people (and TM3)

\end{itemize}

\end{description}

\sphinxAtStartPar
DEFAULT \sphinxhyphen{} Attribute flag for handling default global attrib (TM3)
SINGLETHREAD \sphinxhyphen{} Attribute flag to handle single\sphinxhyphen{}threading \$commands
ATRLOCK \sphinxhyphen{} Attribute flag to handle global attribute locking


\paragraph{Toggles}
\label{\detokenize{changelog:id14}}\begin{description}
\item[{MUXPAGE \sphinxhyphen{} Toggle to allow mux\sphinxhyphen{}like paging for Penn/MUX/TM3 compatibility.}] \leavevmode\begin{itemize}
\item {} 
\sphinxAtStartPar
Thanks PennMUSH

\end{itemize}

\end{description}

\sphinxAtStartPar
NOGLOBPARENT \sphinxhyphen{} Toggle to disable inheritance of global inheret parents.
NODEFAULT \sphinxhyphen{} Toggle to disable global default handling (TM3)


\subsubsection{Admin Params}
\label{\detokenize{changelog:id15}}\begin{description}
\item[{mail\_verbosity \sphinxhyphen{} Add Subj: to sent mail as well as to disconnected players.}] \leavevmode\begin{itemize}
\item {} 
\sphinxAtStartPar
Thanks Saffron \& Lyle @ Paris

\end{itemize}

\end{description}

\sphinxAtStartPar
mail\_anonymous \sphinxhyphen{} Default name for anonymous mail (Default: \sphinxstyleemphasis{Anonymous})
sidefx\_maxcalls \sphinxhyphen{} (1000 default) for max sideffects allowed/command.
oattr\_enable\_altname \sphinxhyphen{} to enable/disable alt name usage in odrop/ofail/osucc.
\begin{description}
\item[{oattr\_uses\_altname \sphinxhyphen{} for alt names sent to odrop/ofail/osucc.}] \leavevmode\begin{itemize}
\item {} 
\sphinxAtStartPar
Thanks Rook @ Ennerseas (default \_NPC)

\end{itemize}

\item[{empower\_fulltel \sphinxhyphen{} Offer two methods for FULLTEL (‘self’ \& anything not cloaked)}] \leavevmode\begin{itemize}
\item {} 
\sphinxAtStartPar
Thanks \sphinxhref{mailto:Punk@FantasyMoon}{Punk@FantasyMoon}

\end{itemize}

\item[{spam\_msg \sphinxhyphen{} Message sent to spammers}] \leavevmode\begin{itemize}
\item {} 
\sphinxAtStartPar
Ambrosia

\end{itemize}

\item[{spam\_objmsg \sphinxhyphen{} Message sent to spammers of objects}] \leavevmode\begin{itemize}
\item {} 
\sphinxAtStartPar
Ambrosia

\end{itemize}

\item[{room\_aconnect \sphinxhyphen{} Aconnects on individual rooms work (cloak/dark checked)}] \leavevmode\begin{itemize}
\item {} 
\sphinxAtStartPar
Lensman

\end{itemize}

\item[{room\_adisconnect \sphinxhyphen{} Adisconnects on individual rooms work (cloak/dark checked)}] \leavevmode\begin{itemize}
\item {} 
\sphinxAtStartPar
Lensman

\end{itemize}

\end{description}

\sphinxAtStartPar
player\_attr\_default \sphinxhyphen{} Sets default @\textless{}attrib\textgreater{} handler for did\_it() attribs (TM3)
thing\_attr\_default \sphinxhyphen{} Sets default @\textless{}attrib\textgreater{} handler for did\_it() attribs (TM3)
exit\_attr\_default \sphinxhyphen{} Sets default @\textless{}attrib\textgreater{} handler for did\_it() attribs (TM3)
room\_attr\_default \sphinxhyphen{} Sets default @\textless{}attrib\textgreater{} handler for did\_it() attribs (TM3)
global\_clone\_obj \sphinxhyphen{} Sets default dbref\# for cloning attributes
global\_clone\_player \sphinxhyphen{} Sets default debref\# for cloning attributes (TM3)
global\_clone\_thing \sphinxhyphen{} Sets default debref\# for cloning attributes (TM3)
global\_clone\_room \sphinxhyphen{} Sets default debref\# for cloning attributes (TM3)
global\_clone\_exit \sphinxhyphen{} Sets default debref\# for cloning attributes (TM3)
global\_attrdefault \sphinxhyphen{} Sets global locker for attribute sets/clears
nobroadcast\_host \sphinxhyphen{} Define what sites will be ‘snuffed’ from MONITOR


\subsubsection{Bug Fixes}
\label{\detokenize{changelog:id16}}\begin{description}
\item[{Bug in wizhelp with ‘mail\_lockdown’ toggle.  \sphinxhyphen{} Fixed}] \leavevmode\begin{itemize}
\item {} 
\sphinxAtStartPar
Thanks Dervish

\end{itemize}

\item[{Multiple help/wizhelp fixes.  \sphinxhyphen{} Fixed}] \leavevmode\begin{itemize}
\item {} 
\sphinxAtStartPar
Thanks Dervish

\end{itemize}

\item[{Bug with @mvattr and QUIET flag \sphinxhyphen{} Fixed}] \leavevmode\begin{itemize}
\item {} 
\sphinxAtStartPar
Thanks Xandar

\end{itemize}

\end{description}

\sphinxAtStartPar
Objects could use the brandy toggle to send mail.  \sphinxhyphen{} Fixed
Security issue with autoregistration \sphinxhyphen{} Fixed
goto didn’t have hooks before/after right \sphinxhyphen{} Fixed
\begin{description}
\item[{SIGSEGV on autozone add if player didn’t belong to zone \sphinxhyphen{} Fixed}] \leavevmode\begin{itemize}
\item {} 
\sphinxAtStartPar
Thanks Rook \& Sylph @ Ennerseas

\end{itemize}

\item[{Bug with teleporting and permissions \sphinxhyphen{} Fixed}] \leavevmode\begin{itemize}
\item {} 
\sphinxAtStartPar
Thanks Mach \sphinxhref{mailto:Speed@FantasyMoon}{Speed@FantasyMoon}

\end{itemize}

\item[{NASTY bug that could corrupt registries with @freeze/@thaw/@wait (RARE) \sphinxhyphen{} Fixed}] \leavevmode\begin{itemize}
\item {} 
\sphinxAtStartPar
Thanks Rook @ Ennerseas

\end{itemize}

\item[{Fix help entry with @convert/@quota cross\sphinxhyphen{}matching. \sphinxhyphen{} Fixed}] \leavevmode\begin{itemize}
\item {} 
\sphinxAtStartPar
Thanks Dervish

\end{itemize}

\item[{Fix alloc corruption with ‘page’ from liberal nulls. (p15 introduced) \sphinxhyphen{} Fixed}] \leavevmode\begin{itemize}
\item {} 
\sphinxAtStartPar
Thanks Xandar

\end{itemize}

\item[{Fix issue where if in @program string sent to global\_error\_obj incorrect \sphinxhyphen{} Fixed}] \leavevmode\begin{itemize}
\item {} 
\sphinxAtStartPar
Thanks Zivilyn

\end{itemize}

\item[{Fix for orflags()/andflags() being broke with flag rewrite. \sphinxhyphen{} Fixed}] \leavevmode\begin{itemize}
\item {} 
\sphinxAtStartPar
Thinks Rosalind \& Vulcan @ Nevermore

\end{itemize}

\end{description}

\sphinxAtStartPar
Fix for @hooks to now successfully work on goto \sphinxhyphen{} Fixed
Fix for @uptime to show time up longer than a year \sphinxhyphen{} Fixed
\begin{description}
\item[{Fix for @hook on goto.  Did not handle /permit or /ignore right \sphinxhyphen{} Fixed}] \leavevmode\begin{itemize}
\item {} 
\sphinxAtStartPar
Thanks Rook

\end{itemize}

\item[{Fix for legacy bug in @list functions for blowing a buffer if too many user defined functions have been defined.  Wow, talk about old.}] \leavevmode\begin{itemize}
\item {} 
\sphinxAtStartPar
Thanks \sphinxhref{mailto:Aalita@Ennersea}{Aalita@Ennersea}

\end{itemize}

\end{description}

\sphinxAtStartPar
Fix for ZENTY\_ANSI compiletime with a possible buffer overrun. \sphinxhyphen{} Fixed
Fix for ZENTY\_ANSI with handling safebuff() \sphinxhyphen{} Fixed.
Fix for two rhosts running same debugmon debug\_id \sphinxhyphen{} Fixed.
Fix for mis\sphinxhyphen{}matched DPUSH/RETURN for DPOP in door.c \sphinxhyphen{} Fixed
\begin{description}
\item[{Fix for pemit()/npemit() with argument evaluation \sphinxhyphen{} Fixed}] \leavevmode\begin{itemize}
\item {} 
\sphinxAtStartPar
Thanks \sphinxhref{mailto:Matthew@Draconis}{Matthew@Draconis}

\end{itemize}

\item[{Fix for legacy bug with vattr initialization \sphinxhyphen{} Fixed}] \leavevmode\begin{itemize}
\item {} 
\sphinxAtStartPar
Thanks \sphinxhref{mailto:Matthew@Draconis}{Matthew@Draconis}

\end{itemize}

\item[{Fix for library overflow issues regarding system call on ptimefmt() \sphinxhyphen{} Fixed}] \leavevmode\begin{itemize}
\item {} 
\sphinxAtStartPar
Thanks Shari

\end{itemize}

\item[{Fix for default() and edefault() returning improperly for invalid dbref\# \sphinxhyphen{} Fixed}] \leavevmode\begin{itemize}
\item {} 
\sphinxAtStartPar
Thanks Matthew

\end{itemize}

\end{description}

\sphinxAtStartPar
lock(), rxlevel(), txlevel(), and parent() didn’t increment the sidefx counter the correct way.  \sphinxhyphen{} Fixed
\begin{description}
\item[{wizhelp entry with no\_move had grammer mistakes. \sphinxhyphen{} Fixed}] \leavevmode\begin{itemize}
\item {} 
\sphinxAtStartPar
Matthew

\end{itemize}

\item[{Possible overrun on the stack with regards to iter() during certain config opts.}] \leavevmode\begin{itemize}
\item {} 
\sphinxAtStartPar
Thanks \sphinxhref{mailto:Illithid@Ennersea}{Illithid@Ennersea}

\end{itemize}

\end{description}


\subsection{RhostMUSH 3.2.4 p14 Update}
\label{\detokenize{changelog:rhostmush-3-2-4-p14-update}}
\sphinxAtStartPar
{[}07/10/2002{]}


\subsubsection{Changes}
\label{\detokenize{changelog:id17}}\begin{description}
\item[{round() excepts negative args (60) for rounding values to whole numbers. (MUX2) The SPOOF flag is now inheritable.}] \leavevmode\begin{itemize}
\item {} 
\sphinxAtStartPar
Thanks Milk \& Nyssa

\end{itemize}

\end{description}

\sphinxAtStartPar
Added /nosub switch to @pemit so \#\# and \#@ arn’t subbed.
\begin{description}
\item[{Add /preserve to @wipe that wipes all \_but\_ the match}] \leavevmode\begin{itemize}
\item {} 
\sphinxAtStartPar
Thanks Mikhail Mikhailov

\end{itemize}

\item[{Force an ANSI\_NORMAL at the end of @extansi calls @oxtport/@o\textless{}blah\textgreater{} messages don’t show if a null string.  This is handy if you want to process sideeffects but don’t want a result to show.}] \leavevmode\begin{itemize}
\item {} 
\sphinxAtStartPar
Thanks Nyssa

\end{itemize}

\end{description}

\sphinxAtStartPar
Add to cpu\sphinxhyphen{}slamming an optional way to register\sphinxhyphen{}site/forbid\sphinxhyphen{}site the person.
set() now handles ansi.
setq()/setr()/r() now handles ansi.
If owner set FLOATING, floating messages not returned.
Improved chksum methods on @freeze/@thaw
\begin{description}
\item[{Added /basic switch to @lock for PENN compatibility.}] \leavevmode\begin{itemize}
\item {} 
\sphinxAtStartPar
Thanks \sphinxhref{mailto:Trispis@M*U*S*H}{Trispis@M*U*S*H}

\end{itemize}

\item[{Attributes starting with a ‘\textasciitilde{}’ are now supported if ATTR\_HACK enabled.}] \leavevmode\begin{itemize}
\item {} 
\sphinxAtStartPar
Thanks \sphinxhref{mailto:Trispis@M*U*S*H}{Trispis@M*U*S*H}

\end{itemize}

\item[{Added wildcard matches to @list user\_attributes.}] \leavevmode\begin{itemize}
\item {} 
\sphinxAtStartPar
Thanks \sphinxhref{mailto:Brazil@MUX2}{Brazil@MUX2}

\end{itemize}

\end{description}

\sphinxAtStartPar
Added a\sphinxhyphen{}z setq() registers for MUX/TM3 compatibility. (very MEMORY intensive)
@list alloc now shows additional stack/lbuf information.
iter() and it’s ilk now use Brazil’s replace\_token() call for \#\# and \#@ (MUX2)
citer() now has an output seperator
lwho() takes argument of ‘2’ to list JUST the ports.
Add internal attribute SpamMonitor to store history of command(s).
Modified MONITOR sitecons so it shows the remote port they’re connecting from.
Modify sin(), tan(), etc (ala MUX) for conversions.  Backward compatible (MUX2)
\begin{description}
\item[{Zenty’s ANSI modifications.}] \leavevmode\begin{itemize}
\item {} 
\sphinxAtStartPar
Thanks \sphinxhref{mailto:Zenty@RhostMUSH}{Zenty@RhostMUSH}

\end{itemize}

\item[{Added way to convert PENN 1.7.5 flatfiles to RhostMUSH native.}] \leavevmode\begin{itemize}
\item {} 
\sphinxAtStartPar
Thanks Mac

\end{itemize}

\item[{Added SHS password encryption and plantext to crypt/SHS conversion on the fly.}] \leavevmode\begin{itemize}
\item {} 
\sphinxAtStartPar
Thanks \sphinxhref{mailto:Azhdeen@RhostMUSH}{Azhdeen@RhostMUSH}

\end{itemize}

\end{description}


\subsubsection{Additions}
\label{\detokenize{changelog:id18}}

\paragraph{Functions}
\label{\detokenize{changelog:id19}}\begin{description}
\item[{txlevel() \sphinxhyphen{} sideeffect that sets @txlevel (or displays)}] \leavevmode\begin{itemize}
\item {} 
\sphinxAtStartPar
Thanks \sphinxhref{mailto:accela@AniMUSH}{accela@AniMUSH}

\end{itemize}

\item[{rxlevel() \sphinxhyphen{} sideeffect that sets @rxlevel (or displays)}] \leavevmode\begin{itemize}
\item {} 
\sphinxAtStartPar
Thanks \sphinxhref{mailto:accela@AniMUSH}{accela@AniMUSH}

\end{itemize}

\item[{rset() \sphinxhyphen{} sideeffect that sets attribute and returns value.}] \leavevmode\begin{itemize}
\item {} 
\sphinxAtStartPar
Thanks \sphinxhref{mailto:Trispis@M*U*S*H}{Trispis@M*U*S*H}

\end{itemize}

\end{description}

\sphinxAtStartPar
pedit() \sphinxhyphen{} used to mimic Penn’s edit() functionality (Penn)

\sphinxAtStartPar
ptimefmt() \sphinxhyphen{} used to mimic Penn/Mux’s timefmt() functionality (Penn/MUX2)
\begin{description}
\item[{textfile() \sphinxhyphen{} works like dynhelp() but pushes onto buffer (Penn)}] \leavevmode\begin{itemize}
\item {} 
\sphinxAtStartPar
Thanks Raevnos \& PennMUSH

\end{itemize}

\item[{lattrp() \sphinxhyphen{} lattr() for parent checks}] \leavevmode\begin{itemize}
\item {} 
\sphinxAtStartPar
Idea from Jake \& MUX2

\end{itemize}

\end{description}

\sphinxAtStartPar
ctu() \sphinxhyphen{} function that does deg/rad/grad conversion
visiblemux() \sphinxhyphen{} works like mux’s visible()


\paragraph{Commands}
\label{\detokenize{changelog:id20}}
\sphinxAtStartPar
@hook to show/display/change individual hooks.  Switches are:

\begin{sphinxVerbatim}[commandchars=\\\{\}]
         \PYG{o}{/}\PYG{n}{permit}    \PYG{o}{\PYGZhy{}} \PYG{n}{Return} \PYG{l+s+s1}{\PYGZsq{}}\PYG{l+s+s1}{Permission denied.}\PYG{l+s+s1}{\PYGZsq{}} \PYG{k}{if} \PYG{n}{fail} \PYG{n}{lock} \PYG{p}{(}\PYG{l+m+mi}{1}\PYG{o}{/}\PYG{l+m+mi}{0} \PYG{n}{boolean}\PYG{p}{)}
         \PYG{o}{/}\PYG{n}{ignore}    \PYG{o}{\PYGZhy{}} \PYG{n}{Fall} \PYG{n}{through} \PYG{n}{command} \PYG{n}{check} \PYG{k}{if} \PYG{n}{fail} \PYG{n}{lock} \PYG{p}{(}\PYG{l+m+mi}{1}\PYG{o}{/}\PYG{l+m+mi}{0} \PYG{n}{boolean}\PYG{p}{)}
         \PYG{o}{/}\PYG{n}{before}    \PYG{o}{\PYGZhy{}} \PYG{n}{Process} \PYG{n}{functionality} \PYG{n}{before} \PYG{n}{command} \PYG{n}{execution}\PYG{o}{.}
         \PYG{o}{/}\PYG{n}{after}     \PYG{o}{\PYGZhy{}} \PYG{n}{Process} \PYG{n}{functionality} \PYG{n}{after} \PYG{n}{command} \PYG{n}{execution}\PYG{o}{.}
         \PYG{o}{/}\PYG{n}{igswitch}  \PYG{o}{\PYGZhy{}} \PYG{n}{Mark} \PYG{n}{command} \PYG{n}{to} \PYG{n}{ignore} \PYG{n}{failed} \PYG{o}{/}\PYG{n}{switch} \PYG{n}{matches}\PYG{o}{.}

\PYG{o}{\PYGZhy{}} \PYG{n}{Thanks} \PYG{n}{Moe}\PYG{n+nd}{@BrazilMUX} \PYG{p}{(}\PYG{o}{/}\PYG{n}{igswitch} \PYG{n}{idea}\PYG{p}{)}
\end{sphinxVerbatim}


\paragraph{Flags}
\label{\detokenize{changelog:id21}}
\sphinxAtStartPar
ATTRIBUTE FLAG: uselock \sphinxhyphen{} when set on an attribute with a \$command, will look for a matching \textasciitilde{}\textless{}attribute\textgreater{} to eval the lock.  This does BOOLEAN evaluation.  1 pass, 0 fail.

\sphinxAtStartPar
SPAMMONITOR \sphinxhyphen{} mark if player and/or target item check for spam (60 cmds/sec)


\paragraph{Toggles}
\label{\detokenize{changelog:id22}}
\sphinxAtStartPar
ZONEINHERIT \sphinxhyphen{} Allows zonemasters to have attributes inherited to children.


\subsubsection{Admin Params}
\label{\detokenize{changelog:id23}}\begin{description}
\item[{muddb\_name \sphinxhyphen{} admin param for db names to seperate from ‘mud\_name’}] \leavevmode\begin{itemize}
\item {} 
\sphinxAtStartPar
Thanks \sphinxhref{mailto:Milk@MattRhost}{Milk@MattRhost}

\end{itemize}

\item[{global\_error\_obj \sphinxhyphen{} evaluate the VA attribute on the object if defined.  this will evaluate the ‘huh?’ message(s).}] \leavevmode\begin{itemize}
\item {} 
\sphinxAtStartPar
Thanks \sphinxhref{mailto:Zenty@RhostMUSH}{Zenty@RhostMUSH}

\end{itemize}

\end{description}

\sphinxAtStartPar
mail\_autodeltime \sphinxhyphen{} specifies when mail is globally purged (default 21 days)
global\_parent\_room \sphinxhyphen{} globally inherit attributes to room w/o @parent
global\_parent\_thing \sphinxhyphen{} globally inherit attributes to thing w/o @parent
global\_parent\_player \sphinxhyphen{} globally inherit attributes to player w/o @parent
global\_parent\_exit \sphinxhyphen{} globally inherit attributes to exit w/o @parent
hook\_obj \sphinxhyphen{} globally define the ‘hook’ object.
hook\_cmd \sphinxhyphen{} process ‘hooks’ on specified commands:

\begin{sphinxVerbatim}[commandchars=\\\{\}]
         \PYG{n}{PERMIT}   \PYG{o}{\PYGZhy{}} \PYG{n}{to} \PYG{k}{pass} \PYG{n}{who} \PYG{n}{can} \PYG{n}{use}\PYG{o}{/}\PYG{n}{etc} \PYG{p}{(}\PYG{n}{bitmask} \PYG{l+m+mi}{1}\PYG{p}{)}
         \PYG{n}{IGNORE}   \PYG{o}{\PYGZhy{}} \PYG{n}{to} \PYG{n}{IGNORE} \PYG{n}{who} \PYG{n}{can} \PYG{n}{use}\PYG{o}{/}\PYG{n}{etc} \PYG{p}{(}\PYG{n}{bitmask} \PYG{l+m+mi}{2}\PYG{p}{)}
         \PYG{n}{BEFORE}   \PYG{o}{\PYGZhy{}} \PYG{n}{to} \PYG{k}{pass} \PYG{n}{what} \PYG{o+ow}{is} \PYG{n}{done} \PYG{n}{before} \PYG{n}{command} \PYG{p}{(}\PYG{n}{bitmask} \PYG{l+m+mi}{4}\PYG{p}{)}
         \PYG{n}{AFTER}    \PYG{o}{\PYGZhy{}} \PYG{n}{to} \PYG{k}{pass} \PYG{n}{what} \PYG{o+ow}{is} \PYG{n}{done} \PYG{n}{after} \PYG{n}{command} \PYG{p}{(}\PYG{n}{bitmask} \PYG{l+m+mi}{8}\PYG{p}{)}
         \PYG{n}{IGSWITCH} \PYG{o}{\PYGZhy{}} \PYG{n}{bypass} \PYG{n}{error} \PYG{n}{control} \PYG{n}{on} \PYG{n}{non}\PYG{o}{\PYGZhy{}}\PYG{n}{existant} \PYG{n}{switches}\PYG{o}{.}

\PYG{o}{\PYGZhy{}} \PYG{n}{Thanks} \PYG{n}{Moe}\PYG{n+nd}{@BrazilMUX} \PYG{p}{(}\PYG{n}{igswitch} \PYG{n}{idea}\PYG{p}{)}
\end{sphinxVerbatim}

\sphinxAtStartPar
look\_moreflags \sphinxhyphen{} if enabled, will show global flags on things with examine.
stack\_limit \sphinxhyphen{} nest check for ‘stack’ to throttle back a given amount. (Penn)


\subsubsection{Bug Fixes}
\label{\detokenize{changelog:id24}}\begin{description}
\item[{HELPFILE \sphinxhyphen{} trigger() was not a command but was shown in help. \sphinxhyphen{}Fixed}] \leavevmode\begin{itemize}
\item {} 
\sphinxAtStartPar
Thanks \sphinxhref{mailto:accela@AniMUSH}{accela@AniMUSH}

\end{itemize}

\item[{HELPFILE \sphinxhyphen{} trim() had examples with args reversed.  \sphinxhyphen{} Fixed}] \leavevmode\begin{itemize}
\item {} 
\sphinxAtStartPar
Thanks \sphinxhref{mailto:accela@AniMUSH}{accela@AniMUSH}

\end{itemize}

\item[{HELPFILE \sphinxhyphen{} locate() had ‘I’ instead of ‘i’.  \sphinxhyphen{} Fixed}] \leavevmode\begin{itemize}
\item {} 
\sphinxAtStartPar
Thanks \sphinxhref{mailto:DOSBoots@AniMUSH}{DOSBoots@AniMUSH}

\end{itemize}

\item[{HELPFILE \sphinxhyphen{} Help on substitutions incorrect with \%\sphinxhyphen{}subs.  \sphinxhyphen{} Fixed}] \leavevmode\begin{itemize}
\item {} 
\sphinxAtStartPar
Thanks \sphinxhref{mailto:DOSBoots@AniMUSH}{DOSBoots@AniMUSH}

\end{itemize}

\item[{HELPFILE \sphinxhyphen{} wizhelp didn’t list ‘news’ and ‘newsdb’ in the main list. \sphinxhyphen{} Fixed}] \leavevmode\begin{itemize}
\item {} 
\sphinxAtStartPar
Thanks \sphinxhref{mailto:Nyssa@Everywhere}{Nyssa@Everywhere}

\end{itemize}

\item[{HELPFILE \sphinxhyphen{} help didn’t show the /quiet switch to @trigger. \sphinxhyphen{} Fixed}] \leavevmode\begin{itemize}
\item {} 
\sphinxAtStartPar
Thanks Deus \sphinxhref{mailto:Maximas@TaintedEarth}{Maximas@TaintedEarth}

\end{itemize}

\item[{HELPFILE \sphinxhyphen{} modify help for twinklock to point to NOMODIFY flag.}] \leavevmode\begin{itemize}
\item {} 
\sphinxAtStartPar
Thanks Lyle

\end{itemize}

\end{description}

\sphinxAtStartPar
Option incorrectly shown in @list options mail. \sphinxhyphen{} Fixed
SESSION didn’t cut the name off at 16 chars (formatting issue) \sphinxhyphen{} Fixed
mailquick()’s arg didn’t totally mirror MUX’s mail()  \sphinxhyphen{} Fixed
Fix for if who\_unfindable disabled, player\_dark disabled, and allow\_whodark enabled you’d never get the connect flag of a wizard. \sphinxhyphen{} Fixed
\begin{description}
\item[{Linux and other weird unix systems tended to hang on AUTH lookups still. \sphinxhyphen{} Fixed}] \leavevmode\begin{itemize}
\item {} 
\sphinxAtStartPar
\sphinxhref{mailto:Thorin@RhostMUSH}{Thorin@RhostMUSH}

\end{itemize}

\item[{v() wouldn’t handle special chars as first char if enabled via ATTR\_HACK \sphinxhyphen{} Fixed}] \leavevmode\begin{itemize}
\item {} 
\sphinxAtStartPar
Thanks \sphinxhref{mailto:Trispis@M*U*S*H}{Trispis@M*U*S*H}

\end{itemize}

\end{description}

\sphinxAtStartPar
parent() when used to set a new parent did not return the dbref\# \sphinxhyphen{} Fixed
Fixed legacy bug in QUEUE which effected a\sphinxhyphen{}z setq() regs. \sphinxhyphen{} Fixed
‘home’ check was broke if set disabled and ignore at the same time \sphinxhyphen{} Fixed
\begin{description}
\item[{@thaw didn’t handle semaphores properly. \sphinxhyphen{} Fixed}] \leavevmode\begin{itemize}
\item {} 
\sphinxAtStartPar
Thanks Dervish

\end{itemize}

\item[{Zone\sphinxhyphen{}ignores didn’t work for QUIT, LOGOUT, and it’s ilk. \sphinxhyphen{} Fixed}] \leavevmode\begin{itemize}
\item {} 
\sphinxAtStartPar
Thanks Nyssa

\end{itemize}

\item[{Debug Monitor stack had a conditional off\sphinxhyphen{}by\sphinxhyphen{}one state on @reboots \sphinxhyphen{} Fixed}] \leavevmode\begin{itemize}
\item {} 
\sphinxAtStartPar
Thanks Mac and Ambrosia

\end{itemize}

\item[{@extansi was broke when ZENTY\_ANSI not defined \sphinxhyphen{} Fixed}] \leavevmode\begin{itemize}
\item {} 
\sphinxAtStartPar
Thanks Ambrosia \& Lensman

\end{itemize}

\item[{plushelp\_file and plushelp\_index missing from wizhelp \sphinxhyphen{} Fixed}] \leavevmode\begin{itemize}
\item {} 
\sphinxAtStartPar
Thanks Ronan

\end{itemize}

\item[{wildcard matching could become problematic \sphinxhyphen{} Fixed.}] \leavevmode\begin{itemize}
\item {} 
\sphinxAtStartPar
Thanks \sphinxhref{mailto:Sketch@M*U*S*H}{Sketch@M*U*S*H} \& Javelin

\end{itemize}

\end{description}


\subsection{RhostMUSH 3.2.4 p13 Update}
\label{\detokenize{changelog:rhostmush-3-2-4-p13-update}}
\sphinxAtStartPar
{[}02/01/2002{]}


\subsubsection{Changes}
\label{\detokenize{changelog:id25}}
\sphinxAtStartPar
Help was missing for @admin parameter areg\_lim.  \sphinxhyphen{} Added
Help was inconsistant for die() and dice().  \sphinxhyphen{} Changed
\begin{description}
\item[{/quiet switch to @notify.  Also added /quiet switch to @drain.}] \leavevmode\begin{itemize}
\item {} 
\sphinxAtStartPar
Thanks \sphinxhref{mailto:Hellspawn@MUX2}{Hellspawn@MUX2}

\end{itemize}

\end{description}

\sphinxAtStartPar
@cpattr modified so if no source given, assumes enactor (player) to be source.
lattr() now takes 3rd argument. ‘\$’ for all \$commands, ‘\textasciicircum{}’ for listens.
/clear switch added to @toggle.  It only clears what you have access to clear.
If there are more than 20 zones on an item, it displays the dbref\#’s only.
/\#\# notation added to @site and all @admin site information. (0\sphinxhyphen{}32 range)
ANSI highlight of @edit substitutions added.  Follows NO\_ANSI\_EX toggle.
Multiple ‘types’ now allowed in function remtype()
On buffer problems, it records the line number and file that it happened.
@list options handle sub\sphinxhyphen{}options now (config, mail, boolean, values)
Name field increased from 16 to 22 characters (MUX2/TM3/PENN)
/instant switch to @destroy (compatibility) \sphinxhyphen{} is the ‘default’ behavior. (TM3)
Immortals/\#1 can use mail/recall to see mail sent by others.
3rd argument to lrooms()’s to show level you’re in (MUX2)
LOGGING option to @flagdef.  Logs all setting/removing (configurable)
\begin{description}
\item[{Allow specifying filename to @dump/flat (filename restricted and ends in .flat)}] \leavevmode\begin{itemize}
\item {} 
\sphinxAtStartPar
Thanks \sphinxhref{mailto:Reptile@CotM}{Reptile@CotM}

\end{itemize}

\end{description}

\sphinxAtStartPar
Modification to @lock/GetFrom to include target’s location as well.
Options added to mail{[}/read{]} for more flexability (ball, nall, uall)
Ports listing added to lwho() (boolean 1/0 \sphinxhyphen{} 0 (null) default)
Ports added to idle() (boolean 1/0 and optional specified port)
Ports added to conn() (boolean 1/0 and optional specified port)
cmds() modified to handle target port
lattr() handles optional target of *player and player now.


\subsubsection{Additions}
\label{\detokenize{changelog:id26}}

\paragraph{Functions}
\label{\detokenize{changelog:id27}}
\sphinxAtStartPar
safebuff() \sphinxhyphen{} function to go back in the function until the first matched delimiter if strlen() is equal to or greater than the maximum.  (3998 characters)

\begin{sphinxadmonition}{note}{Note:}
\sphinxAtStartPar
this \_WILL\_ strip ansi.
\end{sphinxadmonition}

\sphinxAtStartPar
floordiv() \sphinxhyphen{} return the ‘floor’ (rounded down) of result of division. (MUX2)
last() \sphinxhyphen{} return last word specified by delimiter (MUX2)
singletime() \sphinxhyphen{} return time rounded to lowest element (s, m, h, d, etc) (MUX2)
parenmatch() \sphinxhyphen{} (2 args, 1 just for error) (Idea from ChaoticMUX) ansifies all bracket/paren/brace matches and RED’s nonmatch.
lrand() \sphinxhyphen{} (4 arguments) \sphinxhyphen{} returns random numbers between two points (MUX2)
keeptype() \sphinxhyphen{} does reverse of remtype()
lcmds() \sphinxhyphen{} List all commands ‘\$’ or listens ‘\textasciicircum{}’ on object (MUX2)
pack() \sphinxhyphen{} convert a number to base 2\sphinxhyphen{}64. (MUX2)
unpack() \sphinxhyphen{} convert a pack()’d number back to base 10 (MUX2)
crc32() \sphinxhyphen{} return a crc32 code for the specified string. (MUX2)
toggle() \sphinxhyphen{} works like the @toggle command.  Follows SIDEFX restrictions.
moneyname() \sphinxhyphen{} returns the singular/plural name based on argument (Discordia)
config() \sphinxhyphen{} if no argument, displays all parameters (you have access to), otherwise display the value of the specified parameter.


\paragraph{Commands}
\label{\detokenize{changelog:id28}}
\sphinxAtStartPar
@eval \sphinxhyphen{} force evaluation of functionality (TM3)


\paragraph{De\sphinxhyphen{}Powers}
\label{\detokenize{changelog:de-powers}}\begin{description}
\item[{mortal\_examine \sphinxhyphen{} if set, target always examines like a mortal. (@decompile/etc)}] \leavevmode\begin{itemize}
\item {} 
\sphinxAtStartPar
Thanks \sphinxhref{mailto:Belial@Armageddon}{Belial@Armageddon}

\end{itemize}

\end{description}


\paragraph{Toggles}
\label{\detokenize{changelog:id29}}
\sphinxAtStartPar
mail\_lockdown \sphinxhyphen{} target is restricted in mail ‘monitoring’ like a mortal.
muxpage \sphinxhyphen{} allows ‘p \textless{}blah\textgreater{}’ to work like in MUX/TM3


\subsubsection{Admin Params}
\label{\detokenize{changelog:id30}}
\sphinxAtStartPar
log (parameter) god \sphinxhyphen{} log all activity of \#1
heavy\_cpu\_max \sphinxhyphen{} ceilings heavilly used cpu\sphinxhyphen{}intensive functions.
lastsite\_paranoia \sphinxhyphen{} enable auto\sphinxhyphen{}register/auto\sphinxhyphen{}forbid of hosts spamming site.
max\_lastsite\_cnt \sphinxhyphen{} specify \# of connects in period of time to allow from site.
min\_con\_attempt \sphinxhyphen{} specify the wait between ‘first’ connect and subquent cons.
lattr\_default\_oldstyle \sphinxhyphen{} (default 0) snuffs the ‘\#\sphinxhyphen{}1 NO MATCH’ (TM3)
formats\_are\_local \sphinxhyphen{} localize @nameformat, @conformat, @exitformat
mail\_def\_object \sphinxhyphen{} default object for global aliases. Handle ‘alias.name’ and ‘comment.name’.
wizard\_queue\_limit \sphinxhyphen{} distinguish between wizard and mortals for queues.
max\_pcreate\_time \sphinxhyphen{} time range allowed before max\_pcreates reached.
max\_pcreate\_lim \sphinxhyphen{} number of pcreates allowed in given timeframe
pcreate\_paranoia \sphinxhyphen{} level of action you want to take against infidels (0\sphinxhyphen{}2)
global\_parent\_obj \sphinxhyphen{} global parent that attributes are inherited off of.


\subsubsection{Bug Fixes}
\label{\detokenize{changelog:id31}}
\sphinxAtStartPar
convtime() used daylight savings \sphinxhyphen{} no longer uses daylight savings \sphinxhyphen{} Fixed
IDESC wouldn’t work with Reality Levels \sphinxhyphen{} Fixed
Bad memory could corrupt command parsing with sockets \sphinxhyphen{} Fixed
@list alloc would scroll values negative \sphinxhyphen{} Fixed
@teleport and movement could ‘hide’ from wizards in inventories \sphinxhyphen{} Fixed
home, if @icmd’d, wouldn’t allow the command to be overridden \sphinxhyphen{} Fixed
The CLOAK flag would give ‘has left’ messages \sphinxhyphen{} Fixed
The lookup\_player() code now handles ‘*’ as well as not. \sphinxhyphen{} Fixed
If recycling was not enabled, you could still @destroy \sphinxhyphen{} Fixed
\begin{description}
\item[{Trace output was broke with how it displayed enactor/target \sphinxhyphen{} Fixed}] \leavevmode\begin{itemize}
\item {} 
\sphinxAtStartPar
Thanks \sphinxhref{mailto:Raevnos@M*U*S*H}{Raevnos@M*U*S*H}

\end{itemize}

\item[{NOSTOP flag on objects in the master room could cause an infinite loop \sphinxhyphen{} Fixed}] \leavevmode\begin{itemize}
\item {} 
\sphinxAtStartPar
Thanks \sphinxhref{mailto:Selene@TaintedEarth}{Selene@TaintedEarth}

\end{itemize}

\end{description}

\sphinxAtStartPar
QueueMax attribute would not work unless owned by the target. \sphinxhyphen{} Fixed
\begin{description}
\item[{DNS/AUTH lookups could have unforseen behavior with non\sphinxhyphen{}printable chars \sphinxhyphen{} Fixed}] \leavevmode\begin{itemize}
\item {} 
\sphinxAtStartPar
Thanks \sphinxhref{mailto:Morgan@RhostMUSH}{Morgan@RhostMUSH}

\end{itemize}

\item[{setqmatch() does not work like the help dictated it should. \sphinxhyphen{} Fixed}] \leavevmode\begin{itemize}
\item {} 
\sphinxAtStartPar
Thanks \sphinxhref{mailto:Belial@Armageddon}{Belial@Armageddon}

\end{itemize}

\item[{hastoggle() was wizard\sphinxhyphen{}only for no reason \sphinxhyphen{} Fixed}] \leavevmode\begin{itemize}
\item {} 
\sphinxAtStartPar
Thanks \sphinxhref{mailto:Belial@Armageddon}{Belial@Armageddon}

\end{itemize}

\item[{atof() could have buffer issues if string greater than 100 characters. \sphinxhyphen{} Fixed}] \leavevmode\begin{itemize}
\item {} 
\sphinxAtStartPar
Thanks \sphinxhref{mailto:Brazil@BrazilMUX}{Brazil@BrazilMUX}

\end{itemize}

\item[{Counts for input, output, and lost fields were not accurate \sphinxhyphen{} Fixed}] \leavevmode\begin{itemize}
\item {} 
\sphinxAtStartPar
Thanks \sphinxhref{mailto:Amos@RhostMUSH}{Amos@RhostMUSH}

\end{itemize}

\item[{@conncheck could have unpredicatable results with the name of the door. \sphinxhyphen{} Fixed}] \leavevmode\begin{itemize}
\item {} 
\sphinxAtStartPar
Thanks \sphinxhref{mailto:Amos@RhostMUSH}{Amos@RhostMUSH}

\end{itemize}

\item[{MONITOR information had some uninitialized values that could crash \sphinxhyphen{} Fixed}] \leavevmode\begin{itemize}
\item {} 
\sphinxAtStartPar
Thanks \sphinxhref{mailto:Amos@RhostMUSH}{Amos@RhostMUSH}

\end{itemize}

\item[{DARK flags with wizards so it won’t show wiz as being connected. \sphinxhyphen{} Fixed}] \leavevmode\begin{itemize}
\item {} 
\sphinxAtStartPar
Thanks \sphinxhref{mailto:Ian@Everywhere}{Ian@Everywhere}

\end{itemize}

\end{description}

\sphinxAtStartPar
mid() didn’t work like it did in MUX2/PENN.  It’s now an alias \sphinxhyphen{} Fixed
\begin{description}
\item[{type() could bypass cloaked targets \sphinxhyphen{} Fixed}] \leavevmode\begin{itemize}
\item {} 
\sphinxAtStartPar
Thanks \sphinxhref{mailto:Brazil@BrazilMUX}{Brazil@BrazilMUX}

\end{itemize}

\end{description}

\sphinxAtStartPar
quick\_wild() could cause some SIGSEGV’s if not accurately verified \sphinxhyphen{} Fixed
\begin{description}
\item[{inventory doesn’t handle Reality Levles corrected \sphinxhyphen{} Fixed}] \leavevmode\begin{itemize}
\item {} 
\sphinxAtStartPar
Thanks \sphinxhref{mailto:Zenty@BrazilMUX}{Zenty@BrazilMUX}

\end{itemize}

\end{description}

\sphinxAtStartPar
lcon(), next(), con(), lexits(), lcon(), xcon(), and sees() with reguards to handling Reality Levles were broke. \sphinxhyphen{} Fixed
\begin{description}
\item[{Help for trim() had examples with it’s 2nd and 3rds arguments reversed \sphinxhyphen{} Fixed}] \leavevmode\begin{itemize}
\item {} 
\sphinxAtStartPar
Thanks \sphinxhref{mailto:accela@AniMUSH}{accela@AniMUSH}

\end{itemize}

\end{description}


\subsection{RhostMUSH 3.2.4 p12 Update}
\label{\detokenize{changelog:rhostmush-3-2-4-p12-update}}
\sphinxAtStartPar
{[}06/15/2001{]}


\subsubsection{Changes}
\label{\detokenize{changelog:id32}}\begin{description}
\item[{Introduction of REALMS/Reality Levels}] \leavevmode\begin{itemize}
\item {} 
\sphinxAtStartPar
Thanks to \sphinxhref{mailto:Shade@Mediterranian}{Shade@Mediterranian} for the code

\end{itemize}

\end{description}

\sphinxAtStartPar
randextract() handles 5th arg for output and can handle ‘1’ arg passed
Cleaned up handling of ndbm.h
Improved validation for autoregistration
Improved logging
\begin{description}
\item[{Added total commands on @conncheck}] \leavevmode\begin{itemize}
\item {} 
\sphinxAtStartPar
Thanks \sphinxhref{mailto:Meglos@MyrkaMUX}{Meglos@MyrkaMUX} (Zenty’s concept)

\end{itemize}

\end{description}

\sphinxAtStartPar
Modified @aconnect/@adisconnect behavior.
@aconnect accepts \%0 as 1/0 based on existing connection
@adisconnect accepts \%0 (reason), \%1 (time on), \%2 (1/0 based on conn)
mail/status shows ‘\sphinxhyphen{}‘ on current message read.
@list cmdslogged \sphinxhyphen{} shows current individual commands being logged
@list powers \sphinxhyphen{} shows powers
@list depowers \sphinxhyphen{} shows depowers
@list rlevels \sphinxhyphen{} shows rlevels
Added ‘+proof’ to mail/write command
Cleaned up help with the mail system for easier understanding
Added protection to functions that could be abused with poor coding practices
\begin{description}
\item[{Added HILIGHT’ing to puppets when displaying messages.}] \leavevmode\begin{itemize}
\item {} 
\sphinxAtStartPar
Thanks to \sphinxhref{mailto:Rynos@Armageddon}{Rynos@Armageddon}

\end{itemize}

\item[{Added LastIP attribute to house the IP address the person last connected from.}] \leavevmode\begin{itemize}
\item {} 
\sphinxAtStartPar
Thanks \sphinxhref{mailto:Jeff@Sandbox}{Jeff@Sandbox}

\end{itemize}

\item[{Help fixes for wizhelp on permissions.}] \leavevmode\begin{itemize}
\item {} 
\sphinxAtStartPar
Thanks \sphinxhref{mailto:Jeff@Sandbox}{Jeff@Sandbox}

\end{itemize}

\end{description}

\sphinxAtStartPar
Help fixes for help with regards to arbitrary commands.
Improved DoS protection vers. race conditions on connect screen.
switch() and switchall() now recognize \textgreater{} and \textless{} as math\sphinxhyphen{}args (config param)


\subsubsection{Additions}
\label{\detokenize{changelog:id33}}

\paragraph{Functions}
\label{\detokenize{changelog:id34}}
\sphinxAtStartPar
rxlevel() \sphinxhyphen{} return RX reality level for source check
txlevel() \sphinxhyphen{} return TX reality level for target check
listrlevels() \sphinxhyphen{} list all levels
hasrxlevel() \sphinxhyphen{} return ‘1’ if target has rxlevel()
hastxlevel() \sphinxhyphen{} return ‘1’ if target has txlevel()
cansee() \sphinxhyphen{} return ‘1’ if target can see source (reality level based only)


\paragraph{Commands}
\label{\detokenize{changelog:id35}}
\sphinxAtStartPar
@rxlevel \sphinxhyphen{} set/remove reality level source check
@txlevel \sphinxhyphen{} set/remove reality level target check
mrpage (mrp) \sphinxhyphen{} respond to the player list that you received in a page.  like rpage, this is seperate from lpage


\paragraph{Flags}
\label{\detokenize{changelog:id36}}
\sphinxAtStartPar
NONAME \sphinxhyphen{} if set on a target, the name is not displayed.


\subsubsection{Admin Params}
\label{\detokenize{changelog:id37}}
\sphinxAtStartPar
reality\_level     \sphinxhyphen{} define new reality level(s).  32 max.
wiz\_always\_real   \sphinxhyphen{} a wizard is defined as always seeing everything as ‘real’
def\_exit\_rx       \sphinxhyphen{} default exit RX value (1 default)
def\_exit\_tx       \sphinxhyphen{} default exit TX value (1 default)
def\_room\_rx       \sphinxhyphen{} default room RX value (1 default)
def\_room\_tx       \sphinxhyphen{} default room TX value (1 default)
def\_player\_rx     \sphinxhyphen{} default room RX value (1 default)
def\_player\_tx     \sphinxhyphen{} default room TX value (1 default)
def\_thing\_rx      \sphinxhyphen{} default thing RX value (1 default)
def\_thing\_tx      \sphinxhyphen{} default thing TX value (1 default)
\begin{description}
\item[{validate\_host     \sphinxhyphen{} Specify invalid site\sphinxhyphen{}masks for autoregistration emails}] \leavevmode\begin{itemize}
\item {} 
\sphinxAtStartPar
Thanks \sphinxhref{mailto:Zara@UnderGround}{Zara@UnderGround} Labyrinth

\end{itemize}

\end{description}

\sphinxAtStartPar
log\_command\_list  \sphinxhyphen{} Specify commands to individually log
partial\_conn      \sphinxhyphen{} Enable/disable @aconnect on partial connects
partial\_deconn    \sphinxhyphen{} Enable/disable @adisconnects on partial disconnects
secure\_functions  \sphinxhyphen{} Enable/disable security on listed functions FOREACH, WHILE, FOLD, FILTER, MAP, STEP, and MIX
max\_logins\_allowed (mudstate) \sphinxhyphen{} absolute ceiling of connections allowed to avoid any type of DoS based attack.  This will always be 10 less than the OS can handle.
penn\_switches     \sphinxhyphen{} if enabled, switch() and switchall() will work like PENN and accept \textless{} and \textgreater{} as mathmatical operands.


\subsubsection{Bug Fixes}
\label{\detokenize{changelog:id38}}\begin{description}
\item[{filter() would accept ‘1\textless{}blah\textgreater{}’ as well as ‘1’.  \sphinxhyphen{} Fixed.}] \leavevmode\begin{itemize}
\item {} 
\sphinxAtStartPar
Thanks \sphinxhref{mailto:Raevnos@M*U*S*H}{Raevnos@M*U*S*H}

\end{itemize}

\end{description}

\sphinxAtStartPar
possible pointer\sphinxhyphen{}misalignment with internal messaging with MONITOR \sphinxhyphen{} Fixed
Rare occurance of getting inside a garbage object (non\sphinxhyphen{}crash\sphinxhyphen{}bug) \sphinxhyphen{} Fixed
@mvattr had small bug with copying over itself \sphinxhyphen{} Fixed
@mvattr had small bug with keeping original copy \sphinxhyphen{} Fixed
Semaphores didn’t do wait time correctly (introduced in p11) \sphinxhyphen{} Fixed
Time/Create stamps visible by anyone. Shouldn’t be. \sphinxhyphen{} Fixed
LBUF not freed in mail/quota \sphinxhyphen{} Fixed
@dynhelp not able to parse subdirectories \sphinxhyphen{} Fixed (using a ‘\textasciicircum{}’)
@icmd not inherited in all instances \sphinxhyphen{} Fixed for everything
Some side effects had wrong security/permission checks \sphinxhyphen{} Fixed
Side effects did not check inheritance for permission \sphinxhyphen{} Fixed
Iter()/list() and suite did not handle ansi properly \sphinxhyphen{} Fixed


\subsection{RhostMUSH 3.2.4 p11 Update}
\label{\detokenize{changelog:rhostmush-3-2-4-p11-update}}
\sphinxAtStartPar
{[}03/15/2001{]}


\subsubsection{Changes}
\label{\detokenize{changelog:id39}}\begin{description}
\item[{elist() has 5th argument (for string seperator)}] \leavevmode\begin{itemize}
\item {} 
\sphinxAtStartPar
Thanks \sphinxhref{mailto:Talek@M*U*S*H}{Talek@M*U*S*H}

\end{itemize}

\end{description}

\sphinxAtStartPar
@icmd has new switches to handle location/zone overriding:

\begin{sphinxVerbatim}[commandchars=\\\{\}]
\PYG{o}{/}\PYG{n}{droom} \PYG{o}{\PYGZhy{}} \PYG{n}{disable} \PYG{n}{commands} \PYG{n}{on} \PYG{n}{room}
\PYG{o}{/}\PYG{n}{iroom} \PYG{o}{\PYGZhy{}} \PYG{n}{ignore} \PYG{n}{commands} \PYG{n}{on} \PYG{n}{room}
\PYG{o}{/}\PYG{n}{croom} \PYG{o}{\PYGZhy{}} \PYG{n}{clear} \PYG{n}{commands} \PYG{n}{on} \PYG{n}{room}
\PYG{o}{/}\PYG{n}{lroom} \PYG{o}{\PYGZhy{}} \PYG{n+nb}{list} \PYG{n}{commands} \PYG{n}{on} \PYG{n}{room}
\PYG{o}{/}\PYG{n}{lallroom} \PYG{o}{\PYGZhy{}} \PYG{n+nb}{list} \PYG{n+nb}{all} \PYG{n}{commands} \PYG{n}{at} \PYG{n}{current} \PYG{n}{location}
\end{sphinxVerbatim}

\sphinxAtStartPar
DoS Protection increased for vattr’s and object creation/destruction
@wait giving new /until switch to specify exact static time
strmath() has two new args.  First handles start location, second count
Mail uses \textasciitilde{} to evaluate attributes for player targets
\begin{description}
\item[{Enhanced email authorization for autoregistration.}] \leavevmode\begin{itemize}
\item {} 
\sphinxAtStartPar
Thanks \sphinxhref{mailto:Jeff@SandBox}{Jeff@SandBox}

\end{itemize}

\item[{Enhanced page to handle alias displaying}] \leavevmode\begin{itemize}
\item {} 
\sphinxAtStartPar
Thanks \sphinxhref{mailto:Wolfie@CotM}{Wolfie@CotM}

\end{itemize}

\end{description}

\sphinxAtStartPar
Pagelock to follow normal methodology when wizard\sphinxhyphen{}overrides in effect
\begin{description}
\item[{Improved mail/write +list (proof) interface for mail with BRANDY\_MAIL toggle.}] \leavevmode\begin{itemize}
\item {} 
\sphinxAtStartPar
Thanks \sphinxhref{mailto:Belial@Armegeddon}{Belial@Armegeddon}

\end{itemize}

\end{description}

\sphinxAtStartPar
@pemit has /silent switch for PENN compatibility


\subsubsection{Additions}
\label{\detokenize{changelog:id40}}

\paragraph{Functions}
\label{\detokenize{changelog:id41}}
\sphinxAtStartPar
foreach() \sphinxhyphen{} (MUX) added for compatibility
ilev() \sphinxhyphen{} Returns nest level of an iter()


\paragraph{Commands}
\label{\detokenize{changelog:id42}}
\sphinxAtStartPar
@limit {[}/vmod /dmod /list /reset{]} \sphinxhyphen{} to limit max @destroy/vattr creation


\paragraph{Toggles}
\label{\detokenize{changelog:id43}}
\sphinxAtStartPar
IGNOREZONE \sphinxhyphen{} toggles zone/location to enable zone/location command disable/ignore
VPAGE \sphinxhyphen{} toggles aliases to show/not show in pages you receive.
\begin{description}
\item[{PAGELOCK \sphinxhyphen{} toggles pages to normal methodologies when wizard lock overriding}] \leavevmode\begin{itemize}
\item {} 
\sphinxAtStartPar
Thanks \sphinxhref{mailto:Jeff@Sandbox}{Jeff@Sandbox}

\end{itemize}

\end{description}

\sphinxAtStartPar
MAIL\_NOPARSE \sphinxhyphen{} seperate toggle to break up translating \%r’s and \%t’s in mail viewing.


\subsubsection{Admin Params}
\label{\detokenize{changelog:id44}}
\sphinxAtStartPar
ignore\_zone \sphinxhyphen{} globally specify zone ignoring for a command

\begin{sphinxadmonition}{note}{Note:}
\sphinxAtStartPar
ignore\_zone allows all other ignore\_* params for level specifiers.)
\end{sphinxadmonition}

\sphinxAtStartPar
disable\_zone \sphinxhyphen{} globally specify zone disabling for a command
global\_ansimask \sphinxhyphen{} globally define what ansi codes to allow/deny
expand\_goto \sphinxhyphen{} force exit movement to use a ‘goto’.
max\_dest\_limit \sphinxhyphen{} specify maximum @destroys allowed per player
max\_vattr\_limit \sphinxhyphen{} specify maximum \_NEW\_ vattrs allowed per player
\begin{description}
\item[{hide\_nospoof \sphinxhyphen{} if ‘1’ you have to control target to see NOSPOOF flag}] \leavevmode\begin{itemize}
\item {} 
\sphinxAtStartPar
Thanks \sphinxhref{mailto:Jeff@SandBox}{Jeff@SandBox}

\end{itemize}

\end{description}


\subsubsection{Bug Fixes}
\label{\detokenize{changelog:id45}}
\sphinxAtStartPar
buffer issue with @decompile \sphinxhyphen{} fixed
\begin{description}
\item[{remove lmath() from help. \sphinxhyphen{} fixed}] \leavevmode\begin{itemize}
\item {} 
\sphinxAtStartPar
Thanks \sphinxhref{mailto:Raevnos@M*U*S*H}{Raevnos@M*U*S*H}

\end{itemize}

\item[{mail showed cloaked wizzes being connected \sphinxhyphen{} fixed}] \leavevmode\begin{itemize}
\item {} 
\sphinxAtStartPar
Thanks \sphinxhref{mailto:Morgan@RhostMUSH}{Morgan@RhostMUSH} Team

\end{itemize}

\item[{moon() didn’t handle full moons properly \sphinxhyphen{} fixed}] \leavevmode\begin{itemize}
\item {} 
\sphinxAtStartPar
Thanks \sphinxhref{mailto:Jeff@SandBox}{Jeff@SandBox}

\end{itemize}

\item[{isdbref() thought ‘\#’ was a valid dbref. \sphinxhyphen{} fixed}] \leavevmode\begin{itemize}
\item {} 
\sphinxAtStartPar
Thanks \sphinxhref{mailto:Raevnos@M*U*S*H}{Raevnos@M*U*S*H}

\end{itemize}

\item[{Attribute setting issue introduced with P11}] \leavevmode\begin{itemize}
\item {} 
\sphinxAtStartPar
MUCH Thanks \sphinxhref{mailto:Jeff@SandBox}{Jeff@SandBox}

\end{itemize}

\item[{Logging options were a bit skewered with on/off settings. \sphinxhyphen{} fixed}] \leavevmode\begin{itemize}
\item {} 
\sphinxAtStartPar
Thanks \sphinxhref{mailto:Sorien@Delphi}{Sorien@Delphi}

\end{itemize}

\end{description}

\sphinxAtStartPar
Attempted to free mbufs in @disable login area twice.  \sphinxhyphen{} fixed
\begin{description}
\item[{Parenting fix with possible recursion \sphinxhyphen{} fixed}] \leavevmode\begin{itemize}
\item {} 
\sphinxAtStartPar
Thanks \sphinxhref{mailto:Brazil@MUX2}{Brazil@MUX2}

\end{itemize}

\end{description}

\sphinxAtStartPar
@lock/chown not documented in help \sphinxhyphen{} fixed
\begin{description}
\item[{@lock/chown required on CHOWN\_OK object.  Shouldn’t be.  \sphinxhyphen{} Fixed}] \leavevmode
\sphinxAtStartPar
\sphinxhyphen{}Thanks \sphinxhref{mailto:Rynos@Armageddon}{Rynos@Armageddon}

\item[{@icmd for players broke with room/location addition \sphinxhyphen{} fixed.}] \leavevmode
\sphinxAtStartPar
\sphinxhyphen{}Thanks \sphinxhref{mailto:Jeff@SandBox}{Jeff@SandBox}

\item[{Cloaked items didn’t trigger @startup \sphinxhyphen{} fixed.}] \leavevmode
\sphinxAtStartPar
\sphinxhyphen{}Thanks \sphinxhref{mailto:Rynos@Armageddon}{Rynos@Armageddon}

\end{description}


\subsection{RhostMUSH 3.2.4 p10 Update}
\label{\detokenize{changelog:rhostmush-3-2-4-p10-update}}
\sphinxAtStartPar
{[}02/15/2001{]}


\subsubsection{Changes}
\label{\detokenize{changelog:id46}}
\sphinxAtStartPar
improved CPU handler for DoS protection


\subsubsection{Additions}
\label{\detokenize{changelog:id47}}

\paragraph{Functions}
\label{\detokenize{changelog:id48}}
\sphinxAtStartPar
strfunc(\textless{}function\textgreater{},\textless{}string of args\textgreater{}{[},\textless{}optional seperator for args\textgreater{}) \sphinxhyphen{} for lists


\subsubsection{Admin Params}
\label{\detokenize{changelog:id49}}
\sphinxAtStartPar
max\_cpu\_cycles admin param (default 3)
cpu\_secure\_lvl admin param (default 0)


\subsubsection{Bug Fixes}
\label{\detokenize{changelog:id50}}
\sphinxAtStartPar
you could still get into a garbage object \sphinxhyphen{} fixed (MINOR)
forwardlists were not cleared on @wipe \sphinxhyphen{} fixed


\subsection{RhostMUSH 3.2.4 p9 Update}
\label{\detokenize{changelog:rhostmush-3-2-4-p9-update}}
\sphinxAtStartPar
{[}12/15/2000{]}


\subsubsection{Changes}
\label{\detokenize{changelog:id51}}
\sphinxAtStartPar
mail/write +acc added to allow extending player lists and not replacing.
\begin{description}
\item[{mail/recall{[}/all{]} \textless{}playerlist\textgreater{} to list messages by player\sphinxhyphen{}name you sent to}] \leavevmode\begin{itemize}
\item {} 
\sphinxAtStartPar
Suggested by \sphinxhref{mailto:Julius@Bermuda}{Julius@Bermuda}

\end{itemize}

\end{description}

\sphinxAtStartPar
functionality for @flagdef for flagname filtering on display and paging.  Cleaned up some functionality more cases to valid() (name, attrname, playername)
\begin{description}
\item[{filler to columns() (new argument)}] \leavevmode\begin{itemize}
\item {} 
\sphinxAtStartPar
Thanks \sphinxhref{mailto:Morgan@BrazilMUX}{Morgan@BrazilMUX}

\end{itemize}

\item[{‘save’ to helpfile showing that you do not need to save}] \leavevmode\begin{itemize}
\item {} 
\sphinxAtStartPar
Suggested by \sphinxhref{mailto:Jamie@AdminMUSH}{Jamie@AdminMUSH}

\end{itemize}

\end{description}


\subsubsection{Additions}
\label{\detokenize{changelog:id52}}

\paragraph{Functions}
\label{\detokenize{changelog:id53}}
\sphinxAtStartPar
caseall() \sphinxhyphen{} works like switchall() but for case (Taken from PENN)
cand() \sphinxhyphen{} works like and() but stops processing on first ‘FALSE’ (from PENN)
cor() \sphinxhyphen{} works like or() but stops processing on first ‘TRUE’ (from PENN)
moon() \sphinxhyphen{} based on the POM code from berkley
isword() \sphinxhyphen{} returns ‘TRUE’ if entire string is all alpha (or has a hyphon)
itext() \sphinxhyphen{} returns nth arg (text ‘\#\#’) of an iter() (from PENN)
inum() \sphinxhyphen{} returns nth arg (num ‘\#@’) of an iter() (from PENN)
modulo() \sphinxhyphen{} returns the modulus of the numbers (from PENN)


\paragraph{Flags}
\label{\detokenize{changelog:id54}}
\sphinxAtStartPar
NO\_PARSE \sphinxhyphen{} attribute flag that stops processing/evaluation of \%0\sphinxhyphen{}\%9 in \$commands.
SAFE \sphinxhyphen{} attribute flag that stops modification of attribute it’s set on (TM 3.0)
SHOWFAILCMD \sphinxhyphen{} when set on object any matching failed \$cmd uses the @ufail suite.


\paragraph{Toggles}
\label{\detokenize{changelog:id55}}
\sphinxAtStartPar
MAIL\_STRIPRETURN \sphinxhyphen{} when combining lines uses spaces instead of carrage returns
PENN\_MAIL \sphinxhyphen{} when sending mail, use PENN like style


\subsubsection{Admin Params}
\label{\detokenize{changelog:id56}}
\sphinxAtStartPar
guest\_namelist \sphinxhyphen{} specifies a dynamic namelist for guests (with protection)
hackattr\_nowiz \sphinxhyphen{} defines if ‘\_attr’ is wiz only or follows normal rules
hackattr\_see \sphinxhyphen{} defines if ‘\_attr’ is viewable by wizard only.
penn\_playercmds \sphinxhyphen{} \$commands on player work only for that player or inventory (from PENN)


\subsubsection{Bug Fixes}
\label{\detokenize{changelog:id57}}\begin{description}
\item[{Bug with dynhelp().  Player (3rd arg) never resolved correctly \sphinxhyphen{} fixed}] \leavevmode\begin{itemize}
\item {} 
\sphinxAtStartPar
Thanks \sphinxhref{mailto:Svlatmaer@BTW}{Svlatmaer@BTW}

\end{itemize}

\end{description}

\sphinxAtStartPar
Possible (and very rare) overflow conditions with certain wiz commands \sphinxhyphen{} fixed
SIGSEGV bug with recursive @functions \sphinxhyphen{} fixed
When @toggled MONITOR\_BAD, MONITOR\_FAIL doubled up failed connections \sphinxhyphen{} fixed
Malaligned pointer in command.c.  Non\sphinxhyphen{}crash issue.  \sphinxhyphen{} fixed
\begin{description}
\item[{setq() registers wern’t cleared for extreamly fast command input \sphinxhyphen{} fixed}] \leavevmode\begin{itemize}
\item {} 
\sphinxAtStartPar
Thanks \sphinxhref{mailto:Brazil@BrazilMUX}{Brazil@BrazilMUX}

\end{itemize}

\end{description}

\sphinxAtStartPar
ex thing/\sphinxstyleemphasis{sta} returns ‘no attribs found’ if tstamps enabled \sphinxhyphen{} fixed
enhanced security for .txt file reading/verification (possible DoS) \sphinxhyphen{} fixed
attrib flag DARK could not be unset by non\sphinxhyphen{}god.  Only \#1 can set it now \sphinxhyphen{} fixed
encrypt/decrypt really mangled with how it passed key \sphinxhyphen{} fixed


\subsection{RhostMUSH 3.2.4 p8 Update}
\label{\detokenize{changelog:rhostmush-3-2-4-p8-update}}
\sphinxAtStartPar
{[}10/12/2000{]}


\subsubsection{Changes}
\label{\detokenize{changelog:id58}}
\sphinxAtStartPar
lnum() and lnum2() now can handle negative numbers.
@list options shows if the COMMAND flag is used.
NOMODIFY can be configured to be settable/unsettable/modifiable by imm only.
vector functions (vadd, etc) that return a list now recognize output seps.
MAX\_ARGS in ‘externs.h’ may be changed to increase the total number of arguments functions like switch() may take.  MAX\_ARGS is currently defaulted to ‘30’ though MUX 2.0 has it set to ‘100’.  This HAS to be a compile time change else possible SIGSEGV’s could happen.
mail/write +editall=old,new \sphinxhyphen{} argument to mail/write added. Modifies ALL lines.
mail/write +feditall=old,new \sphinxhyphen{} modify ALL matches on ALL lines.
mail/write +fedit \#=old,new \sphinxhyphen{} modify ALL matches on given line.
mail/write +cc/+bcc=new \sphinxhyphen{} if FORWARDING or SENDING (not reply!) will allow the user to redefine who the message is being sent out to.
\_ATTR can be compile\sphinxhyphen{}time added to be wiz\sphinxhyphen{}only settable/modifiable attributes.
Modified mail/write +justify, +insert, +edit, +editall to rehandle line count
Made option to make harder password guessing.
dice() takes ‘1’ for expanded, ‘2’ for expanded w/o totals, and optional output seperator.  Also has optional argument for low\sphinxhyphen{}limit Can mimic TM 3.0’s lrand() function.
modified connecting to @program based on toggle.
added modified/created with ‘examine’ and ‘examine/brief’ and ‘examine thing/*’


\subsubsection{Additions}
\label{\detokenize{changelog:id59}}

\paragraph{Functions}
\label{\detokenize{changelog:id60}}\begin{description}
\item[{mailalias() \sphinxhyphen{} returns the dbref\# list of players in the given global mail alias}] \leavevmode\begin{itemize}
\item {} 
\sphinxAtStartPar
Suggested by \sphinxhref{mailto:Stormwolf@CotF}{Stormwolf@CotF}

\end{itemize}

\end{description}

\sphinxAtStartPar
programmer() \sphinxhyphen{} returns the dbref\# of thing that put target in @program
vcross() \sphinxhyphen{} taken from TinyMUX 2.0 (with permission)
switchall() \sphinxhyphen{} idea taken from TinyMUSH 3.0 (coded from scratch)
brackets() \sphinxhyphen{} idea taken from PENN 1.7.3
@@() \sphinxhyphen{} works like null() but does not evaluate (idea from PENN 1.7.3)


\paragraph{Commands}
\label{\detokenize{changelog:id61}}
\sphinxAtStartPar
@flagdef \sphinxhyphen{} define flag permission levels (/set, /unset, /see)
@dynhelp \sphinxhyphen{} dynamically read a specified \textless{}file\textgreater{}.txt file (needs matching .indx!)


\paragraph{Flags}
\label{\detokenize{changelog:id62}}\begin{description}
\item[{COMMANDS \sphinxhyphen{} optional \#ifdef to define what uses commands or not (TinyMUSH compat) uses \sphinxhyphen{}DENABLE\_COMMAND\_FLAG}] \leavevmode\begin{itemize}
\item {} 
\sphinxAtStartPar
Idea from TinyMUSH 2.2.4

\end{itemize}

\end{description}

\sphinxAtStartPar
NO\_CLONE \sphinxhyphen{} attribute flag to stop attributes from copying over when @cloned
\begin{description}
\item[{MARKER0 through MARKER9}] \leavevmode\begin{itemize}
\item {} 
\sphinxAtStartPar
idea from TinyMUSH 3.0

\end{itemize}

\end{description}

\sphinxAtStartPar
BOUNCE \sphinxhyphen{} allows players to pass what they hear to their contents w/o effecting @ahear/@amhear/etc.


\paragraph{Toggles}
\label{\detokenize{changelog:id63}}
\sphinxAtStartPar
MONITOR\_BAD \sphinxhyphen{} monitor failed connections/creations to non\sphinxhyphen{}existant players
PROG\_ON\_CONNECT \sphinxhyphen{} reverses the current global logic of program and connecting.


\subsubsection{Admin Params}
\label{\detokenize{changelog:id64}}
\sphinxAtStartPar
imm\_nomod \sphinxhyphen{} when enabled (default disabled) specifies that only immortal can set/unset the NOMODIFY flag and only immortals can modify things set NOMODIFY.  This will allow NOMODIFY to mimic CONSTANT ala TM 3.0

\sphinxAtStartPar
paranoid\_exit\_linking \sphinxhyphen{} when enabled, you MUST control exit to link.  This includes UNLINKED exits.  Also, exits won’t be @chowned to you automatically. (Default disabled)

\sphinxAtStartPar
notonerr\_return \sphinxhyphen{} when DISABLED (default enabled) not() returns a ‘0’ for any function returning \#\sphinxhyphen{}1.
\begin{description}
\item[{safer\_passwords \sphinxhyphen{} requires passwords be 5+ chars long and have one upper, one lower, and one special character.}] \leavevmode\begin{itemize}
\item {} 
\sphinxAtStartPar
Idea from TinyMUSH 2.2.4

\end{itemize}

\end{description}

\sphinxAtStartPar
max\_sitecons \sphinxhyphen{} (default 50) specifies the maximum \# of port connections to site of that given site.


\subsubsection{Bug Fixes}
\label{\detokenize{changelog:id65}}
\sphinxAtStartPar
NOMODIFY wasn’t strict enough \sphinxhyphen{} fixed
BACKSTAGE/NOBACKSTAGE didn’t handle @zones \sphinxhyphen{} fixed
create() didn’t check command permissions first for @dig, @open, and @pcreate
put a temp fix for a possible SIGSEGV bug.


\subsection{RhostMUSH 3.2.4 p7 Update}
\label{\detokenize{changelog:rhostmush-3-2-4-p7-update}}
\sphinxAtStartPar
{[}07/15/2000{]}


\subsubsection{Changes}
\label{\detokenize{changelog:id66}}
\sphinxAtStartPar
@set now recognizes the following attribute flags:

\begin{sphinxVerbatim}[commandchars=\\\{\}]
\PYG{n}{GOD}
\PYG{n}{IMMORTAL}
\PYG{n}{WIZARD} \PYG{p}{(}\PYG{n}{suggested} \PYG{n}{by} \PYG{n}{Stormwolf}\PYG{n+nd}{@CotF}\PYG{p}{)}
\PYG{n}{ROYALTY} \PYG{p}{(}\PYG{n}{alias} \PYG{k}{for} \PYG{n}{WIZARD}\PYG{p}{)}
\PYG{n}{COUNCILOR}
\PYG{n}{ARCHITECT}
\PYG{n}{GUILDMASTER}
\PYG{n}{HIDDEN}
\PYG{n}{DARK}
\end{sphinxVerbatim}


\subsubsection{Admin Params}
\label{\detokenize{changelog:id67}}

\subsubsection{Bug Fixes}
\label{\detokenize{changelog:id68}}\begin{description}
\item[{@list options did not show RESTRICT\_HOME config setting.  Fixed}] \leavevmode\begin{itemize}
\item {} 
\sphinxAtStartPar
Reported by \sphinxhref{mailto:Mercutio@ShatteredCathedrals}{Mercutio@ShatteredCathedrals}

\end{itemize}

\end{description}


\subsection{RhostMUSH 3.2.4 p6 Update}
\label{\detokenize{changelog:rhostmush-3-2-4-p6-update}}
\sphinxAtStartPar
{[}06/16/2000{]}


\subsubsection{Changes}
\label{\detokenize{changelog:id69}}
\sphinxAtStartPar
@list site\_information now shows DNS blocking


\subsubsection{Admins Params}
\label{\detokenize{changelog:admins-params}}
\sphinxAtStartPar
forbid\_host \sphinxhyphen{} allows DNS entry/modify for forbidding hosts
register\_host \sphinxhyphen{} allows DNS entry/modify for registered hosts
autoreg\_host \sphinxhyphen{} allows DNS entry/modify for disallowing autoreg to hosts
noguest\_host \sphinxhyphen{} allows DNS entry/modify for disallowing guests to hosts
suspect\_host \sphinxhyphen{} allows DNS entry/modify for suspect hosts


\subsubsection{Bug Fixes}
\label{\detokenize{changelog:id70}}\begin{description}
\item[{ansi() didn’t recognize ‘u’ for underline.  Fixed.}] \leavevmode\begin{itemize}
\item {} 
\sphinxAtStartPar
Reported by \sphinxhref{mailto:LadyDraconis@BermudaByNight}{LadyDraconis@BermudaByNight}

\end{itemize}

\end{description}

\sphinxAtStartPar
@program had small issue with prompt storage.  Fixed.
\begin{description}
\item[{autoreg file inclusion didn’t count total lines right.  Fixed.}] \leavevmode\begin{itemize}
\item {} 
\sphinxAtStartPar
Reported by \sphinxhref{mailto:Rachel@AdminMUSH}{Rachel@AdminMUSH}

\end{itemize}

\end{description}

\sphinxAtStartPar
hastoggles(), haspowers(), and hasdepowers() didn’t work.  Fixed.


\subsection{RhostMUSH 3.2.4 p5 Update}
\label{\detokenize{changelog:rhostmush-3-2-4-p5-update}}
\sphinxAtStartPar
{[}06/01/2000{]}


\subsubsection{Changes}
\label{\detokenize{changelog:id71}}
\sphinxAtStartPar
New /noauth and /nodns switches to @site
Better timeout conditions for AUTH lookups.
@list site\_information now shows additional information


\subsubsection{Admin Params}
\label{\detokenize{changelog:id72}}
\sphinxAtStartPar
noauth\_site \textendash{} specifies site and mask that AUTH lookups are not to be performed
nodns\_site  \textendash{} specifies site and mask that DNS lookups are not to be performed


\subsubsection{Bug Fixes}
\label{\detokenize{changelog:id73}}
\sphinxAtStartPar
AUTH hung if remote server was configured wrong.  Fix put in around the OS
limitation to handle this.
Command substitution (\%c/\%x) could cause a SIGSEGV at random times.  Fixed.


\subsection{RhostMUSH 3.2.4 p4 Update}
\label{\detokenize{changelog:rhostmush-3-2-4-p4-update}}
\sphinxAtStartPar
{[}04/15/2000{]}


\subsubsection{Changes}
\label{\detokenize{changelog:id74}}
\sphinxAtStartPar
wizhelp has new topic for setting up guests (GUEST SETUP)
\begin{description}
\item[{help has help for differences/useful (DIFFERENCE/USEFUL)}] \leavevmode\begin{itemize}
\item {} 
\sphinxAtStartPar
Suggested by \sphinxhref{mailto:Sycorax@ShatteredCathedrals}{Sycorax@ShatteredCathedrals}

\end{itemize}

\end{description}

\sphinxAtStartPar
filter() now supports an output seperator
\begin{description}
\item[{lnum() and lnum2() optionally return NULL if given a ‘0’.}] \leavevmode\begin{itemize}
\item {} 
\sphinxAtStartPar
Suggested by \sphinxhref{mailto:Gorath@Rhostshyl}{Gorath@Rhostshyl}

\end{itemize}

\end{description}

\sphinxAtStartPar
mask() now takes ‘\textasciitilde{}’ for adding 1’s comp, ‘1’ for 1’s and ‘2’ for 2’s.
mailquick() takes 3rd argument for MUX mail() compatibility.
news/articlelife takes ‘forever’ as a valid argument.
mail/status takes /subject, (U)nread, (N)ew, (B)oth new/unread, (S)aved, (M)arked, and (O)ld mail as optional search params.
mail/number takes same new args as mail/status
news/read now marks messages as ‘read’.
@ansiname now allows raw ansi (under controlled circumstances)
\begin{description}
\item[{dig(), create(), open(), clone() all optionally return dbref\#’s.}] \leavevmode\begin{itemize}
\item {} 
\sphinxAtStartPar
Suggested by \sphinxhref{mailto:Corum@Underground}{Corum@Underground}

\end{itemize}

\end{description}

\sphinxAtStartPar
@list options now shows more (and valuable) information.
@function/list now shows flags for privalaged/preserved functions.
mail/status and mail/read now show connected players.
mail/write and \sphinxhyphen{} now show how many characters you have written.
mail/forward and mail/reply now recognize the BRANDY\_MAIL @toggle.


\subsubsection{Additions}
\label{\detokenize{changelog:id75}}

\paragraph{Functions}
\label{\detokenize{changelog:id76}}
\sphinxAtStartPar
listmatch(\textless{}string\textgreater{},\textless{}wildcard\textgreater{}{[},\textless{}delimiter\textgreater{}{]})
setqmatch(\textless{}string\textgreater{},\textless{}wildcard\textgreater{}{[},\textless{}delimiter\textgreater{}{]})
listnewsgroups({[}\textless{}player\textgreater{}{]})
inprogram(player)


\paragraph{Commands}
\label{\detokenize{changelog:id77}}
\sphinxAtStartPar
@program (idea from MUX)
@quitprogram (idea from MUX)
\begin{description}
\item[{@progprompt}] \leavevmode\begin{itemize}
\item {} 
\sphinxAtStartPar
Suggested by \sphinxhref{mailto:Zara@Underground}{Zara@Underground}

\end{itemize}

\end{description}

\sphinxAtStartPar
@extansi
\begin{description}
\item[{train}] \leavevmode\begin{itemize}
\item {} 
\sphinxAtStartPar
Suggested by \sphinxhref{mailto:Trey@GameHendge}{Trey@GameHendge}

\end{itemize}

\end{description}

\sphinxAtStartPar
@function/preserve (idea from MUX)
+help (hardcode) (idea from MUX)


\paragraph{Toggles}
\label{\detokenize{changelog:id78}}
\sphinxAtStartPar
PROG (idea from MUX)
NOSHPROG
IMMPROG


\subsubsection{Alias Additions}
\label{\detokenize{changelog:alias-additions}}

\paragraph{Commands}
\label{\detokenize{changelog:id79}}
\sphinxAtStartPar
@prog aliased to @program


\subsubsection{Admin Params}
\label{\detokenize{changelog:id80}}
\sphinxAtStartPar
login\_to\_prog
\begin{description}
\item[{noshell\_prog}] \leavevmode\begin{itemize}
\item {} 
\sphinxAtStartPar
Suggested by \sphinxhref{mailto:Draken-Korin@Underground}{Draken\sphinxhyphen{}Korin@Underground}

\end{itemize}

\end{description}

\sphinxAtStartPar
sidefx\_returnval
nospam\_connect (idea from PENN)
\begin{description}
\item[{noregist\_onwho}] \leavevmode\begin{itemize}
\item {} 
\sphinxAtStartPar
Suggested by \sphinxhref{mailto:Rachel@AdminMUSH}{Rachel@AdminMUSH}

\end{itemize}

\end{description}

\sphinxAtStartPar
lnum\_compat
\begin{description}
\item[{mailinclude\_file}] \leavevmode\begin{itemize}
\item {} 
\sphinxAtStartPar
Suggested by \sphinxhref{mailto:Rachel@AdminMUSH}{Rachel@AdminMUSH}

\end{itemize}

\end{description}

\sphinxAtStartPar
must\_unlquota


\subsubsection{Bug Fixes}
\label{\detokenize{changelog:id81}}\begin{description}
\item[{news/articlelife not in wizhelp}] \leavevmode\begin{itemize}
\item {} 
\sphinxAtStartPar
Reported by \sphinxhref{mailto:Rachel@AdminMUSH}{Rachel@AdminMUSH}

\end{itemize}

\item[{filter() did not correctly parse arguments in some instances \textendash{} fixed}] \leavevmode\begin{itemize}
\item {} 
\sphinxAtStartPar
Reported by \sphinxhref{mailto:Troll@Bermuda}{Troll@Bermuda}

\end{itemize}

\item[{@set thing/attr=\_thing/attr would on rare occurances not work \textendash{} fixed}] \leavevmode\begin{itemize}
\item {} 
\sphinxAtStartPar
Reported by \sphinxhref{mailto:Troll@Bermuda}{Troll@Bermuda}

\end{itemize}

\item[{lock() would not parse the second argument in some occurances \textendash{} fixed}] \leavevmode\begin{itemize}
\item {} 
\sphinxAtStartPar
Reported by \sphinxhref{mailto:Cerebus@Bermuda}{Cerebus@Bermuda}

\end{itemize}

\end{description}

\sphinxAtStartPar
mail/reply didn’t function right when BRANDY\_MAIL toggled \textendash{} fixed
\begin{description}
\item[{setunion() would not parse 3rd/4th args correctly on rare occurances \textendash{}fixed}] \leavevmode\begin{itemize}
\item {} 
\sphinxAtStartPar
Reported by \sphinxhref{mailto:Tyr@Forgotten}{Tyr@Forgotten} (Thanks for the patch)

\end{itemize}

\end{description}

\sphinxAtStartPar
news/articlelife wouldn’t reset articlelife with ‘\sphinxhyphen{}1’.  \textendash{} fixed


\subsection{RhostMUSH 3.2.4 p3 Update}
\label{\detokenize{changelog:rhostmush-3-2-4-p3-update}}
\sphinxAtStartPar
{[}11/15/1999{]}


\subsubsection{Changes}
\label{\detokenize{changelog:id82}}
\begin{sphinxadmonition}{note}{Note:}
\sphinxAtStartPar
Make sure to make the appropiate aliases
\end{sphinxadmonition}
\begin{description}
\item[{@flag is not listed in ‘wizhelp commands’}] \leavevmode\begin{itemize}
\item {} 
\sphinxAtStartPar
reported by \sphinxhref{mailto:Stormwolf@Children}{Stormwolf@Children}

\end{itemize}

\item[{remtype(\textless{}string\textgreater{},\textless{}type\textgreater{}{[},\textless{}sep\textgreater{},\textless{}osep\textgreater{}{]})}] \leavevmode\begin{itemize}
\item {} 
\sphinxAtStartPar
suggested by \sphinxhref{mailto:Stormwolf@Children}{Stormwolf@Children}

\end{itemize}

\end{description}

\sphinxAtStartPar
name(\textless{}target\textgreater{}{[},\textless{}newname\textgreater{}{]})
examine/quick \sphinxhyphen{} Previous /brief functionality
\begin{description}
\item[{examine/brief \sphinxhyphen{} Modified for MUX/PENN compatibility}] \leavevmode\begin{itemize}
\item {} 
\sphinxAtStartPar
suggested by \sphinxhref{mailto:Rachel@AdminMUSH}{Rachel@AdminMUSH}

\end{itemize}

\end{description}

\sphinxAtStartPar
@function{[}/list{]} \sphinxhyphen{} for PENN compatibility

\sphinxAtStartPar
@pemit/zone{[}/list{]} \textless{}zone(s)\textgreater{}
\begin{quote}

\sphinxAtStartPar
if zonemaster \sphinxhyphen{} all rooms in zone.
if !zonemaster \sphinxhyphen{} specified zone.
\end{quote}

\sphinxAtStartPar
@dolist/notify
‘mail/next \sphinxhyphen{}‘ \sphinxhyphen{} reads previous mail
‘mail/zap \sphinxhyphen{}‘ \sphinxhyphen{} marks current message and reads previous mail


\subsubsection{Additions}
\label{\detokenize{changelog:id83}}

\paragraph{Functions}
\label{\detokenize{changelog:id84}}
\sphinxAtStartPar
mailquick(\textless{}player\textgreater{}{[},\textless{}folder\textgreater{}{]}) (wiz only)
eval(\textless{}object\textgreater{},\textless{}attr\textgreater{}) or eval(\textless{}string\textgreater{})
translate(\textless{}string\textgreater{},\textless{}(s)pace/un(p)arse\textgreater{})
valid(name,\textless{}string\textgreater{})
entrances(target{[},(a)ll/(r)oom/(t)hing/(p)layer/(e)xit{]})
graball(\textless{}string\textgreater{},\textless{}wildcard\textgreater{}{[},\textless{}sep\textgreater{}{]})
remit(\textless{}list of rooms\textgreater{},\textless{}string\textgreater{})
rnum(\textless{}perspective\textgreater{},\textless{}target\textgreater{})
wipe(\textless{}obj\textgreater{}{[}/attr{]})
destroy(\textless{}obj\textgreater{})
step() (borrowed and modified from TinyMUSH 3.0)
localize(\textless{}string\textgreater{})
null(\textless{}string\textgreater{})
ladd(\textless{}string\textgreater{}{[},\textless{}sep\textgreater{}{]})
lsub(\textless{}string\textgreater{}{[},\textless{}sep\textgreater{}{]})
lmul(\textless{}string\textgreater{}{[},\textless{}sep\textgreater{}{]})
ldiv(\textless{}string\textgreater{}{[},\textless{}sep\textgreater{}{]})
land(\textless{}string\textgreater{}{[},\textless{}sep\textgreater{}{]})
lavg(\textless{}string\textgreater{}{[},\textless{}sep\textgreater{}{]})
lmax(\textless{}string\textgreater{}{[},\textless{}sep\textgreater{}{]})
lmin(\textless{}string\textgreater{}{[},\textless{}sep\textgreater{}{]})
lor(\textless{}string\textgreater{}{[},\textless{}sep\textgreater{}{]})
lxor(\textless{}string\textgreater{}{[},\textless{}sep\textgreater{}{]})
lnor(\textless{}string\textgreater{}{[},\textless{}sep\textgreater{}{]})
lxnor(\textless{}string\textgreater{}{[},\textless{}sep\textgreater{}{]})
lastcreate(\textless{}target\textgreater{},\textless{}(r)oom, (t)hing, (e)xit, (p)layer\textgreater{})
ncomp(\textless{}num\textgreater{},\textless{}num\textgreater{})
streq(\textless{}str\textgreater{},\textless{}str\textgreater{})
while() (borrowed and modified from TinyMUSH 3.0)=20
xcon(\textless{}target\textgreater{}{[}/\textless{}switch\textgreater{}{]},\textless{}start\textgreater{},\textless{}count\textgreater{})
modifystamp(\textless{}target\textgreater{})
createdstamp(\textless{}target\textgreater{})
inzone(\textless{}zone\textgreater{})
zemit(\textless{}list of zones\textgreater{},\textless{}string\textgreater{})
zwho(\textless{}zone\textgreater{})
zfun({[}\textless{}zone\textgreater{}/{]}\textless{}attr\textgreater{}{[},\textless{}args\textgreater{}{]})
zfun2({[}\textless{}zone\textgreater{}/{]}\textless{}attr\textgreater{}{[},\textless{}args\textgreater{}{]})
zfunlocal({[}\textless{}zone\textgreater{}/{]}\textless{}attr\textgreater{}{[},\textless{}args\textgreater{}{]})
zfun2local({[}\textless{}zone\textgreater{}/{]}\textless{}attr\textgreater{}{[},\textless{}args\textgreater{}{]})
zfundefault({[}\textless{}zone\textgreater{}/{]}\textless{}attr\textgreater{},\textless{}default\textgreater{}{[},\textless{}args\textgreater{}{]})
zfun2default({[}\textless{}zone\textgreater{}/{]}\textless{}attr\textgreater{},\textless{}default\textgreater{}{[},\textless{}args\textgreater{}{]})
zfuneval({[}\textless{}zone\textgreater{}/{]}\textless{}attr\textgreater{}/\textless{}level\textgreater{}{[},\textless{}args\textgreater{}{]})
zfunldefault({[}\textless{}zone\textgreater{}/{]}\textless{}attr\textgreater{},\textless{}default\textgreater{}{[},\textless{}args\textgreater{}{]})
zfunl2default({[}\textless{}zone\textgreater{}/{]}\textless{}attr\textgreater{},\textless{}default\textgreater{}{[},\textless{}args\textgreater{}{]})


\paragraph{Commands}
\label{\detokenize{changelog:id85}}
\sphinxAtStartPar
@hide{[}/on/off{]} \sphinxhyphen{} For PENN compatibility
@saystring \sphinxhyphen{} define what is substituted instead of ‘says’


\paragraph{Flags}
\label{\detokenize{changelog:id86}}
\sphinxAtStartPar
NOWHO (internal) \sphinxhyphen{} mark who is @hidden
LOGIN \sphinxhyphen{} bypass @disable logins
ZONECONTENTS \sphinxhyphen{} makes zonemaster behave like master room
BACKSTAGE \sphinxhyphen{} check to see if auto\sphinxhyphen{}inherited to ownership
NOBACKSTAGE \sphinxhyphen{} marker for backstage checks
ANONYMOUS \sphinxhyphen{} return ‘Someone’ when cloaked when talk/say


\paragraph{Powers}
\label{\detokenize{changelog:powers}}
\sphinxAtStartPar
NOWHO \sphinxhyphen{} specify who can @hide
EXAMINE\_FULL \sphinxhyphen{} examine anything but \#1, cloaked, and noexamine things
FULLTEL \sphinxhyphen{} teleport anywhere but \#1 and cloaked locations.


\paragraph{Toggles}
\label{\detokenize{changelog:id87}}
\sphinxAtStartPar
BRANDY\_MAIL \sphinxhyphen{} send mail with ‘mail user\sphinxhyphen{}list=3Dsubject’ methodology
FORCEHALTED \sphinxhyphen{} force someone who is halted


\subsubsection{Alias Additions}
\label{\detokenize{changelog:id88}}

\paragraph{Functions}
\label{\detokenize{changelog:id89}}
\sphinxAtStartPar
subeval() aliased to eval()
grepi() aliased to grep()
zone() aliased to lzone()
element aliased to elements()
if() aliased to ifelse()
matchall() aliased to totmatch()
nonzero() aliased to ifelse()
filterbool() aliased to filter()
landbool() aliased to land()
lorbool() aliased to lor()
lnorbool() aliased to lnor()
andbool() aliased to and()
notbool() aliased to not()
xorbool() aliased to xor()
loop() aliased to list()
enumerate() aliased to elist()
mean() aliased to avg()


\paragraph{Commands}
\label{\detokenize{changelog:id90}}
\sphinxAtStartPar
@brief aliased to examine/brief
@lemit aliased to @emit/room
@remit aliased to @pemit/contents/list
@zemit aliased to @pemit/zone/list


\paragraph{Flags}
\label{\detokenize{changelog:id91}}
\sphinxAtStartPar
NO\_COMMAND alias to NOCOMMAND


\subsubsection{Admin Params}
\label{\detokenize{changelog:id92}}
\sphinxAtStartPar
mail\_tolist (default 0) \sphinxhyphen{} enable/disable To: \textless{}players\textgreater{} in mail = automatically
mail\_default (default 0) \sphinxhyphen{} change ‘mail’ from mail/quick to mail/status


\subsubsection{Bug Fixes}
\label{\detokenize{changelog:id93}}
\sphinxAtStartPar
depower \sphinxhyphen{} tel\_anywhere (broke \sphinxhyphen{} fixed)
look \sphinxhyphen{} could see cloaked things if given long\sphinxhyphen{}finger ability (fixed)
recover objects \sphinxhyphen{} could be examined to find names of them and owner (fixed)
grab() \sphinxhyphen{} crashed server if given only one argument \sphinxhyphen{} fixed
\begin{description}
\item[{go \sphinxhyphen{} couldn’t use ‘go’ through parent exits before}] \leavevmode\begin{itemize}
\item {} 
\sphinxAtStartPar
Reported by \sphinxhref{mailto:Medwyn@Underground}{Medwyn@Underground}

\end{itemize}

\end{description}

\sphinxAtStartPar
ueval() \sphinxhyphen{} fixed missing LBUF free
mail/write +list \sphinxhyphen{} gave erraneous results \sphinxhyphen{} fixed
\begin{description}
\item[{lexits() \sphinxhyphen{} won’t show exits set PRIVATE at the ‘home’ of those exits.}] \leavevmode\begin{itemize}
\item {} 
\sphinxAtStartPar
Reported by \sphinxhref{mailto:Medwyn@Underground}{Medwyn@Underground}

\end{itemize}

\item[{min() \sphinxhyphen{} value shoved a double in an int}] \leavevmode\begin{itemize}
\item {} 
\sphinxAtStartPar
Reported by \sphinxhref{mailto:Stormwolf@Children}{Stormwolf@Children}

\end{itemize}

\item[{max() \sphinxhyphen{} value shoved a double in an int}] \leavevmode\begin{itemize}
\item {} 
\sphinxAtStartPar
Reported by \sphinxhref{mailto:Stormwolf@Children}{Stormwolf@Children}

\end{itemize}

\item[{wmail/size \sphinxhyphen{} tried to free null pointer}] \leavevmode\begin{itemize}
\item {} 
\sphinxAtStartPar
Reported by \sphinxhref{mailto:Mercutio@Shattered}{Mercutio@Shattered}

\end{itemize}

\end{description}

\sphinxAtStartPar
host sites over 50 characters would cause the host parameter to not properly null terminate.  \sphinxhyphen{} fixed.
@open would, under rare circumstances, clobber the return exit from how tprintf() buffers. \sphinxhyphen{}fixed.
andflags() and orflags() didn’t escape out ‘2’ or ‘1’ right. \sphinxhyphen{} fixed


\section{Historical Non\sphinxhyphen{}Disclosure Agreement}
\label{\detokenize{nda:historical-non-disclosure-agreement}}\label{\detokenize{nda::doc}}
\sphinxAtStartPar
The following NDA comes from the time when RhostMUSH was not publially
available. It is preserved here for historical reasons. We are suckers for
looking back at things. :)
\textendash{}Ambrosia
\begin{enumerate}
\sphinxsetlistlabels{\arabic}{enumi}{enumii}{}{)}%
\item {} 
\sphinxAtStartPar
I agree, to not give out the code, in part or in full, in any form of
medium, to anyone or anything not previously allowed by the developers.

\item {} 
\sphinxAtStartPar
I agree, to not let others look at the code, in part or in full, in
any form of medium, to anyone or anything not previously allowed by the
developers.

\item {} 
\sphinxAtStartPar
I am aware that any modifications I make to the code is \_FULLY\_
permitted, and that I do \_NOT\_ have to return said patches to the
developers.

\end{enumerate}

\begin{sphinxadmonition}{note}{Note:}
\sphinxAtStartPar
The Rhost developers would like to see what was added, and possibly
look at adding them to the main distro if we see others would like
it (with full credits to you), but we belive once you have the code,
you should be allowed to play with it fully as long as the first two
rules are kept.
\end{sphinxadmonition}


\chapter{Copyright}
\label{\detokenize{index:copyright}}
\sphinxAtStartPar
Copyright © 1990, 1991, 1992, 1993, 1994, 1995, 1996, 1997, 1998, 1999,
2000, 2001, 2002, 2003, 2004, 2005, 2006, 2007, 2008, 2009, 2010, 2011, 2012,
2013, 2014, 2015, 2016, 2017, 2018, 2019, 2020, 2021

\sphinxAtStartPar
Seawolf, Thorin, Ashen\sphinxhyphen{}Shugar, Kale, Lensman, Morgan, Odin, Kage, Ambrosia, Rook

\sphinxAtStartPar
All rights, reserved.

\sphinxAtStartPar
The copyright includes but is not limited to all changes and modifications to
the code, the help files, and all information included in this code.
Copying of these changes is not permitted without prior approval.
Borrowing of ideas require notification of where idea originated.
Please use ‘RhostMUSH’ when identifying source. Modification of code is allowed
as long as contact and acceptance is made prior to changes with one (or more) of
the original writers of the code in writing.

\sphinxAtStartPar
This copyright does not include the original code that was given as GNU freeware

\sphinxAtStartPar
This copyright information may not be changed, altered, or omitted.


\section{Indices and tables}
\label{\detokenize{index:indices-and-tables}}\begin{itemize}
\item {} 
\sphinxAtStartPar
\DUrole{xref,std,std-ref}{genindex}

\item {} 
\sphinxAtStartPar
\DUrole{xref,std,std-ref}{modindex}

\item {} 
\sphinxAtStartPar
\DUrole{xref,std,std-ref}{search}

\end{itemize}



\renewcommand{\indexname}{Index}
\printindex
\end{document}