%% Generated by Sphinx.
\def\sphinxdocclass{report}
\documentclass[letterpaper,10pt,english]{sphinxmanual}
\ifdefined\pdfpxdimen
   \let\sphinxpxdimen\pdfpxdimen\else\newdimen\sphinxpxdimen
\fi \sphinxpxdimen=.75bp\relax

\PassOptionsToPackage{warn}{textcomp}
\usepackage[utf8]{inputenc}
\ifdefined\DeclareUnicodeCharacter
% support both utf8 and utf8x syntaxes
  \ifdefined\DeclareUnicodeCharacterAsOptional
    \def\sphinxDUC#1{\DeclareUnicodeCharacter{"#1}}
  \else
    \let\sphinxDUC\DeclareUnicodeCharacter
  \fi
  \sphinxDUC{00A0}{\nobreakspace}
  \sphinxDUC{2500}{\sphinxunichar{2500}}
  \sphinxDUC{2502}{\sphinxunichar{2502}}
  \sphinxDUC{2514}{\sphinxunichar{2514}}
  \sphinxDUC{251C}{\sphinxunichar{251C}}
  \sphinxDUC{2572}{\textbackslash}
\fi
\usepackage{cmap}
\usepackage[T1]{fontenc}
\usepackage{amsmath,amssymb,amstext}
\usepackage{babel}



\usepackage{times}
\expandafter\ifx\csname T@LGR\endcsname\relax
\else
% LGR was declared as font encoding
  \substitutefont{LGR}{\rmdefault}{cmr}
  \substitutefont{LGR}{\sfdefault}{cmss}
  \substitutefont{LGR}{\ttdefault}{cmtt}
\fi
\expandafter\ifx\csname T@X2\endcsname\relax
  \expandafter\ifx\csname T@T2A\endcsname\relax
  \else
  % T2A was declared as font encoding
    \substitutefont{T2A}{\rmdefault}{cmr}
    \substitutefont{T2A}{\sfdefault}{cmss}
    \substitutefont{T2A}{\ttdefault}{cmtt}
  \fi
\else
% X2 was declared as font encoding
  \substitutefont{X2}{\rmdefault}{cmr}
  \substitutefont{X2}{\sfdefault}{cmss}
  \substitutefont{X2}{\ttdefault}{cmtt}
\fi


\usepackage[Bjarne]{fncychap}
\usepackage{sphinx}

\fvset{fontsize=\small}
\usepackage{geometry}


% Include hyperref last.
\usepackage{hyperref}
% Fix anchor placement for figures with captions.
\usepackage{hypcap}% it must be loaded after hyperref.
% Set up styles of URL: it should be placed after hyperref.
\urlstyle{same}

\addto\captionsenglish{\renewcommand{\contentsname}{Table of Contents}}

\usepackage{sphinxmessages}
\setcounter{tocdepth}{1}



\title{RhostMUSH Handbook}
\date{Apr 22, 2021}
\release{}
\author{wstarter}
\newcommand{\sphinxlogo}{\vbox{}}
\renewcommand{\releasename}{}
\makeindex
\begin{document}

\pagestyle{empty}
\sphinxmaketitle
\pagestyle{plain}
\sphinxtableofcontents
\pagestyle{normal}
\phantomsection\label{\detokenize{index::doc}}


\sphinxAtStartPar
The RhostMUSH source tree offers many abilities and options
not normally found in any other flavor of mush.  This doesn\textquotesingle{}t
make it better than other servers (though we think so \sphinxstyleemphasis{j/k})
but it does give you a wider selection of configurability,
which, as you know, is the best part of setting up a mush.
(yea, right)
\begin{itemize}
\item {} 
\sphinxAtStartPar
A high\sphinxhyphen{}performance dual\sphinxhyphen{}quota system.

\item {} 
\sphinxAtStartPar
A complete rewrite of key areas and referbishments of all the other areas.

\item {} 
\sphinxAtStartPar
An on\sphinxhyphen{}line recover tool for accidental db destruction.

\item {} 
\sphinxAtStartPar
Multi\sphinxhyphen{}wizard architecture for better control of staff.

\item {} 
\sphinxAtStartPar
Multi\sphinxhyphen{}power system to tweek abilities of players.

\item {} 
\sphinxAtStartPar
Multi\sphinxhyphen{}zone system where people can belong to multiple zones.

\item {} 
\sphinxAtStartPar
Built in error correction for db corruption or other misuse.

\item {} 
\sphinxAtStartPar
Built in accounting for those annoying twinks who try to hack.

\item {} 
\sphinxAtStartPar
Many new functions and commands not seen anywhere else.

\item {} 
\sphinxAtStartPar
A lot more that could drag this document out for pages.

\end{itemize}


\chapter{Installing RhostMUSH}
\label{\detokenize{install:installing-rhostmush}}\label{\detokenize{install:id1}}\label{\detokenize{install::doc}}
\begin{sphinxShadowBox}
\sphinxstyletopictitle{Table of Contents}
\begin{itemize}
\item {} 
\sphinxAtStartPar
\phantomsection\label{\detokenize{install:id10}}{\hyperref[\detokenize{install:rhostmush-requirements}]{\sphinxcrossref{RhostMUSH Requirements}}}
\begin{itemize}
\item {} 
\sphinxAtStartPar
\phantomsection\label{\detokenize{install:id11}}{\hyperref[\detokenize{install:system-requirements}]{\sphinxcrossref{System Requirements}}}

\item {} 
\sphinxAtStartPar
\phantomsection\label{\detokenize{install:id12}}{\hyperref[\detokenize{install:software-requirements}]{\sphinxcrossref{Software Requirements}}}
\begin{itemize}
\item {} 
\sphinxAtStartPar
\phantomsection\label{\detokenize{install:id13}}{\hyperref[\detokenize{install:optional-packages}]{\sphinxcrossref{Optional Packages}}}

\end{itemize}

\item {} 
\sphinxAtStartPar
\phantomsection\label{\detokenize{install:id14}}{\hyperref[\detokenize{install:hosting-requirements}]{\sphinxcrossref{Hosting Requirements}}}

\end{itemize}

\item {} 
\sphinxAtStartPar
\phantomsection\label{\detokenize{install:id15}}{\hyperref[\detokenize{install:obtaining-rhostmush-source-code}]{\sphinxcrossref{Obtaining RhostMUSH Source Code}}}

\item {} 
\sphinxAtStartPar
\phantomsection\label{\detokenize{install:id16}}{\hyperref[\detokenize{install:options-for-making-a-mush}]{\sphinxcrossref{Options for making a MUSH}}}
\begin{itemize}
\item {} 
\sphinxAtStartPar
\phantomsection\label{\detokenize{install:id17}}{\hyperref[\detokenize{install:compile-time-options}]{\sphinxcrossref{Compile time options}}}

\item {} 
\sphinxAtStartPar
\phantomsection\label{\detokenize{install:id18}}{\hyperref[\detokenize{install:configuration-file-options}]{\sphinxcrossref{Configuration file options}}}

\item {} 
\sphinxAtStartPar
\phantomsection\label{\detokenize{install:id19}}{\hyperref[\detokenize{install:starting-database-options}]{\sphinxcrossref{Starting database options}}}

\item {} 
\sphinxAtStartPar
\phantomsection\label{\detokenize{install:id20}}{\hyperref[\detokenize{install:the-choices-we-make}]{\sphinxcrossref{The Choices We Make}}}

\end{itemize}

\item {} 
\sphinxAtStartPar
\phantomsection\label{\detokenize{install:id21}}{\hyperref[\detokenize{install:compiling-rhostmush}]{\sphinxcrossref{Compiling RhostMush}}}
\begin{itemize}
\item {} 
\sphinxAtStartPar
\phantomsection\label{\detokenize{install:id22}}{\hyperref[\detokenize{install:setup-directory-permissions}]{\sphinxcrossref{Setup directory permissions}}}

\item {} 
\sphinxAtStartPar
\phantomsection\label{\detokenize{install:id23}}{\hyperref[\detokenize{install:compile-the-source-code}]{\sphinxcrossref{Compile the source code}}}
\begin{itemize}
\item {} 
\sphinxAtStartPar
\phantomsection\label{\detokenize{install:id24}}{\hyperref[\detokenize{install:saving-your-compile-options}]{\sphinxcrossref{Saving your compile options}}}

\item {} 
\sphinxAtStartPar
\phantomsection\label{\detokenize{install:id25}}{\hyperref[\detokenize{install:troubleshooting-compile-errors}]{\sphinxcrossref{Troubleshooting compile errors}}}

\item {} 
\sphinxAtStartPar
\phantomsection\label{\detokenize{install:id26}}{\hyperref[\detokenize{install:recompiling-the-source-code}]{\sphinxcrossref{Recompiling the source code}}}

\end{itemize}

\end{itemize}

\item {} 
\sphinxAtStartPar
\phantomsection\label{\detokenize{install:id27}}{\hyperref[\detokenize{install:configuring-the-game}]{\sphinxcrossref{Configuring the game}}}
\begin{itemize}
\item {} 
\sphinxAtStartPar
\phantomsection\label{\detokenize{install:id28}}{\hyperref[\detokenize{install:persistent-configurable-game-options}]{\sphinxcrossref{Persistent configurable game options}}}

\item {} 
\sphinxAtStartPar
\phantomsection\label{\detokenize{install:id29}}{\hyperref[\detokenize{install:starting-the-game}]{\sphinxcrossref{Starting the game}}}

\item {} 
\sphinxAtStartPar
\phantomsection\label{\detokenize{install:id30}}{\hyperref[\detokenize{install:first-login-to-the-game}]{\sphinxcrossref{First login to the game}}}

\end{itemize}

\item {} 
\sphinxAtStartPar
\phantomsection\label{\detokenize{install:id31}}{\hyperref[\detokenize{install:creating-rhostmush-with-a-provided-db}]{\sphinxcrossref{Creating RHostMUSH with a Provided DB}}}
\begin{itemize}
\item {} 
\sphinxAtStartPar
\phantomsection\label{\detokenize{install:id32}}{\hyperref[\detokenize{install:important-before-you-actually-start-building}]{\sphinxcrossref{Important before you actually start building}}}

\item {} 
\sphinxAtStartPar
\phantomsection\label{\detokenize{install:id33}}{\hyperref[\detokenize{install:using-the-prebuilt-flatfile}]{\sphinxcrossref{Using the prebuilt flatfile}}}
\begin{itemize}
\item {} 
\sphinxAtStartPar
\phantomsection\label{\detokenize{install:id34}}{\hyperref[\detokenize{install:to-load-a-prebuilt-flatfile}]{\sphinxcrossref{To load a prebuilt flatfile}}}

\end{itemize}

\end{itemize}

\item {} 
\sphinxAtStartPar
\phantomsection\label{\detokenize{install:id35}}{\hyperref[\detokenize{install:basic-instructions-for-starting-a-new-rhostmush}]{\sphinxcrossref{Basic Instructions for starting a new RhostMUSH}}}
\begin{itemize}
\item {} 
\sphinxAtStartPar
\phantomsection\label{\detokenize{install:id36}}{\hyperref[\detokenize{install:manual-configuration-of-source-code}]{\sphinxcrossref{Manual configuration of source code}}}

\item {} 
\sphinxAtStartPar
\phantomsection\label{\detokenize{install:id37}}{\hyperref[\detokenize{install:loading-a-database-for-your-mush}]{\sphinxcrossref{Loading a database for your MUSH}}}
\begin{itemize}
\item {} 
\sphinxAtStartPar
\phantomsection\label{\detokenize{install:id38}}{\hyperref[\detokenize{install:option-only-perform-these-steps-if-using-a-provided-database}]{\sphinxcrossref{Option: Only perform these steps if using a provided database}}}

\end{itemize}

\item {} 
\sphinxAtStartPar
\phantomsection\label{\detokenize{install:id39}}{\hyperref[\detokenize{install:option-things-to-do-once-you-have-connected-if-you-did-not-use-a-provided-database}]{\sphinxcrossref{Option: Things to do once you have connected if you did NOT use a provided database}}}

\end{itemize}

\item {} 
\sphinxAtStartPar
\phantomsection\label{\detokenize{install:id40}}{\hyperref[\detokenize{install:customtize-the-textfiles-for-your-game}]{\sphinxcrossref{Customtize the textfiles for your game}}}

\item {} 
\sphinxAtStartPar
\phantomsection\label{\detokenize{install:id41}}{\hyperref[\detokenize{install:three-options-for-starting-a-mush}]{\sphinxcrossref{Three Options for Starting a MUSH}}}
\begin{itemize}
\item {} 
\sphinxAtStartPar
\phantomsection\label{\detokenize{install:id42}}{\hyperref[\detokenize{install:option-1-creating-a-new-game-with-a-blank-database}]{\sphinxcrossref{Option 1: Creating a new game with a blank database}}}

\item {} 
\sphinxAtStartPar
\phantomsection\label{\detokenize{install:id43}}{\hyperref[\detokenize{install:option-2-creating-a-new-game-with-the-generic-default-database}]{\sphinxcrossref{Option 2: Creating a new game with the generic default database}}}

\item {} 
\sphinxAtStartPar
\phantomsection\label{\detokenize{install:id44}}{\hyperref[\detokenize{install:option-3-creating-a-new-game-with-ambrosia-s-default-database}]{\sphinxcrossref{Option 3: Creating a new game with Ambrosia\textquotesingle{}s default database}}}

\end{itemize}

\end{itemize}
\end{sphinxShadowBox}


\section{RhostMUSH Requirements}
\label{\detokenize{install:rhostmush-requirements}}\label{\detokenize{install:id2}}

\subsection{System Requirements}
\label{\detokenize{install:system-requirements}}\label{\detokenize{install:id3}}\begin{itemize}
\item {} 
\sphinxAtStartPar
Operating System: Unix\sphinxhyphen{}like
\begin{itemize}
\item {} 
\sphinxAtStartPar
Almost any variant of Unix or Linux should compile cleanly and run

\item {} 
\sphinxAtStartPar
Preliminary support for Windows with an equivalent development environment

\end{itemize}

\item {} 
\sphinxAtStartPar
Memory: 12\sphinxhyphen{}100 MB
\begin{itemize}
\item {} 
\sphinxAtStartPar
Depending on size of database and buffers, as well as selected options

\item {} 
\sphinxAtStartPar
1 GB (memory and swap combined) is the minimum required to compile

\end{itemize}

\item {} 
\sphinxAtStartPar
Storage: 100 MB and up
\begin{itemize}
\item {} 
\sphinxAtStartPar
Depending on size of database and backup retention policy

\end{itemize}

\end{itemize}

\begin{sphinxadmonition}{note}{Note:}
\sphinxAtStartPar
Third\sphinxhyphen{}party applications can use considerably more resources
\end{sphinxadmonition}


\subsection{Software Requirements}
\label{\detokenize{install:software-requirements}}\label{\detokenize{install:id4}}
\sphinxAtStartPar
RhostMUSH is a Linux or Unix based server software that runs as a daemon on the host.
In order to build this software, you will need the bare minimum of the ability to run \textquotesingle{}make\textquotesingle{} commands.

\sphinxAtStartPar
Package requirements are as follows:
\begin{itemize}
\item {} 
\sphinxAtStartPar
bash/ksh/dash (or compatible shell \sphinxhyphen{} for use with build menu)

\item {} 
\sphinxAtStartPar
glibc and gcc/clang (compiling the code)

\item {} 
\sphinxAtStartPar
git (to clone the source and maintain patches)

\item {} 
\sphinxAtStartPar
libcrypt (for password encryption \sphinxhyphen{} this is usually standard on unix based systems)

\end{itemize}


\subsubsection{Optional Packages}
\label{\detokenize{install:optional-packages}}
\sphinxAtStartPar
RhostMUSH also offers optional linking and library attachments.
For some of these libraries it will attempt to do auto\sphinxhyphen{}detection,
but in a worse case scenario, there exists override hashes in the menu to disable options it thinks exist that do not.

\sphinxAtStartPar
Optional packages are as follows:
\begin{itemize}
\item {} 
\sphinxAtStartPar
libpcre (if you wish to use system pcre libraries instead of the built\sphinxhyphen{}in ones)

\item {} 
\sphinxAtStartPar
mysql client \& mysql\_config (required for mysql capabilities)

\item {} 
\sphinxAtStartPar
openssl dev libraries/headers (for MUX password compatibility, and digest() and advanced cryptology functionality.

\item {} 
\sphinxAtStartPar
ruby/perl/python/etc (for custom interactive dynamic custom functions with the execscript() feature)

\item {} 
\sphinxAtStartPar
sqlite3 libraries (required for sqlite capabilities)

\end{itemize}


\subsection{Hosting Requirements}
\label{\detokenize{install:hosting-requirements}}\label{\detokenize{install:id5}}\begin{itemize}
\item {} 
\sphinxAtStartPar
You will need a stable host and access to open a single port number of your choice on the firewall.

\item {} 
\sphinxAtStartPar
Most games choose a number between 1025 and 9999, by convention.

\item {} 
\sphinxAtStartPar
Please make sure your debug\_id matches the port number + 5.
\begin{itemize}
\item {} 
\sphinxAtStartPar
So if your port is 4201, your debug\_id will be 42015.

\item {} 
\sphinxAtStartPar
The debug\_id is for use in the API daemon that runs Rhost as a container to keep track of heap, stack, and execution location.

\end{itemize}

\end{itemize}


\section{Obtaining RhostMUSH Source Code}
\label{\detokenize{install:obtaining-rhostmush-source-code}}\label{\detokenize{install:obtaining-rhostmush}}
\sphinxAtStartPar
The only official source for obtaining RhostMUSH is through the \textquotesingle{}RhostMUSH\textquotesingle{}
github account. If the source code was obtained in some other manner, there
are potentially any number of unintentional or intentional issues that you
might run into.

\sphinxAtStartPar
The recommended method of obtaing RhostMUSH is to clone it\textquotesingle{}s git reposistory:

\begin{sphinxVerbatim}[commandchars=\\\{\}]
\PYG{n}{git} \PYG{n}{clone} \PYG{n}{https}\PYG{p}{:}\PYG{o}{/}\PYG{o}{/}\PYG{n}{github}\PYG{o}{.}\PYG{n}{com}\PYG{o}{/}\PYG{n}{RhostMUSH}\PYG{o}{/}\PYG{n}{trunk} \PYG{n}{Rhost}
\end{sphinxVerbatim}

\sphinxAtStartPar
It is possible, but not recommended to download RhostMUSH via a web browser:

\begin{sphinxVerbatim}[commandchars=\\\{\}]
\PYG{n}{https}\PYG{p}{:}\PYG{o}{/}\PYG{o}{/}\PYG{n}{github}\PYG{o}{.}\PYG{n}{com}\PYG{o}{/}\PYG{n}{RhostMUSH}\PYG{o}{/}\PYG{n}{trunk}\PYG{o}{/}\PYG{n}{archive}\PYG{o}{/}\PYG{n}{master}\PYG{o}{.}\PYG{n}{zip}
\end{sphinxVerbatim}

\begin{sphinxadmonition}{note}{Note:}
\sphinxAtStartPar
This documentation generally assumes that you obtained the RhostMUSH source
code by cloning it\textquotesingle{}s git repository or at the very least downloading an
archive of the source code from the GitHub website.

\sphinxAtStartPar
It also assumes that the base directory from which all commands are run
and all files are looked for is that git repo\textquotesingle{}s \sphinxcode{\sphinxupquote{Server/}} directory,
unless specifically noted otherwise.
\end{sphinxadmonition}


\section{Options for making a MUSH}
\label{\detokenize{install:options-for-making-a-mush}}
\sphinxAtStartPar
There is a \sphinxstyleemphasis{lot} of options once your start making your MUSH, but there are
also a few big choices right as you get started making a MUSH. No matter which
choices you ultimately make, there are certain things you will need to know how
to do. This chapter is going to walk you through making the simplest possible
version of RhostMUSH. It\textquotesingle{}s going to show you the things you would have to do
no matter which choices you were making.

\sphinxAtStartPar
In the process of making that simplest possible RhostMUSH you\textquotesingle{}ll also learn
what the basics of those choices are and when and how you would make them.
Later chapters will get much more into all of those choices, but they will also
assume you know everything in this chapter already, or at least have it as a
handy point of reference.

\sphinxAtStartPar
While there are lots of little options, there are three big areas where you
make those choices.


\subsection{Compile time options}
\label{\detokenize{install:compile-time-options}}
\sphinxAtStartPar
One of the first steps of making a MUSH or really any computer program is
to compile the source code. This takes what\textquotesingle{}s basically text files full of code
and turns them into a program you can run. Within that source code are a
number of options to choose from, but any time you want to change one of them,
you have to recompile the source code and then restart the MUSH.

\sphinxAtStartPar
These choices are typically made through the \sphinxcode{\sphinxupquote{confsource}} menu which you\textquotesingle{}ll
be seeing momentarily. There are some pretty big choices here from whether or
not you want hardcoded +help and comsys, deciding between a more secure server
and certtain powerful but potentially dangerous MUSHcode options, and then
whether or not you want to be able to connect to a variety of external programs
like databases, webservers, and even other programming languages.


\subsection{Configuration file options}
\label{\detokenize{install:configuration-file-options}}
\sphinxAtStartPar
While there are some options in RhostMUSH that can only be changed through
recompiling the source code, there are way more options that can be changed
without having to recompile. These choices are mostly made through the
\sphinxcode{\sphinxupquote{netrhost.conf}} file. Whenever a MUSH starts up or gets rebooted, it\textquotesingle{}s
going to look to what\textquotesingle{}s in that file. A few of those choices relate to
further configuring the choices you made with \sphinxcode{\sphinxupquote{confsource}}

\sphinxAtStartPar
The \sphinxcode{\sphinxupquote{netrhost.conf}} file has some aesthetic options like what if anything
it says when the database is being saved or whether or not your MUSH will
allow ANSI color, both in general, but also in things like people\textquotesingle{}s names.
It has a lot of very esoteric options for tuning the performance and safety
of your MUSH. It also is where you define things that connect to your database
like your master room and guests. It let\textquotesingle{}s you determine which powers your
staff does or does not have, and it\textquotesingle{}s also where you have an option to change
the password for \#1 should you forget it. There are ways to change some of
these options from within the MUSH and even have those changes become new
defaults that survive a reboot. One thing you set there that you definitely
can\textquotesingle{}t change from within the MUSH, is which port it runs on.


\subsection{Starting database options}
\label{\detokenize{install:starting-database-options}}
\sphinxAtStartPar
This last big choice is one that you probably are well aware of at least some
of the things it allows for, mostly because the database is basically where
everyone on a MUSH lives. Most of the choices you ever make about your MUSH
will happen in the database and it\textquotesingle{}s something that\textquotesingle{}s basically always going
to be changing in more ways than any one person could follow. No matter how
vast the database of a MUSH gets, they all started somewhere, and that\textquotesingle{}s the
last big choice you have to make.

\sphinxAtStartPar
Depending on how you look at it, there\textquotesingle{}s somewhere between thousands of choices
and two choices for a starting database. What I mean is that you have the
choice of starting with a brand new database to populate, only a room (\#0) and
you (\#1), even the Master Room is something you\textquotesingle{}d have to add to it. The other
option is to import an existing database, though to choose that option you\textquotesingle{}d
also have to have access to an existing database.

\sphinxAtStartPar
Besides the brand new database that can be made on\sphinxhyphen{}demand, RhostMUSH comes with
two databases that you can use to get your start. One is called \textquotesingle{}Minimal\textquotesingle{} and
the other is called \textquotesingle{}Ambrosia\textquotesingle{} after the lead RhostMUSH developed that made it.
Despite the name, they\textquotesingle{}re both fairly minimal, there\textquotesingle{}s no grid in either, but
what you get is a lot of pre\sphinxhyphen{}installed softcode and security, as well as a
\sphinxcode{\sphinxupquote{netrhost.conf\textasciigrave{}}} file that has been tuned to work well with it. Even if you
don\textquotesingle{}t use those databases you can take ideas for the \sphinxcode{\sphinxupquote{netrhost.conf}} file for
tuning your MUSH or even use some RhostMUSH commands to import that softcode
into your database.


\subsection{The Choices We Make}
\label{\detokenize{install:the-choices-we-make}}
\sphinxAtStartPar
Well, not you\textquotesingle{}re aware of those three big choices, what they are, and where
you\textquotesingle{}ll run across them. Later in this Handbook we\textquotesingle{}ll be going through those
choices in\sphinxhyphen{}depth. In this chapter, what we\textquotesingle{}ll be doing is using the default
options for for \sphinxcode{\sphinxupquote{confsource}} and \sphinxcode{\sphinxupquote{netrhost.conf}} and a brand new database.
Those options and that blank slate are necessary so that you have in front of
you a working MUSH with only \#0 and \#1 and get shown the way of building it
into whatever is your dream MUSH, and I know we all have one.

\sphinxAtStartPar
There will be a chance to make all those other choices later, in fact that
will be happening very soon. There\textquotesingle{}s a lot of very interesting choices that
you can make with RhostMUSH, too many for anyone to ever be able to use them
all. If you don\textquotesingle{}t start with mastering the basics, you\textquotesingle{}ll never end up knowing
what are the most RhostMUSH options. I can\textquotesingle{}t tell you what they are, it\textquotesingle{}s truly
something you have to discover on your own. Remember, you want to build your
dream MUSH, not mine.

\begin{sphinxadmonition}{note}{Note:}
\sphinxAtStartPar
Unless you are intending to start with a brand new database, you must be
aware of it\textquotesingle{}s needs and expectations for the settings of \sphinxcode{\sphinxupquote{confsource}} and
\sphinxcode{\sphinxupquote{netrhost.conf}} Starter databases tend to distribute with them config
files of at least the options they expect set or not set during the process
of configuring and compiling the server.

\sphinxAtStartPar
Knowing which database you intend to use is the first choice made of these
initial major configuration options, but that last that is made part of the
MUSH.
\end{sphinxadmonition}


\section{Compiling RhostMush}
\label{\detokenize{install:compiling-rhostmush}}\label{\detokenize{install:id6}}

\subsection{Setup directory permissions}
\label{\detokenize{install:setup-directory-permissions}}
\sphinxAtStartPar
In order to both compile and run, all of the RhostMUSH files and directories
need to have the proper permissions set. If you obtained the source code
directly from GitHub, it is likely that this step is not required, but there
is no harm in running it anyway:

\begin{sphinxVerbatim}[commandchars=\\\{\}]
\PYG{o}{.}\PYG{o}{/}\PYG{n}{dirsetup}\PYG{o}{.}\PYG{n}{sh}
\end{sphinxVerbatim}

\sphinxAtStartPar
If you did NOT obtain the source code directly from GitHub, it is possible that
even the above script will fail to run with \textquotesingle{}permission denied\textquotesingle{} or similar
errors. It is recommended that you obtain the source code from there, but if
for whatever reason this is not an option, manually adjust your permissions
and then re\sphinxhyphen{}run the automated permission script:

\begin{sphinxVerbatim}[commandchars=\\\{\}]
\PYG{n}{chmod} \PYG{o}{+}\PYG{n}{rx} \PYG{n+nb}{bin}\PYG{o}{/}\PYG{o}{*}\PYG{o}{.}\PYG{n}{sh} \PYG{n}{src}\PYG{o}{/}\PYG{o}{*}\PYG{o}{.}\PYG{n}{sh} \PYG{n}{game}\PYG{o}{/}\PYG{o}{*}\PYG{o}{.}\PYG{n}{sh} \PYG{n}{game}\PYG{o}{/}\PYG{n}{Startmush} \PYG{n}{game}\PYG{o}{/}\PYG{n}{db\PYGZus{}}\PYG{o}{*}
\PYG{o}{.}\PYG{o}{/}\PYG{n}{dirsetup}\PYG{o}{.}\PYG{n}{sh}
\end{sphinxVerbatim}


\subsection{Compile the source code}
\label{\detokenize{install:compile-the-source-code}}
\sphinxAtStartPar
Once the source code has been obtained and the proper file and directory
permissions have been set, the RhostMUSH source code is ready to be compiled.
This is typically done through an interactive program where you configure the
options you want to have available to your installation:

\begin{sphinxVerbatim}[commandchars=\\\{\}]
\PYG{n}{make} \PYG{n}{confsource}
\end{sphinxVerbatim}

\begin{sphinxadmonition}{note}{Note:}
\sphinxAtStartPar
It is recommended that if you are just starting out with RhostMUSH that you
compile for the first time using the default compile options which have
specifically been tuned to be closest to what the average person would need
or expect. Changing these options before you have a grood grasp of what
they mean and how RhostMUSH works on a deeper level can potentially cause
security issues, reduce compatibility with softcode rom other types of MUSH
servers, as well as waste system resources.
\end{sphinxadmonition}


\subsubsection{Saving your compile options}
\label{\detokenize{install:saving-your-compile-options}}
\sphinxAtStartPar
\sphinxcode{\sphinxupquote{make confsource}} will remember the most recent options you used to compile
the source code for the next time you use \sphinxcode{\sphinxupquote{make confsource}} It might still
be a good idea to more permanently save the options you used to attempt to
compile. However, you still might want to have these options saved more
permanently, just in case. At the \sphinxcode{\sphinxupquote{make confsource}} menu there is an
option to save your current settings to a file. If you choose to do this,
you will find the the save file in the \sphinxcode{\sphinxupquote{bin/}} directory.


\subsubsection{Troubleshooting compile errors}
\label{\detokenize{install:troubleshooting-compile-errors}}
\sphinxAtStartPar
Should this result in an error, a script has been included to correct the most
common errors, after which you can once more try to compile:

\begin{sphinxVerbatim}[commandchars=\\\{\}]
\PYG{o}{.}\PYG{o}{/}\PYG{n+nb}{bin}\PYG{o}{/}\PYG{n}{script\PYGZus{}setup}\PYG{o}{.}\PYG{n}{sh}
\PYG{n}{make} \PYG{n}{confsource}
\end{sphinxVerbatim}

\sphinxAtStartPar
Once the compile process successfully complete, you should be able to start\sphinxhyphen{}up
your new RhostMUSH server. If it complains about missing binaries make sure
they are linked. The provided script will fix this issue, and is not harmful
to run in any situation:

\begin{sphinxVerbatim}[commandchars=\\\{\}]
\PYG{n}{make} \PYG{n}{links}
\end{sphinxVerbatim}


\subsubsection{Recompiling the source code}
\label{\detokenize{install:recompiling-the-source-code}}
\sphinxAtStartPar
If you plan to use \sphinxcode{\sphinxupquote{make confsource}} to recompile your source, you should
first issue a \sphinxcode{\sphinxupquote{make clean}} before re\sphinxhyphen{}issuing a \sphinxcode{\sphinxupquote{make confsource}}

\sphinxAtStartPar
A failure to issue \sphinxcode{\sphinxupquote{make clean}} prior to re\sphinxhyphen{}compiling with \sphinxcode{\sphinxupquote{make confsource}}
can potentially leave stale object files which may cause unforseen issues when
running code, including but not limited to random crashes.  Generally whenever
recompiling it\textquotesingle{}s good to always make clean first.

\begin{sphinxadmonition}{note}{Note:}
\sphinxAtStartPar
You may also issue \sphinxcode{\sphinxupquote{make source}} if the \sphinxcode{\sphinxupquote{Makefile}} is already defined how
you want it to be.  Please remember to \sphinxcode{\sphinxupquote{make clean}} before \sphinxcode{\sphinxupquote{make source\textasciigrave{}}}
whenever you alter the code or import new source code.
\end{sphinxadmonition}


\section{Configuring the game}
\label{\detokenize{install:configuring-the-game}}

\subsection{Persistent configurable game options}
\label{\detokenize{install:persistent-configurable-game-options}}
\sphinxAtStartPar
Upon compiling a RhostMUSH server, if it doesn\textquotesingle{}t already exist, a
\sphinxcode{\sphinxupquote{netrhost.conf}} is copied into the \sphinxcode{\sphinxupquote{game/}} directory for your game. It
includes useful defaults for most compile\sphinxhyphen{}time options that will work well for
most games, particularly ones using both the default \sphinxcode{\sphinxupquote{confsource}} options and
related database.

\sphinxAtStartPar
This configuration is derived from \sphinxcode{\sphinxupquote{defaults/game/netrhost.conf.default}}

\sphinxAtStartPar
While this \sphinxcode{\sphinxupquote{netrhost.conf}} is very well documented and quite easy to change
in some places, but there are also some rather technical options that you it\textquotesingle{}s
important to know the full implications of this.

\begin{sphinxadmonition}{note}{Note:}
\sphinxAtStartPar
The default \sphinxcode{\sphinxupquote{netrhost.conf}} starts the game running on the port \sphinxstyleemphasis{4021} of
the server. If this is your time creating a MUSH, it is recommended that this
setting is the only one that you potentially change, and only if there is a
good reason to. Ports below 1024 are priviliged ports and can not be used for
this purpose.
\end{sphinxadmonition}


\subsection{Starting the game}
\label{\detokenize{install:starting-the-game}}
\sphinxAtStartPar
Once done, you start up the system with the following command:

\begin{sphinxVerbatim}[commandchars=\\\{\}]
\PYG{o}{.}\PYG{o}{/}\PYG{n}{Startmush}
\end{sphinxVerbatim}

\sphinxAtStartPar
It will prompt you to start a new db if it doesn\textquotesingle{}t find one.

\sphinxAtStartPar
You may also do the commands individually:

\begin{sphinxVerbatim}[commandchars=\\\{\}]
\PYG{p}{[}\PYG{n}{csh}\PYG{p}{]} \PYG{n}{netrhost} \PYG{o}{\PYGZhy{}}\PYG{n}{s} \PYG{n}{netrhost}\PYG{o}{.}\PYG{n}{conf} \PYG{o}{\PYGZgt{}}\PYG{o}{\PYGZam{}} \PYG{n}{netrhost}\PYG{o}{.}\PYG{n}{log} \PYG{o}{\PYGZam{}}
\PYG{p}{[}\PYG{n}{sh}\PYG{p}{]}  \PYG{n}{netrhost} \PYG{o}{\PYGZhy{}}\PYG{n}{s} \PYG{n}{netrhost}\PYG{o}{.}\PYG{n}{conf} \PYG{o}{\PYGZgt{}} \PYG{n}{netrhost}\PYG{o}{.}\PYG{n}{log} \PYG{l+m+mi}{2}\PYG{o}{\PYGZgt{}}\PYG{o}{\PYGZam{}}\PYG{l+m+mi}{1} \PYG{o}{\PYGZam{}}
\end{sphinxVerbatim}


\subsection{First login to the game}
\label{\detokenize{install:first-login-to-the-game}}
\sphinxAtStartPar
Once started, log in the \#1 character (Wizard) with it\textquotesingle{}s appropiate
password (no, not \textquotesingle{}potrzebie\textquotesingle{}, but \textquotesingle{}Nyctasia\textquotesingle{}).  There were private
reasons for the password change.

\sphinxAtStartPar
Once in, do a @shutdown to save the database.  Then you can run Startup
normally.   You may make a backup of your database at anytime on\sphinxhyphen{}line by
utilizing the @dump/flat option.  A script comes with this distribution
that allows the ability of auto\sphinxhyphen{}archiving your database for a configurable
number of backups.


\section{Creating RHostMUSH with a Provided DB}
\label{\detokenize{install:creating-rhostmush-with-a-provided-db}}

\subsection{Important before you actually start building}
\label{\detokenize{install:important-before-you-actually-start-building}}
\sphinxAtStartPar
The main parts of making your RhostMUSH, easy pleasy:
\begin{enumerate}
\sphinxsetlistlabels{\arabic}{enumi}{enumii}{}{.}%
\item {} 
\sphinxAtStartPar
The stunnel directory contains TLS/SSL connectivity.  This has to be linked to another port and will tunnel to the mush port.  The README file explains how to set up and configure your TLS/SSL connection.

\item {} 
\sphinxAtStartPar
./patch.sh \sphinxhyphen{}\sphinxhyphen{} This makes sure you have the latest code.  If you got this by git clone \sphinxurl{https://github.com/RhostMUSH/trunk} then you can ignore patching.  You can use ./patch.sh at any time to update your code.  It ignores local.c incase you make your own modules.

\item {} 
\sphinxAtStartPar
make confsource.  Yup, it\textquotesingle{}s menu driven, nifty eh?
\begin{enumerate}
\sphinxsetlistlabels{\arabic}{enumii}{enumiii}{}{.}%
\item {} 
\sphinxAtStartPar
Options you may want to select (other than the defaults):

\item {} 
\sphinxAtStartPar
5  (\%c is selected by default, but choose \%x as well for MUX/TM3 compat)

\item {} 
\sphinxAtStartPar
9  (if you want \$commands to require the COMMAND flag)

\item {} 
\sphinxAtStartPar
16 (if you want a wider WHO listing like older versions of MUX)

\item {} 
\sphinxAtStartPar
22 (if you\textquotesingle{}re converting a TinyMUSH3 or TinyMUX/MUX2 flatfile)

\item {} 
\sphinxAtStartPar
24 (if you have issues with \sphinxhyphen{}lssl not being found)

\item {} 
\sphinxAtStartPar
B3 (for 64 character attribute names)

\item {} 
\sphinxAtStartPar
B6 (select 8K for Penn/MUX2/TM3 default, up to 32K.  64K is network intensive)

\item {} 
\sphinxAtStartPar
B5 (will be autoselected if you choose 8K or more.  Pick this anyway)

\item {} 
\sphinxAtStartPar
B4 (if you have sqlite libraries and wish to use this)

\end{enumerate}

\item {} 
\sphinxAtStartPar
\textquotesingle{}r\textquotesingle{} to compile with the settings you selected.

\item {} 
\sphinxAtStartPar
Modify your netrhost.conf file as specified.  Make sure to align your port and debug\_id as shown in the netrhost.conf file.

\item {} 
\sphinxAtStartPar
If you wish to port in an old flatfile, please refer to the readme directory on how to port your flatfile in (README.DBLOADING).

\end{enumerate}


\subsection{Using the prebuilt flatfile}
\label{\detokenize{install:using-the-prebuilt-flatfile}}
\sphinxAtStartPar
There are pre\sphinxhyphen{}loaded flatfile databases you can use at this point.  The netrhost.db.flat
and corrisponding netrhost.conf file will be located in the minimal\sphinxhyphen{}DBs/minimal\_db directory.

\sphinxAtStartPar
You may auto\sphinxhyphen{}load the minimal db and corresponding netrhost.conf file with the command:

\begin{sphinxVerbatim}[commandchars=\\\{\}]
\PYG{o}{.}\PYG{o}{/}\PYG{n}{minimal}\PYG{o}{.}\PYG{n}{sh}
\end{sphinxVerbatim}

\sphinxAtStartPar
This is ran from within the \textquotesingle{}game\textquotesingle{} directory.  Once this is ran, you will need
to customize the netrhost.conf file for your purposes.  The port and debug\_id must
be changed at the very least.  Keep the debug\_id coordinated to the port as described.


\subsubsection{To load a prebuilt flatfile}
\label{\detokenize{install:to-load-a-prebuilt-flatfile}}\begin{enumerate}
\sphinxsetlistlabels{\arabic}{enumi}{enumii}{}{.}%
\item {} 
\sphinxAtStartPar
Make a backup of your existing netrhost.conf file:

\begin{sphinxVerbatim}[commandchars=\\\{\}]
\PYG{n}{cp} \PYG{n}{game}\PYG{o}{/}\PYG{n}{netrhost}\PYG{o}{.}\PYG{n}{conf} \PYG{n}{game}\PYG{o}{/}\PYG{n}{netrhost}\PYG{o}{.}\PYG{n}{conf}\PYG{o}{.}\PYG{n}{backup}
\end{sphinxVerbatim}

\item {} 
\sphinxAtStartPar
Copy the netrhost.conf file into your game directory:

\begin{sphinxVerbatim}[commandchars=\\\{\}]
\PYG{n}{cp} \PYG{o}{\PYGZhy{}}\PYG{n}{f} \PYG{o}{.}\PYG{o}{/}\PYG{n}{minimal}\PYG{o}{\PYGZhy{}}\PYG{n}{DBs}\PYG{o}{/}\PYG{n}{minimal\PYGZus{}db}\PYG{o}{/}\PYG{n}{netrhost}\PYG{o}{.}\PYG{n}{conf} \PYG{o}{.}\PYG{o}{/}\PYG{n}{game}\PYG{o}{/}\PYG{n}{netrhost}\PYG{o}{.}\PYG{n}{conf}
\end{sphinxVerbatim}

\item {} 
\sphinxAtStartPar
At this point you can modify your netrhost.conf file settings in your game directory.
Using an editor modify the \textquotesingle{}port\textquotesingle{} and \textquotesingle{}debug\_id\textquotesingle{} respectively in your netrhost.conf as state.
The \textquotesingle{}port\textquotesingle{} will be the port the mush listens on, the debug\_id is for the debug\sphinxhyphen{}stack and is
your port with a \textquotesingle{}5\textquotesingle{} at the end.  So if your port is 4444, the debug\_id is 44445

\end{enumerate}

\begin{sphinxadmonition}{note}{\label{\detokenize{install:id7}}Todo:}
\sphinxAtStartPar
Clean up the below section
\end{sphinxadmonition}
\begin{enumerate}
\sphinxsetlistlabels{\arabic}{enumi}{enumii}{}{.}%
\item {} 
\sphinxAtStartPar
Load in the flatfile into the mush (You could do this in the Startmush as well)
Manually:

\begin{sphinxVerbatim}[commandchars=\\\{\}]
\PYG{n}{cd} \PYG{n}{game}
\PYG{o}{.}\PYG{o}{/}\PYG{n}{db\PYGZus{}load} \PYG{n}{data}\PYG{o}{/}\PYG{n}{netrhost}\PYG{o}{.}\PYG{n}{gdbm} \PYG{o}{.}\PYG{o}{.}\PYG{o}{/}\PYG{n}{minimal}\PYG{o}{\PYGZhy{}}\PYG{n}{DBs}\PYG{o}{/}\PYG{n}{minimal\PYGZus{}db}\PYG{o}{/}\PYG{n}{netrhost}\PYG{o}{.}\PYG{n}{db}\PYG{o}{.}\PYG{n}{flat} \PYG{n}{data}\PYG{o}{/}\PYG{n}{netrhost}\PYG{o}{.}\PYG{n}{db}\PYG{o}{.}\PYG{n}{new} \PYG{n}{dwF}
\end{sphinxVerbatim}

\sphinxAtStartPar
Start your mush:

\begin{sphinxVerbatim}[commandchars=\\\{\}]
\PYG{o}{.}\PYG{o}{/}\PYG{n}{Startmush}
\end{sphinxVerbatim}

\sphinxAtStartPar
This will load the db that you loaded.

\sphinxAtStartPar
\sphinxhyphen{}\sphinxhyphen{}\sphinxhyphen{}\sphinxhyphen{}\sphinxhyphen{}\sphinxhyphen{}\sphinxhyphen{}\sphinxhyphen{}\sphinxhyphen{}\sphinxhyphen{}\sphinxhyphen{}\sphinxhyphen{}\sphinxhyphen{}\sphinxhyphen{}\sphinxhyphen{}OR\sphinxhyphen{}\sphinxhyphen{}\sphinxhyphen{}\sphinxhyphen{}\sphinxhyphen{}\sphinxhyphen{}\sphinxhyphen{}

\sphinxAtStartPar
From Startmush when prompted, hit \textless{}RETURN\textgreater{} for searching then select the number of the netrhost.db.flat that is listed as \textasciitilde{}/minimal\sphinxhyphen{}DBs/minimal\_db/netrhost.db.flat:

\begin{sphinxVerbatim}[commandchars=\\\{\}]
\PYG{o}{.}\PYG{o}{/}\PYG{n}{Startmush}
\end{sphinxVerbatim}

\end{enumerate}


\section{Basic Instructions for starting a new RhostMUSH}
\label{\detokenize{install:basic-instructions-for-starting-a-new-rhostmush}}

\subsection{Manual configuration of source code}
\label{\detokenize{install:manual-configuration-of-source-code}}
\sphinxAtStartPar
To do manual configuration (skip if the previous step worked for you) And yes, this is a bit of a pain in the bottom, hopefully you will not need this.

\sphinxAtStartPar
You need the following definitions defined to make this work:
\begin{enumerate}
\sphinxsetlistlabels{\arabic}{enumi}{enumii}{}{.}%
\item {} 
\sphinxAtStartPar
TINY\_U, USE\_SIDEEFFECTS, MUX\_INCDEC, ATTR\_HACK

\item {} 
\sphinxAtStartPar
(u()/u2() switched)

\item {} 
\sphinxAtStartPar
(sideeffects)

\item {} 
\sphinxAtStartPar
(inc()/xinc() switched)

\item {} 
\sphinxAtStartPar
(support for \_/\textasciitilde{} attribs)

\end{enumerate}

\sphinxAtStartPar
You only need to do this if you received the RhostMUSH src.  If you received a binary, continue on to the next part.

\sphinxAtStartPar
To compile the code, just type \textquotesingle{}make confsource\textquotesingle{}.  It will prompt you with settings on what you need to do.  If you just want to quickly hand edit the Makefile, it is in the directory src (full path src/Makefile).  Then you may just run \textquotesingle{}make source\textquotesingle{}, if you so choose to hand\sphinxhyphen{}edit the Makefile.

\sphinxAtStartPar
After the compile process is done, type:

\begin{sphinxVerbatim}[commandchars=\\\{\}]
\PYG{n}{make} \PYG{n}{links}
\end{sphinxVerbatim}


\subsection{Loading a database for your MUSH}
\label{\detokenize{install:loading-a-database-for-your-mush}}
\sphinxAtStartPar
You now have a choice of optionally starting at a provided database or starting from scratch.


\subsubsection{Option: Only perform these steps if using a provided database}
\label{\detokenize{install:option-only-perform-these-steps-if-using-a-provided-database}}
\sphinxAtStartPar
Copy an existing flatfile and corresponding netrhost.conf file Default provied example:

\begin{sphinxVerbatim}[commandchars=\\\{\}]
\PYG{n}{cp} \PYG{n}{game}\PYG{o}{/}\PYG{n}{netrhost}\PYG{o}{.}\PYG{n}{conf} \PYG{n}{game}\PYG{o}{/}\PYG{n}{netrhost}\PYG{o}{.}\PYG{n}{conf}\PYG{o}{.}\PYG{n}{backup}
\PYG{n}{cp} \PYG{o}{\PYGZhy{}}\PYG{n}{f} \PYG{n}{minimal}\PYG{o}{\PYGZhy{}}\PYG{n}{DBs}\PYG{o}{/}\PYG{n}{minimal\PYGZus{}db}\PYG{o}{/}\PYG{n}{netrhost}\PYG{o}{.}\PYG{n}{conf} \PYG{n}{game}\PYG{o}{/}\PYG{n}{netrhost}\PYG{o}{.}\PYG{n}{conf}
\PYG{n}{cd} \PYG{n}{game}
\PYG{o}{.}\PYG{o}{/}\PYG{n}{db\PYGZus{}load} \PYG{n}{data}\PYG{o}{/}\PYG{n}{netrhost}\PYG{o}{.}\PYG{n}{gdbm} \PYG{o}{.}\PYG{o}{.}\PYG{o}{/}\PYG{n}{minimal}\PYG{o}{\PYGZhy{}}\PYG{n}{DBs}\PYG{o}{/}\PYG{n}{minimal\PYGZus{}db}\PYG{o}{/}\PYG{n}{netrhost}\PYG{o}{.}\PYG{n}{db}\PYG{o}{.}\PYG{n}{flat} \PYG{n}{data}\PYG{o}{/}\PYG{n}{netrhost}\PYG{o}{.}\PYG{n}{db}\PYG{o}{.}\PYG{n}{new}
\end{sphinxVerbatim}


\subsection{Option: Things to do once you have connected if you did NOT use a provided database}
\label{\detokenize{install:option-things-to-do-once-you-have-connected-if-you-did-not-use-a-provided-database}}\begin{enumerate}
\sphinxsetlistlabels{\arabic}{enumi}{enumii}{}{.}%
\item {} 
\sphinxAtStartPar
@dig your master room and in your netrhost.conf file define master\_room to this dbref (without the \#.  So like master\_room 2)

\item {} 
\sphinxAtStartPar
Create an immortal holder charater (@pcreate then @set immortal) Feel free to set up holder characters for all the bittypes which are: GUILDMASTER, ARCHITECT, COUNCILOR, WIZARD, IMMORTAL

\item {} 
\sphinxAtStartPar
@chown/preserve the master room and \#0 to the immortal holder character.

\item {} 
\sphinxAtStartPar
Log into the immortal character

\item {} 
\sphinxAtStartPar
@pcreate all your guest characters and set them up properly.  My suggestion:

\begin{sphinxVerbatim}[commandchars=\\\{\}]
\PYG{n+nd}{@dolist} \PYG{n}{lnum}\PYG{p}{(}\PYG{l+m+mi}{1}\PYG{p}{,}\PYG{l+m+mi}{10}\PYG{p}{)}\PYG{o}{=}\PYG{p}{\PYGZob{}}\PYG{n+nd}{@pcreate} \PYG{n}{Guest}\PYG{c+c1}{\PYGZsh{}\PYGZsh{}=guest;@set *Guest\PYGZsh{}\PYGZsh{}=guest;@desc *Guest\PYGZsh{}\PYGZsh{}=A guest player.;@adisconnect *Guest\PYGZsh{}\PYGZsh{}=home;@lock *Guest\PYGZsh{}\PYGZsh{}=*Guest\PYGZsh{}\PYGZsh{}\PYGZcb{}}
\end{sphinxVerbatim}

\end{enumerate}

\begin{sphinxadmonition}{note}{Note:}
\sphinxAtStartPar
@list guest will show your guest characters and if they\textquotesingle{}re set up properly.
\end{sphinxadmonition}
\begin{enumerate}
\sphinxsetlistlabels{\arabic}{enumi}{enumii}{}{.}%
\item {} 
\sphinxAtStartPar
Any master room code you load in from your immholder character (or @chown/preserve to it) The readme directory has softfunctions.minmax that has MUX/Penn compatability functions and comsys.  All other softcode (like mail wrappers) can be found on \sphinxurl{https://github.com/RhostMUSH/trunk} in Mushcode.

\item {} 
\sphinxAtStartPar
Setup new character, staff, and take tasks that can only be accomplished by \#1

\item {} 
\sphinxAtStartPar
Set up any other characters you want.  Anyone immortal can issue @function, @admin, or anything \#1 can do.

\end{enumerate}

\begin{sphinxadmonition}{note}{\label{\detokenize{install:id8}}Todo:}
\sphinxAtStartPar
Figure out what they were trying to say by having those above two sentences right after each other.
\end{sphinxadmonition}


\section{Customtize the textfiles for your game}
\label{\detokenize{install:customtize-the-textfiles-for-your-game}}
\sphinxAtStartPar
All connect.txt and customized files can be found in the \textasciitilde{}/Server/game/txt directory.  There is a
README file there that explains their purposes in more detail.  You can see more information on
all files and how they inter\sphinxhyphen{}relate with \textquotesingle{}wizhelp file\textquotesingle{}.


\section{Three Options for Starting a MUSH}
\label{\detokenize{install:three-options-for-starting-a-mush}}
\sphinxAtStartPar
The RhostMUSH Git Repository comes with three options for starting your Mush.

\begin{sphinxadmonition}{note}{\label{\detokenize{install:id9}}Todo:}
\sphinxAtStartPar
Well, there\textquotesingle{}s also some other pre\sphinxhyphen{}existing DB, upgrading, and ansible so let\textquotesingle{}s try to make this a little more coherent.
\end{sphinxadmonition}


\subsection{Option 1: Creating a new game with a blank database}
\label{\detokenize{install:option-1-creating-a-new-game-with-a-blank-database}}
\sphinxAtStartPar
Modify your ./game/netrhost.conf file or what settings you want.
Don\textquotesingle{}t feel overwhelmed, it\textquotesingle{}s all very well documented.


\subsection{Option 2: Creating a new game with the generic default database}
\label{\detokenize{install:option-2-creating-a-new-game-with-the-generic-default-database}}
\sphinxAtStartPar
Copy the netrhost.conf from minimal\sphinxhyphen{}DBs/minimal\_db to your game directory.

\sphinxAtStartPar
When ./Startmush prompts you to load a flatfile, say \textquotesingle{}yes\textquotesingle{} and hit \textless{}RETURN\textgreater{}
to have it search for flatfiles, then select netrhost.db.flat from under
the minimal\sphinxhyphen{}DBs/minimal\_db directory.


\subsection{Option 3: Creating a new game with Ambrosia\textquotesingle{}s default database}
\label{\detokenize{install:option-3-creating-a-new-game-with-ambrosia-s-default-database}}
\sphinxAtStartPar
This option is covered in detail here: {\hyperref[\detokenize{ambrosiadb:ambrosiadb-installation}]{\sphinxcrossref{\DUrole{std,std-ref}{Ambrosia\textquotesingle{}s Minimal Rhost DB}}}}


\chapter{What RhostMUSH is about and what\textquotesingle{}s so great about RhostMUSH}
\label{\detokenize{features:what-rhostmush-is-about-and-what-s-so-great-about-rhostmush}}\label{\detokenize{features::doc}}
\sphinxAtStartPar
RhostMUSH was founded in 1989, originally by Natasha Davis (Nyctasia) and as
a branch from the original release of TinyMUD code.  It was her desire to make
a game that was flexible, with multiple levels of progression and highly
customizeable.  She lost time and interest and passed the game to
Steve Shivers (Seawolf), Mike McDermott (Thorin), and Jace Hoppel (Ashen\sphinxhyphen{}Shugar)

\sphinxAtStartPar
Through their work, the stability improved, we fixed it to be multi\sphinxhyphen{}platform
and as bug free as we could possibly make it.  We introduced several methods both
in game and in source that allowed consistent memory bounds checking and
various alerts for any mischievous naughtyness in\sphinxhyphen{}game or possibilities of any
hacks, patches, or alterations in the code causing leaks or issues.

\sphinxAtStartPar
While not perfect, it has allowed us to have an absolutely outstanding
turn around for any bugs sent our way, which anyone who uses RhostMUSH will
attest to.

\sphinxAtStartPar
Over the years, others have joined the RhostMUSH team, including Ambrosia
(who is the current dev lead), Lensman, Kage (who kindly provided the
UTF8/unicode port), Jeff/Loki, Rook, Noltar, and Odin.

\sphinxAtStartPar
We also have had hundreds of people who have offered (and provided) help,
patches, suggestions, bug fixes, and alternations all on their own and
every single one will have had their name mentioned in the RHOST.CHANGES
file in the readme directory.  It\textquotesingle{}s far too large to have in the online
help files.

\sphinxAtStartPar
RhostMUSH today provides an amazing tool that allows nearly entire
customization in\sphinxhyphen{}game of every single feature available in Rhost without
having the requirement to modify the hardcode.  This includes but is
not limited to:


\section{Recycle bin}
\label{\detokenize{features:recycle-bin}}
\sphinxAtStartPar
Yup, you guessed it.  RhostMUSH has a windows like recycle bin.
This means the objects you @nuke and @destroy become \textquotesingle{}destroyed\textquotesingle{}
but not recycled until they are @purged.  If you use the Myrddin
CRON in the Mushcode directory, it by default sets up a job
to purge anything over 30 days old, which should be more than
sufficient for any needs.  The goodness of this?  You can recover
nuked things from any period of time, as long as they were not
@purged first.

\sphinxAtStartPar
Commands: @purge, @nuke, @destroy, @recover, @reclist


\section{@snapshot}
\label{\detokenize{features:snapshot}}
\sphinxAtStartPar
Live image snapshots to unload or load to and from
disk.  As many snapshots as you want, as often as you want.
It essentially does a flatfile dump of a dbref\#.  Great for
backups or cross\sphinxhyphen{}Rhost portability.

\sphinxAtStartPar
Command: @snapshot


\section{Wizard and Immortals by default}
\label{\detokenize{features:wizard-and-immortals-by-default}}\begin{itemize}
\item {} 
\sphinxAtStartPar
are spoofable.  Meaning all their @pemits by default will not
trigger NOSPOOF.  If you do not wish this, set the SPOOF flag
this applies to anyone below their level.

\item {} 
\sphinxAtStartPar
override all locks.  There\textquotesingle{}s two flags to disable this.
NO\_OVERRIDE to stop overriding all locks (including attribs)
and NO\_USELOCK to just stop overriding uselocks.
This applies to anything their level and lower.

\item {} 
\sphinxAtStartPar
optionally cloak from all non\sphinxhyphen{}immortals/God player.
This can be highly abused if not careful and there
is a @depower to disable cloaking and/or dark that will
disable this.

\item {} 
\sphinxAtStartPar
immortals can optionally supercloak from even wizards.
this can not be disabled, and you must consider that immortals
should be treated as the God player (\#1) since they are
effectively \#1 in nearly every way.

\end{itemize}


\section{Titles and Captions to a player\textquotesingle{}s name}
\label{\detokenize{features:titles-and-captions-to-a-player-s-name}}
\sphinxAtStartPar
@caption and @titlecaption


\section{Have an alternate name with locks for NPC obfuscation}
\label{\detokenize{features:have-an-alternate-name-with-locks-for-npc-obfuscation}}
\sphinxAtStartPar
@altname
@lock/altname


\section{Have multiple player aliases}
\label{\detokenize{features:have-multiple-player-aliases}}
\sphinxAtStartPar
As well as a method to reserve player names per player w/o revealing who has what name.

\sphinxAtStartPar
@protect


\section{Actively control how dark works both game\sphinxhyphen{}wide and individually}
\label{\detokenize{features:actively-control-how-dark-works-both-game-wide-and-individually}}
\sphinxAtStartPar
@depower dark

\sphinxAtStartPar
@admin allow\_whodark, sweep\_dark, command\_dark, lcon\_checks\_dark,
secure\_dark, see\_owned\_dark, idle\_wiz\_dark, player\_dark

\sphinxAtStartPar
@toggle snuffdark

\sphinxAtStartPar
@flagdef to redefine who and what can set the DARK flag


\section{Make config file changes in\sphinxhyphen{}game without having to reboot or have shell access}
\label{\detokenize{features:make-config-file-changes-in-game-without-having-to-reboot-or-have-shell-access}}
\sphinxAtStartPar
@admin admin\_object


\section{Execute any binary or script as a localized function}
\label{\detokenize{features:execute-any-binary-or-script-as-a-localized-function}}
\sphinxAtStartPar
EXECSCRIPT (power), SIDEFX (flag)


\section{Customized percent substitutions (like \%n, \%\#, etc)}
\label{\detokenize{features:customized-percent-substitutions-like-n-etc}}
\sphinxAtStartPar
@admin sub\_include, @hook


\section{Redefine percent substitutions (like \%n, \%\#, etc)}
\label{\detokenize{features:redefine-percent-substitutions-like-n-etc}}
\sphinxAtStartPar
@admin sub\_override, @hook


\section{Localize command and function overrides in a sandbox}
\label{\detokenize{features:localize-command-and-function-overrides-in-a-sandbox}}
\sphinxAtStartPar
@icmd, @lfunction, subeval(), sandbox()


\section{Multiple Zones}
\label{\detokenize{features:multiple-zones}}
\sphinxAtStartPar
Have multiple zones which can optionally belong to multiple targets (multiple zones per target allowable!)

\sphinxAtStartPar
@zone, zones, lzone(), zonecmd()


\section{Optionally control, enable, or disable sideeffects}
\label{\detokenize{features:optionally-control-enable-or-disable-sideeffects}}
\sphinxAtStartPar
@admin sideeffects, SIDEFX (flag)


\section{Have 31 cross\sphinxhyphen{}interactive realities for locations}
\label{\detokenize{features:have-31-cross-interactive-realities-for-locations}}
\sphinxAtStartPar
This works as a truly independant and self\sphinxhyphen{}contained environment.
A room can have 31 \textquotesingle{}layers\textquotesingle{}, each \textquotesingle{}layer\textquotesingle{} is a reality in
the same physical space.  These layers can work independently
or allow interaction with other layers for vast customization.
This affects all methods within the game including all matching,
looking, \$commands, listens, movement, interaction, pretty
much every single aspect of mushing.

\sphinxAtStartPar
REALITY LEVELS


\section{Override any command with softcode}
\label{\detokenize{features:override-any-command-with-softcode}}
\sphinxAtStartPar
@admin access (check ignore)

\sphinxAtStartPar
Master room \$commands to then override the hardcode


\section{The abilility to raise or lower permissions on the various}
\label{\detokenize{features:the-abilility-to-raise-or-lower-permissions-on-the-various}}
\sphinxAtStartPar
staff bitlevels for each player.

\sphinxAtStartPar
@power, @depower, TOGGLES, FLAGS


\section{Customize new commands on the connect screen}
\label{\detokenize{features:customize-new-commands-on-the-connect-screen}}
\sphinxAtStartPar
@admin file\_object2


\section{Softcode any txt file (like connect.txt)}
\label{\detokenize{features:softcode-any-txt-file-like-connect-txt}}
\sphinxAtStartPar
and have it evaluate in\sphinxhyphen{}game.  It evaluates as the object it is on.

\sphinxAtStartPar
@admin file\_object


\section{Advanced tracing methods for debugging your code including labels!}
\label{\detokenize{features:advanced-tracing-methods-for-debugging-your-code-including-labels}}
\sphinxAtStartPar
Commands: @label

\sphinxAtStartPar
Functions: parenmatch(), trace()

\sphinxAtStartPar
Toggles: CPUTIME

\sphinxAtStartPar
Flags: TRACE

\sphinxAtStartPar
Attributes: TRACE\_GREP, TRACE, TRACE\_COLOR, TRACE\_COLOR\_\textless{}attr\textgreater{}

\sphinxAtStartPar
Substitutions: \%\_


\section{Built in pretty\sphinxhyphen{}printing of attributes with the parenmatch() function}
\label{\detokenize{features:built-in-pretty-printing-of-attributes-with-the-parenmatch-function}}
\sphinxAtStartPar
Example Code Output:

\begin{sphinxVerbatim}[commandchars=\\\{\}]
\PYG{n+nd}{@emit} \PYG{p}{[}\PYG{n}{add}\PYG{p}{(}\PYG{l+m+mi}{1}\PYG{p}{,}\PYG{n}{sub}\PYG{p}{(}\PYG{l+m+mi}{2}\PYG{p}{,}\PYG{n}{div}\PYG{p}{(}\PYG{l+m+mi}{3}\PYG{p}{,}\PYG{l+m+mi}{4}\PYG{p}{)}\PYG{p}{)}\PYG{p}{,}\PYG{l+m+mi}{5}\PYG{p}{)}\PYG{p}{]}\PYG{p}{;}\PYG{n+nd}{@emit} \PYG{p}{[}\PYG{n}{extract}\PYG{p}{(}\PYG{n}{get}\PYG{p}{(}\PYG{n}{me}\PYG{o}{/}\PYG{n}{foo}\PYG{p}{)}\PYG{p}{,}\PYG{l+m+mi}{3}\PYG{p}{,}\PYG{l+m+mi}{1}\PYG{p}{)}

\PYG{n}{Example} \PYG{n}{Pretty} \PYG{n}{Print}\PYG{p}{:}
\PYG{n+nd}{@emit} \PYG{p}{[}
   \PYG{n}{add}\PYG{p}{(}
      \PYG{l+m+mi}{1}\PYG{p}{,}\PYG{n}{sub}\PYG{p}{(}
         \PYG{l+m+mi}{2}\PYG{p}{,}\PYG{n}{div}\PYG{p}{(}
            \PYG{l+m+mi}{3}\PYG{p}{,}\PYG{l+m+mi}{4}
         \PYG{p}{)}
      \PYG{p}{)}\PYG{p}{,}\PYG{l+m+mi}{5}
   \PYG{p}{)}
\PYG{p}{]}\PYG{p}{;}\PYG{n+nd}{@emit} \PYG{p}{[}
   \PYG{n}{extract}\PYG{p}{(}
      \PYG{n}{get}\PYG{p}{(}
         \PYG{n}{me}\PYG{o}{/}\PYG{n}{foo}
      \PYG{p}{)}\PYG{p}{,}\PYG{l+m+mi}{3}\PYG{p}{,}\PYG{l+m+mi}{1}
   \PYG{p}{)}
\PYG{p}{]}
\end{sphinxVerbatim}


\section{Plenty more not mentioned!}
\label{\detokenize{features:plenty-more-not-mentioned}}
\sphinxAtStartPar
The flexibility to customize RhostMUSH is what is most daunting.
Don\textquotesingle{}t fret, you don\textquotesingle{}t need to do it to run RhostMUSH successfully.
In fact, the default configuration is mostly compatible with
MUSH and will work correctly out of the box for most needs.  For those
wishing to play, of course the sky is the limit of what you want to
do.


\section{Advanced features of RhostMUSH}
\label{\detokenize{features:advanced-features-of-rhostmush}}

\subsection{Debugging/Tracing}
\label{\detokenize{features:debugging-tracing}}\begin{itemize}
\item {} 
\sphinxAtStartPar
Debugging in Rhost allows for advanced features like expressing where and
when to do debugging via a trace() function, with toggled labels, and the
ability to grep content from trace output.  There also exists features to
color\sphinxhyphen{}match parenthesis, braces, and brackets in\sphinxhyphen{}game as well as pretty print
the output of commands and functions.
\begin{itemize}
\item {} 
\sphinxAtStartPar
help trace

\item {} 
\sphinxAtStartPar
help \%\_

\item {} 
\sphinxAtStartPar
help trace()

\item {} 
\sphinxAtStartPar
help parenmatch()

\item {} 
\sphinxAtStartPar
help parenstr()

\end{itemize}

\end{itemize}


\subsection{Zoning}
\label{\detokenize{features:zoning}}\begin{itemize}
\item {} 
\sphinxAtStartPar
Zoning in Rhost allows the same functionality of Penn and MUX, though the
syntax is different.  It also allows the ability to belong to multiple
zones at the same time and take advantage of mulitple zones at once.
This allows for increased levels of complexity.
\begin{itemize}
\item {} 
\sphinxAtStartPar
help zones

\item {} 
\sphinxAtStartPar
help @zone

\item {} 
\sphinxAtStartPar
help zonecmd()

\item {} 
\sphinxAtStartPar
help lzone()

\item {} 
\sphinxAtStartPar
help @Lock type twink

\item {} 
\sphinxAtStartPar
help @lock type zone

\end{itemize}

\end{itemize}


\subsection{Reality Levels}
\label{\detokenize{features:reality-levels}}\begin{itemize}
\item {} 
\sphinxAtStartPar
Reality levels allows for the ability to have a sandboxed \textquotesingle{}existance\textquotesingle{}
in each physical location across the entirity of the mush.  Each
reality is its own sandbox and can either stand alone or work
dependently with other realities.  A person can belong to multiple
realities at the same time, and realities is geared to a method for
send and receive.  Each \textquotesingle{}method\textquotesingle{} requires to be in the given reality
to affect it.
\begin{itemize}
\item {} 
\sphinxAtStartPar
help reality levels

\item {} 
\sphinxAtStartPar
wizhelp chkreality

\item {} 
\sphinxAtStartPar
wizhelp reaity level

\item {} 
\sphinxAtStartPar
help @Lock type user

\end{itemize}

\end{itemize}


\subsection{Function and Command Overriding}
\label{\detokenize{features:function-and-command-overriding}}\begin{itemize}
\item {} 
\sphinxAtStartPar
Functions and commands can both be overridden with softcode.  To
override a hardcoded command you first set the command ignore.
There are various levels of ignoring so that you could have it
ignored from mortals but have it executed fine for non\sphinxhyphen{}mortals.
This allows you to use the actual physical command within a
softcode override.  You may also use @Hook for altering how
a command works.   Functions are overridden by setting the
function in question ignored, then writing a softcode alternative
that is then executed and fetched appropriately.

\sphinxAtStartPar
Commands:
\begin{itemize}
\item {} 
\sphinxAtStartPar
wizhelp @admin

\item {} 
\sphinxAtStartPar
wizhelp access

\item {} 
\sphinxAtStartPar
wizhelp permissions

\item {} 
\sphinxAtStartPar
wizhelp @Hook

\item {} 
\sphinxAtStartPar
wizhelp hook setup

\end{itemize}

\sphinxAtStartPar
Functions:
\begin{itemize}
\item {} 
\sphinxAtStartPar
wizhelp @admin

\item {} 
\sphinxAtStartPar
wizhelp function\_access

\item {} 
\sphinxAtStartPar
wizhelp @function

\item {} 
\sphinxAtStartPar
help @lfunction

\item {} 
\sphinxAtStartPar
wizhelp bypass()

\end{itemize}

\end{itemize}


\subsection{The Recycle Bin}
\label{\detokenize{features:the-recycle-bin}}\begin{itemize}
\item {} 
\sphinxAtStartPar
Rhost has a recycle bin that works a bit like a windows recycle bin.
Whenever you destroy something within the mush, it is stacked onto
the recycle bin and marked unavailable within the mush.  This marks
the dbref\# as garbage in any sense of the word.  However, the object
is not able to be reused until purged.  Once purged, it is put onto
a free list that can then be reassigned to a new object.
\begin{itemize}
\item {} 
\sphinxAtStartPar
wizhelp @nuke

\item {} 
\sphinxAtStartPar
wizhelp @destroy

\item {} 
\sphinxAtStartPar
wizhelp @toad

\item {} 
\sphinxAtStartPar
wizhelp @turtle

\item {} 
\sphinxAtStartPar
wizhelp @purge

\item {} 
\sphinxAtStartPar
wizhelp @recover

\item {} 
\sphinxAtStartPar
wizhelp @reclist

\end{itemize}

\end{itemize}


\subsection{Percent Substitution Adding/Overriding}
\label{\detokenize{features:percent-substitution-adding-overriding}}\begin{itemize}
\item {} 
\sphinxAtStartPar
Rhost allows the ability to both override percent substitutions as
well as creating new ones.  This is done with @Hook and admin
params and issues softcode overriding.  Due to how it is evaluated
there is no risk of recursion.
\begin{itemize}
\item {} 
\sphinxAtStartPar
wizhelp @hook

\item {} 
\sphinxAtStartPar
wizhelp hook\_cmd

\item {} 
\sphinxAtStartPar
wizhelp sub\_include

\item {} 
\sphinxAtStartPar
wizhelp sub\_override

\end{itemize}

\end{itemize}


\subsection{Hooking}
\label{\detokenize{features:hooking}}\begin{itemize}
\item {} 
\sphinxAtStartPar
Hooking allows you to have advanced methods to manipulate commands
including adding customized switches to them via softcode.
\begin{itemize}
\item {} 
\sphinxAtStartPar
wizhelp @hook

\item {} 
\sphinxAtStartPar
wizhelp hook\_cmd

\item {} 
\sphinxAtStartPar
wizhelp hook\_obj

\item {} 
\sphinxAtStartPar
wizhelp hook setup

\end{itemize}

\end{itemize}


\subsection{Command based uselocks}
\label{\detokenize{features:command-based-uselocks}}\begin{itemize}
\item {} 
\sphinxAtStartPar
This allows you to have unique uselocks per \$command.  This is done
through the use of the USELOCK attribute flag, then you set up
a matching attribute name with a prefix of a \textasciitilde{} to specify how
the lock is to be evaluated.  This works in the same manner
as an evaluation lock.  To be able to use the USELOCK attribute flag
you must be empowered to do so with the \textquotesingle{}ATRUSE\textquotesingle{} @toggle.  You may
also use the secure\_atruselock config parameter to globally enable
this and not require the toggle to be set.
\begin{itemize}
\item {} 
\sphinxAtStartPar
wizhelp atruse toggle

\item {} 
\sphinxAtStartPar
help attribute uselocks

\end{itemize}

\end{itemize}


\subsection{Differentating between command and listen locks}
\label{\detokenize{features:differentating-between-command-and-listen-locks}}\begin{itemize}
\item {} 
\sphinxAtStartPar
We distinguish between commands and listens with uselocks by passing
an optional argument to all locks that are uselocks.  This optional
argument is 0 for a default lock, 1 for a command lock and 2 for
a listen lock.
\begin{itemize}
\item {} 
\sphinxAtStartPar
help @lock type uselock

\end{itemize}

\end{itemize}


\subsection{Wizard auto\sphinxhyphen{}overiding and how to disable it}
\label{\detokenize{features:wizard-auto-overiding-and-how-to-disable-it}}\begin{itemize}
\item {} 
\sphinxAtStartPar
By default wizards override all locks, including attribute locks,
can see all dark exits, and bypass pagelocks.  This can be
troublesome, and even abusive, so there\textquotesingle{}s ways to disable this.
\begin{itemize}
\item {} 
\sphinxAtStartPar
wizhelp @depower (for those abusing it)

\item {} 
\sphinxAtStartPar
wizhelp no\_override (disable overiding locks)

\item {} 
\sphinxAtStartPar
wizhelp no\_uselock (disable just uselock overriding)

\item {} 
\sphinxAtStartPar
wizhelp pagelock toggle (disable pagelock overriding)

\end{itemize}

\end{itemize}


\subsection{Advanced FLAG and TOGGLE control}
\label{\detokenize{features:advanced-flag-and-toggle-control}}\begin{itemize}
\item {} 
\sphinxAtStartPar
Flags and toggles can be controlled to have multiple permissions
and enable/disable targets of how the flags are allowed to be
set.  This is done through commands in\sphinxhyphen{}game or you can use
conf file options to do so.
\begin{itemize}
\item {} 
\sphinxAtStartPar
wizhelp @flagdef

\item {} 
\sphinxAtStartPar
wizhelp @toggledef

\item {} 
\sphinxAtStartPar
@admin @flagdef alternatives
\begin{itemize}
\item {} 
\sphinxAtStartPar
wizhelp flag\_access\_set

\item {} 
\sphinxAtStartPar
wizhelp flag\_access\_unset

\item {} 
\sphinxAtStartPar
wizhelp flag\_access\_see

\item {} 
\sphinxAtStartPar
wizhelp flag\_access\_type

\end{itemize}

\item {} 
\sphinxAtStartPar
@admin @toggledef alternatives
\begin{itemize}
\item {} 
\sphinxAtStartPar
wizhelp toggle\_access\_set

\item {} 
\sphinxAtStartPar
wizhelp toggle\_access\_unset

\item {} 
\sphinxAtStartPar
wizhelp toggle\_access\_see

\item {} 
\sphinxAtStartPar
wizhelp toggle\_access\_type

\end{itemize}

\end{itemize}

\end{itemize}


\subsection{Advanced site control}
\label{\detokenize{features:advanced-site-control}}\begin{itemize}
\item {} 
\sphinxAtStartPar
We allow advanced site control by not only blocking various sites
but we can specify how many times a player can be connected at the
same time as well as how many times sites are able to connect at
the same time.  This is done through normal site manipulation.
\begin{itemize}
\item {} 
\sphinxAtStartPar
wizhelp forbid\_host

\item {} 
\sphinxAtStartPar
wizhelp register\_host

\item {} 
\sphinxAtStartPar
wizhelp noguest\_host

\item {} 
\sphinxAtStartPar
wizhelp @list (site option)

\end{itemize}

\end{itemize}


\subsection{Auto\sphinxhyphen{}Registration}
\label{\detokenize{features:auto-registration}}\begin{itemize}
\item {} 
\sphinxAtStartPar
Autoregistration is the method that a player can auto\sphinxhyphen{}register
by providing their email on the connect screen.  It will email
them a password and an optional document that the administrator
provides.  This is well described in the wizhelp.
\begin{itemize}
\item {} 
\sphinxAtStartPar
wizhelp autoregistration

\end{itemize}

\end{itemize}


\subsection{Which bit level is best?}
\label{\detokenize{features:which-bit-level-is-best}}\begin{itemize}
\item {} 
\sphinxAtStartPar
This is something that should be discussed by you and the staff
of your game.  As a good rule of thumb, only provide the bitlevel
that is required to do the job.  Too much power is always risky.
In essence, each tier of bit can do everything the previous bitlevel
can do, and then additional stuff on top of it.  The highest bitlevel
is \#1 itself, being bitlevel 7.  Then immortal, which should be
considered the \#1 character in most cases and is bitlevel 6.  Then
the royalty character, which is equal to wizard on penn, mux, or
other codebases.  For most things, this is the bitlevel you want
to assign players.  The exception will likely be game owners or
people who control the master room code.

\sphinxAtStartPar
There\textquotesingle{}s a bunch of readme files and online wizhelp that goes into
detail of the various bitlevels and what each can do.
\begin{itemize}
\item {} 
\sphinxAtStartPar
wizhelp control

\end{itemize}

\end{itemize}


\subsection{What are the limits for size/growth for RhostMUSH?}
\label{\detokenize{features:what-are-the-limits-for-size-growth-for-rhostmush}}
\sphinxAtStartPar
While using QDBM, there\textquotesingle{}s really no set limits for most things.
The limits that we have are as followed:
\begin{itemize}
\item {} 
\sphinxAtStartPar
LBUF \sphinxhyphen{} 64K.  It is recommended to only use 32K as there is some issues with networking with 64K lbufs.
\begin{itemize}
\item {} 
\sphinxAtStartPar
Compile time option with the menu configurator

\end{itemize}

\item {} 
\sphinxAtStartPar
SBUF \sphinxhyphen{} 64 characters (if configured \sphinxhyphen{}\sphinxhyphen{} it\textquotesingle{}s suggested you do).
\begin{itemize}
\item {} 
\sphinxAtStartPar
Compile time option with the menu configurator

\end{itemize}

\item {} 
\sphinxAtStartPar
MBUF \sphinxhyphen{} 200 characters.  Not able to be changed.

\item {} 
\sphinxAtStartPar
MAX CONNECTIONS \sphinxhyphen{} Limited by the total number of open sockets and descriptors on the account and server running on.  There are various tools to limit connection DoS attemps and other such nastiness.  This is well documented in the netrhost.conf file.
\begin{itemize}
\item {} 
\sphinxAtStartPar
wizhelp max\_players

\item {} 
\sphinxAtStartPar
\textasciitilde{}/game/netrhost.conf

\end{itemize}

\item {} 
\sphinxAtStartPar
MEMORY \sphinxhyphen{} no limit.  Generally runs between 8\sphinxhyphen{}50M depending on the size of the mush and the LBUF size specified.

\item {} 
\sphinxAtStartPar
CPU  \sphinxhyphen{} no limit, but has built in cpu abort in code.  The netrhost.conf file documents this well for customizing.  the default values are usually good enough.
\begin{itemize}
\item {} 
\sphinxAtStartPar
wizhelp max\_cpu\_cycles

\item {} 
\sphinxAtStartPar
wizhelp cpuintervalchk

\item {} 
\sphinxAtStartPar
wizhelp cputimechk

\item {} 
\sphinxAtStartPar
wizhelp cpu\_secure\_lvl

\item {} 
\sphinxAtStartPar
wizhelp heavy\_cpu\_max

\item {} 
\sphinxAtStartPar
\textasciitilde{}/game/netrhost.conf

\end{itemize}

\item {} 
\sphinxAtStartPar
DISK \sphinxhyphen{} no limit.  Generally will be 75\sphinxhyphen{}200M depending on size, number of backups and if you leave your compiled object files in.

\item {} 
\sphinxAtStartPar
DB Size \sphinxhyphen{} (20000 default) There is no limit on the number of objects the db can have.  By default it\textquotesingle{}s soft limited to 20K objects, which can be changed by a netrhost.conf file change.  We have had this up past 1.5 million objects, and other than a second or two of lag for complex searches we had no real problem.
\begin{itemize}
\item {} 
\sphinxAtStartPar
wizhelp maximum\_size

\item {} 
\sphinxAtStartPar
help @quota

\item {} 
\sphinxAtStartPar
wizhelp @quota

\item {} 
\sphinxAtStartPar
wizhelp @limit

\end{itemize}

\item {} 
\sphinxAtStartPar
Attribute Size \sphinxhyphen{} 10K as a hard limit.  750 as a soft limit.  You can increase this but it can\textquotesingle{}t exceed 10000 attributes.  This is to avoid DoS style attacks.
\begin{itemize}
\item {} 
\sphinxAtStartPar
wizhelp vlimit

\item {} 
\sphinxAtStartPar
wizhelp @limit

\end{itemize}

\end{itemize}


\subsection{Sqlite and MySQL/Maria setup and why use it?}
\label{\detokenize{features:sqlite-and-mysql-maria-setup-and-why-use-it}}\begin{itemize}
\item {} 
\sphinxAtStartPar
Both of these can be configured separately or conjointly to
run in parallel.  This can be done through the RhostMUSH
configuration utility.  You generally want to use SQL for
external data storage or accessing a central repository
of data to share between multiple projects.  Like, for
example between a wiki, a forum, and the mush.

\end{itemize}


\subsection{Executing outside scripts and binaries within RhostMUSH}
\label{\detokenize{features:executing-outside-scripts-and-binaries-within-rhostmush}}\begin{itemize}
\item {} 
\sphinxAtStartPar
Rhost has execscript() which allows executing outside binaries
or scripts as a native function.  All effort has been done to
avoid any type of DoS based issue or hang by doing this, however
the guidelines presented should be followed before doing so.
\begin{itemize}
\item {} 
\sphinxAtStartPar
wizhelp execscript

\item {} 
\sphinxAtStartPar
wizhelp power execscript

\item {} 
\sphinxAtStartPar
help sidefx

\item {} 
\sphinxAtStartPar
wizhelp writing scripts

\end{itemize}

\end{itemize}


\subsection{Pulling external data into RhostMUSH}
\label{\detokenize{features:pulling-external-data-into-rhostmush}}\begin{itemize}
\item {} 
\sphinxAtStartPar
You are capable of pulling external data in to RhostMUSH using several methods.  These are by using:
\begin{itemize}
\item {} 
\sphinxAtStartPar
SQL (mysql or sqlite)

\item {} 
\sphinxAtStartPar
execscript

\item {} 
\sphinxAtStartPar
cron (wizhelp signal)

\end{itemize}

\end{itemize}


\subsection{Integrating a unix cron right into RhostMUSH}
\label{\detokenize{features:integrating-a-unix-cron-right-into-rhostmush}}\begin{itemize}
\item {} 
\sphinxAtStartPar
The unix cron can be used to integrate with Rhost fairly
easilly by use of signals.  By using SIGUSR1 you can specify
Rhost to execute code in\sphinxhyphen{}game, which part of that could be
to pick up a pre\sphinxhyphen{}designed list of commands that the unix cron
has set up.
\begin{itemize}
\item {} 
\sphinxAtStartPar
wizhelp signal

\item {} 
\sphinxAtStartPar
wizhelp signal\_cron

\end{itemize}

\end{itemize}


\subsection{Signal handling, how it works, and when and why use it}
\label{\detokenize{features:signal-handling-how-it-works-and-when-and-why-use-it}}\begin{itemize}
\item {} 
\sphinxAtStartPar
Signals are used to do different things for the mush.  By default,
the following signals are recognized by the mush and will do
the following as defaults.
\begin{itemize}
\item {} 
\sphinxAtStartPar
SIGUSR1 \sphinxhyphen{} will do a reboot of the mush.  This is also customizable so that you can have it execute code in\sphinxhyphen{}mush if you want.

\item {} 
\sphinxAtStartPar
SIGUSR2 \sphinxhyphen{} will do a clean shutdown of the mush.

\item {} 
\sphinxAtStartPar
SIGTERM \sphinxhyphen{} will immediately scram the mush as cleanly and fast as possible.  It will avoid dumping anything to the database to speed up scramming, but write a TERM flat file to be loaded in if corruption.

\end{itemize}

\end{itemize}


\subsection{Setting up global parents, global @parents, global attribute formatting, and other global setups}
\label{\detokenize{features:setting-up-global-parents-global-parents-global-attribute-formatting-and-other-global-setups}}\begin{itemize}
\item {} 
\sphinxAtStartPar
Global parents are useful when you want to have a global \textquotesingle{}parent\textquotesingle{}
without actually having a defined \sphinxhref{mailto:'@parent}{\textquotesingle{}@parent}\textquotesingle{}.  It always will be the
highest tier in a lookup.  The way lookups will go will be:

\begin{sphinxVerbatim}[commandchars=\\\{\}]
\PYG{n+nb+bp}{self}\PYG{o}{\PYGZhy{}}\PYG{o}{\PYGZgt{}}\PYG{n+nd}{@parent}\PYG{p}{(}\PYG{n}{s}\PYG{p}{)}\PYG{o}{\PYGZhy{}}\PYG{o}{\PYGZgt{}}\PYG{n+nd}{@zone}\PYG{p}{(}\PYG{n}{s}\PYG{p}{)}\PYG{o}{\PYGZhy{}}\PYG{o}{\PYGZgt{}}\PYG{n}{GlobalParent}
\end{sphinxVerbatim}

\end{itemize}
\begin{quote}

\sphinxAtStartPar
The type of the parent does not have to match the target.

\sphinxAtStartPar
These global parents can be defined either by using a global
generic parent or by using the type.  If a type is specified it
overrides the generic.  The following parameters are used:
\begin{itemize}
\item {} 
\sphinxAtStartPar
global\_parent\_obj     \sphinxhyphen{} The generic global parent (if defined)

\item {} 
\sphinxAtStartPar
global\_parent\_room    \sphinxhyphen{} The room global parent

\item {} 
\sphinxAtStartPar
global\_parent\_exit    \sphinxhyphen{} The exit global parent

\item {} 
\sphinxAtStartPar
global\_parent\_thing   \sphinxhyphen{} The thing global parent

\item {} 
\sphinxAtStartPar
global\_parent\_player  \sphinxhyphen{} The player global parent

\end{itemize}
\end{quote}
\begin{itemize}
\item {} 
\sphinxAtStartPar
Global @parents are different than global parents in that any new
item of similar type that is created is automatically assigned this
physical @parent.  It\textquotesingle{}s obviously more limiting since it sets
the literal physical parent defined.

\sphinxAtStartPar
The type of the parent does not have to match the target.

\sphinxAtStartPar
The following parameters are used:
\begin{itemize}
\item {} 
\sphinxAtStartPar
room\_parent           \sphinxhyphen{} The target that new rooms are @parented

\item {} 
\sphinxAtStartPar
exit\_parent           \sphinxhyphen{} The target that new exits are @parented

\item {} 
\sphinxAtStartPar
thing\_parent          \sphinxhyphen{} The target that new things are @parented

\item {} 
\sphinxAtStartPar
player\_parent         \sphinxhyphen{} The target that new players are @parented

\end{itemize}

\item {} 
\sphinxAtStartPar
Global attribute formatting is a method define a wrapper, of sorts,
where attributes like @desc, @odesc, @succ, and anything similar
can be processed through this.  All attributes will be either
\&FORMAT\textless{}attribute\textgreater{} or \&\textless{}attribute\textgreater{}FORMAT based on the current
configuration.  Example: \&FORMATDESC or \&DESCFORMAT localy, or
use the following global objects for global formatting.  Local
formatting has priority.

\sphinxAtStartPar
The type of the parent does not have to match the target.
\begin{itemize}
\item {} 
\sphinxAtStartPar
room\_attr\_default     \sphinxhyphen{} Target for room formatting

\item {} 
\sphinxAtStartPar
exit\_attr\_default     \sphinxhyphen{} Target for exit formatting

\item {} 
\sphinxAtStartPar
thing\_attr\_default    \sphinxhyphen{} Target for thing formatting

\item {} 
\sphinxAtStartPar
player\_attr\_default   \sphinxhyphen{} Target for player formatting

\end{itemize}

\end{itemize}


\subsection{RhostMUSH limitations and how to get around them}
\label{\detokenize{features:rhostmush-limitations-and-how-to-get-around-them}}
\sphinxAtStartPar
While Rhost is insanely configurable and quite powerful, there are
limitations that exist within it.
\begin{itemize}
\item {} 
\sphinxAtStartPar
Function invocations.  Sometimes you will hit a ceiling on evaluation.
You may want to tweak values to allow more functions or commands
to execute.  The following controls that:
\begin{itemize}
\item {} 
\sphinxAtStartPar
function\_invocation\_limit {[}25000 default{]} \sphinxhyphen{} specifies the total functions you can execute per command.

\item {} 
\sphinxAtStartPar
function\_recursion\_limit {[}50{]} \sphinxhyphen{} specifies the total times a function can call itself over and over.  Rarely should this be increased and doing so can effect your stack depth.

\end{itemize}

\item {} 
\sphinxAtStartPar
Command queue limits.  Sometimes you want more to be queued up for
players or wizards.
\begin{itemize}
\item {} 
\sphinxAtStartPar
player\_queue\_limit  \sphinxhyphen{} Max number of entries a player can queue

\item {} 
\sphinxAtStartPar
wizard\_queue\_limit  \sphinxhyphen{} Max number of entries a wizard can queue

\end{itemize}

\item {} 
\sphinxAtStartPar
@limit is a wonderful way to lock down limitations per player or global.  Lots of power is available here.
\begin{itemize}
\item {} 
\sphinxAtStartPar
@limit

\item {} 
\sphinxAtStartPar
vattr\_limit\_checkwiz \sphinxhyphen{} Enable @limit checks for wizards

\item {} 
\sphinxAtStartPar
wizmax\_vattr\_limit   \sphinxhyphen{} Set wizard global VATTR limits

\item {} 
\sphinxAtStartPar
wizmax\_dest\_limit    \sphinxhyphen{} Set wizard global @destroy limits

\item {} 
\sphinxAtStartPar
max\_vattr\_limit      \sphinxhyphen{} Set player global VATTR limits

\item {} 
\sphinxAtStartPar
max\_dest\_limit       \sphinxhyphen{} Set player global @destroy limits

\end{itemize}

\item {} 
\sphinxAtStartPar
Lots of trace output can be cut off.  You can modify this with:
\begin{itemize}
\item {} 
\sphinxAtStartPar
trace\_output\_limit   \sphinxhyphen{} Set lines of trace output shown

\end{itemize}

\item {} 
\sphinxAtStartPar
To define how many commands a minute a player set SPAMMONITOR can use
\begin{itemize}
\item {} 
\sphinxAtStartPar
spam\_limit \sphinxhyphen{}\sphinxhyphen{} default 120

\end{itemize}

\item {} 
\sphinxAtStartPar
If you examine things and see \textquotesingle{}Output cut off\textquotesingle{} messages, you may want
to increase your output limit, funny enough, the name of this is
similar
\begin{itemize}
\item {} 
\sphinxAtStartPar
output\_limit \sphinxhyphen{} You should set this no less than 4 times the current size of your LBUF.

\end{itemize}

\item {} 
\sphinxAtStartPar
Attributes names can not exceed 64 characters.  Sorry, it\textquotesingle{}s a hard limit

\item {} 
\sphinxAtStartPar
Sometime you may find a single may not work for an escape.  You can
in most cases use a \% instead or double escape the to make it work.
Also look at lit() as a solution.

\end{itemize}


\subsection{Advanced guest setup}
\label{\detokenize{features:advanced-guest-setup}}\begin{itemize}
\item {} 
\sphinxAtStartPar
After you set up your guests, you can set unique names to each guest
if you so want after defining the dbref\#\textquotesingle{}s your guests use.  This is
done by defining them in the guest\_namelist parameter.  You can also
increase guests (or decrease them) between 0\sphinxhyphen{}31 guests.
\begin{itemize}
\item {} 
\sphinxAtStartPar
wizhelp guest\_namelist

\item {} 
\sphinxAtStartPar
wizhelp num\_guests

\end{itemize}

\end{itemize}


\subsection{Attribute permission masking and the joys of the power behind it}
\label{\detokenize{features:attribute-permission-masking-and-the-joys-of-the-power-behind-it}}\begin{itemize}
\item {} 
\sphinxAtStartPar
Attribute contentlocks can be set up so you can lock the actual
content that you can set (or even unset!) into an attribute.
The beauty of this is that you can specify case sensitive
information, lock different ways contents in attributes are set
based on who is setting it, or even on where it\textquotesingle{}s being set.
The sky\textquotesingle{}s the limit.
\begin{itemize}
\item {} 
\sphinxAtStartPar
global\_attrdefault    \sphinxhyphen{} Target for defining content locks

\end{itemize}

\end{itemize}


\subsection{The amazing @cluster and what it can do for you}
\label{\detokenize{features:the-amazing-cluster-and-what-it-can-do-for-you}}\begin{itemize}
\item {} 
\sphinxAtStartPar
Clusters is the way to virtually assign multiple objects into
a single physical object.  It essentially chains together two
or more objects to share attributes between them, so that any
attribute set on any object in that cluster can be set or fetched
as if it was a singular entity.  This allows some amazing ability
to distribute attribute content or even have a farm of a massive
amount of attributes without paying a hefty penalty on object bloat.
\begin{itemize}
\item {} 
\sphinxAtStartPar
help cluster  \sphinxhyphen{}\sphinxhyphen{} Gives a fantastic overview of how clusters work.

\end{itemize}

\end{itemize}


\subsection{What we plan for the future}
\label{\detokenize{features:what-we-plan-for-the-future}}\begin{itemize}
\item {} 
\sphinxAtStartPar
Things to look forward to the future with RhostMUSH.
\begin{itemize}
\item {} 
\sphinxAtStartPar
Full Unicode/UTF8 in Rhost 4.0

\item {} 
\sphinxAtStartPar
A fully featured tag system in Rhost 4.0

\item {} 
\sphinxAtStartPar
Built in Python API handler in Rhost 4.0

\item {} 
\sphinxAtStartPar
Hopefully a built in Ruby and Perl API in Rhost 4.0/4.1

\item {} 
\sphinxAtStartPar
Cross\sphinxhyphen{}Mush execution between mushes in Rhost 4.1

\item {} 
\sphinxAtStartPar
More as we think about it :)

\end{itemize}

\end{itemize}


\subsection{Additional features not covered otherwise}
\label{\detokenize{features:additional-features-not-covered-otherwise}}\begin{itemize}
\item {} 
\sphinxAtStartPar
+/\sphinxhyphen{} 5.4 million years can be utilized with the built in time functions
which includes timefmt(), secs(), convtime(), convsecs(), and moon().  Party on!

\item {} 
\sphinxAtStartPar
Changing permission levels in the middle of execution for evaluation.
\begin{itemize}
\item {} 
\sphinxAtStartPar
see help on the streval and ueval function\textquotesingle{}

\end{itemize}

\item {} 
\sphinxAtStartPar
Full features in\sphinxhyphen{}game customization of near every aspect of the game.

\end{itemize}


\section{Random notes and things to know about RhostMUSH}
\label{\detokenize{features:random-notes-and-things-to-know-about-rhostmush}}
\sphinxAtStartPar
Here are some things to know about RhostMUSH and what you may or may not
want to do.  Things here are not covered in other documents:

\sphinxAtStartPar
Admin toggles to configure the WHO, various things you\textquotesingle{}re used to, etc is in the
\textquotesingle{}netrhost.conf\textquotesingle{} file.  descriptions Notes in the autoconf.h file is in the
README.AUTOCONF file.


\subsection{Note on bits, their levels, and things they do}
\label{\detokenize{features:note-on-bits-their-levels-and-things-they-do}}\begin{itemize}
\item {} 
\sphinxAtStartPar
IMMORTAL \sphinxhyphen{} They can do anything.  Treat this as \#1 and only give to
people you trust.  Period.   You don\textquotesingle{}t have to use this bit
if you do not want to and just assume \#1.

\item {} 
\sphinxAtStartPar
ROYALTY \sphinxhyphen{} Unlike PENN/MUX, this is \sphinxstyleemphasis{not} a sub\sphinxhyphen{}wizard, this is a
FULL wizard.  Plus, they can do a bit more.

\item {} 
\sphinxAtStartPar
COUNCILOR \sphinxhyphen{} Like royalty on PENN/MUX but they can modify.

\item {} 
\sphinxAtStartPar
ARCHITECT \sphinxhyphen{} Can\textquotesingle{}t do as much as councilor, but lot more than BUILDER.

\item {} 
\sphinxAtStartPar
GUILDMASTER \sphinxhyphen{} Very limited.  Sees dbrief\#\textquotesingle{}s, can ex things their
level and lower and @quota players.

\end{itemize}


\subsection{You need to @pcreate your guest characters and set them GUEST}
\label{\detokenize{features:you-need-to-pcreate-your-guest-characters-and-set-them-guest}}
\sphinxAtStartPar
It doesn\textquotesingle{}t create them on the fly but we considered this better.
You have 31 total you can have.  It defaults to 10 in the
netrhost.conf file.  You can rename the guests anything you want,
but before you do so, you must add the dbref\#\textquotesingle{}s to the param
guest\_namelist


\subsection{@powers are INHERITED}
\label{\detokenize{features:powers-are-inherited}}
\sphinxAtStartPar
Therefore, you need power\_objects enabled (@admin)
to make this work properly for non\sphinxhyphen{}plauyers.
A power is taken before a bit level ONLY if higher than that bit.
Yes, powers are multi\sphinxhyphen{}level.


\subsection{@depowers are automatically checked first before anything else}
\label{\detokenize{features:depowers-are-automatically-checked-first-before-anything-else}}
\sphinxAtStartPar
This is also meaningless on objects.


\subsection{Zones are unique}
\label{\detokenize{features:zones-are-unique}}
\sphinxAtStartPar
You can have things in multiple zones.


\subsection{The db auto\sphinxhyphen{}repares itself when it can}
\label{\detokenize{features:the-db-auto-repares-itself-when-it-can}}
\sphinxAtStartPar
It does this by purging anything
it can\textquotesingle{}t identify.  Dataloss is better than unrecoverable data.
Yes, any such \textquotesingle{}repairing\textquotesingle{} is logged so you know if something is up.


\subsection{You can get your connect.txt to parse ansi}
\label{\detokenize{features:you-can-get-your-connect-txt-to-parse-ansi}}
\sphinxAtStartPar
See ansi\_txtfiles in wizhelp.

\sphinxAtStartPar
You can also override it with softcode if you so wanted.

\sphinxAtStartPar
See file\_object in wizhelp for more information on this.


\subsection{Re\sphinxhyphen{}compiled binaries do not require an @shutdown}
\label{\detokenize{features:re-compiled-binaries-do-not-require-an-shutdown}}
\sphinxAtStartPar
When re\sphinxhyphen{}compiling the binaries, all you have to remember is when done, issue

\sphinxAtStartPar
@reboot on the game and @readcache.

\sphinxAtStartPar
You do not need to @shutdown.


\subsection{Softcode emulations of functions from other servers are included}
\label{\detokenize{features:softcode-emulations-of-functions-from-other-servers-are-included}}
\sphinxAtStartPar
Load the file softfunctions into the mush once it\textquotesingle{}s set up.  This are
@functions that will alias the functions that PENN, MUX, and TinyMUSH have
that is either named differently or we don\textquotesingle{}t have for one reason or another.


\chapter{What FLAGS, TOGGLES, POWERS, and DEPOWERS mean in RhostMUSH}
\label{\detokenize{features:what-flags-toggles-powers-and-depowers-mean-in-rhostmush}}

\section{What are Flags?}
\label{\detokenize{features:what-are-flags}}
\sphinxAtStartPar
Flags are pretty much exactly the same as any other mush.  It\textquotesingle{}s a flag
that you set or unset on a target which then enables/disables or
alters something that target can do.  There\textquotesingle{}s help on all the flags
in help and wizhelp.


\section{What are Toggles?}
\label{\detokenize{features:what-are-toggles}}
\sphinxAtStartPar
Toggles were designed as a single point flag that immediately enables
or disables a set ability or condition, thus a \textquotesingle{}toggle\textquotesingle{}.  It works
exactly like a flag and was originally designed for two reasons.  To
distinguish from the multi\sphinxhyphen{}meaning of a \textquotesingle{}flag\textquotesingle{}, and because frankly
we ran out of flag space :)


\section{What is an @power?}
\label{\detokenize{features:what-is-an-power}}
\sphinxAtStartPar
A power is similar to a power on other mushes, but unlike them, our
powers are multi\sphinxhyphen{}tier.  This means that they can be customized to
empower something at a given bitlevel.  You may empower something
from guildmaster up to councilor level.  There are some powers
with a power level of N/A meaning they are a toggle power granting
an absolute power level as specified in the help for that power.
This requires the INHERIT flag for non\sphinxhyphen{}players to inherit powers,
however, a specific object can be granted a power as well.


\section{What is an @depower?}
\label{\detokenize{features:what-is-an-depower}}
\sphinxAtStartPar
This is the anti\sphinxhyphen{}thesis of @power.  Also, depowers do not require
inheritance.  They also have priority over flags, toggles, and
powers.  You can use depower to remove or lower abilities and
control from a target, even a full wizard (royalty) can be
depowered.
\phantomsection\label{\detokenize{features/bitlevels:bitlevels}}
\index{Bitlevels@\spxentry{Bitlevels}}\ignorespaces 

\chapter{Multi\sphinxhyphen{}tiered bitlevel systems}
\label{\detokenize{features/bitlevels:multi-tiered-bitlevel-systems}}\label{\detokenize{features/bitlevels:index-0}}\label{\detokenize{features/bitlevels::doc}}

\section{Altering bitlevels}
\label{\detokenize{features/bitlevels:altering-bitlevels}}
\sphinxAtStartPar
Please keep in mind each of these bitlevels can be tweeked with the @admin
parameters and with the @powers (accessable by royalty) or @depowers (only
by immortal and higher).

\sphinxAtStartPar
RhostMUSH offers a multi\sphinxhyphen{}tier bitlevel system.  They go in order of presidence
You do not have to use all these bits, only use what you want.


\section{ghod (\#1) \textless{}bitlevel 7\textgreater{}}
\label{\detokenize{features/bitlevels:ghod-1-bitlevel-7}}
\sphinxAtStartPar
This bitlevel can do everything.  Only those who you trust with absolute power should have this.  Period.


\section{Immortal(i) \textless{}Level 6\textgreater{} \sphinxhyphen{} Basically \#1}
\label{\detokenize{features/bitlevels:immortal-i-level-6-basically-1}}
\sphinxAtStartPar
The only thing this bitlevel can not do is directly effect \#1,
set/unset some internal flags/attributes, and set/unset the
immortal flag.  These players can do EVERYTHING else.  Treat
this bit as you would treat \#1.  Only give it to those you know
without a doubt you can trust.
\begin{itemize}
\item {} 
\sphinxAtStartPar
Can do everything except set some internal flags, effect \#1, and set/remove the immortal flag.

\end{itemize}


\section{Royalty(W) \textless{}Level 5\textgreater{} \sphinxhyphen{} FULL wizbit level}
\label{\detokenize{features/bitlevels:royalty-w-level-5-full-wizbit-level}}
\sphinxAtStartPar
This is your standard wizard.  They can do everything you\textquotesingle{}re
used to on other mushes that wizards can do.  In addition, they
also override all locks by default (this can be disabled), and
they have an enhanced wizcloaking ability (which also can be
disabled).  They can also set all the lower bitlevels.
\begin{itemize}
\item {} 
\sphinxAtStartPar
All things of Level 4 and lower

\item {} 
\sphinxAtStartPar
Ability to set more flags: STOP, NOSTOP, FUBAR

\item {} 
\sphinxAtStartPar
Ability to @attribute,

\item {} 
\sphinxAtStartPar
Ability to WIZCLOAK

\end{itemize}


\section{Councilor(a) \textless{}Level 4\textgreater{} \sphinxhyphen{} High wizbit level}
\label{\detokenize{features/bitlevels:councilor-a-level-4-high-wizbit-level}}
\sphinxAtStartPar
This is your almost\sphinxhyphen{}but\sphinxhyphen{}not\sphinxhyphen{}quite wizard.  They have access to
about 80\% of the wizard commands.  This includes @nuke, @toad,
@newpassword and the like.  The only things they can\textquotesingle{}t do that
wizards can is cloak, override locks, and use some of the
database manipulation tools in wizhelp.
\begin{itemize}
\item {} 
\sphinxAtStartPar
All things of Level 3 and lower

\item {} 
\sphinxAtStartPar
Ability to set more flags: NOCONNECT, WANDERER, FREE

\item {} 
\sphinxAtStartPar
Ability to @nuke, @toad, @boot, @chownall, @dbck, @poor, @newpassword, @pcreate, slay

\end{itemize}


\section{Architect(B) \textless{}Level 3\textgreater{} \sphinxhyphen{} Middle wizbit level}
\label{\detokenize{features/bitlevels:architect-b-level-3-middle-wizbit-level}}
\sphinxAtStartPar
This is your sub\sphinxhyphen{}wizard.  They still have the ability to control
anything their bitlevel and lower (including @chown, @destroy, etc)
but do not have any control of other players (like @nuke, @toad, etc)
but they can set the slave flag.  Otherwise, all things their level
and lower they can treat as if they owned it.
\begin{itemize}
\item {} 
\sphinxAtStartPar
All things of Level 2

\item {} 
\sphinxAtStartPar
Ability to fully control and modify anything their level and lower (including @cloning, @destroying, etc)

\item {} 
\sphinxAtStartPar
Ability to use @tel on anything their level and lower.

\item {} 
\sphinxAtStartPar
Ability to bypass jump\_ok rooms on anything their level \& lower.

\item {} 
\sphinxAtStartPar
Ability to set some restricted flags: SLAVE, NO\_YELL

\item {} 
\sphinxAtStartPar
Has infinite quota and money

\item {} 
\sphinxAtStartPar
Able to give negative money (Steal)

\item {} 
\sphinxAtStartPar
Able to @toggle the MONITOR

\end{itemize}


\section{Guildmaster(g) \textless{}Level 2\textgreater{} \sphinxhyphen{} Lowest wizbit level}
\label{\detokenize{features/bitlevels:guildmaster-g-level-2-lowest-wizbit-level}}
\sphinxAtStartPar
This is the lowest wiz bit.  They only have moderate abilities.
They can examine/decompile anything their level and lower, they can
@guild/@quota people, and they have a few other minor abilities.
They don\textquotesingle{}t have free money however.
\begin{itemize}
\item {} 
\sphinxAtStartPar
Ability to access things remotely (long\_fingers)

\item {} 
\sphinxAtStartPar
Things are FREE for them in the queue.

\item {} 
\sphinxAtStartPar
Can see dbref\#\textquotesingle{}s of things their level and lower

\item {} 
\sphinxAtStartPar
Can examine/decompile things their level and lower.

\item {} 
\sphinxAtStartPar
Can set @quota/@guild on their level and lower.

\end{itemize}


\section{Wanderer \textless{}bitlevel 0\textgreater{}}
\label{\detokenize{features/bitlevels:wanderer-bitlevel-0}}
\sphinxAtStartPar
This is a hinderance flag.  This flag is automatically set on new
players that are created (which can be disabled).  This flag stops
the player from creating/destroying any database information.   In
effect it stops them from any type of building type commands.  They
still are allowed to set/unset locks/attributes/etc though without
hinderance.


\section{Guest \textless{}bitlevel 0 as well\textgreater{}}
\label{\detokenize{features/bitlevels:guest-bitlevel-0-as-well}}
\sphinxAtStartPar
This is a bigger hinderance flag.  By default all guests should be
set this.  This flag stops the player from ANY database manipulation
along with @teleporting, and many other advanced commands.  It\textquotesingle{}s
extreamly dehibilatating.


\chapter{Comparison of modern MUSH servers}
\label{\detokenize{comparison:comparison-of-modern-mush-servers}}\label{\detokenize{comparison::doc}}

\section{Differences to expect to the end user between PennMUSH, MUX, and RhostMUSH}
\label{\detokenize{comparison:differences-to-expect-to-the-end-user-between-pennmush-mux-and-rhostmush}}
\sphinxAtStartPar
The largest end user experience will mostly resolve around some look and feel.
The general look and feel of how to set attributes, work with commands and
functions, and getting around a grid will be identical between the mush flavors.
@nuke and @destroy will work somewhat differently between the three codebases
and some effort should be looked at on how different it behaves.


\subsection{Differences with PennMUSH}
\label{\detokenize{comparison:differences-with-pennmush}}\begin{itemize}
\item {} 
\sphinxAtStartPar
The way the comsystem (hardcode) works with the latest PennMUSH has some
compatibility now with MUX\textquotesingle{}s comsystem.

\item {} 
\sphinxAtStartPar
The @mail system is different to MUX and Rhost.

\item {} 
\sphinxAtStartPar
Debugging code uses DEBUG and has an indented format.

\item {} 
\sphinxAtStartPar
The parser for code doesn\textquotesingle{}t require nested escaping like other codebases, but has issues with the pre\sphinxhyphen{}parser and nesting.

\item {} 
\sphinxAtStartPar
All standard movement, attribute setting, other should be similar

\item {} 
\sphinxAtStartPar
Penn allows empty attributes.

\item {} 
\sphinxAtStartPar
Penn supports attribute trees.

\item {} 
\sphinxAtStartPar
Penn has the standard set of bitlevel flags with on/off @powers

\end{itemize}


\subsection{Differences with MUX}
\label{\detokenize{comparison:differences-with-mux}}\begin{itemize}
\item {} 
\sphinxAtStartPar
The comsystem (hardcode) is unique to MUX/TinyMUSH3 and is not compatible with Penn.

\item {} 
\sphinxAtStartPar
The @mail system is unique to MUX/TinyMUSH3 and is not compatible with Penn.

\item {} 
\sphinxAtStartPar
Debugging uses TRACE and is the old standard for debugging.

\item {} 
\sphinxAtStartPar
The parser requires extra escaping for nested escaping but handles pre\sphinxhyphen{}parsing and nesting fine.

\item {} 
\sphinxAtStartPar
MUX does not allow empty attributes.

\item {} 
\sphinxAtStartPar
MUX does not support attribute trees.

\item {} 
\sphinxAtStartPar
MUX has the standard set of bitlevel flags with on/off @powers

\end{itemize}


\subsection{Differences with RhostMUSH}
\label{\detokenize{comparison:differences-with-rhostmush}}\begin{itemize}
\item {} 
\sphinxAtStartPar
The comsystem (softcode) is in the Mushcode directory and is compatible to both PennMUSH and MUX.

\item {} 
\sphinxAtStartPar
The mail system is unique to Rhost but there exists mail wrappers that allow MUX and Penn compatibility.

\item {} 
\sphinxAtStartPar
Debugging uses TRACE and is the old standard for debugging.  It allows advanced features like labeling and grepping for content.

\item {} 
\sphinxAtStartPar
The parser requires extra escaping for nested escaping but handles pre\sphinxhyphen{}parsing and nesting fine.

\item {} 
\sphinxAtStartPar
Rhost does not allow empty attributes.

\item {} 
\sphinxAtStartPar
Rhost marginally supports attribute trees.

\item {} 
\sphinxAtStartPar
Rhost has a multi\sphinxhyphen{}level set of bitlevel flags with multi\sphinxhyphen{}tier @powers, @depowers, and varying other tools for permissions.

\end{itemize}


\section{Comparison of features between RhostMUSH, PennMUSH, and MUX2}
\label{\detokenize{comparison:comparison-of-features-between-rhostmush-pennmush-and-mux2}}
\sphinxAtStartPar
There are differences in the initial set up from Rhost to PennMUSH (and MUX2).

\sphinxAtStartPar
A lot of people use PennMUSH and if you are one of them, this may help you.
Some people use MUX2, hopefully this will help you as well.

\begin{sphinxadmonition}{note}{Note:}
\sphinxAtStartPar
If you plan to use sideeffects, you must set the SIDEFX flag to allow
the target to use it
\end{sphinxadmonition}

\begin{sphinxadmonition}{note}{Note:}
\sphinxAtStartPar
Rhost does support UNICODE/UTF8, but it\textquotesingle{}s currently not part of the main
branch as of this writing.  Please talk to Kage on the dev site for more
information on when it\textquotesingle{}ll be released.
\end{sphinxadmonition}


\section{Organization of help and wizhelp}
\label{\detokenize{comparison:organization-of-help-and-wizhelp}}
\sphinxAtStartPar
First, on PennMUSH, help is separated into \textquotesingle{}help\textquotesingle{} and \textquotesingle{}wizhelp\textquotesingle{}.  This can
be confusing. If you want to \textquotesingle{}combine\textquotesingle{} them, you may use the following snippit:

\begin{sphinxVerbatim}[commandchars=\\\{\}]
\PYGZdl{}ahelp*:@pemit \PYGZpc{}\PYGZsh{}=[switch([!!words(\PYGZpc{}0)][match(/search,\PYGZpc{}0*)],0?,[setq(0,help)][setq(1,0)],11,[setq(0,rest(\PYGZpc{}0))][setq(1,2)],setq(0,trim(\PYGZpc{}0)))][setq(a,textfile(help,\PYGZpc{}q0,\PYGZpc{}q1))][setq(b,textfile(wizhelp,\PYGZpc{}q0,\PYGZpc{}q1))][setq(2,)][ifelse(!strmatch(\PYGZpc{}qa,No entry*),\PYGZpc{}qa[setq(2,\PYGZpc{}r)])][ifelse(!strmatch(\PYGZpc{}qb,No entry*),\PYGZpc{}q2\PYGZpc{}qb[setq(2,\PYGZpc{}r)])][ifelse(!\PYGZdl{}r(2),No topics found for \PYGZsq{}\PYGZpc{}q0\PYGZsq{}.)]
\end{sphinxVerbatim}

\sphinxAtStartPar
This will display help from help and wizhelp for any matching topic.  It
should make life easier for you.  This also honors the /search switch.


\section{Mail, comsystem, guests, master rooms, etc..}
\label{\detokenize{comparison:mail-comsystem-guests-master-rooms-etc}}
\sphinxAtStartPar
The comsystem is softcode and included in the readme directory. (comsys).
Just load it in.  The mail wrappers can be gotten from:

\begin{sphinxVerbatim}[commandchars=\\\{\}]
\PYG{n}{The} \PYG{n}{Mushcode} \PYG{n}{directory} \PYG{o+ow}{in} \PYG{n}{the} \PYG{n}{Rhost} \PYG{n}{Distribution}\PYG{o}{.}
\end{sphinxVerbatim}

\sphinxAtStartPar
Please see the other readme files on setting other things up like
guests, master rooms, and so forth.


\section{Comparisons of features}
\label{\detokenize{comparison:comparisons-of-features}}
\sphinxAtStartPar
Comparisons were done based on the following:
RhostMUSH 3.9.5p2
MUX 2.12.0.2
PennMUSH 1.8.6p0


\section{@commands with equivalents}
\label{\detokenize{comparison:commands-with-equivalents}}
\sphinxAtStartPar
The following are used for compatibility to keep in mind:

\begin{sphinxVerbatim}[commandchars=\\\{\}]
Pennmush                RhostMUSH                      MUX2

@@                      @@                             @@
@ALLHALT                @halt/all                      @HALT/ALL
@ALLQUOTA               @quota/all                     @QUOTA/ALL
@ALIAS                  @alias/@protect                @ALIAS
N/A                     @apply\PYGZus{}marked                  @APPLY\PYGZus{}MARKED
@ASSERT                 @assert                        @ASSERT
@ATRCHOWN               @chown                         @CHOWN
@ATRLOCK                @lock                          @LOCK
@ATTRIBUTE              @attribute                     @ATTRIBUTE
N/A                     (see backup\PYGZus{}flat.sh)           @BACKUP
@BOOT                   @boot                          @BOOT
@BREAK                  @break                         @BREAK
N/A                     N/A                            @CCHARGE
N/A                     N/A                            @CCHOWN
@CEMIT                  (softcode)                     @CEMIT
@CHANNEL                N/A                            N/A
@CHAT                   N/A                            N/A
@CHOWNALL               @chownall                      @CHOWNALL
@CHZONE                 @zone                          @CHZONE
@CHZONEALL              @zone                          N/A
@CLOCK                  N/A                            N/A
@CLONE                  @clone                         @CLONE
N/A                     N/A                            @CCREATE
N/A                     N/A                            @CDESTROY
N/A                     N/A                            @CWHO
@COMMAND                N/A                            N/A
@CONFIG                 @admin                         @ADMIN
N/A                     N/A                            @COFLAGS
N/A                     N/A                            @CPFLAGS
N/A                     N/A                            @CSET
@CPATTR                 @cpattr                        @CPATTR
@CREATE                 @create                        @CREATE
N/A                     @cut                           @CUT
@DBCK                   @dbck                          @DBCK
@DECOMPILE              @decompile                     @DECOMPILE
@DESTROY                @destroy                       @DESTROY
@DIG                    @dig                           @DIG
@DISABLE                @disable                       @DISABLE
@DOING                  @doing                         @DOING
@DOLIST                 @dolist                        @DOLIST
@DRAIN                  @drain                         @DRAIN
@DUMP                   @dump                          @DUMP
@EDIT                   @edit                          @EDIT
@ELOCK                  @lock/enter                    @LOCK/ENTER
@EMIT                   @emit                          @EMIT
N/A                     N/A                            @EMAIL
@ENABLE                 @enable                        @ENABLE
@ENTRANCES              @entrances                     @ENTRANCES
@EUNLOCK                @unlock/enter                  @UNLOCK/ENTER
N/A                     @eval                          @EVAL
N/A                     @femit                         @FEMIT
N/A                     @fpose                         @FPOSE
N/A                     @fsay                          @FSAY
@FIND                   @find                          @FIND
@FIRSTEXIT              N/A                            N/A
@FLAG                   @flag                          @FLAG
@FORCE                  @force                         @FORCE
N/A                     folder                         @FOLDER
@FUNCTION               @function/@lfunction           @FUNCTION
@GREP                   @grep                          N/A
@HALT                   @halt                          @HALT
@HIDE                   @hide                          N/A
@HOOK                   @hook                          @HOOK
@INCLUDE                @include                       N/A
N/A                     @skip/ifelse                   @IF
@KICK                   @kick                          @KICK
N/A                     @last                          @LAST
@LEMIT                  @lemit                         @LEMIT
@LINK                   @link                          @LINK
@LIST                   @list                          @LIST
N/A                     @list\PYGZus{}file                     @LIST\PYGZus{}FILE
@LISTMOTD               @listmotd                      @LISTMOTD
@LOCK                   @lock                          @LOCK
@LOG                    @log                           @LOG
@LOGWIPE                N/A                            N/A
@LSET                   @set                           @SET
N/A                     @mark                          @MARK
N/A                     @mark\PYGZus{}all                      @MARK\PYGZus{}ALL
@MAIL                   mail                           @MAIL
@MALIAS                 wmail/alias                    @MALIAS
@MAPSQL                 N/A                            N/A
@MESSAGE                @pemit/@remit + parsestr()     N/A
@MONIKER                @extansi                       @MONIKER
@MOTD                   @motd                          @MOTD
@MVATTR                 @mvattr                        @MVATTR
@NAME                   @name                          @NAME
N/A                     @emit/noeval                   @NEMIT
N/A                     @pemit/noeval                  @NPEMIT
@NEWPASSWORD            @newpassword                   @NEWPASSWORD
@NOTIFY                 @notify                        @NOTIFY
@NSCEMIT                N/A                            N/A
@NSEMIT                 @emit                          @emit
@NSLEMIT                @lemit                         @LEMIT
@NSOEMIT                @oemit                         @OEMIT
@NSPEMIT                @pemit                         @PEMIT
@NSPROMPT               N/A                            N/A
@NSREMIT                @remit                         @REMIT
@NSZEMIT                @zemit                         N/A
@NUKE                   @destroy/@nuke                 @DESTROY/@NUKE
@OEMIT                  @oemit                         @OEMIT
@OPEN                   @open                          @OPEN
@PARENT                 @parent                        @PARENT
@PASSWORD               @password                      @PASSWORD
@PCREATE                @pcreate                       @PCREATE
@PEMIT                  @pemit                         @PEMIT
@POLL                   @doing/header                  @POLL
@POOR                   @poor                          @POOR
@POWER                  @power                         @POWER
@PROMPT                 N/A (@program?)                N/A (@program?)
N/A                     @program                       @PROGRAM
@PS                     @ps                            @PS
@PURGE                  @timewarp/dump 1               @TIMEWARP/DUMP 1
N/A                     @quitprogram                   @QUITPROGRAM
@QUOTA                  @quota                         @QUOTA
N/A                     N/A                            @QUERY
@READCACHE              @readcache                     @READCACHE
@RECYCLE                @purge                         N/A
N/A                     N/A                            @REFERENCE
N/A                     @robot                         @ROBOT
@REJECTMOTD             @rejectmotd                    @REJECTMOTD
@REMIT                  @remit                         @REMIT
@RESTART                @reboot                        @RESTART
@RETRY                  N/A                            N/A
@RWALL                  @wall/wiz                      @WALL/WIZ
@SCAN                   (see softcode)                 N/A
@SEARCH                 @search                        @SEARCH
@SELECT                 @switch/first                  @SWITCH/FIRST
@SET                    @set                           @SET
@SHUTDOWN               @shutdown                      @SHUTDOWN
@SITELOCK               @admin forbid\PYGZus{}host/forbid\PYGZus{}site @admin forbid\PYGZus{}site
@SLAVE                  N/A                            @STARTSLAVE
@SOCKSET                N/A                            N/A
@SQL                    (only if MySQL enabled)        @QUERY
@SQUOTA                 @quota                         N/A
@STATS                  @stats                         @STATS
@SWEEP                  @sweep                         @SWEEP
@SWITCH                 @switch                        @SWITCH
N/A                     @timewarp                      @TIMEWARP
@TELEPORT               @teleport                      @TELEPORT
N/A                     @timecheck                     @TIMECHECK
N/A                     @toad                          @TOAD
@TRIGGER                @trigger                       @TRIGGER
@ULOCK                  @lock/use                      @LOCK/USE
@UNDESTROY              @recover                       N/A
@UNLINK                 @unlink                        @UNLINK
@UNLOCK                 @unlock                        @UNLOCK
@UNRECYCLE              @recover                       N/A
@UPTIME                 @uptime                        @UPTIME
@UUNLOCK                @unlock/use                    @UNLOCK/USE
@VERB                   @verb                          @VERB
@VERSION                @version                       VERSION
@WAIT                   @wait                          @WAIT
@WALL                   @wall                          @WALL
@WARNINGS               N/A                            N/A
@WCHECK                 N/A                            N/A
@WEBPASSWD              N/A                            N/A
@WHEREIS                @whereis                       N/A
@WIPE                   @wipe                          @WIPE
@WIZMOTD                @wizmotd                       @WIZMOTD
@WIZWALL                @wall/wiz                      @WALL/WIZ
@ZEMIT                  @zemit                         N/A
N/A                     (softcode)                     ALLCOM
N/A                     (softcode)                     COMLIST
N/A                     (softcode)                     DELCOM
N/A                     (softcode)                     ADDCOM
N/A                     (softcode)                     COMTITLE
ANEWS                   @dynhelp                       N/A
ATTRIB\PYGZus{}SET              (@hook on S)                   (@hook on S)
BRIEF                   ex/brief                       EX/BRIEF
BUY                     N/A                            N/A
N/A                     N/A                            CLEARCOM
DESERT                  (see follow softcode)          N/A
DISMISS                 (see follow softcode)          N/A
DOING                   doing                          DOING
DROP                    drop                           DROP
EMPTY                   @tel/list lcon(target)=me      @tel/list lcon(target)=me
ENTER                   enter                          ENTER
EXAMINE                 examine                        EXAMINE
FOLLOW                  (see follow softcode)          N/A
GET                     get                            GET
GIVE                    give                           GIVE
GOTO                    goto                           GOTO
HELP                    help/wizhelp                   HELP/WIZHELP
HOME                    home                           HOME
HUH\PYGZus{}COMMAND             @admin global\PYGZus{}error\PYGZus{}obj        @admin global\PYGZus{}error\PYGZus{}obj
INFO                    INFO                           INFO
INVENTORY               inventory                      INVENTORY
KILL                    kill                           KILL
LEAVE                   leave                          LEAVE
LOGOUT                  logout                         LOGOUT
LOOK                    look                           LOOK
NEWS                    news                           NEWS
N/A                     outputprefix                   OUTPUTPREFIX
N/A                     outputsuffix                   OUTPUTSUFFIX
PAGE                    page/lpage/rpage/mrpage        PAGE
POSE                    pose                           POSE
N/A                     N/A                            PUEBLOCLIENT
QUIT                    quit                           QUIT
N/A                     N/A                            REPORT
SAY                     say                            SAY
SCORE                   score                          SCORE
SEMIPOSE                pose/nospace                   POSE/NOSPACE
SESSION                 session                        SESSION
SLAY                    slay                           SLAY
TEACH                   train                          TRAIN
THINK                   think                          THINK
UNFOLLOW                (see follow softcode)          N/A
UNIMPLEMENTED\PYGZus{}COMMAND   @admin global\PYGZus{}error\PYGZus{}obj        @admin global\PYGZus{}error\PYGZus{}obj
USE                     use                            USE
WARN\PYGZus{}ON\PYGZus{}MISSING         N/A                            N/A
WHISPER                 whisper                        WHISPER
WHO                     who                            WHO
WITH                    N/A                            N/A
\end{sphinxVerbatim}


\section{@commands unique to RhostMUSH}
\label{\detokenize{comparison:commands-unique-to-rhostmush}}
\sphinxAtStartPar
Commands that exist in Rhost that have no PennMUSH equivelant:

\begin{sphinxVerbatim}[commandchars=\\\{\}]
\PYG{n+nd}{@aflags}                      \PYG{n+nd}{@apply\PYGZus{}marked}                 \PYG{n+nd}{@areg}
\PYG{n+nd}{@blacklist}                   \PYG{n+nd}{@cluster}                      \PYG{n+nd}{@conncheck}
\PYG{n+nd}{@convert}                     \PYG{n+nd}{@cut}                          \PYG{n+nd}{@dbclean}
\PYG{n+nd}{@depower}                     \PYG{n+nd}{@dynhelp}                      \PYG{n+nd}{@eval}
\PYG{n+nd}{@femit}                       \PYG{n+nd}{@fixdb}                        \PYG{n+nd}{@fpose}
\PYG{n+nd}{@freeze}                      \PYG{n+nd}{@fsay}                         \PYG{n+nd}{@icmd}
\PYG{n+nd}{@last}                        \PYG{n+nd}{@lfunction}                    \PYG{n+nd}{@limit}
\PYG{n+nd}{@logrotate}                   \PYG{n+nd}{@mark}                         \PYG{n+nd}{@mark\PYGZus{}all}
\PYG{n+nd}{@money}                       \PYG{n+nd}{@pipe}                         \PYG{n+nd}{@program}
\PYG{n+nd}{@progreset}                   \PYG{n+nd}{@protect}                      \PYG{n+nd}{@quitprogram}
\PYG{n+nd}{@reclist}                     \PYG{n+nd}{@recover}                      \PYG{n+nd}{@register}
\PYG{n+nd}{@remote}                      \PYG{n+nd}{@robot}                        \PYG{n+nd}{@rxlevel}
\PYG{n+nd}{@skip}                        \PYG{n+nd}{@snapshot}                     \PYG{n+nd}{@snoop}
\PYG{n+nd}{@thaw}                        \PYG{n+nd}{@timewarp}                     \PYG{n+nd}{@toad}
\PYG{n+nd}{@toggle}                      \PYG{n+nd}{@toggledef}                    \PYG{n+nd}{@tor}
\PYG{n+nd}{@turtle}                      \PYG{n+nd}{@txlevel}                      \PYG{n+nd}{@whereall}
\PYG{n}{grab}                         \PYG{n}{join}                          \PYG{n}{listen}
\PYG{n}{mrpage}                       \PYG{n}{newsdb}                        \PYG{n}{rpage}
\PYG{n}{smell}                        \PYG{n}{taste}                         \PYG{n}{touch}
\PYG{n}{wielded}                      \PYG{n}{worn}                          \PYG{o}{+}\PYG{n}{players}
\end{sphinxVerbatim}


\section{@lock equivalents}
\label{\detokenize{comparison:lock-equivalents}}
\begin{sphinxVerbatim}[commandchars=\\\{\}]
\PYG{n}{PennMUSH}               \PYG{n}{RhostMUSH}                                   \PYG{n}{MUX2}

\PYG{n}{BASIC}                  \PYG{n}{BASIC}\PYG{o}{/}\PYG{n}{DEFAULT}                               \PYG{n}{DEFAULT}
\PYG{n}{ENTER}                  \PYG{n}{ENTER}                                       \PYG{n}{ENTER}
\PYG{n}{TELEPORT}               \PYG{n}{TPORT}                                       \PYG{n}{TPORT}
\PYG{n}{USE}                    \PYG{n}{USE}                                         \PYG{n}{USE}
\PYG{n}{PAGE}                   \PYG{n}{PAGE}                                        \PYG{n}{PAGE}
\PYG{n}{ZONE}                   \PYG{n}{ZONEWIZLOCK}\PYG{o}{/}\PYG{n}{ZONETOLOCK}\PYG{o}{/}\PYG{n}{TWINKLOCK}            \PYG{n}{N}\PYG{o}{/}\PYG{n}{A}
\PYG{n}{PARENT}                 \PYG{n}{PARENT}                                      \PYG{n}{PARENT}
\PYG{n}{LINK}                   \PYG{n}{LINK}                                        \PYG{n}{LINK}
\PYG{n}{OPEN}                   \PYG{n}{OPEN}                                        \PYG{n}{OPEN}
\PYG{n}{MAIL}                   \PYG{n}{mail}\PYG{o}{/}\PYG{n}{lock}                                   \PYG{n}{MAIL}
\PYG{n}{USER}                   \PYG{n}{USER}                                        \PYG{n}{USER}
\PYG{n}{USER}\PYG{p}{:}\PYG{o}{\PYGZlt{}}\PYG{n}{dynamicname}\PYG{o}{\PYGZgt{}}     \PYG{n}{lockencode}\PYG{p}{(}\PYG{p}{)}\PYG{o}{/}\PYG{n}{lockdecode}\PYG{p}{(}\PYG{p}{)}\PYG{o}{/}\PYG{n}{lockcheck}\PYG{p}{(}\PYG{p}{)}       \PYG{n}{N}\PYG{o}{/}\PYG{n}{A}
\PYG{n}{SPEECH}                 \PYG{n}{SPEECH}                                      \PYG{n}{SPEECH}
\PYG{n}{LISTEN}                 \PYG{n}{USE} \PYG{p}{(}\PYG{n}{see} \PYG{n}{listen} \PYG{n}{argument}\PYG{p}{)}                   \PYG{n}{N}\PYG{o}{/}\PYG{n}{A}
\PYG{n}{COMMAND}                \PYG{n}{USE} \PYG{p}{(}\PYG{n}{commands} \PYG{n}{are} \PYG{n}{default}\PYG{p}{)}                  \PYG{n}{N}\PYG{o}{/}\PYG{n}{A}
\PYG{n}{LEAVE}                  \PYG{n}{LEAVE}                                       \PYG{n}{LEAVE}
\PYG{n}{DROP}                   \PYG{n}{DROP}                                        \PYG{n}{DROP}
\PYG{n}{DROPIN}                 \PYG{n}{DROPTO}                                      \PYG{n}{N}\PYG{o}{/}\PYG{n}{A}
\PYG{n}{GIVE}                   \PYG{n}{GIVE}                                        \PYG{n}{GIVE}
\PYG{n}{FROM}                   \PYG{n}{GIVETO}                                      \PYG{n}{N}\PYG{o}{/}\PYG{n}{A}
\PYG{n}{PAY}                    \PYG{n}{N}\PYG{o}{/}\PYG{n}{A}                                         \PYG{n}{N}\PYG{o}{/}\PYG{n}{A}
\PYG{n}{RECEIVE}                \PYG{n}{RECEIVE}                                     \PYG{n}{RECEIVE}
\PYG{n}{FOLLOW}                 \PYG{p}{(}\PYG{n}{See} \PYG{n}{softcoded} \PYG{n}{follow} \PYG{n}{code}\PYG{p}{)}                 \PYG{n}{N}\PYG{o}{/}\PYG{n}{A}
\PYG{n}{EXAMINE}                \PYG{n}{See} \PYG{n}{NO\PYGZus{}MODIFY}\PYG{o}{/}\PYG{n}{NO\PYGZus{}EXAMINE}\PYG{o}{/}\PYG{n}{TWINKLOCK}          \PYG{n}{N}\PYG{o}{/}\PYG{n}{A}
\PYG{n}{CHZONE}                 \PYG{n}{ZONETOLOCK}\PYG{o}{/}\PYG{n}{ZONEWIZLOCK}\PYG{o}{/}\PYG{n}{TWINKLOCK}            \PYG{n}{N}\PYG{o}{/}\PYG{n}{A}
\PYG{n}{FORWARD}                \PYG{n}{N}\PYG{o}{/}\PYG{n}{A}                                         \PYG{n}{N}\PYG{o}{/}\PYG{n}{A}
\PYG{n}{FILTER}                 \PYG{n}{N}\PYG{o}{/}\PYG{n}{A}                                         \PYG{n}{N}\PYG{o}{/}\PYG{n}{A}
\PYG{n}{INFILTER}               \PYG{n}{N}\PYG{o}{/}\PYG{n}{A}                                         \PYG{n}{N}\PYG{o}{/}\PYG{n}{A}
\PYG{n}{CONTROL}                \PYG{n}{CONTROL}                                     \PYG{n}{N}\PYG{o}{/}\PYG{n}{A}
\PYG{n}{DROPTO}                 \PYG{n}{DROPTO}                                      \PYG{n}{N}\PYG{o}{/}\PYG{n}{A}
\PYG{n}{DESTROY}                \PYG{n}{See}\PYG{p}{:} \PYG{n+nd}{@recover}\PYG{o}{/}\PYG{n+nd}{@purge}\PYG{o}{/}\PYG{n}{INDESTRUCTIBLE}\PYG{o}{/}\PYG{n}{SAFE}    \PYG{n}{N}\PYG{o}{/}\PYG{n}{A}
\PYG{n}{INTERACT}               \PYG{n}{N}\PYG{o}{/}\PYG{n}{A} \PYG{p}{(}\PYG{n}{See}\PYG{p}{:} \PYG{n}{Reality} \PYG{n}{Levels}\PYG{p}{)}                   \PYG{n}{N}\PYG{o}{/}\PYG{n}{A} \PYG{p}{(}\PYG{n}{See}\PYG{p}{:} \PYG{n}{Reality} \PYG{n}{Levels}\PYG{p}{)}
\PYG{n}{TAKE}                   \PYG{n}{GETFROM}                                     \PYG{n}{GETFROM}
\PYG{n}{MAILFORWARD}            \PYG{n}{mail}\PYG{o}{/}\PYG{n}{lock}\PYG{p}{,} \PYG{n}{mail}\PYG{o}{/}\PYG{n}{autofor}                     \PYG{n}{N}\PYG{o}{/}\PYG{n}{A}
\PYG{n}{N}\PYG{o}{/}\PYG{n}{A}                    \PYG{n}{TELOUT}                                      \PYG{n}{TELOUT}
\PYG{n}{N}\PYG{o}{/}\PYG{n}{A}                    \PYG{n}{DARK}                                        \PYG{n}{VISIBLE}
\end{sphinxVerbatim}


\section{@locks that exist in Rhost that have no PennMUSH equivelant}
\label{\detokenize{comparison:locks-that-exist-in-rhost-that-have-no-pennmush-equivelant}}
\begin{sphinxVerbatim}[commandchars=\\\{\}]
\PYG{n}{TELOUTLOCK}                   \PYG{n}{TWINKLOCK}                     \PYG{n}{DARKLOCK}
\PYG{n}{ALTNAME}                      \PYG{n}{CHOWN}
\end{sphinxVerbatim}


\section{Flag and toggle equivalents}
\label{\detokenize{comparison:flag-and-toggle-equivalents}}
\begin{sphinxVerbatim}[commandchars=\\\{\}]
\PYG{n}{Pennmush}                \PYG{n}{RhostMUSH}                              \PYG{n}{MUX2}

\PYG{n}{ABODE}                   \PYG{n}{ABODE}                                  \PYG{n}{ABODE}
\PYG{n}{N}\PYG{o}{/}\PYG{n}{A}                     \PYG{n}{N}\PYG{o}{/}\PYG{n}{A}                                    \PYG{n}{ASCII}
\PYG{n}{ANSI}                    \PYG{n}{ANSI}                                   \PYG{n}{ANSI}
\PYG{n}{AUDIBLE}                 \PYG{n}{AUDIBLE}                                \PYG{n}{AUDIBLE}
\PYG{p}{(}\PYG{n}{Not} \PYG{n}{Needed}\PYG{p}{)}            \PYG{p}{(}\PYG{n}{Not} \PYG{n}{Needed}\PYG{p}{)}                           \PYG{n}{BLEED}
\PYG{n}{N}\PYG{o}{/}\PYG{n}{A}                     \PYG{n}{AUDITORIUM}                             \PYG{n}{AUDITORIUM}
\PYG{n}{N}\PYG{o}{/}\PYG{n}{A}                     \PYG{n}{BLIND}                                  \PYG{n}{BLIND}
\PYG{n}{N}\PYG{o}{/}\PYG{n}{A}                     \PYG{n}{COMMANDS}                               \PYG{n}{COMMANDS}
\PYG{n}{CHAN\PYGZus{}USEFIRSTMATCH}      \PYG{n}{N}\PYG{o}{/}\PYG{n}{A}                                    \PYG{n}{N}\PYG{o}{/}\PYG{n}{A}
\PYG{n}{CHOWN\PYGZus{}OK}                \PYG{n}{CHOWN\PYGZus{}OK}                               \PYG{n}{CHOWN\PYGZus{}OK}
\PYG{n}{CLOUDY}                  \PYG{n}{N}\PYG{o}{/}\PYG{n}{A}                                    \PYG{n}{N}\PYG{o}{/}\PYG{n}{A}
\PYG{n}{COLOR}                   \PYG{n}{ANSICOLOR}                              \PYG{n}{N}\PYG{o}{/}\PYG{n}{A}
\PYG{n}{CONNECTED}               \PYG{n}{CONNECTED}                              \PYG{n}{CONNECTED}
\PYG{n}{DARK}                    \PYG{n}{DARK}                                   \PYG{n}{DARK}
\PYG{n}{DEBUG}                   \PYG{n}{TRACE}                                  \PYG{n}{TRACE}
\PYG{n}{DESTROY\PYGZus{}OK}              \PYG{n}{DESTROY\PYGZus{}OK}                             \PYG{n}{DESTROY\PYGZus{}OK}
\PYG{n}{ENTER\PYGZus{}OK}                \PYG{n}{ENTER\PYGZus{}OK}                               \PYG{n}{ENTER\PYGZus{}OK}
\PYG{n}{FIXED}                   \PYG{n}{NO\PYGZus{}TEL}                                 \PYG{n}{FIXED}
\PYG{n}{FLOATING}                \PYG{n}{FLOATING}                               \PYG{n}{FLOATING}
\PYG{n}{GAGGED}                  \PYG{n}{FUBAR}                                  \PYG{n}{GAGGED}
\PYG{n}{GOING}                   \PYG{n}{GOING}                                  \PYG{n}{GOING}
\PYG{n}{HALT}                    \PYG{n}{HALT}                                   \PYG{n}{HALT}
\PYG{n}{HAVEN}                   \PYG{n}{HAVEN}                                  \PYG{n}{HAVEN}
\PYG{p}{(}\PYG{n}{see} \PYG{n+nd}{@flag}\PYG{p}{)}             \PYG{p}{(}\PYG{n}{marker0}\PYG{o}{\PYGZhy{}}\PYG{n}{marker9}\PYG{p}{)}                      \PYG{n}{HEAD}
\PYG{n}{HEAR\PYGZus{}CONNECT}            \PYG{n}{MONITOR} \PYG{p}{(}\PYG{n+nd}{@toggle}\PYG{p}{)}                      \PYG{n}{SITECON}
\PYG{n}{HEAVY}                   \PYG{n}{NO\PYGZus{}TEL}\PYG{o}{/}\PYG{n+nd}{@lock}\PYG{o}{\PYGZhy{}}\PYG{n}{teleport}                  \PYG{n}{N}\PYG{o}{/}\PYG{n}{A}
\PYG{n}{N}\PYG{o}{/}\PYG{n}{A}                     \PYG{n}{N}\PYG{o}{/}\PYG{n}{A}                                    \PYG{n}{HTML}
\PYG{n}{N}\PYG{o}{/}\PYG{n}{A}                     \PYG{n}{FREE}                                   \PYG{n}{IMMORTAL}
\PYG{n}{N}\PYG{o}{/}\PYG{n}{A}                     \PYG{n}{INHERIT}                                \PYG{n}{INHERIT}
\PYG{n}{JUMP\PYGZus{}OK}                 \PYG{n}{JUMP\PYGZus{}OK}                                \PYG{n}{JUMP\PYGZus{}OK}
\PYG{n}{KEEPALIVE}               \PYG{n}{KEEPALIVE} \PYG{p}{(}\PYG{n+nd}{@toggle}\PYG{p}{)}                    \PYG{n}{KEEPALIVE}
\PYG{n}{N}\PYG{o}{/}\PYG{n}{A}                     \PYG{n}{KEY}                                    \PYG{n}{KEY}
\PYG{n}{LIGHT}                   \PYG{n}{LIGHT}                                  \PYG{n}{LIGHT}
\PYG{n}{LINK\PYGZus{}OK}                 \PYG{n}{LINK\PYGZus{}OK}                                \PYG{n}{LINK\PYGZus{}OK}
\PYG{n}{LISTEN\PYGZus{}PARENT}           \PYG{p}{(}\PYG{n+nd}{@admin} \PYG{n}{listen\PYGZus{}parents}\PYG{p}{)}                \PYG{n}{N}\PYG{o}{/}\PYG{n}{A}
\PYG{n}{LOUD}                    \PYG{n}{NO\PYGZus{}OVERRIDE}\PYG{o}{/}\PYG{n}{NO\PYGZus{}USELOCK}                 \PYG{n}{N}\PYG{o}{/}\PYG{n}{A}
\PYG{p}{(}\PYG{n}{see} \PYG{n+nd}{@flag}\PYG{p}{)}             \PYG{n}{MARKER0}\PYG{o}{\PYGZhy{}}\PYG{n}{MARKER9}                        \PYG{n}{MARKER0}\PYG{o}{\PYGZhy{}}\PYG{n}{MARKER9}
\PYG{n}{MISTRUST}                \PYG{n}{GUILDOBJ}\PYG{o}{/}\PYG{n}{NO\PYGZus{}GOBJ}\PYG{o}{/}\PYG{n}{BACKSTAGE}\PYG{o}{/}\PYG{n}{NOBACKSTAGE} \PYG{n}{N}\PYG{o}{/}\PYG{n}{A}
\PYG{n}{MONIKER}                 \PYG{n}{EXTANSI} \PYG{p}{(}\PYG{n+nd}{@toggle}\PYG{p}{)}                      \PYG{n}{N}\PYG{o}{/}\PYG{n}{A}
\PYG{n}{MONITOR}                 \PYG{n}{MONITOR}                                \PYG{n}{MONITOR}
\PYG{n}{MYOPIC}                  \PYG{n}{MYOPIC}                                 \PYG{n}{MYOPIC}
\PYG{n}{NOACCENTS}               \PYG{n}{ACCENTS} \PYG{p}{(}\PYG{n+nd}{@toggle}\PYG{p}{)}                      \PYG{n}{ACCENTS}
\PYG{p}{(}\PYG{n}{Not} \PYG{n}{Needed}\PYG{p}{)}            \PYG{p}{(}\PYG{n}{Not} \PYG{n}{Needed}\PYG{p}{)}                           \PYG{n}{NO\PYGZus{}BLEED}
\PYG{n}{NOSPOOF}                 \PYG{n}{NOSPOOF}                                \PYG{n}{NOSPOOF}
\PYG{p}{(}\PYG{n}{See} \PYG{n+nd}{@ns}\PYG{o}{\PYGZhy{}}\PYG{n}{commands}\PYG{p}{)}      \PYG{n}{Auto}\PYG{o}{\PYGZhy{}}\PYG{n}{Enabled} \PYG{k}{for} \PYG{n}{Wiz}\PYG{o}{+}                  \PYG{n}{SPOOF}
\PYG{n}{NO\PYGZus{}COMMAND}              \PYG{n}{NO\PYGZus{}COMMAND}                             \PYG{n}{NO\PYGZus{}COMMAND}
\PYG{n}{NO\PYGZus{}LEAVE}                \PYG{n+nd}{@icmd} \PYG{n}{leave}\PYG{o}{/}\PYG{n+nd}{@lock}\PYG{o}{\PYGZhy{}}\PYG{n}{leave}                \PYG{n+nd}{@icmd} \PYG{n}{leave}\PYG{o}{/}\PYG{n+nd}{@lock}\PYG{o}{\PYGZhy{}}\PYG{n}{leave}
\PYG{n}{NO\PYGZus{}TEL}                  \PYG{n}{NO\PYGZus{}TEL}                                 \PYG{n}{N}\PYG{o}{/}\PYG{n}{A}
\PYG{n}{ON}\PYG{o}{\PYGZhy{}}\PYG{n}{VACATION}             \PYG{n}{MARKER0}\PYG{o}{\PYGZhy{}}\PYG{n}{MARKER9}                        \PYG{n}{VACATION}
\PYG{n}{OPAQUE}                  \PYG{n}{OPAQUE}                                 \PYG{n}{OPAQUE}
\PYG{n}{OPEN\PYGZus{}OK}                 \PYG{n+nd}{@lock}\PYG{o}{\PYGZhy{}}\PYG{n}{openfrom}                         \PYG{n}{OPEN\PYGZus{}OK}
\PYG{n}{ORPHAN}                  \PYG{n}{NOGLOBPARENT} \PYG{p}{(}\PYG{n+nd}{@toggle}\PYG{p}{)}                 \PYG{n}{N}\PYG{o}{/}\PYG{n}{A}
\PYG{n}{N}\PYG{o}{/}\PYG{n}{A}                     \PYG{n}{PARENT\PYGZus{}OK}                              \PYG{n}{PARENT\PYGZus{}OK}
\PYG{n}{PUPPET}                  \PYG{n}{PUPPET}                                 \PYG{n}{PUPPET}
\PYG{n}{QUIET}                   \PYG{n}{QUIET}                                  \PYG{n}{QUIET}
\PYG{n}{N}\PYG{o}{/}\PYG{n}{A}                     \PYG{n}{ROBOT}                                  \PYG{n}{ROBOT}
\PYG{n}{ROYALTY}                 \PYG{n}{GUILDMASTER}\PYG{o}{/}\PYG{n}{ARCHITECT}\PYG{o}{/}\PYG{n}{COUNCILOR}        \PYG{n}{ROYALTY}
\PYG{p}{(}\PYG{n+nd}{@see} \PYG{n+nd}{@flag}\PYG{p}{)}            \PYG{n}{MARKER0}\PYG{o}{\PYGZhy{}}\PYG{l+m+mi}{9}                              \PYG{n}{STAFF}
\PYG{n}{SAFE}                    \PYG{n}{SAFE}                                   \PYG{n}{SAFE}
\PYG{n}{N}\PYG{o}{/}\PYG{n}{A}                     \PYG{n}{SLAVE}                                  \PYG{n}{SLAVE}
\PYG{n}{N}\PYG{o}{/}\PYG{n}{A}                     \PYG{n}{MONITOR} \PYG{p}{(}\PYG{n+nd}{@toggle}\PYG{p}{)}                      \PYG{n}{SITEMON}
\PYG{n}{STICKY}                  \PYG{n}{STICKY}                                 \PYG{n}{STICKY}
\PYG{n}{N}\PYG{o}{/}\PYG{n}{A}                     \PYG{n}{SUSPECT}                                \PYG{n}{SUSPECT}
\PYG{n}{TERSE}                   \PYG{n}{TERSE}                                  \PYG{n}{TERSE}
\PYG{n}{TRANSPARENT}             \PYG{n}{TRANSPARENT}                            \PYG{n}{TRANSPARENT}
\PYG{n}{UNFINDABLE}              \PYG{n}{UNFINDABLE}                             \PYG{n}{UNFINDABLE}
\PYG{n}{N}\PYG{o}{/}\PYG{n}{A}                     \PYG{n}{N}\PYG{o}{/}\PYG{n}{A}                                    \PYG{n}{UNICODE}
\PYG{p}{(}\PYG{n}{See} \PYG{n+nd}{@flag}\PYG{p}{)}             \PYG{n}{WANDERER}                               \PYG{n}{UNINSPECTED}
\PYG{n}{VERBOSE}                 \PYG{n}{VERBOSE}                                \PYG{n}{VERBOSE}
\PYG{n}{VISUAL}                  \PYG{n}{VISUAL}                                 \PYG{n}{VISUAL}
\PYG{n}{WIZARD}                  \PYG{n}{WIZARD}\PYG{o}{/}\PYG{n}{IMMORTAL}                        \PYG{n}{WIZARD}
\PYG{n}{XTERM256}                \PYG{n}{XTERMCOLOR}                             \PYG{n}{COLOR256}
\end{sphinxVerbatim}


\section{Flags and toggles that only exist in RhostMUSH}
\label{\detokenize{comparison:flags-and-toggles-that-only-exist-in-rhostmush}}
\sphinxAtStartPar
Flags:

\begin{sphinxVerbatim}[commandchars=\\\{\}]
\PYG{n}{ALTQUOTA}                     \PYG{n}{ANONYMOUS}                     \PYG{n}{ARCHITECT}
\PYG{n}{AUDITORIUM}                   \PYG{n}{BLIND}                         \PYG{n}{BOUNCE}
\PYG{n}{CLOAK}                        \PYG{n}{COUNCILOR}                     \PYG{n}{FUBAR}
\PYG{n}{GUILDMASTER}                  \PYG{n}{GUILDOBJ}                      \PYG{n}{IMMORTAL}
\PYG{n}{INDESTRUCTABLE}               \PYG{n}{NO\PYGZus{}ANSINAME}                   \PYG{n}{NO\PYGZus{}BACKSTAGE}
\PYG{n}{NO\PYGZus{}CODE}                      \PYG{n}{NO\PYGZus{}CONNECT}                    \PYG{n}{NO\PYGZus{}EXAMINE}
\PYG{n}{NO\PYGZus{}FLASH}                     \PYG{n}{NO\PYGZus{}GOBJ}                       \PYG{n}{NO\PYGZus{}MODIFY}
\PYG{n}{NO\PYGZus{}MOVE}                      \PYG{n}{NO\PYGZus{}NAME}                       \PYG{n}{NO\PYGZus{}OVERRIDE}
\PYG{n}{NO\PYGZus{}PESTER}                    \PYG{n}{NO\PYGZus{}POSSESS}                    \PYG{n}{NO\PYGZus{}STOP}
\PYG{n}{NO\PYGZus{}UNDERLINE}                 \PYG{n}{NO\PYGZus{}USELOCK}                    \PYG{n}{NO\PYGZus{}WALLS}
\PYG{n}{NO\PYGZus{}YELL}                      \PYG{n}{PRIVATE}                       \PYG{n}{ROBOT}
\PYG{n}{SCLOAK}                       \PYG{n}{SEE\PYGZus{}OEMIT}                     \PYG{n}{SHOWFAILCMD}
\PYG{n}{SIDEFX}                       \PYG{n}{SPAMMONITOR}                   \PYG{n}{SPOOF}
\PYG{n}{STOP}                         \PYG{n}{WANDERER}                      \PYG{n}{ZONECONTENTS}
\PYG{n}{ZONEPARENT}
\end{sphinxVerbatim}

\sphinxAtStartPar
Toggles:

\begin{sphinxVerbatim}[commandchars=\\\{\}]
\PYG{n}{ATRUSE}                       \PYG{n}{CHKREALITY}                    \PYG{n}{CPUTIME}
\PYG{n}{EXFULLWIZATTR}                \PYG{n}{FORCEHALTED}                   \PYG{n}{HIDEIDLE}
\PYG{n}{IGNOREZONE}                   \PYG{n}{IMMPROG}                       \PYG{n}{LOGROOM}
\PYG{n}{MAILVALIDATE}                 \PYG{n}{MAIL\PYGZus{}LOCKDOWN}                 \PYG{n}{MAIL\PYGZus{}NOPARSE}
\PYG{n}{MAIL\PYGZus{}STRIPRETURN}             \PYG{n}{MONITOR\PYGZus{}AREG}                  \PYG{n}{MONITOR\PYGZus{}BAD}
\PYG{n}{MONITOR\PYGZus{}CONN}                 \PYG{n}{MONITOR\PYGZus{}CPU}                   \PYG{n}{MONITOR\PYGZus{}DISREASON}
\PYG{n}{MONITOR\PYGZus{}FAIL}                 \PYG{n}{MONITOR\PYGZus{}SITE}                  \PYG{n}{MONITOR\PYGZus{}STATS}
\PYG{n}{MONITOR\PYGZus{}TIME}                 \PYG{n}{MONITOR\PYGZus{}USERID}                \PYG{n}{MONITOR\PYGZus{}VLIMIT}
\PYG{n}{MORTALREALITY}                \PYG{n}{NODEFAULT}                     \PYG{n}{NOSHPROG}
\PYG{n}{NOZONEPARENT}                 \PYG{n}{NO\PYGZus{}ANSI\PYGZus{}EX}                    \PYG{n}{NO\PYGZus{}ANSI\PYGZus{}EXIT}
\PYG{n}{NO\PYGZus{}ANSI\PYGZus{}PLAYER}               \PYG{n}{NO\PYGZus{}ANSI\PYGZus{}ROOM}                  \PYG{n}{NO\PYGZus{}ANSI\PYGZus{}THING}
\PYG{n}{NO\PYGZus{}FORMAT}                    \PYG{n}{NO\PYGZus{}TIMESTAMP}                  \PYG{n}{PAGELOCK}
\PYG{n}{PROG}                         \PYG{n}{PROG\PYGZus{}ON\PYGZus{}CONNECT}               \PYG{n}{SAFELOG}
\PYG{n}{SEE\PYGZus{}SUSPECT}                  \PYG{n}{SILENTEFFECT}                  \PYG{n}{SNUFFDARK}
\PYG{n}{ZONECMDCHK}                   \PYG{n}{ZONE\PYGZus{}AUTOADD}                  \PYG{n}{ZONE\PYGZus{}AUTOADDALL}
\end{sphinxVerbatim}


\section{@power equivalents}
\label{\detokenize{comparison:power-equivalents}}
\begin{sphinxVerbatim}[commandchars=\\\{\}]
Pennmush                RhostMUSH                                          MUX2

Announce                FREE\PYGZus{}WALL                                          Announce
Boot                    BOOT                                               Boot
Builder                 @quota !WANDERER (flag)                            Builder
CAN\PYGZus{}DARK                @admin player\PYGZus{}dark/@depower dark                   N/A
Can\PYGZus{}spoof               N/A \PYGZhy{} Wizard+ auto\PYGZhy{}spoof                           N/A
Cemit                   N/A                                                N/A
N/A                     CHOWN\PYGZus{}OTHER                                        chown\PYGZus{}anything
N/A                     @lock/twink                                        control\PYGZus{}all
N/A                     WIZ\PYGZus{}WHO                                            expanded\PYGZus{}who
Chat\PYGZus{}Privs              N/A                                                comm\PYGZus{}all
DEBIT                   STEAL                                              Steal\PYGZus{}money
Functions               (See @lfunctions)                                  Wizard+ only
Guest                   GUEST (flag)                                       Guest
HOOK                    Wizard+ only                                       Wizard+ only
Halt                    HALT\PYGZus{}QUEUE/HALT\PYGZus{}QUEUE\PYGZus{}ALL                          Halt
Hide                    NOWHO                                              Hide
Idle                    @timeout player to \PYGZhy{}1                              Idle
N/A                     NO\PYGZus{}MODIFY (flag)                                   Immutable
Link\PYGZus{}Anywhere           N/A (security risk)                                N/A
Login                   LOGIN (flag)                                       LOGIN (flag)
Long\PYGZus{}Fingers            LONG\PYGZus{}FINGERS                                       Long\PYGZus{}fingers
MANY\PYGZus{}ATTRIBS            (@admin vlimit)                                    N/A
N/A                     MONITOR (@toggle)                                  Monitor
No\PYGZus{}Pay                  FREE (flag)                                        Free\PYGZus{}money
No\PYGZus{}Quota                FREE\PYGZus{}QUOTA                                         Free\PYGZus{}quota
Open\PYGZus{}Anywhere           N/A (security risk)                                N/A
N/A                     (Wiz+ Automatic)                                   Pass\PYGZus{}locks
PICK\PYGZus{}DBREFS             Wizard+ only                                       N/A
PUEBLO\PYGZus{}SEND             N/A                                                N/A
Pemit\PYGZus{}All               LONG\PYGZus{}FINGERS                                       N/A
Player\PYGZus{}Create           PCREATE                                            N/A
Poll                    N/A \PYGZhy{}\PYGZhy{} Softcode @doing/header                      Poll
N/A                     PROG (@toggle)                                     Prog
Queue                   SEE\PYGZus{}QUEUE/SEE\PYGZus{}QUEUE\PYGZus{}ALL/HALT\PYGZus{}QUEUE/HALT\PYGZus{}QUEUE\PYGZus{}ALL  N/A
Quotas                  CHANGE\PYGZus{}QUOTAS                                      N/A
SQL\PYGZus{}OK                  N/A                                                N/A
Search                  SEARCH\PYGZus{}ANY                                         Search
See\PYGZus{}All                 EXAMINE\PYGZus{}FULL                                       See\PYGZus{}all
N/A                     WHO\PYGZus{}UNFIND                                         See\PYGZus{}hidden
N/A                     SHUTDOWN                                           Siteadmin
See\PYGZus{}Queue               SEE\PYGZus{}QUEUE/SEE\PYGZus{}QUEUE\PYGZus{}ALL                            N/A
N/A                     STAT\PYGZus{}ANY                                           Stat\PYGZus{}any
Tport\PYGZus{}Anything          TEL\PYGZus{}ANYTHING                                       Tel\PYGZus{}anything
Tport\PYGZus{}Anywhere          TEL\PYGZus{}ANYWHERE                                       Tel\PYGZus{}anywhere
Unkillable              NOKILL                                             Unkillable
\end{sphinxVerbatim}


\subsection{@power unique to RhostMUSH}
\label{\detokenize{comparison:power-unique-to-rhostmush}}
\sphinxAtStartPar
Depowers are unique in Rhost and PennMUSH has no equivelant.

\sphinxAtStartPar
Powers that exist in RhostMUSH that have no match in PennMUSH:

\begin{sphinxVerbatim}[commandchars=\\\{\}]
\PYG{n}{CHOWN\PYGZus{}ME}                     \PYG{n}{WIZ\PYGZus{}WHO}                       \PYG{n}{NOFORCE}
\PYG{n}{FREE\PYGZus{}QUOTA}                   \PYG{n}{JOIN\PYGZus{}PLAYER}                   \PYG{n}{NO\PYGZus{}BOOT}
\PYG{n}{STAT\PYGZus{}ANY}                     \PYG{n}{WHO\PYGZus{}UNFIND}                    \PYG{n}{SHUTDOWN}
\PYG{n}{PURGE}                        \PYG{n}{CHOWN\PYGZus{}ANYWHERE}                \PYG{n}{CHOWN\PYGZus{}OTHER}
\PYG{n}{GRAB\PYGZus{}PLAYER}                  \PYG{n}{SECURITY}                      \PYG{n}{WRAITH}
\PYG{n}{HIDEBIT}
\end{sphinxVerbatim}


\section{Functions equivalents}
\label{\detokenize{comparison:functions-equivalents}}
\begin{sphinxVerbatim}[commandchars=\\\{\}]
Pennmush                RhostMUSH                    MUX

@@                      @@                           @@
ABS                     ABS                          ABS/IABS
ACCENT                  ACCENT                       ACCENT
ACCNAME                 CNAME                        MONIKER
ACOS                    ACOS                         ACOS
ADD                     ADD                          ADD
AFTER                   AFTER                        AFTER
ALIAS                   get(\PYGZsh{}db/alias)/LISTPROTECT   get(\PYGZsh{}db/alias)
ALIGN                   PRINTF                       N/A
ALLOF                   OFPARSE                      N/A
ALPHAMAX                ALPHAMAX                     ALPHAMAX
ALPHAMIN                ALPHAMIN                     ALPHAMIN
AND                     AND                          AND/ANDBOOL
ANDFLAGS                ANDFLAGS                     ANDFLAGS
ANDLFLAGS               ANDFLAG                      N/A
ANDLPOWERS              @function (softfunctions)    N/A
ANSI                    ANSI                         ANSI
APOSS                   APOSS                        APOSS
ART                     ART                          ART
ASIN                    ASIN                         ASIN
ATAN                    ATAN                         ATAN
ATAN2                   ATAN2                        ATAN2
N/A                     ATTRCNT                      ATTRCNT
ATRLOCK                 HASFLAG(\PYGZsh{}obj/attr,LOCK)      HASFLAG(\PYGZsh{}obj/attr,LOCK)
ATTRIB\PYGZus{}SET              SET                          SET
BAND                    MASK                         BAND
BASECONV                PACK/UNPACK                  BASECONV
BEEP                    BEEP                         BEEP
BEFORE                  BEFORE                       BEFORE
N/A                     BITTYPE                      BITTYPE
BENCHMARK               CPUTIME (@toggle)            N/A
BNAND                   MASK                         BNAND
BNOT                    MASK                         N/A
BOR                     BOR                          BOR
BOUND                   BOUND/FBOUND                 N/A
BRACKETS                BRACKETS                     N/A
BXOR                    MASK                         BXOR
CAND                    CAND                         CAND/CANDBOOL
CAPSTR                  CAPSTR                       CAPSTR
CASE                    CASE                         CASE
CASEALL                 CASEALL                      N/A
CAT                     CAT                          CAT
CBUFFER                 N/A                          N/A
CBUFFERADD              N/A                          N/A
CDESC                   N/A                          N/A
CEIL                    CEIL                         CEIL
CEMIT                   N/A                          CEMIT
CENTER                  CENTER                       CENTER
CFLAGS                  N/A                          N/A
CHANNELS                N/A                          CHANNELS
CHECKPASS               CHECKPASS                    N/A
CHILDREN                CHILDREN                     CHILDREN
N/A                     N/A                          CHOOSE
CHR                     CHR                          CHR
CLFLAGS                 N/A                          N/A
CLOCK                   N/A                          N/A
CLONE                   CLONE                        N/A
CMDS                    CMDS                         CMDS
CMOGRIFIER              N/A                          N/A
CMSGS                   N/A                          N/A
COLORS                  COLORS                       N/A
N/A                     N/A                          COLORDEPTH
N/A                     @function (softfunctions)    COLUMNNS
N/A                     N/A                          COMALIAS
COMP                    COMP                         COMP
N/A                     N/A                          COMTITLE
CON                     CON                          CON
COND                    @function (softfunctions)    N/A
CONDALL                 @function (softfunctions)    N/A
CONFIG                  CONFIG                       CONFIG
CONN                    CONN                         CONN
convsecs(get(\PYGZsh{}db/last)) convsecs(get(\PYGZsh{}db/last))      CONNLAST
N/A                     N/A                          CONNLEFT
N/A                     N/A                          CONNMAX
N/A                     N/A                          CONNNUM
N/A                     N/A                          CONNRECORD
N/A                     N/A                          CONNTOTAL
CONTROLS                CONTROLS                     CONTROLS
CONVSECS                CONVSECS                     CONVSECS
CONVTIME                CONVTIME                     CONVTIME
CONVUTCSECS             CONVSECS                     CONVSECS
CONVUTCTIME             CONVTIME                     CONVTIME
COR                     COR                          COR/CORBOOL
COS                     COS                          COS
ALIGN                   PRINTF                       CPAD
N/A                     CRC32                        CRC32
COWNER                  N/A                          N/A
CREATE                  CREATE                       CREATE
CRECALL                 N/A                          N/A
CSECS                   N/A                          N/A
CSTATUS                 N/A                          N/A
CTIME                   N/A                          CTIME
CTITLE                  N/A                          N/A
CTU                     CTU                          CTU
CUSERS                  N/A                          N/A
CWHO                    N/A                          CWHO
DEC                     DEC/XDEC                     DEC
DECODE64                DECODE64                     N/A
DECOMPOSE               TRANSLATE                    TRANSLATE
DECRYPT                 DECRYPT                      DECRYPT
DEFAULT                 DEFAULT                      DEFAULT
N/A                     DESTROY                      DESTROY
DIE                     DICE                         DIE
DIG                     DIG                          CREATE(with \PYGZsq{}r\PYGZsq{})
DIGEST                  DIGEST                       DIGEST
N/A                     TIMEFMT                      DIGITTIME
DIST2D                  DIST2D                       DIST2D
DIST3D                  DIST3D                       DIST3D
@function               @function                    DISTRIBUTE
DIV                     DIV                          IDIV
DOING                   DOING                        DOING
N/A                     N/A                          DUMPING
E                       E                            E
EDEFAULT                EDEFAULT                     EDEFAULT
EDIT                    PEDIT/EDIT                   EDIT
ELEMENT                 MATCH                        MATCH
ELEMENTS                ELEMENTSMUX/ELEMENTS         ELEMENTS
ELIST                   ELIST                        ITEMIZE
ELOCK                   ELOCK                        ELOCK
EMIT                    EMIT                         EMIT
ENCODE64                ENCODE64                     N/A
ENCRYPT                 ENCRYPT                      ENCRYPT
ENDTAG                  N/A                          N/A
ENTRANCES               ENTRANCES                    ENTRANCES
EQ                      EQ                           EQ
N/A                     ERROR                        ERROR
ESCAPE                  ESCAPE                       ESCAPE
ETIME                   @function (softfunctions)    N/A
ETIMEFMT                TIMEFMT                      ETIMEFMT
EVAL                    EVAL                         EVAL
EXIT                    EXIT                         EXIT
N/A                     EXP                          EXP
EXTRACT                 EXTRACT                      EXTRACT
\PYGZpc{}+                      \PYGZpc{}+                           FCOUNT
\PYGZpc{}+                      \PYGZpc{}+                           FDEPTH
FDIV                    FDIV                         FDIV
FILTER                  FILTER                       FILTER
FILTERBOOL              FILTER                       FILTERBOOL
FINDABLE                FINDABLE                     FINDABLE
FIRST                   FIRST                        FIRST
FIRSTOF                 OFPARSE                      N/A
FLAGS                   FLAGS                        FLAGS
FLIP                    REVERSE                      REVERSE
FLOOR                   FLOOR                        FLOOR
FLOORDIV                FLOORDIV                     FLOORDIV
FMOD                    FMOD                         FMOD
FN                      BYPASS                       N/A
FOLD                    FOLD                         FOLD
FOLDERSTATS             FOLDERLIST                   N/A
FOLLOWERS               N/A (softcode available)     N/A
FOLLOWING               N/A (softcode available)     N/A
FOREACH                 FOREACH                      FOREACH
FRACTION                N/A                          N/A
FULLALIAS               ALIAS + LISTPROTECT          N/A
FULLNAME                FULLNAME                     FULLNAME
FUNCTIONS               LISTFUNCTIONS                N/A
GET                     GET                          GET
GETPIDS                 PIDS                         N/A
GET\PYGZus{}EVAL                GET\PYGZus{}EVAL                     GET\PYGZus{}EVAL
GRAB                    GRAB                         GRAB
GRABALL                 GRABALL                      GRABALL
GREP                    GREP                         GREP
GREPI                   GREP                         GREPI
GT                      GT                           GT
GTE                     GTE                          GTE
HASATTR                 HASATTR                      HASATTR
HASATTRP                HASATTRP                     HASATTRP
HASATTRPVAL             HASATTRP                     HASATTRP
HASATTRVAL              HASATTR                      HASATTR
HASFLAG                 HASFLAG                      HASFLAG
HASPOWER                HASPOWER                     HASPOWER
N/A                     HASQUOTA                     HASQUOTA
HASTYPE                 HASTYPE                      HASTYPE
HEIGHT                  @function (softfunctions)    HEIGHT
HIDDEN                  HIDDEN                       N/A
HOME                    HOME                         HOME
HOST                    LOOKUP\PYGZus{}SITE                  HOST
HTML                    N/A                          N/A
IBREAK                  IBREAK                       N/A
IDLE                    IDLE                         IDLE
IF                      IFELSE                       IF
IFELSE                  IFELSE                       IFELSE
ILEV                    ILEV                         ILEV
INAME                   NAME                         NAME
INC                     INC/XINC                     INC
INDEX                   INDEX                        INDEX
INUM                    INUM/INUM2                   INUM
N/A                     INZONE                       INZONE
IPADDR                  LOOKUP\PYGZus{}SITE                  N/A
ISDAYLIGHT              TIMEFMT                      N/A
ISDBREF                 ISDBREF                      ISDBREF
ISINT                   ISINT                        ISINT
ISNUM                   ISNUM                        ISNUM
N/A                     N/A                          ISRAT
ISOBJID                 N/A                          N/A
ISREGEXP                N/A                          N/A
ISWORD                  ISWORD                       ISWORD
ITEMIZE                 ELIST                        ITEMIZE
ITEMS                   WORDS                        WORDS
ITER                    ITER                         ITER
ITEXT                   ITEXT                        ITEXT
LALIGN                  PRINTF                       N/A
STRMATH                 LADD                         LADD
N/A                     LAND                         LAND
LAST                    LAST                         LAST
N/A                     LASTCREATE                   LASTCREATE
LATTR                   LATTR                        LATTR
N/A                     LATTR                        LATTRCMDS
LATTRP                  LATTRP                       LATTRP
N/A                     LCMDS                        LCMDS
LCON                    LCON                         LCON
LCSTR                   LCSTR                        LCSTR
LDELETE                 LDELETE                      LDELETE
LEFT                    LEFT                         STRTRUNC
LEMIT                   LEMIT                        N/A
LETQ                    @function (softfunctions)    N/A
LEXITS                  LEXITS                       LEXITS
LFLAGS                  LFLAGS                       LFLAGS
LINK                    LINK                         LINK
N/A                     LIST (like iter())           LIST (like iter())
LINSERT                 INSERT                       INSERT
LIST                    LISTPOWERS, FLAGS, etc       N/A
LISTQ                   N/A                          N/A
LIT                     LIT                          LIT
LJUST                   LJUST                        LJUST
LLOCKFLAGS              N/A                          N/A
LLOCKS                  LOCKS                        LOCKS
LMATH                   STRFUNC                      N/A
LN                      LN                           LN
LNUM                    LNUM/LNUM2                   LNUM
LOC                     LOC                          LOC
LOCALIZE                LOCALIZE                     LOCALIZE
LOCATE                  LOCATE                       LOCATE
LOCK                    LOCK                         LOCK
N/A                     LOG2FILE                     LOG
LOCKFILTER              LOCKCHECK                    N/A
LOCKFLAGS               FLAGS                        FLAGS
LOCKOWNER               OWNER                        OWNER
LOCKS                   LOCK                         LOCK
LOG                     LOG                          LOG
LPARENT                 PARENTS                      LPARENT
LPIDS                   PIDS                         N/A
LPLAYERS                LCON                         LCON
N/A                     LOR                          LOR
LPORTS                  PORT                         PORTS
ALIGN                   PRINTF                       LPAD
LPOS                    LPOS                         LPOS
DIE                     DICE                         LRAND
N/A                     LROOMS                       LROOMS
LREPLACE                REPLACE                      REPLACE
LSEARCH                 SEARCH/SEARCHNG              SEARCH
LSEARCHR                revwords(search())           revwords(search())
LSET                    SET                          SET
LSTATS                  STATS                        STATS
LT                      LT                           LT
LTE                     LTE                          LTE
LTHINGS                 LCON                         LCON
LVCON                   LCON + STREVAL at mortal     N/A
LVEXITS                 LCON + STREVAL at mortal     N/A
LVPLAYERS               LCON + STREVAL at mortal     N/A
LVTHINGS                LCON + STREVAL at mortal     N/A
LWHO                    LWHO                         LWHO
LWHOID                  N/A                          N/A
MAIL                    MAILREAD/MAILSEND            MAIL
MAILDSTATS              MAILSIZE/MAILQUOTA           MAILSIZE
MAILFROM                MAILREAD                     MAILFROM
MAILFSTATS              FOLDERLIST/FOLDERCURRENT     N/A
MAILLIST                MAILREAD                     N/A
MAILSEND                MAILSEND                     N/A
MAILSTATS               MAILSIZE/MAILQUOTA           MAILSIZE
MAILSTATUS              MAILSIZE/MAILQUOTA           MAILSIZE
MAILSUBJECT             MAILREAD                     MAILSUBJ
MAILTIME                MAILREAD                     N/A
MALIAS                  MAILREAD                     N/A
MAP                     MAP                          MAP
MAPSQL                  N/A                          N/A
MATCH                   MATCH                        MATCH
MATCHALL                MATCHALL                     MATCHALL
MAX                     MAX                          MAX
MEAN                    AVG                          AVG
MEDIAN                  AVG                          AVG
MEMBER                  MEMBER                       MEMBER
MERGE                   MERGE                        MERGE
MESSAGE                 PARSESTR + PEMIT/REMIT       N/A
MID                     MID                          MID
MIN                     MIN                          MIN
MIX                     MIX                          MIX
MODULO                  MOD                          MOD
MONEY                   MONEY                        MONEY
MONIKER                 CNAME                        MONIKER
N/A                     N/A                          MOTD
MSECS                   MODIFYTIME + CONVTIME        MTIME + CONVTIME
MTIME                   MODIFYTIME                   MTIME
MUDNAME                 MUDNAME                      MUDNAME
MUDURL                  N/A                          N/A
MUL                     MUL                          MUL
MUNGE                   MUNGE                        MUNGE
MWHO                    LWHO + STREVAL at mortal     N/A
MWHOID                  N/A                          N/A
NAME                    NAME                         NAME
NAMEGRAB                @function (softfunctions)    N/A
NAMEGRABALL             @function (softfunctions)    N/A
NAMELIST                @function (softfunctions)    N/A
NAND                    NAND                         N/A
NATTR                   ATTRCNT                      ATTRCNT
NATTRP                  ATTRCNT                      ATTRCNT
NCAND                   !CAND                        NOT(CAND())
NCHILDREN               CHILDREN                     CHILDREN
NCON                    WORDS + LCON                 WORDS + LCON
NCOND                   @function (softfunctions)    N/A
NCONDALL                @function (softfunctions)    N/A
NCOR                    !COR                         NOT(COR())
NEARBY                  NEARBY                       NEARBY
NEQ                     NEQ                          NEQ
NEXITS                  WORDS + LEXITS               WORDS + LEXITS
NEXT                    NEXT                         NEXT
NEXTDBREF               N/A                          N/A
NLSEARCH                WORDS + SEARCH               WORDS + SEARCH
NMWHO                   WORDS+LWHO+STREVAL at mort   N/A
NOR                     NOR                          N/A
NOT                     NOT or !                     NOT
NPLAYERS                WORDS + LCON                 WORDS + LCON
NSCEMIT                 N/A                          N/A
NSEARCH                 WORDS + SEARCH               WORDS + SEARCH
NSEMIT                  EMIT                         EMIT
NSLEMIT                 LEMIT                        N/A
NSOEMIT                 OEMIT                        N/A
NSPEMIT                 PEMIT                        N/A
NSPROMPT                N/A (@program?)              N/A (@program?)
NSREMIT                 REMIT                        N/A
NSZEMIT                 ZEMIT                        N/A
NTHINGS                 WORDS + LCON                 WORDS + LCON
NULL                    NULL                         NULL
NUM                     NUM                          NUM
NUMVERSION              N/A                          N/A
NVCON                   WORDS+LCON+STREVAL at mort   N/A
NVEXITS                 WORDS+LEXITS+STREVAL at mo   N/A
NVPLAYERS               WORDS+LCON+STREVAL at mort   N/A
NVTHINGS                WORDS+LCON+STREVAL at mort   N/A
NWHO                    WORDS + LWHO                 WORDS + LWHO
OBJ                     OBJ                          OBJ
OBJEVAL                 OBJEVAL                      OBJEVAL
OBJID                   N/A                          N/A
OBJMEM                  SIZE                         OBJMEM
OEMIT                   OEMIT                        OEMIT
OPEN                    OPEN                         N/A
OR                      OR                           OR/ORBOOL
ORD                     ASC                          ORD
ORDINAL                 N/A                          N/A
ORFLAGS                 ORFLAGS                      ORFLAGS
ORLFLAGS                ORFLAG                       N/A
ORLPOWERS               N/A (easy to @function)      N/A
OWNER                   OWNER                        OWNER
PARENT                  PARENT                       PARENT
PCREATE                 CREATE                       CREATE
PEMIT                   PEMIT                        PEMIT
PFUN                    U + PARENT                   U + PARENT
PI                      PI                           PI
PIDINFO                 PID                          N/A
PLAYER                  before(grab(lwho(1),*:\PYGZpc{}0),:) N/A
PLAYERMEM               SIZE                         PLAYMEM
PMATCH                  PMATCH                       PMATCH
POLL                    DOING                        POLL
PORTS                   PORT                         PORTS
POS                     POS                          POS
POSS                    POSS                         POSS
POWER                   POWER                        POWER
POWERS                  LPOWERS                      POWERS
PROMPT                  N/A (@program?)              N/A (@program?)
PUEBLO                  N/A                          N/A
QUOTA                   QUOTA                        N/A
R                       R                            R
RAND                    RAND                         RAND
RANDWORD                PICKRAND                     PICKRAND
RECV                    CHARIN                       N/A
REGEDIT                 REGEDIT                      N/A
REGEDITALL              REGEDITALL                   N/A
REGEDITALLI             REGEDITALLI                  N/A
REGEDITI                REGEDITI                     N/A
REGISTERS               N/A                          N/A
REGLATTR                LATTR                        N/A
REGLATTRP               LATTRP                       N/A
REGLMATCH               REGLMATCH                    N/A
REGLMATCHALL            REGLMATCHALL                 N/A
REGLMATCHALLI           REGLMATCHALLI                N/A
REGLMATCHI              REGLMATCHI                   N/A
REGMATCH                REGMATCH                     REGMATCH
REGMATCHI               REGMATCHI                    REGMATCHI
REGNATTR                WORDS + ATTR                 N/A
REGNATTRP               WORDS + ATTRP                N/A
REGRAB                  REGRAB                       REGRAB
REGRABALL               REGRABALL                    REGRABALL
REGRABALLI              REGRABALLI                   REGRABALLI
REGRABI                 REGRABI                      REGRABI
REGREP                  REGREP                       N/A
REGREPI                 REGREPI                      N/A
REGXATTR                ATTR                         N/A
REGXATTRP               ATTRP                        N/A
REMAINDER               REMAINDER                    REMAINDER
REMIT                   REMIT                        REMIT
REMOVE                  REMOVE                       REMOVE
RENDER                  N/A                          N/A
REPEAT                  REPEAT                       REPEAT
REST                    REST                         REST
RESTARTS                N/A                          RESTARTS
RESTARTTIME             REBOOTTIME                   RESTARTTIME
CONVTIME(RESTARTTIME))  CONVTIME(REBOOTTIME())       RESTARTSECS
RESWITCH                RESWITCH                     N/A
RESWITCHALL             RESWITCHALL                  N/A
RESWITCHALLI            RESWITCHALLI                 N/A
RESWITCHI               RESWITCHI                    N/A
REVWORDS                REVWORDS                     REVWORDS
RIGHT                   RIGHT                        RIGHT
RJUST                   RJUST                        RJUST
RLOC                    RLOC                         RLOC
N/A                     ROMAN                        ROMAN
RNUM                    RNUM                         N/A
ROOM                    ROOM                         ROOM
ROOT                    N/A                          N/A
ROUND                   ROUND                        ROUND
ALIGN                   PRINTF                       RPAD
S                       S                            S
SCAN                    N/A                          N/A
SCRAMBLE                SCRAMBLE                     SCRAMBLE
SECS                    SECS                         SECS
SECURE                  SECURE/SECUREX               SECURE
SENT                    CHAROUT                      N/A
SET                     SET                          SET
SETDIFF                 SETDIFF                      SETDIFF
SETINTER                SETINTER                     SETINTER
SETQ                    SETQ                         SETQ
SETR                    SETR                         SETR
SETUNION                SETUNION                     SETUNION
SHA0                    DIGEST                       DIGEST
DIGEST                  DIGEST                       SHA1
SHL                     SHL                          SHL
SHR                     SHR                          SHR
SHUFFLE                 SHUFFLE                      SHUFFLE
SIGN                    NCOMP(\PYGZpc{}0,0)                  SIGN
SIN                     SIN                          SIN
SLEV                    N/A                          N/A
@function               @function                    SITEINFO
SORT                    SORT                         SORT
SORTBY                  SORTBY                       SORTBY
SORTKEY                 @function (softfunctions)    N/A
SOUNDEX                 SOUNDEX                      N/A
SOUNDSLIKE              SOUNDXLIKE                   N/A
SPACE                   SPACE                        SPACE
SPEAK                   PARSESTR                     N/A
SPEAKPENN               PARSESTR                     N/A
SPELLNUM                SPELLNUM                     SPELLNUM
SPLICE                  SPLICE                       SPLICE
SQL                     (if MYSQL enabled)           N/A (ASYNC db)
SQLESCAPE               (if MYSQL enabled)           N/A (ASYNC db)
SQRT                    SQRT                         SQRT
SQUISH                  SQUISH                       SQUISH
SSL                     N/A                          N/A
STARTTIME               STARTTIME                    STARTTIME
CONVTIME(STARTTIME))    CONVTIME(STARTTIME())        STARTSECS
N/A                     STATS                        STATS
STDDEV                  AVG                          AVG
STEP                    STEP                         STEP
STEXT                   N/A                          N/A
STRALLOF                OFPARSE                      N/A
STRCAT                  STRCAT                       STRCAT
N/A                     STRIP                        STRIP
STRDELETE               CREPLACE/DELETE              DELETE
STRFIRSTOF              OFPARSE                      N/A
STRINGSECS              @function (softfunctions)    N/A
STRINSERT               CREPLACE                     N/A
STRIPACCENTS            STRIPACCENTS                 STRIPACCENTS
STRIPANSI               STRIPANSI                    STRIPANSI
STRLEN                  STRLEN                       STRLEN
STRMATCH                STRMATCH                     STRMATCH
N/A                     STRLENRAW                    STRMEM
STRREPLACE              CREPLACE/REPLACE             REPLACE
SUB                     SUB                          SUB
N/A                     ESCAPEX                      SUBEVAL
SUBJ                    SUBJ                         SUBJ
N/A                     N/A                          SUCCESSES
SWITCH                  SWITCH                       SWITCH
SWITCHALL               SWITCHALL                    N/A
T                       T                            T
TABLE                   @function (softfunctions)    TABLE
TAG                     N/A                          N/A
TAGWRAP                 N/A                          N/A
TAN                     TAN                          TAN
TEL                     TEL                          TEL
TERMINFO                N/A                          TERMINFO
TESTLOCK                LOCKCHECK                    N/A
TEXTENTRIES             WORDS + TEXTFILE             WORDS + TEXTFILE
TEXTFILE                TEXTFILE                     TEXTFILE
TIME                    TIME                         TIME
TIMEFMT                 PTIMEFMT                     TIMEFMT
TIMESTRING              SINGLETIME/TIMEFMT           SINGLETIME
TR                      TR                           TR
TRIM                    TRIM                         TRIM
TRIMPENN                TRIM                         TRIM
TRIMTINY                TRIM                         TRIM
N/A                     N/A                          TRIGGER
TRUNC                   TRUNC                        TRUNC
TYPE                    TYPE                         TYPE
UCSTR                   UCSTR                        UCSTR
UDEFAULT                UDEFAULT                     UDEFAULT
UFUN                    U                            U
ULAMBDA                 U + \PYGZsh{}lambda                  N/A
ULDEFAULT               ULDEFAULT                    N/A
ULOCAL                  ULOCAL                       ULOCAL
UNIQUE                  LISTDIFF/LISTUNION/LISTINTER N/A
UNSETQ                  N/A                          N/A
UPTIME                  N/A                          N/A
UTCTIME                 TIME                         TIME
V                       V                            V
VADD                    VADD                         VADD
VALID                   VALID                        VALID
VCROSS                  VCROSS                       VCROSS
VDIM                    VDIM                         VDIM
VDOT                    VDOT                         VDOT
VERSION                 VERSION                      VERSION
VISIBLE                 VISIBLE                      VISIBLE
VMAG                    VMAG                         VMAG
VMAX                    SORTLIST                     N/A
VMIN                    SORTLIST                     N/A
VMUL                    VMUL                         VMUL
VSUB                    VSUB                         VSUB
VUNIT                   VUNIT                        VUNIT
WHERE                   WHERE                        WHERE
WIDTH                   @function (softfunctions)    WIDTH
WILDGREP                GREP                         N/A
WILDGREPI               GREPI                        N/A
WIPE                    WIPE                         WIPE
WORDPOS                 WORDPOS                      WORDPOS
WORDS                   WORDS/MWORDS                 WORDS
WRAP                    WRAP                         WRAP
N/A                     MODIFYTIME                   WRITETIME
XATTR                   ATTR                         N/A
XATTRP                  ATTRP                        N/A
XCON                    XCON                         N/A
XEXITS                  LEXITS + EXTRACT             N/A
XGET                    XGET                         XGET
XMWHO                   LWHO+EXTRACT+STREVAL at mor  N/A
XMWHOID                 N/A                          N/A
XOR                     XOR                          XOR
XPLAYERS                XCON                         N/A
XTHINGS                 XCON                         N/A
XVCON                   XCON + STREVAL at mortal     N/A
XVEXITS                 LEXITS + STREVAL at mortal   N/A
XVPLAYERS               XCON + STREVAL at mortal     N/A
XVTHINGS                XCON + STREVAL at mortal     N/A
XWHO                    LWHO + EXTRACT               LWHO + EXTRACT
XWHOID                  N/A                          N/A
ZEMIT                   ZEMIT                        N/A
ZFUN                    ZFUN                         ZFUN
ZMWHO                   ZWHO + STREVAL at mortal     N/A
ZONE                    LZONE                        ZONE
ZWHO                    ZWHO                         ZWHO
\end{sphinxVerbatim}


\subsection{Functions that only exist in RhostMUSH}
\label{\detokenize{comparison:functions-that-only-exist-in-rhostmush}}
\sphinxAtStartPar
Functions that exist in Rhost that do not have a match in PennMUSH:

\begin{sphinxVerbatim}[commandchars=\\\{\}]
\PYG{n}{AIINDEX}                      \PYG{n}{AINDEX}                        \PYG{n}{ANDCHR}
\PYG{n}{ARRAY}                        \PYG{n}{ATTRCNT}                       \PYG{n}{BETWEEN}
\PYG{n}{BITTYPE}                      \PYG{n}{CANSEE}                        \PYG{n}{CAPLIST}
\PYG{n}{CHKGARBAGE}                   \PYG{n}{CHKREALITY}                    \PYG{n}{CHKTRACE}
\PYG{n}{CHOMP}                        \PYG{n}{CITER}                         \PYG{n}{CLOAK}
\PYG{n}{CLUSTER\PYGZus{}ADD}                  \PYG{n}{CLUSTER\PYGZus{}ATTRCNT}               \PYG{n}{CLUSTER\PYGZus{}DEFAULT}
\PYG{n}{CLUSTER\PYGZus{}EDEFAULT}             \PYG{n}{CLUSTER\PYGZus{}FLAGS}                 \PYG{n}{CLUSTER\PYGZus{}GET}
\PYG{n}{CLUSTER\PYGZus{}GET\PYGZus{}EVAL}             \PYG{n}{CLUSTER\PYGZus{}GREP}                  \PYG{n}{CLUSTER\PYGZus{}HASATTR}
\PYG{n}{CLUSTER\PYGZus{}HASFLAG}              \PYG{n}{CLUSTER\PYGZus{}LATTR}                 \PYG{n}{CLUSTER\PYGZus{}REGREP}
\PYG{n}{CLUSTER\PYGZus{}REGREPI}              \PYG{n}{CLUSTER\PYGZus{}SET}                   \PYG{n}{CLUSTER\PYGZus{}STATS}
\PYG{n}{CLUSTER\PYGZus{}U}                    \PYG{n}{CLUSTER\PYGZus{}U2}                    \PYG{n}{CLUSTER\PYGZus{}U2DEFAULT}
\PYG{n}{CLUSTER\PYGZus{}U2LDEFAULT}           \PYG{n}{CLUSTER\PYGZus{}U2LOCAL}               \PYG{n}{CLUSTER\PYGZus{}UDEFAULT}
\PYG{n}{CLUSTER\PYGZus{}UEVAL}                \PYG{n}{CLUSTER\PYGZus{}ULDEFAULT}             \PYG{n}{CLUSTER\PYGZus{}ULOCAL}
\PYG{n}{CLUSTER\PYGZus{}VATTRCNT}             \PYG{n}{CLUSTER\PYGZus{}WIPE}                  \PYG{n}{CLUSTER\PYGZus{}XGET}
\PYG{n}{COLUMNS}                      \PYG{n}{COSH}                          \PYG{n}{COUNTSPECIAL}
\PYG{n}{CRC32}                        \PYG{n}{DELEXTRACT}                    \PYG{n}{DESTROY}
\PYG{n}{EDITANSI}                     \PYG{n}{EE}                            \PYG{n}{ERROR}
\PYG{n}{EXP}                          \PYG{n}{FBETWEEN}                      \PYG{n}{FBOUND}
\PYG{n}{GARBLE}                       \PYG{n}{GLOBALROOM}                    \PYG{n}{GUILD}
\PYG{n}{HASDEPOWER}                   \PYG{n}{HASQUOTA}                      \PYG{n}{HASRXLEVEL}
\PYG{n}{HASTOGGLE}                    \PYG{n}{HASTXLEVEL}                    \PYG{n}{INPROGRAM}
\PYG{n}{INZONE}                       \PYG{n}{ISALNUM}                       \PYG{n}{ISALPHA}
\PYG{n}{ISCLUSTER}                    \PYG{n}{ISDIGIT}                       \PYG{n}{ISHIDDEN}
\PYG{n}{ISLOWER}                      \PYG{n}{ISPUNCT}                       \PYG{n}{ISSPACE}
\PYG{n}{ISUPPER}                      \PYG{n}{ISXDIGIT}                      \PYG{n}{KEEPFLAGS}
\PYG{n}{KEEPTYPE}                     \PYG{n}{LAND}                          \PYG{n}{LAVG}
\PYG{n}{LCMDS}                        \PYG{n}{LDEPOWERS}                     \PYG{n}{LISTMATCH}
\PYG{n}{LISTNEWSGROUPS}               \PYG{n}{LISTRLEVELS}                   \PYG{n}{LISTTOGGLES}
\PYG{n}{LLOC}                         \PYG{n}{LMAX}                          \PYG{n}{LMIN}
\PYG{n}{LMUL}                         \PYG{n}{LNOR}                          \PYG{n}{LOCALFUNC}
\PYG{n}{LOCKDECODE}                   \PYG{n}{LOCKENCODE}                    \PYG{n}{LOGSTATUS}
\PYG{n}{LOGTOFILE}                    \PYG{n}{LOR}                           \PYG{n}{LRAND}
\PYG{n}{LROOMS}                       \PYG{n}{LTOGGLES}                      \PYG{n}{LXNOR}
\PYG{n}{LXOR}                         \PYG{n}{MONEYNAME}                     \PYG{n}{MOON}
\PYG{n}{MOVE}                         \PYG{n}{NAMEQ}                         \PYG{n}{NOSTR}
\PYG{n}{NOTCHR}                       \PYG{n}{NSLOOKUP}                      \PYG{n}{ORCHR}
\PYG{n}{PARENMATCH}                   \PYG{n}{PFIND}                         \PYG{n}{PGREP}
\PYG{n}{POWER10}                      \PYG{n}{PRIVATIZE}                     \PYG{n}{PROGRAMMER}
\PYG{n}{PUSHREGS}                     \PYG{n}{RACE}                          \PYG{n}{RANDMATCH}
\PYG{n}{RANDPOS}                      \PYG{n}{REGEDITALLILIT}                \PYG{n}{REGEDITALLLIT}
\PYG{n}{REGEDITILIT}                  \PYG{n}{REGEDITLIT}                    \PYG{n}{REGNUMMATCH}
\PYG{n}{REGNUMMATCHI}                 \PYG{n}{REMFLAGS}                      \PYG{n}{REMTYPE}
\PYG{n}{ROMAN}                        \PYG{n}{ROTL}                          \PYG{n}{ROTR}
\PYG{n}{RSET}                         \PYG{n}{RXLEVEL}                       \PYG{n}{SAFEBUFF}
\PYG{n}{SEES}                         \PYG{n}{SETQMATCH}                     \PYG{n}{SHIFT}
\PYG{n}{SINH}                         \PYG{n}{SORTLISAT}                     \PYG{n}{STR}
\PYG{n}{STRDISTANCE}                  \PYG{n}{STREQ}                         \PYG{n}{STREVAL}
\PYG{n}{STRFUNC}                      \PYG{n}{STRIP}                         \PYG{n}{STRLENRAW}
\PYG{n}{STRLENVIS}                    \PYG{n}{STRMATH}                       \PYG{n}{SUBNETMATCH}
\PYG{n}{TANH}                         \PYG{n}{TOGGLE}                        \PYG{n}{TOTCMDS}
\PYG{n}{TRACE}                        \PYG{n}{TXLEVEL}                       \PYG{n}{UEVAL}
\PYG{n}{WHILE}                        \PYG{n}{WILDMATCH}                     \PYG{n}{WRAPCOLUMNS}
\PYG{n}{WRITABLE}                     \PYG{n}{XNOR}                          \PYG{n}{XORCHR}
\PYG{n}{XORFLAG}                      \PYG{n}{ZFUNDEFAULT}                   \PYG{n}{ZFUNEVAL}
\PYG{n}{ZFUNLDEFAULT}                 \PYG{n}{ZFUNLOCAL}
\end{sphinxVerbatim}


\section{What may need to be modified to get softcode from PennMUSH, TinyMUSH2, TinyMUSH3, or MUX2 to work on Rhost}
\label{\detokenize{comparison:what-may-need-to-be-modified-to-get-softcode-from-pennmush-tinymush2-tinymush3-or-mux2-to-work-on-rhost}}
\sphinxAtStartPar
RhostMUSH, for the most part, will work out of the box with most softcode gotten
from other codebases.  There are, however, exceptions.  Most of these exceptions
will be minor code differences between how ANSI is processed, the variences
of arguments or switches to commands or functions, or required flags.

\sphinxAtStartPar
Most changes will revolve around the ones listed in this document.


\subsection{Problematic functions between codebases}
\label{\detokenize{comparison:problematic-functions-between-codebases}}
\sphinxAtStartPar
lsearch()/search(), align()/printf(), *attrval()


\subsection{Problematic features between codebases}
\label{\detokenize{comparison:problematic-features-between-codebases}}
\sphinxAtStartPar
named variables for regexp patterns in \$commands are not supported.
@aliases on non\sphinxhyphen{}players are not supported.  Frankly I find them redundant.


\subsection{Problematic commands}
\label{\detokenize{comparison:problematic-commands}}
\sphinxAtStartPar
@mapsql, hardcoded required comssytem commands (some are redundant)


\subsection{SIDEFX flag}
\label{\detokenize{comparison:sidefx-flag}}\begin{quote}

\sphinxAtStartPar
Anything that uses sideeffects \sphinxhyphen{}\sphinxhyphen{}DIRECTLY\sphinxhyphen{}\sphinxhyphen{} requires this flag.
Sideeffects are like set(), pemit(), and so forth.  list(), while a
side\sphinxhyphen{}effect, does not require this flag as it is considered passive and safe.
\end{quote}


\subsection{Variable exits}
\label{\detokenize{comparison:variable-exits}}\begin{quote}

\sphinxAtStartPar
Rhost handles them slightly different.  You do not link
exits to \#\sphinxhyphen{}4.  That\textquotesingle{}s an invalid destination.  I always found it, frankly,
stupid to save any data in the database that was literally invalid.  So,
you link the exit as you normally would, then @toggle the exit variable.
At that point you use @exitto like you would any other codebase.
\end{quote}


\subsection{Zones}
\label{\detokenize{comparison:zones}}\begin{quote}

\sphinxAtStartPar
Zones actually can work near exactly as you would expect them to
work on TinyMUSH, MUX, or Penn.  Either at once or at different times.
We recognize multiple zones, zone masters, zone inheritance, zone
parenting, zone command processing, and the ability to bypass zones
entirely.  There\textquotesingle{}s a ton of flexbility with this.  However, the syntax
for adding/removing zones is different so the commands will have to be
ported to Rhost.
\end{quote}


\subsection{@powers}
\label{\detokenize{comparison:powers}}\begin{quote}

\sphinxAtStartPar
Powers work a bit differently in Rhost and they\textquotesingle{}re named
differently, which should not be that big a surprise as they\textquotesingle{}re different
between all the codebases anyway.  The big difference is our powers are
tiered, meaning the can be limited or grown to a given bitlevel and are
not just toggle powers like the other codebases.  We also have @depower
that is the anti\sphinxhyphen{}thesis of @power
\end{quote}


\subsection{Attribute length}
\label{\detokenize{comparison:attribute-length}}\begin{quote}

\sphinxAtStartPar
While we have 64 character attribute capabilities like
most other codebases, PennMUSH allows 1024 attribute length attributes.
Why you need one that long boggles the mind, but if you do use attribs
that long you need to make sure they are cut down to the proper length.
\end{quote}


\subsection{Attribute contents}
\label{\detokenize{comparison:attribute-contents}}\begin{quote}

\sphinxAtStartPar
You\textquotesingle{}ll be happy to know that Rhost allows upwards
to 64,000 bytes of data to be assigned an LBUF.  We strongly recommand
to cap at 32,000 however as the various TCP socket protocols play nicer
with that value.
\end{quote}


\subsection{256 color}
\label{\detokenize{comparison:color}}
\sphinxAtStartPar
Yup!  We got it.


\subsection{Unicode/UTF8}
\label{\detokenize{comparison:unicode-utf8}}\begin{quote}

\sphinxAtStartPar
Yup!  We got this too.  Not quiet yet in the main branch,
but download Kage\textquotesingle{}s branch, you won\textquotesingle{}t be dissapointed.  We will have
UTF8 in Rhost 4.0 when released.
\end{quote}


\subsection{Attributes per object}
\label{\detokenize{comparison:attributes-per-object}}\begin{quote}

\sphinxAtStartPar
This is configurable with the VLIMIT @admin
command, however, it is absolutely hard\sphinxhyphen{}limited at 10000 maximum.
This is to avoid any DoS type situation and because frankly there
should never be a reason to exceed that.  If you need more, use
@clusters.
\end{quote}


\subsection{Destroying}
\label{\detokenize{comparison:destroying}}\begin{quote}

\sphinxAtStartPar
@nuke only works on players.  @destroy works on non\sphinxhyphen{}players.
Never the two will meet.  We also have a built in recycle bin meaning
anything destroyed will not be automatically recycled.  If you want it
recycled, you have to @purge it.  Yes, if you use  Myrddin\textquotesingle{}s CRON, it
has a built in entry to auto\sphinxhyphen{}purge anything older than 30 days.  This
also means you can on\sphinxhyphen{}line recover anything destroyed before that 30
days.  Groovy, eh?
\end{quote}


\subsection{object id\textquotesingle{}s}
\label{\detokenize{comparison:object-id-s}}
\sphinxAtStartPar
Yup, we got them.  Even in searches, and, well, everything.


\subsection{lsearch() and search()}
\label{\detokenize{comparison:lsearch-and-search}}\begin{quote}

\sphinxAtStartPar
lsearch() in Penn is not syntacically similar to non\sphinxhyphen{}Penn search().
This will have to be altered.  In addition, search() in non\sphinxhyphen{}penn games
have to have special consideration for escaping out the evaled args.
\end{quote}


\subsection{@locks can be different}
\label{\detokenize{comparison:locks-can-be-different}}\begin{quote}

\sphinxAtStartPar
We have many more lock capabilities and options
so this should be a non\sphinxhyphen{}issue.
\end{quote}


\subsection{Customer user\sphinxhyphen{}locks}
\label{\detokenize{comparison:customer-user-locks}}\begin{quote}

\sphinxAtStartPar
We do not have custom user\sphinxhyphen{}locks like Penn.  We do, however, have the way
to set encapsulated lock data into an attribute to fetch and compare
against which I find more useful and far more flexible.
See: lockencode(), lockdecode(), and lockcheck()
\end{quote}


\subsection{Attribute trees}
\label{\detokenize{comparison:attribute-trees}}\begin{quote}

\sphinxAtStartPar
Unlike Penn, we don\textquotesingle{}t really have attribute trees.  We do support the
basic capabilities of it for compatibility if you load in softcode that
uses it, but it doesn\textquotesingle{}t have the advanced features of attribute trees.
Please see \textquotesingle{}help attribute tree\textquotesingle{} for more information.
\end{quote}


\subsection{Prefix permission locking}
\label{\detokenize{comparison:prefix-permission-locking}}\begin{quote}

\sphinxAtStartPar
We do allow prefix permission locking, and some very advanced abilities
of it.  Please see wizhelp on @aflags for more information.
\sphinxhyphen{} wizhelp @aflags
\sphinxhyphen{} wizhelp atrperms\_max
\sphinxhyphen{} wizhelp atrlock
\sphinxhyphen{} wizhelp atrperms
\end{quote}


\subsection{align() and printf()}
\label{\detokenize{comparison:align-and-printf}}\begin{quote}

\sphinxAtStartPar
We do not have align().  Most of the code that uses align() will have to
be converted to our printf() (which is compatible but has different syntax)
\end{quote}


\subsection{MySQL}
\label{\detokenize{comparison:mysql}}\begin{quote}

\sphinxAtStartPar
While we support MySQL, we do not have an async method like MUX2.  This
is just not possible, sorry.
\end{quote}


\subsection{Mail System}
\label{\detokenize{comparison:mail-system}}\begin{quote}

\sphinxAtStartPar
There are mail wrappers to mimic MUX/TM3 and Penn mail systems.
\end{quote}


\subsection{Comsystem}
\label{\detokenize{comparison:comsystem}}\begin{quote}

\sphinxAtStartPar
The softcoded comsystem mimics MUX/TM3 and Penn\textquotesingle{}s comsystem.
\end{quote}


\subsection{Various Functions}
\label{\detokenize{comparison:various-functions}}\begin{quote}

\sphinxAtStartPar
There is a \textquotesingle{}softcode.minmax\textquotesingle{} in the Mushcode directory that loads up a slew
of @function wrappers that will emulate various functions that MUX, Penn, or
TM3 has.  We have the functionality for nearly all of them, but either our
functions have different syntax, or we have different named functions that
duplicate the functionality.  It would be far better to recode it to use
the native functions, but the @function wrappers are there for lazyness :)
\end{quote}


\subsection{Empty Attributes}
\label{\detokenize{comparison:empty-attributes}}\begin{quote}

\sphinxAtStartPar
Penn allows you to have empty attributes.  Non\sphinxhyphen{}penn codebases do not.
Thus, hasattrval and the like are not needed and should likely just point
to hasattr instead.
\end{quote}


\subsection{Player Stats}
\label{\detokenize{comparison:player-stats}}\begin{quote}

\sphinxAtStartPar
MUX has some built in ways for player stats.  We do as well but they\textquotesingle{}re
either done via functions or attribute contents.  Code that requires this
will have to be recoded.
\end{quote}


\subsection{Percent Substitutions}
\label{\detokenize{comparison:percent-substitutions}}\begin{quote}

\sphinxAtStartPar
Some percent substitutions may differ between codebases.  Luckily, Rhost
allows the ability to remap or creaete new ones if this is a problem.
\end{quote}


\subsection{Switches}
\label{\detokenize{comparison:switches}}\begin{quote}

\sphinxAtStartPar
Some switches may not exist in Rhost that do in other codebases, in such
a case, Rhost does allow the ability to @hook a command to define your own
softcoded switch to a hardcoded command and work around the limitation.
\end{quote}


\subsection{Adding Flags}
\label{\detokenize{comparison:adding-flags}}\begin{quote}

\sphinxAtStartPar
Some flags may be missing.  If it\textquotesingle{}s a dummy flag, feel free to use the
marker flags MARKER0 to MARKER9 to set them.  If it\textquotesingle{}s an existing flag
that does similar features, feel free to flag\_alias it or just flag\_name
it to the other name if you want.
\end{quote}


\subsection{Aliases}
\label{\detokenize{comparison:aliases}}\begin{quote}

\sphinxAtStartPar
Multiple aliases are supported via @protect.
\end{quote}


\section{Things other mushes can do that Rhost can not and how to emulate it}
\label{\detokenize{comparison:things-other-mushes-can-do-that-rhost-can-not-and-how-to-emulate-it}}

\subsection{PennMUSH}
\label{\detokenize{comparison:pennmush}}\begin{itemize}
\item {} 
\sphinxAtStartPar
Attribute trees.  These are emulated as a base set and can be duplicated enough to at least port code.

\item {} 
\sphinxAtStartPar
lsearch() will have to be recoded to search()

\item {} 
\sphinxAtStartPar
align() will have to be recoded to printf()

\item {} 
\sphinxAtStartPar
Penn allows empty attributes.  Rhost does not.  Work will have to be done to take this into consideration.

\item {} 
\sphinxAtStartPar
All *val() functions in Penn that are used will have to be remapped to a non\sphinxhyphen{}*val() function.  It should be as simple as function\_alias to the non\sphinxhyphen{}*val.  Example:

\begin{sphinxVerbatim}[commandchars=\\\{\}]
\PYG{n+nd}{@admin} \PYG{n}{function\PYGZus{}alias}\PYG{o}{=}\PYG{n}{hasattrval} \PYG{n+nb}{hasattr}
\end{sphinxVerbatim}

\item {} 
\sphinxAtStartPar
Penn\textquotesingle{}s hardcoded comsystem is emulated with the softcode comsys

\item {} 
\sphinxAtStartPar
Penn\textquotesingle{}s @mail system is workable with mail wrappers

\item {} 
\sphinxAtStartPar
Pueblo is not supported.

\item {} 
\sphinxAtStartPar
json is not supported.

\item {} 
\sphinxAtStartPar
ssl is not natively supported (yet).

\item {} 
\sphinxAtStartPar
This uses @shutdown/restart, Rhost uses @reboot

\end{itemize}


\subsection{MUX}
\label{\detokenize{comparison:mux}}\begin{itemize}
\item {} 
\sphinxAtStartPar
Mux has an async mysql database engine.  This is not possible with Rhost.  You\textquotesingle{}ll have to use the sync method instead.

\item {} 
\sphinxAtStartPar
UTF8 is supported but internally passed differently.

\item {} 
\sphinxAtStartPar
Mux\textquotesingle{}s hardcoded comsystem is emulated with the softcode comsys

\item {} 
\sphinxAtStartPar
MUX\textquotesingle{}s @mail system is workable with mail wrappers

\item {} 
\sphinxAtStartPar
Pueblo is not supported.

\item {} 
\sphinxAtStartPar
This uses @restart, Rhost uses @reboot

\end{itemize}


\chapter{Database}
\label{\detokenize{database:database}}\label{\detokenize{database::doc}}

\section{Loading an existing Database}
\label{\detokenize{database:loading-an-existing-database}}
\sphinxAtStartPar
To load in a previous database, you run the db\_load script.

\sphinxAtStartPar
From the game directory you would type:

\begin{sphinxVerbatim}[commandchars=\\\{\}]
\PYG{o}{.}\PYG{o}{/}\PYG{n}{db\PYGZus{}load} \PYG{n}{data}\PYG{o}{/}\PYG{n}{netrhost}\PYG{o}{.}\PYG{n}{gdbm} \PYG{n}{yourflatfilehere} \PYG{n}{data}\PYG{o}{/}\PYG{n}{netrhost}\PYG{o}{.}\PYG{n}{db}\PYG{o}{.}\PYG{n}{new}
\end{sphinxVerbatim}

\begin{sphinxadmonition}{note}{Note:}
\sphinxAtStartPar
You may also do: ./Startmush
Then you just follow the prompts to load in your flatfile there.
\end{sphinxadmonition}

\sphinxAtStartPar
If you wish to have \#1\textquotesingle{}s password reset to \textquotesingle{}Nyctasia\textquotesingle{} please add this
to the bottom of your netrhost.conf file:

\begin{sphinxVerbatim}[commandchars=\\\{\}]
\PYG{n}{newpass\PYGZus{}god} \PYG{l+m+mi}{777}
\end{sphinxVerbatim}

\sphinxAtStartPar
The caveat is that you must not have any netrhost.db* or netrhost.gdbm* files
in your data directory prior to loading it in.  It\textquotesingle{}ll error out if previous
files exist.  So you will need to move all files that start with netrhost.db*
and all files that start with netrhost.gdbm* to another directory.

\sphinxAtStartPar
Your flatfile tends to be named \textquotesingle{}netrhost.db.flat\textquotesingle{} which is in your data
directory.  You can, however, name your flatfile anything you want and have
it in any directory you want.

\sphinxAtStartPar
To make a flatfile in game, you just issue @dump/flat.  You can specify
a filename after it, otherwise it assumes the name \textquotesingle{}netrhost.db.flat\textquotesingle{}.


\section{Converting a flatfile database for use in RhostMUSH}
\label{\detokenize{database:converting-a-flatfile-database-for-use-in-rhostmush}}
\sphinxAtStartPar
In the \textasciitilde{}/Server/convert directory there is a script called \textquotesingle{}doconvert.sh\textquotesingle{}

\sphinxAtStartPar
This script will convert most flatfiles from existing mush engines to
RhostMUSH.  The exception is PennMUSH 1.8.0 and later.  For this there is a
BETA converter penn18x\_converter.tgz.  This is proven to work, most of the time,
with codebases between 1.8.0 and 1.8.2.  It has not been fully vetted with
the latest PennMUSH databases.  Our apologies.

\sphinxAtStartPar
To convert a non\sphinxhyphen{}pennmush game (or a pennmush 1.7.4 or earlier), you first
need a valid flatfile of the game you\textquotesingle{}re wanting to convert.  Please refer
to the documentatation of the mush engine in question (MUX, Penn 1.7, TM2/3)
on how to do this.  Once you have it type::

\begin{sphinxVerbatim}[commandchars=\\\{\}]
\PYG{o}{.}\PYG{o}{/}\PYG{n}{doconvert}\PYG{o}{.}\PYG{n}{sh} \PYG{n}{FLATFILETOCONVERT} \PYG{n}{FLATFILEOUTPUT}
\end{sphinxVerbatim}

\sphinxAtStartPar
In this instance, FLATFILETOCONVERT will be the filename (with full path) to
the flatfile you are wishing to convert.

\sphinxAtStartPar
The FLATFILEOUTPUT is anyfilename you wish to name the RhostMUSH converted
flatfile.  I suggest netrhost\_converted.db.flat so you know by the name
what it is.

\sphinxAtStartPar
Follow what it asks for and just let it do its thing.


\section{Note about Compiling}
\label{\detokenize{database:note-about-compiling}}
\sphinxAtStartPar
If you are importing a MUX2 flatfile, make ABSOLUTELY SURE that you select
mux passwords as a compatibility option, or you will NOT BE ABLE to log in
to players as the password will not be recognizeable.

\sphinxAtStartPar
Make sure to keep QDBM selected as it\textquotesingle{}s a much more stable database engine
that does not have attribute limit restrictions like GDBM does.

\sphinxAtStartPar
If you are converting from a Penn, TinyMUSH, or MUX database, make sure you
drill down into the LBUF section and select, at minimum, 8K lbufs.  You likely
want that anyway as it gives you far more room for attribute content storage.

\sphinxAtStartPar
You can go up to 32K safely.  While 64k is safe and does work, there are issues
with networking and older routers that use a 32K TCP buffer size that can
at times cut off the data as overflow resulting in output to the end\sphinxhyphen{}point
players not receiving their data.  So it is strongly recommended not to go
above 32K in lbuffer size.

\sphinxAtStartPar
Go ahead and select 64 char attributes.  It allows you to have 64 characters
for attribute names.  It\textquotesingle{}s handy to have.

\sphinxAtStartPar
If you wish at this point to set up mysql and/or sqlite, you  may do so.
Yes, you can use them in parallel without any issue.


\chapter{Getting Started}
\label{\detokenize{gettingstarted:getting-started}}\label{\detokenize{gettingstarted::doc}}

\section{What to type to get the basics running if you did not choose a pre\sphinxhyphen{}existing flatfile}
\label{\detokenize{gettingstarted:what-to-type-to-get-the-basics-running-if-you-did-not-choose-a-pre-existing-flatfile}}
\sphinxAtStartPar
If you decided to get a bare\sphinxhyphen{}bone configuration, you will find your mush has just two things.  The \#1 (God) player and the starting room \#0.  That\textquotesingle{}s it.


\subsection{Login to \#1 from the connect screen}
\label{\detokenize{gettingstarted:login-to-1-from-the-connect-screen}}
\sphinxAtStartPar
Nyctasia is the default password:

\begin{sphinxVerbatim}[commandchars=\\\{\}]
\PYG{n}{co} \PYG{c+c1}{\PYGZsh{}1 Nyctasia}
\end{sphinxVerbatim}


\subsection{Change \#1\textquotesingle{}s password to something you\textquotesingle{}ll remember but is hard to guess}
\label{\detokenize{gettingstarted:change-1-s-password-to-something-you-ll-remember-but-is-hard-to-guess}}
\sphinxAtStartPar
Note: yourpasswordgoeshere can be any password you choose.  Choose well:

\begin{sphinxVerbatim}[commandchars=\\\{\}]
\PYG{n+nd}{@password} \PYG{n}{Nyctasia}\PYG{o}{=}\PYG{n}{YOURPASSWORDGOESHERE}
\end{sphinxVerbatim}


\subsection{Master Room}
\label{\detokenize{gettingstarted:master-room}}
\sphinxAtStartPar
At this point you should create your master room:

\begin{sphinxVerbatim}[commandchars=\\\{\}]
\PYG{n+nd}{@dig} \PYG{n}{Master} \PYG{n}{Room}
\end{sphinxVerbatim}

\begin{sphinxadmonition}{note}{Note:}
\sphinxAtStartPar
Reason: You need a master room to contain global \$commands for players
\textasciicircum{}listens are not global for intentional reasons.  It\textquotesingle{}s far too much overhead for far too minimal gains that few people need or use.
\end{sphinxadmonition}


\subsection{Flag and protect Master Room}
\label{\detokenize{gettingstarted:flag-and-protect-master-room}}
\sphinxAtStartPar
It will return a dbref\#, it should be \#2 if you\textquotesingle{}ve not created anything else:

\begin{sphinxVerbatim}[commandchars=\\\{\}]
\PYG{n+nd}{@set} \PYG{c+c1}{\PYGZsh{}2=safe ind halt float}
\end{sphinxVerbatim}


\subsection{Player Holder Characters}
\label{\detokenize{gettingstarted:player-holder-characters}}
\sphinxAtStartPar
Feel free to change the password to what you want

\begin{sphinxadmonition}{note}{Note:}
\sphinxAtStartPar
Reason: You will want to chown global room or global areas to a given bitlevel and a method to keep organized.
Note: wizhelp control will give you a complete breakdown of what each bit can do.
\end{sphinxadmonition}


\subsubsection{Immortal Holder}
\label{\detokenize{gettingstarted:immortal-holder}}
\begin{sphinxVerbatim}[commandchars=\\\{\}]
@pcreate ImmHolder=abc123
@set *Immholder=no\PYGZus{}connect !wanderer immortal
@badsite *immholder=*
\end{sphinxVerbatim}


\subsubsection{Royalty/Wizard Holder}
\label{\detokenize{gettingstarted:royalty-wizard-holder}}
\begin{sphinxVerbatim}[commandchars=\\\{\}]
@pcreate WizHolder=abc123
@set *wizholder=no\PYGZus{}connect !wanderer royalty
@badsite *wizholder=*
\end{sphinxVerbatim}


\subsubsection{Councilor/Admin Holder}
\label{\detokenize{gettingstarted:councilor-admin-holder}}
\begin{sphinxVerbatim}[commandchars=\\\{\}]
@pcreate AdminHolder=abc123
@set *adminholder=no\PYGZus{}connect !wanderer councilor
@badsite *adminholder=*
\end{sphinxVerbatim}


\subsubsection{Architect/Staff Holder}
\label{\detokenize{gettingstarted:architect-staff-holder}}
\begin{sphinxVerbatim}[commandchars=\\\{\}]
@pcreate StaffHolder=abc123
@set *staffholder=no\PYGZus{}connect !wanderer architect
@badsite *staffholder=*
\end{sphinxVerbatim}


\subsubsection{Guildmaster/Lead Holder}
\label{\detokenize{gettingstarted:guildmaster-lead-holder}}
\begin{sphinxVerbatim}[commandchars=\\\{\}]
@pcreate GuildHolder=abc123
@set *guildholder=no\PYGZus{}connect !wanderer guildmaster
@badsite *guildholder=*
\end{sphinxVerbatim}


\subsection{Chown \#0 (The starting room) and \#2 (The Master room) to immholder}
\label{\detokenize{gettingstarted:chown-0-the-starting-room-and-2-the-master-room-to-immholder}}
\begin{sphinxadmonition}{note}{Note:}
\sphinxAtStartPar
\#0 you can chown to a different bitlevel if you want, but the master room should always be owned by an immortal
\end{sphinxadmonition}

\begin{sphinxVerbatim}[commandchars=\\\{\}]
\PYG{n+nd}{@chown}\PYG{o}{/}\PYG{n}{preserve} \PYG{c+c1}{\PYGZsh{}0=*immholder}
\PYG{n+nd}{@chown}\PYG{o}{/}\PYG{n}{preserve} \PYG{c+c1}{\PYGZsh{}2=*immholder}
\end{sphinxVerbatim}


\subsection{Create yourself your own immortal player then log off \#1 and into this immortal player}
\label{\detokenize{gettingstarted:create-yourself-your-own-immortal-player-then-log-off-1-and-into-this-immortal-player}}
\begin{sphinxadmonition}{note}{Note:}
\sphinxAtStartPar
Pick what you want for playername and playerpassword
\end{sphinxadmonition}

\begin{sphinxVerbatim}[commandchars=\\\{\}]
@pcreate PLAYERNAME=PLAYERPASSWORD
@set *playername=!wanderer immortal
\end{sphinxVerbatim}


\subsection{Log out of \#1 and log into your immortal player}
\label{\detokenize{gettingstarted:log-out-of-1-and-log-into-your-immortal-player}}
\begin{sphinxadmonition}{note}{Note:}
\sphinxAtStartPar
Use the playername and password you created in the step before
\end{sphinxadmonition}

\begin{sphinxVerbatim}[commandchars=\\\{\}]
\PYG{n}{LOGOUT}
\PYG{n}{co} \PYG{n}{PLAYERNAME} \PYG{n}{PLAYERPASSWORD}
\end{sphinxVerbatim}


\subsection{Create your guest characters}
\label{\detokenize{gettingstarted:create-your-guest-characters}}
\begin{sphinxadmonition}{note}{Note:}
\sphinxAtStartPar
Feel free to change the description if you want
\end{sphinxadmonition}

\begin{sphinxVerbatim}[commandchars=\\\{\}]
\PYG{n+nd}{@dolist} \PYG{n}{lnum}\PYG{p}{(}\PYG{l+m+mi}{1}\PYG{p}{,}\PYG{l+m+mi}{10}\PYG{p}{)}\PYG{o}{=}\PYG{p}{\PYGZob{}}\PYG{n+nd}{@pcreate} \PYG{n}{Guest}\PYG{c+c1}{\PYGZsh{}\PYGZsh{}=guest;@set *Guest\PYGZsh{}\PYGZsh{}=guest;@adisconnect *Guest\PYGZsh{}\PYGZsh{}=home;@lock *Guest\PYGZsh{}\PYGZsh{}=*Guest\PYGZsh{}\PYGZsh{};@desc *Guest\PYGZsh{}\PYGZsh{}=A Stranger in a strange land.\PYGZcb{}}
\end{sphinxVerbatim}


\subsection{Dig a closet to store important objects but non\sphinxhyphen{}master room}
\label{\detokenize{gettingstarted:dig-a-closet-to-store-important-objects-but-non-master-room}}
\begin{sphinxadmonition}{note}{Note:}
\sphinxAtStartPar
name it anything you want, just remember it.
\end{sphinxadmonition}

\begin{sphinxVerbatim}[commandchars=\\\{\}]
\PYG{n+nd}{@dig} \PYG{n}{Closet}
\end{sphinxVerbatim}


\subsection{Set the flags on the closet and ownership of it}
\label{\detokenize{gettingstarted:set-the-flags-on-the-closet-and-ownership-of-it}}
\begin{sphinxadmonition}{note}{Note:}
\sphinxAtStartPar
Use the dbref\# that it returned when digging the closet and not \#123
\end{sphinxadmonition}

\begin{sphinxVerbatim}[commandchars=\\\{\}]
\PYG{n+nd}{@set} \PYG{c+c1}{\PYGZsh{}123=inh safe ind float}
\PYG{n+nd}{@chown}\PYG{o}{/}\PYG{n}{pres} \PYG{c+c1}{\PYGZsh{}123=*immholder}
\end{sphinxVerbatim}


\subsection{Create an Admin object for future ease of customization}
\label{\detokenize{gettingstarted:create-an-admin-object-for-future-ease-of-customization}}
\begin{sphinxVerbatim}[commandchars=\\\{\}]
\PYG{n+nd}{@create} \PYG{n}{AdminObject}
\end{sphinxVerbatim}


\subsection{Set the flags on the admin object and ownership and location}
\label{\detokenize{gettingstarted:set-the-flags-on-the-admin-object-and-ownership-and-location}}
\begin{sphinxadmonition}{note}{Note:}
\sphinxAtStartPar
this object must be immortal owned.  Use the dbref\# returned previously instead of \#123
\end{sphinxadmonition}

\begin{sphinxadmonition}{note}{Note:}
\sphinxAtStartPar
Use the closet dbref\# instead of \#234
\end{sphinxadmonition}

\begin{sphinxVerbatim}[commandchars=\\\{\}]
\PYG{n+nd}{@set} \PYG{n}{AdminObject}\PYG{o}{=}\PYG{n}{halt} \PYG{n}{safe} \PYG{n}{ind}
\PYG{n+nd}{@chown}\PYG{o}{/}\PYG{n}{pres} \PYG{c+c1}{\PYGZsh{}123=*immholder}
\PYG{n+nd}{@tel} \PYG{n}{adminobject}\PYG{o}{=}\PYG{c+c1}{\PYGZsh{}234}
\end{sphinxVerbatim}


\subsection{Add admin object to configuration}
\label{\detokenize{gettingstarted:add-admin-object-to-configuration}}
\sphinxAtStartPar
Modify the netrhost.conf file with the following line at the bottom after the line \textquotesingle{}\# define local alises here\textquotesingle{} where you replace 123 with the dbref\# of the admin object that you made:

\begin{sphinxVerbatim}[commandchars=\\\{\}]
\PYG{n}{admin\PYGZus{}object} \PYG{l+m+mi}{123}
\end{sphinxVerbatim}


\subsection{Reboot your mush to load up the change for the admin object}
\label{\detokenize{gettingstarted:reboot-your-mush-to-load-up-the-change-for-the-admin-object}}
\begin{sphinxVerbatim}[commandchars=\\\{\}]
\PYG{n+nd}{@reboot}
\end{sphinxVerbatim}


\subsection{Do @admin/list to see if it shows the admin object}
\label{\detokenize{gettingstarted:do-admin-list-to-see-if-it-shows-the-admin-object}}
\begin{sphinxadmonition}{note}{Note:}
\sphinxAtStartPar
do wizhelp @admin for more information on how to use this
\end{sphinxadmonition}

\begin{sphinxVerbatim}[commandchars=\\\{\}]
\PYG{n+nd}{@admin}\PYG{o}{/}\PYG{n+nb}{list}
\end{sphinxVerbatim}


\subsection{Load in all the various softcode that you want}
\label{\detokenize{gettingstarted:load-in-all-the-various-softcode-that-you-want}}
\sphinxAtStartPar
This is client dependant based on whatever method it uses to load softcode.


\subsubsection{Myrddin MushCron}
\label{\detokenize{gettingstarted:myrddin-mushcron}}
\sphinxAtStartPar
Load in the Myrddin Mush Cron.
It\textquotesingle{}s a very handy piece of software and strongly suggested to do so.  You can find this in the \textquotesingle{}Mushcode\textquotesingle{} directory off the main Rhost directory.
Filename:

\begin{sphinxVerbatim}[commandchars=\\\{\}]
\PYG{o}{\PYGZti{}}\PYG{o}{/}\PYG{n}{Rhost}\PYG{o}{/}\PYG{n}{Mushcode}\PYG{o}{/}\PYG{n}{MyrddinCRON}
\end{sphinxVerbatim}

\begin{sphinxadmonition}{note}{Note:}
\sphinxAtStartPar
The globalroom() function returns the dbref\# of the master room.  Handy!
\end{sphinxadmonition}

\begin{sphinxVerbatim}[commandchars=\\\{\}]
\PYG{n+nd}{@chown}\PYG{o}{/}\PYG{n}{preserve} \PYG{n}{the} \PYG{n}{myrddin} \PYG{n}{mush} \PYG{n}{cron} \PYG{n}{to} \PYG{n}{immholder}\PYG{p}{,} \PYG{n}{then} \PYG{n}{move} \PYG{n}{to} \PYG{n}{maste} \PYG{n}{room}\PYG{o}{.}
\PYG{n+nd}{@chown}\PYG{o}{/}\PYG{n}{pres} \PYG{n}{Myrddin}\PYG{o}{=}\PYG{o}{*}\PYG{n}{Immholder}
\PYG{n+nd}{@tel} \PYG{n}{Myrddin}\PYG{o}{=}\PYG{c+c1}{\PYGZsh{}234 (where \PYGZsh{}234 is the dbref\PYGZsh{} of your code closet)}
\end{sphinxVerbatim}


\subsubsection{AshCom}
\label{\detokenize{gettingstarted:ashcom}}
\sphinxAtStartPar
Load in default softcoded comsystem.  This is PennMUSH and MUX/TM3 compatible.
Filename:

\begin{sphinxVerbatim}[commandchars=\\\{\}]
\PYG{o}{\PYGZti{}}\PYG{o}{/}\PYG{n}{Rhost}\PYG{o}{/}\PYG{n}{Mushcode}\PYG{o}{/}\PYG{n}{comsys}
\end{sphinxVerbatim}

\sphinxAtStartPar
Chown the Comsystem and everything inside it to immholder:

\begin{sphinxVerbatim}[commandchars=\\\{\}]
\PYG{n+nd}{@chown}\PYG{o}{/}\PYG{n}{pres} \PYG{n}{ChanSys}\PYG{o}{=}\PYG{o}{*}\PYG{n}{Immholder}
\PYG{n+nd}{@dolist} \PYG{n}{lcon}\PYG{p}{(}\PYG{n}{chansys}\PYG{p}{)}\PYG{o}{=}\PYG{n+nd}{@chown}\PYG{o}{/}\PYG{n}{pres} \PYG{c+c1}{\PYGZsh{}\PYGZsh{}=*immholder}
\PYG{n+nd}{@tel} \PYG{n}{Chansys}\PYG{o}{=}\PYG{n}{globalroom}\PYG{p}{(}\PYG{p}{)}
\end{sphinxVerbatim}


\subsubsection{Mail Wrappers}
\label{\detokenize{gettingstarted:mail-wrappers}}
\sphinxAtStartPar
Load in mail wrappers if you want MUX/TM3 and/or Penn mail wrapping.
Filename (MUX/TM3):

\begin{sphinxVerbatim}[commandchars=\\\{\}]
\PYG{o}{\PYGZti{}}\PYG{o}{/}\PYG{n}{Rhost}\PYG{o}{/}\PYG{n}{Mushcode}\PYG{o}{/}\PYG{n}{mailwrappers}\PYG{o}{/}\PYG{n}{muxmail}\PYG{o}{.}\PYG{n}{wrap}
\end{sphinxVerbatim}

\sphinxAtStartPar
Filename (Penn):

\begin{sphinxVerbatim}[commandchars=\\\{\}]
\PYG{o}{\PYGZti{}}\PYG{o}{/}\PYG{n}{Rhost}\PYG{o}{/}\PYG{n}{Mushcode}\PYG{o}{/}\PYG{n}{mailwrappers}\PYG{o}{/}\PYG{n}{pennmail}\PYG{o}{.}\PYG{n}{wrap}
\end{sphinxVerbatim}

\sphinxAtStartPar
Chown to immholder:

\begin{sphinxVerbatim}[commandchars=\\\{\}]
\PYG{n+nd}{@chown}\PYG{o}{/}\PYG{n}{pres} \PYG{n}{MUX}\PYG{o}{=}\PYG{o}{*}\PYG{n}{Immholder}
\PYG{n+nd}{@chown}\PYG{o}{/}\PYG{n}{pres} \PYG{n}{Penn}\PYG{o}{=}\PYG{o}{*}\PYG{n}{Immholder}
\PYG{n+nd}{@tel}\PYG{o}{/}\PYG{n+nb}{list} \PYG{n}{mux} \PYG{n}{penn}\PYG{o}{=}\PYG{n}{globalroom}\PYG{p}{(}\PYG{p}{)}
\end{sphinxVerbatim}


\subsubsection{Myrddin BBS}
\label{\detokenize{gettingstarted:myrddin-bbs}}
\sphinxAtStartPar
Load in Myrddin\textquotesingle{}s BBS
Filename:

\begin{sphinxVerbatim}[commandchars=\\\{\}]
\PYG{o}{\PYGZti{}}\PYG{o}{/}\PYG{n}{Rhost}\PYG{o}{/}\PYG{n}{Mushcode}\PYG{o}{/}\PYG{n}{MyrddinBBS}
\end{sphinxVerbatim}

\sphinxAtStartPar
Chown to immholder and the contents of it as well:

\begin{sphinxVerbatim}[commandchars=\\\{\}]
\PYG{n+nd}{@chown}\PYG{o}{/}\PYG{n}{pres} \PYG{n}{Myrddin}\PYG{o}{=}\PYG{o}{*}\PYG{n}{Immholder}
\PYG{n+nd}{@dolist} \PYG{n}{lcon}\PYG{p}{(}\PYG{n}{myrddin}\PYG{p}{)}\PYG{o}{=}\PYG{n+nd}{@chown}\PYG{o}{/}\PYG{n}{pres} \PYG{c+c1}{\PYGZsh{}\PYGZsh{}=*immholder}
\PYG{n+nd}{@tel} \PYG{n}{myrddin}\PYG{o}{=}\PYG{n}{globalroom}\PYG{p}{(}\PYG{p}{)}
\end{sphinxVerbatim}


\subsubsection{Other Mushcode}
\label{\detokenize{gettingstarted:other-mushcode}}
\sphinxAtStartPar
There\textquotesingle{}s other code in the Mushcode directory that you are welcome to install.  You would follow similar procedures
for loading it in, chowning it, and moving to the master room as you did above.

\sphinxAtStartPar
Likewise, any softcode you find on the internet or on other mushes should be portable to RhostMUSH with little to
no changes depending on the complexity of the code in question.


\section{Minimal DB instructions}
\label{\detokenize{gettingstarted:minimal-db-instructions}}
\sphinxAtStartPar
The \textquotesingle{}retired\textquotesingle{} directory has older image files if you\textquotesingle{}re interested

\sphinxAtStartPar
Please use the netrhost.conf file with the database as they\textquotesingle{}re linked.

\sphinxAtStartPar
The flatfile must be loaded in as a new db

\sphinxAtStartPar
This is a minimal db with basic \textquotesingle{}features\textquotesingle{} built in.

\sphinxAtStartPar
Copy the txt files into the Rhost\textquotesingle{}s txt directory off game:

\begin{sphinxVerbatim}[commandchars=\\\{\}]
\PYG{n}{cp} \PYG{n}{txt}\PYG{o}{/}\PYG{o}{*} \PYG{o}{\PYGZti{}}\PYG{o}{/}\PYG{n}{Rhost}\PYG{o}{/}\PYG{n}{Server}\PYG{o}{/}\PYG{n}{game}\PYG{o}{/}\PYG{n}{txt}
\end{sphinxVerbatim}

\sphinxAtStartPar
mkindx the files (substitute FILENAME with the filename):

\begin{sphinxVerbatim}[commandchars=\\\{\}]
\PYG{n}{cd} \PYG{o}{\PYGZti{}}\PYG{o}{/}\PYG{n}{Rhost}\PYG{o}{/}\PYG{n}{Server}\PYG{o}{/}\PYG{n}{game}\PYG{o}{/}\PYG{n}{txt}
\PYG{o}{.}\PYG{o}{.}\PYG{o}{/}\PYG{n}{mkindx} \PYG{n}{FILENAME}\PYG{o}{.}\PYG{n}{txt} \PYG{n}{FILENAME}\PYG{o}{.}\PYG{n}{indx}
\end{sphinxVerbatim}


\subsection{Startup Steps}
\label{\detokenize{gettingstarted:startup-steps}}\begin{enumerate}
\sphinxsetlistlabels{\arabic}{enumi}{enumii}{}{.}%
\item {} 
\sphinxAtStartPar
Using the Startmush utility for the first time, select the load db method

\end{enumerate}

\sphinxAtStartPar
\sphinxhyphen{}\sphinxhyphen{}\sphinxhyphen{} or \sphinxhyphen{}\sphinxhyphen{}\sphinxhyphen{}
\begin{enumerate}
\sphinxsetlistlabels{\arabic}{enumi}{enumii}{}{.}%
\item {} 
\sphinxAtStartPar
copy the netrhost.conf file into the games directory

\item {} 
\sphinxAtStartPar
make any relevant changes you wish

\item {} 
\sphinxAtStartPar
db\_load the flatfile
\begin{enumerate}
\sphinxsetlistlabels{\arabic}{enumii}{enumiii}{}{.}%
\item {} 
\sphinxAtStartPar
go into the game directory

\item {} 
\sphinxAtStartPar
type:

\begin{sphinxVerbatim}[commandchars=\\\{\}]
\PYG{o}{.}\PYG{o}{/}\PYG{n}{db\PYGZus{}load} \PYG{n}{data}\PYG{o}{/}\PYG{n}{netrhost}\PYG{o}{.}\PYG{n}{gdbm} \PYG{o}{.}\PYG{o}{.}\PYG{o}{/}\PYG{n}{minimal}\PYG{o}{\PYGZhy{}}\PYG{n}{DBs}\PYG{o}{/}\PYG{n}{minimal\PYGZus{}db}\PYG{o}{/}\PYG{n}{netrhost}\PYG{o}{.}\PYG{n}{db}\PYG{o}{.}\PYG{n}{flat} \PYG{n}{data}\PYG{o}{/}\PYG{n}{netrhost}\PYG{o}{.}\PYG{n}{db}\PYG{o}{.}\PYG{n}{new}
\end{sphinxVerbatim}

\end{enumerate}

\item {} 
\sphinxAtStartPar
Startmush as expected

\end{enumerate}


\chapter{Ambrosia\textquotesingle{}s Minimal Rhost DB}
\label{\detokenize{ambrosiadb:ambrosia-s-minimal-rhost-db}}\label{\detokenize{ambrosiadb:ambrosiadb-installation}}\label{\detokenize{ambrosiadb::doc}}
\sphinxAtStartPar
Version: 1.0.5          2020\sphinxhyphen{}01\sphinxhyphen{}31


\section{Version history}
\label{\detokenize{ambrosiadb:version-history}}\begin{quote}
\begin{description}
\item[{1.0.0}] \leavevmode\begin{itemize}
\item {} 
\sphinxAtStartPar
Initial database setup.

\end{itemize}

\item[{1.0.1}] \leavevmode\begin{itemize}
\item {} 
\sphinxAtStartPar
Small fixes of objid(), isstaff() and bccheck() permissions and handling.

\item {} 
\sphinxAtStartPar
bittype() access lowered to Architect level

\item {} 
\sphinxAtStartPar
NO\_CODE flag made visual to Architect

\end{itemize}

\item[{1.0.2}] \leavevmode\begin{itemize}
\item {} 
\sphinxAtStartPar
Several convenience changes and fixes: \_ Attributes moved to @aflags
system, allowing Architects to set, Guildmasters to see them.

\item {} 
\sphinxAtStartPar
@flagdef lowered to Royalty level. @quota/max, @quota/unlock and @convert
moved to Architect level.

\item {} 
\sphinxAtStartPar
NO\_CODE flag made settable/unsettable by Architects.

\item {} 
\sphinxAtStartPar
Fixed typo in conf file: ifselse \sphinxhyphen{}\textgreater{} ifelse

\item {} 
\sphinxAtStartPar
Switched \_Attributes to use the @aflags system
See: Guildmaster
Set: Architect

\end{itemize}

\item[{1.0.3}] \leavevmode\begin{itemize}
\item {} 
\sphinxAtStartPar
Removed @flagdefs from in\sphinxhyphen{}game softcode, converted to flag\_access\_*
config options

\item {} 
\sphinxAtStartPar
Lowered mailstatus() access to architect.

\end{itemize}

\item[{1.0.4}] \leavevmode\begin{itemize}
\item {} 
\sphinxAtStartPar
Changed softcoded objid() to tag(), due to Rhost\textquotesingle{}s new hardcoded
objid() which does perform a different functionality.

\item {} 
\sphinxAtStartPar
Added more staff recommendations to this file.

\item {} 
\sphinxAtStartPar
Added Reality TXLevel \textquotesingle{}Admin\textquotesingle{} to all objects in the db except \#1

\end{itemize}

\item[{1.0.5}] \leavevmode\begin{itemize}
\item {} 
\sphinxAtStartPar
Replaced softcoded tag() system with Rhost\textquotesingle{}s new hardcoded @tag/tag()
system. All previous tags are set on the database. The Tag Object
was removed.

\item {} 
\sphinxAtStartPar
@function startup on BC\sphinxhyphen{}Admin\sphinxhyphen{}Royalty fixed \sphinxhyphen{} @wait 1 workaround for
Tags in place.

\item {} 
\sphinxAtStartPar
Places System @startup integrated into BC\sphinxhyphen{}Admin\sphinxhyphen{}Royalty\textquotesingle{}s @startup

\item {} 
\sphinxAtStartPar
Made @dump and @dump/flat available to Councilors in netrhost.conf

\end{itemize}

\item[{1.0.6}] \leavevmode\begin{itemize}
\item {} 
\sphinxAtStartPar
A small typo fix in netrhost.conf. float\_preciiosn \sphinxhyphen{}\textgreater{} precision and
functions\_max \sphinxhyphen{}\textgreater{} function\_max. Thanks to \sphinxhref{mailto:Bobbi@COH}{Bobbi@COH}

\end{itemize}

\end{description}
\end{quote}


\section{AmbrosiaDB Introduction}
\label{\detokenize{ambrosiadb:ambrosiadb-introduction}}
\begin{sphinxadmonition}{note}{Note:}
\sphinxAtStartPar
READ THIS DOCUMENT CAREFULLY!
\end{sphinxadmonition}

\sphinxAtStartPar
Greetings,

\sphinxAtStartPar
This minimal Rhost DB was made with a secure setup, and as a good base to start
a new game off in mind.


\section{AmbrosiaDB Features}
\label{\detokenize{ambrosiadb:ambrosiadb-features}}

\subsection{AmbrosiaDB Configuration}
\label{\detokenize{ambrosiadb:ambrosiadb-configuration}}\begin{itemize}
\item {} 
\sphinxAtStartPar
Limbo, Master Room and Auxiliary room.

\item {} 
\sphinxAtStartPar
BC\sphinxhyphen{}Admin\sphinxhyphen{}\textless{}bitlevel\textgreater{} characters set up for each bitlevel to own global and
data objects, and inherit to.

\item {} 
\sphinxAtStartPar
BC\sphinxhyphen{}Admin\sphinxhyphen{}Mortal is @powered EXAMINE\_ALL(Guildmaster), NOFORCE(Architect) and
LONG\_FINGERS.

\item {} 
\sphinxAtStartPar
@startup on BC\sphinxhyphen{}Admin\sphinxhyphen{}Immortal lowers DARK flag access to Councilor level, and
NO\_CODE visual access to Architect level.

\item {} 
\sphinxAtStartPar
Global Command objects inheriting from each bitlevel, with a separate staff\sphinxhyphen{}only object for each level.

\item {} 
\sphinxAtStartPar
Global Function objects inheriting from each bitlevel.

\item {} 
\sphinxAtStartPar
local Function objects inheriting from each bitlevel.

\item {} 
\sphinxAtStartPar
@function and @hook access lowered to Royalty level to remove immediate need
for Immortals or actual Immortal code.

\item {} 
\sphinxAtStartPar
@rxlevel, @txlevel, bittype() access lowered to Architect level to remove
immediate need for Royalty in many cases.

\item {} 
\sphinxAtStartPar
@startup on BC\sphinxhyphen{}Admin\sphinxhyphen{}Royalty to automatically load @hooks and @functions from
the Global Function objects, based on attribute naming.

\item {} 
\sphinxAtStartPar
Misc Data object to hold general data, like Staff lists etc.

\item {} 
\sphinxAtStartPar
Reality levels \textquotesingle{}Real\textquotesingle{} and \textquotesingle{}Admin\textquotesingle{}.

\end{itemize}

\begin{sphinxadmonition}{note}{Note:}
\sphinxAtStartPar
All created items and players by default are in Recieve\sphinxhyphen{}Level \textquotesingle{}Real\textquotesingle{} and
Transmit\sphinxhyphen{}Levels \textquotesingle{}Real\textquotesingle{} and \textquotesingle{}Admin\textquotesingle{}.
\end{sphinxadmonition}
\begin{itemize}
\item {} 
\sphinxAtStartPar
All globals, Master Room, BCs\sphinxhyphen{}*, and other staff/code\sphinxhyphen{}related objects
currently have only \textquotesingle{}Admin\textquotesingle{} as their Transmit\sphinxhyphen{}Level. This does not prevent
them fromi working properly. The only exception is \#1, who has empty reality
levels.

\item {} 
\sphinxAtStartPar
The supplied netrhost.conf offers a secure setup of options, allows Royalty
to use @hook and @function, and also sets the function\_access of several
functions to !no\_code, which allows NO\_CODE players to use the comsys
properly.

\end{itemize}

\begin{sphinxadmonition}{note}{Note:}
\sphinxAtStartPar
IT IS HIGHLY RECOMMENDED to use this .conf as a base for this DB.
The \textquotesingle{}Port\textquotesingle{} configuration parameter is XXXX\textquotesingle{}d out. Set it first before starting
your game.
\end{sphinxadmonition}
\begin{itemize}
\item {} 
\sphinxAtStartPar
All existing objects have been @set SAFE and INDESTRUCTABLE.

\item {} 
\sphinxAtStartPar
All existing objects have a paranoid series of @locks set on themselves.

\item {} 
\sphinxAtStartPar
The +supersafe command is provided on \#1 as an example of what was used to
easily set this on objects.

\item {} 
\sphinxAtStartPar
Players are @set NO\_CODE and WANDERER by default. They get 40 credits on
creation, and a 1\sphinxhyphen{}credit\sphinxhyphen{}per\sphinxhyphen{}day paycheck.

\item {} 
\sphinxAtStartPar
All *mit sideeffects, as well as set(), create() and list() are enabled. The
latter three are necessary for the Comsystem. The rest of sideeffects are
disabled completely.

\item {} 
\sphinxAtStartPar
Flashing ansi is disabled.

\item {} 
\sphinxAtStartPar
\_Attributes are settable by Architects, and seeable by Guildmasters. Read:
Still invisible and unsettable by mortals.

\item {} 
\sphinxAtStartPar
Architects can set up, unlock, and change alternate @quota on players.

\item {} 
\sphinxAtStartPar
Architects can set/unset the NO\_CODE flag.

\item {} 
\sphinxAtStartPar
Guildmasters can see \_Attributes

\item {} 
\sphinxAtStartPar
Architects can set \_Attributes

\end{itemize}


\subsection{AmbrosiaDB Softcode}
\label{\detokenize{ambrosiadb:ambrosiadb-softcode}}\begin{itemize}
\item {} 
\sphinxAtStartPar
Set\sphinxhyphen{}up compatibility SoftFunctions and @hook object.

\item {} 
\sphinxAtStartPar
Set up Comsys 1.0.9b at Architect level. (+bbhelp command)

\item {} 
\sphinxAtStartPar
Set up Myrddin +bboard at Architect level.

\item {} 
\sphinxAtStartPar
Set up Myrddon Cron at Architect level.

\item {} 
\sphinxAtStartPar
Anomaly Jobs (+jhelp)

\item {} 
\sphinxAtStartPar
SGP Places \& Mutter

\item {} 
\sphinxAtStartPar
Set up Penn\sphinxhyphen{}style follow.

\item {} 
\sphinxAtStartPar
Set up @scan (Up to architect\sphinxhyphen{}level items).

\item {} 
\sphinxAtStartPar
Set up Penn\sphinxhyphen{} and Mux compatibility Mailwrappers. (phelp and mhelp commands)

\item {} 
\sphinxAtStartPar
help .txt files and .indx files for the above.

\item {} 
\sphinxAtStartPar
@dynhelp access lowered to architect to call above helpfiles.

\end{itemize}


\subsection{AmbrosiaDB Functions}
\label{\detokenize{ambrosiadb:ambrosiadb-functions}}\begin{itemize}
\item {} 
\sphinxAtStartPar
isstaff() \sphinxhyphen{} Softcoded function that returns \textquotesingle{}1\textquotesingle{} if its argument matches
a \#dbref in the \textquotesingle{}isstaff\textquotesingle{} attribute on the Misc Data object.

\item {} 
\sphinxAtStartPar
bccheck() \sphinxhyphen{} Softcoded function that returns \textquotesingle{}1\textquotesingle{} if its argument matches
a \#dbref in the \textquotesingle{}bcs\textquotesingle{} attribute on the Misc Data object.

\item {} 
\sphinxAtStartPar
width() \sphinxhyphen{} Softcoded function that returns 78 for now. For cross\sphinxhyphen{}MU*
compatibility.

\item {} 
\sphinxAtStartPar
pass() \sphinxhyphen{} Softcoded function that takes a number as its parameter, and return
a random string of that length. Perfect for setting random passwords.

\item {} 
\sphinxAtStartPar
cmdmessage() \sphinxhyphen{} Softcoded function that takes two strings as its parameters.
Returns \textquotesingle{}\textless{}\textless{} STRING1 \textgreater{}\textgreater{} String2\textquotesingle{}. The \textless{}\textless{}\textgreater{}\textgreater{} part is highlighted red. Good for
all kinds of messages sent by game commands.

\item {} 
\sphinxAtStartPar
header() \sphinxhyphen{} Highly versatile, and a buffer\sphinxhyphen{}saving alternative
to using printf() for centering with ansi borders. HIGHLY recommended to use
instead of printf() for such things.

\end{itemize}

\begin{sphinxVerbatim}[commandchars=\\\{\}]
\PYG{n}{header}\PYG{p}{(}\PYG{n}{text}\PYG{p}{,}\PYG{n}{width}\PYG{p}{,}\PYG{n}{filler}\PYG{p}{,}\PYG{n}{fillercolor}\PYG{p}{,}\PYG{n}{outerpadding}\PYG{p}{,}\PYG{n}{paddingcolor}\PYG{p}{,}
     \PYG{n}{leftinnerpadding}\PYG{p}{,}\PYG{n}{leftinnercolor}\PYG{p}{,}\PYG{n}{rightinnerpadding}\PYG{p}{,}\PYG{n}{rightinnercolor}\PYG{p}{)}
\PYG{n}{text} \PYG{o}{\PYGZhy{}} \PYG{n}{Text} \PYG{n}{to} \PYG{n}{center}
\PYG{n}{width} \PYG{o}{\PYGZhy{}} \PYG{n}{Width} \PYG{n}{of} \PYG{n}{the} \PYG{n}{header}\PYG{p}{,} \PYG{n}{defaults} \PYG{n}{to} \PYG{n}{width}\PYG{p}{(}\PYG{p}{)}
\PYG{n}{filler} \PYG{o}{\PYGZhy{}} \PYG{n}{Character}\PYG{p}{(}\PYG{n}{s}\PYG{p}{)} \PYG{n}{to} \PYG{n}{draw} \PYG{n}{the} \PYG{n}{line} \PYG{k}{with}\PYG{o}{.} \PYG{n}{Default}\PYG{p}{:} \PYG{o}{=}
\PYG{n}{fillercolor} \PYG{o}{\PYGZhy{}} \PYG{n}{ansicode} \PYG{n}{to} \PYG{n}{color} \PYG{n}{the} \PYG{n}{line} \PYG{k}{with}
\PYG{n}{outerpadding} \PYG{o}{\PYGZhy{}} \PYG{n}{Characters} \PYG{n}{to} \PYG{n}{frame} \PYG{n}{the} \PYG{n}{outer} \PYG{n}{ends} \PYG{n}{of} \PYG{n}{the} \PYG{n}{line} \PYG{k}{with}\PYG{o}{.}
\PYG{n}{paddingcolor} \PYG{o}{\PYGZhy{}} \PYG{n}{ansicode} \PYG{n}{to} \PYG{n}{color} \PYG{n}{the} \PYG{n}{outer} \PYG{n}{characters} \PYG{k}{with}
\PYG{n}{leftinnerpadding} \PYG{o}{\PYGZhy{}} \PYG{n}{characters} \PYG{n}{to} \PYG{n}{put} \PYG{n}{at} \PYG{n}{the} \PYG{n}{left} \PYG{n}{side} \PYG{n}{of} \PYG{o}{\PYGZlt{}}\PYG{n}{text}\PYG{o}{\PYGZgt{}}
\PYG{n}{leftinnercolor} \PYG{o}{\PYGZhy{}} \PYG{n}{ansicode} \PYG{n}{to} \PYG{n}{color} \PYG{n}{the} \PYG{n}{leftside} \PYG{n}{characters} \PYG{k}{with}
\PYG{n}{rightinnerpadding} \PYG{o}{\PYGZhy{}} \PYG{n}{characters} \PYG{n}{to} \PYG{n}{put} \PYG{n}{at} \PYG{n}{the} \PYG{n}{right} \PYG{n}{side} \PYG{n}{of} \PYG{o}{\PYGZlt{}}\PYG{n}{text}\PYG{o}{\PYGZgt{}}
\PYG{n}{rightinnercolor} \PYG{o}{\PYGZhy{}} \PYG{n}{ansicode} \PYG{n}{to} \PYG{n}{color} \PYG{n}{the} \PYG{n}{rightside} \PYG{n}{characters} \PYG{k}{with}
\end{sphinxVerbatim}

\begin{sphinxadmonition}{note}{Note:}
\sphinxAtStartPar
ALL of header()\textquotesingle{}s parameters are optional. By default, header() simply draws
a 78\sphinxhyphen{}char wide line of =\textquotesingle{}s. Simply leave parameters empty if you want to set
one of the latter parameters.
\end{sphinxadmonition}


\section{AmbrosiaDB Bitlevels}
\label{\detokenize{ambrosiadb:ambrosiadb-bitlevels}}
\sphinxAtStartPar
The whole DB is highly geared for a low\sphinxhyphen{}bitlevel setup.
I am a strong believer in least\sphinxhyphen{}privileges\sphinxhyphen{}needed to do the job. Bittypes and
powers are tools to do that job, not badges of friendship or trust that get
tossed about.

\begin{sphinxadmonition}{note}{Note:}
\sphinxAtStartPar
Here is my suggested list of powers and bittypes for staffers.
\end{sphinxadmonition}


\subsection{AmbrosiaDB Storytellers}
\label{\detokenize{ambrosiadb:ambrosiadb-storytellers}}
\sphinxAtStartPar
@powered TEL\_ANYWHERE, JOIN\_PLAYER and GRAB\_PLAYER on Guildmaster level.


\subsection{AmbrosiaDB Builder\sphinxhyphen{}BCs}
\label{\detokenize{ambrosiadb:ambrosiadb-builder-bcs}}
\sphinxAtStartPar
Mortal, with @quota and money for their job. There should be one
shared BC for each area of the game, like BC\sphinxhyphen{}Houston. Special
Rooms, items or exits that require privilegs to examine or @tel
a player should belong to a BC\sphinxhyphen{}Houston\sphinxhyphen{}Powered that is @powered
EXAMINE\_ALL, LONG\_FINGERS And TEL\_ANYTHING on Guildmaster level.
If the object actually needs to modify a player directly, have
it use a restricted staff Global, or if you absolutely must,
make a BC\sphinxhyphen{}Houston\sphinxhyphen{}Admin and @set it Architect. Do not give
normal builders access to it, only @chown things to it and @set
them inherit after review.

\sphinxAtStartPar
Both the \sphinxhyphen{}powered and \sphinxhyphen{}admin BCs can have random passwords and
be @set NO\_CONNECT.


\subsection{AmbrosiaDB Building Head}
\label{\detokenize{ambrosiadb:ambrosiadb-building-head}}
\sphinxAtStartPar
@set Guildmaster, powered TEL\_ANYWHERE, TEL\_ANYTHING and
optionally CHOWN\_OTHER on Guildmaster level. Mind that the
latter technically allows them to @chown anything guildmaster\sphinxhyphen{}
and lower\sphinxhyphen{}owned in the master and auxiliary rooms. However,
it allows the Building Head to @chown items between BCs\sphinxhyphen{} and
to the BC\sphinxhyphen{}\textless{}location\textgreater{}\sphinxhyphen{}powered.


\subsection{AmbrosiaDB Enforcers}
\label{\detokenize{ambrosiadb:ambrosiadb-enforcers}}
\sphinxAtStartPar
As Storyteller above, plus being @powered Security at
Guildmaster level, in order to handle problem players.

\sphinxAtStartPar
Optionally always given to Storytellers.


\subsection{AmbrosiaDB Coders}
\label{\detokenize{ambrosiadb:ambrosiadb-coders}}
\sphinxAtStartPar
@set Architect


\subsection{AmbrosiaDB Head Coder}
\label{\detokenize{ambrosiadb:ambrosiadb-head-coder}}
\sphinxAtStartPar
Always trust your head coder.
@set Architect for the everyday bit. Give access to a
maintenance Councilor bit for special code projects. Finished
code along those lines should get @chowned to the
bc\sphinxhyphen{}admin\sphinxhyphen{}councilor.

\sphinxAtStartPar
If you as the MU* Head(s) don\textquotesingle{}t know Softcode well, or want to
leave anything Code to your Head Coder, also give them optional
access to a maintenance Royalty bit in order to properly code
banning/blacklisting +commands and other rare code that requires
Royalty powers. Again, chown finished code to bc\sphinxhyphen{}admin\sphinxhyphen{}royalty.


\subsection{AmbrosiaDB MU* Head(s)}
\label{\detokenize{ambrosiadb:ambrosiadb-mu-head-s}}
\sphinxAtStartPar
You\textquotesingle{}re the boss(es). But please use an Architect bit for your
everyday things. Keep Immortal to yourself. Keep \#1 to yourself.
And seriously avoid using either of them except for creating
a Royalty bit or doing intricate DB maintenance.


\subsection{AmbrosiaDB Site Admins}
\label{\detokenize{ambrosiadb:ambrosiadb-site-admins}}
\sphinxAtStartPar
They already have more powers than any in\sphinxhyphen{}game bit can ever
have ;)

\sphinxAtStartPar
Depending on actual involvement with your game, their abilities
in\sphinxhyphen{}game can range from merely being @powered free\_wall for
notifying players of downtimes and/or being @powered shutdown in
order to shut down the game for maintenance, up to being the
only person with actual access to \#1.


\section{AmbrosiaDB Globals}
\label{\detokenize{ambrosiadb:ambrosiadb-globals}}
\sphinxAtStartPar
Handle necessary functionality for adminning through the admin\sphinxhyphen{}only globals and
softcode.

\sphinxAtStartPar
The setup I personally suggest is to have ALL staffers be AT MAX Architect\sphinxhyphen{}level
for everyday work and communication, with special coders that \sphinxhyphen{}really\sphinxhyphen{} require
it having Councilor\sphinxhyphen{}characters available to log into for maintenance or special
code setup. Technically if everything is done right, the Coder(s) of the game do
not require higher privileges than Architect for the vast majority of things.
Royalty\sphinxhyphen{}level code should be a rare exception, if at all necessary. The MU*
Head(s) or site\sphinxhyphen{}admin should be the only one having access to \#1, Immortals or
perhaps even Royalty. The BC\textquotesingle{}s, Global Function objects and Function objects at
level Royalty and higher have simply been provided as a if\sphinxhyphen{}necessary convenience.

\sphinxAtStartPar
Current objects are only @chowned to certain bitlevels if it is really required
for them to function. Whenever possible, they have been @chowned to the mortal
BC\sphinxhyphen{}Admin\sphinxhyphen{}Mortal. All custom global functions listed above are on the semi\sphinxhyphen{}
\sphinxhyphen{}mortal Global Functions object. The Master Room and Auxiliary Room have been
@chowned to BC\sphinxhyphen{}Admin\sphinxhyphen{}Architect.

\sphinxAtStartPar
The Comsystem and +bboard are owned by BC\sphinxhyphen{}Admin\sphinxhyphen{}Architect, which means that
higher bitlevels might not be able to use those systems if they hide and set
themselves DARK. This is intentional. The Architect bitlevel is enough to freely
set attributes on players, so these systems did not need anything higher, and
it prioritizes Councilor+ as mere mainenance\sphinxhyphen{}duty bitlevels. Even the MU* Head
should log on as an Architect for everyday things.

\sphinxAtStartPar
The Comsystem and BBOARD have been modified to be configurable by Architect and
higher. Both systems have a CANUSE attribute with the according code on them.
Note that if you want both systems to be only configurable by Councilor+, that
instead of @chowning them to a Councilor after changing those attribute for
Councilor or higher, I suggest to simply @set the bboard and comsystem core
objects NO\_MODIFY instead, keeping them at Architect\sphinxhyphen{}power but making them
unmodifyable by Architects.


\section{AmbrosiaDB Quota}
\label{\detokenize{ambrosiadb:ambrosiadb-quota}}
\sphinxAtStartPar
I highly recommend the use of the alternative @quota system. BC\sphinxhyphen{}Admin\sphinxhyphen{}Mortal
and BC\sphinxhyphen{}Admin\sphinxhyphen{}Guildmaster both have this @quota system set up on themselves. Both
of them have a high amount of money for everyday operations. You should not give
them free quota or money.


\section{AmbrosiaDB Adding Functions}
\label{\detokenize{ambrosiadb:ambrosiadb-adding-functions}}
\sphinxAtStartPar
I also recommend to setup most global functions with /Privileged even if they
are mortal\sphinxhyphen{}powered, to make them work even when players are set NO\_CODE and
WANDERER by default.

\sphinxAtStartPar
Enjoy!


\section{AmbrosiaDB Compiling}
\label{\detokenize{ambrosiadb:ambrosiadb-compiling}}
\sphinxAtStartPar
P.S. the \textquotesingle{}bin/asksource.save0\textquotesingle{} file has been supplied for loading in the
\textquotesingle{}make config\textquotesingle{} or \textquotesingle{}make confsource\textquotesingle{} step of Rhost installation. It provides the
settings I have used when creating this database. Some settings, like the ANSI
substitution, are used in the DB.

\sphinxAtStartPar
\sphinxhref{mailto:--Ambrosia@RhostMUSH}{\sphinxhyphen{}\sphinxhyphen{}Ambrosia@RhostMUSH}


\chapter{Security}
\label{\detokenize{security:security}}\label{\detokenize{security::doc}}

\section{Considerations to locking down restrictions in RhostMUSH}
\label{\detokenize{security:considerations-to-locking-down-restrictions-in-rhostmush}}
\sphinxAtStartPar
Sometimes, you want to have things run at various privilage levels and do not
want to have things with too much access.  Weither that is online objects or
players you want to block from connecting to your mush.  Here\textquotesingle{}s things you can
do.

\sphinxAtStartPar
One thing to keep in mind is that RhostMUSH, unlike PennMUSH is not flag
dependant on permission level, it\textquotesingle{}s ownership based.  While setting a wizard
flag on an object would work, it\textquotesingle{}s not recommended and it is instead recommended
to chown the object in question to a wizard (like your wizard holder character)
Then the object must be set inherit to actually inherit the wizard.

\begin{sphinxadmonition}{note}{Note:}
\sphinxAtStartPar
inherit is required to inherit anything from the player.  Flags, powers,
toggles.  The only thing that is inherited automatically is depowers.
\end{sphinxadmonition}


\subsection{Online: Blocking object abilities}
\label{\detokenize{security:online-blocking-object-abilities}}
\sphinxAtStartPar
We have various flag levels.  It is strongly recommanded you check wizhelp
on \textquotesingle{}control\textquotesingle{} for a detailed overview of what each bitlevel can or can not do
prior to giving the ownership to the object.  Things useful for tweaking control
on players and objects.


\subsubsection{FLAGS (access with @set)}
\label{\detokenize{security:flags-access-with-set}}
\begin{sphinxVerbatim}[commandchars=\\\{\}]
\PYG{n}{IMMORTAL}\PYG{p}{,} \PYG{n}{ROYALTY}\PYG{p}{,} \PYG{n}{COUNCILOR}\PYG{p}{,} \PYG{n}{ARCHITECT}\PYG{p}{,} \PYG{n}{GUILDMASTER}\PYG{p}{,}
\PYG{n}{FUBAR}\PYG{p}{,} \PYG{n}{SLAVE}\PYG{p}{,} \PYG{n}{SIDEFX}\PYG{p}{,} \PYG{n}{NO\PYGZus{}CONNECT}\PYG{p}{,} \PYG{n}{WANDERER}\PYG{p}{,} \PYG{n}{SAFE}\PYG{p}{,}
\PYG{n}{AUDITORIUM}\PYG{p}{,} \PYG{n}{BACKSTAGE}\PYG{p}{,} \PYG{n}{NOBACKSTAGE}\PYG{p}{,} \PYG{n}{INDESTRUCTIBLE}\PYG{p}{,}
\PYG{n}{INHERIT}\PYG{p}{,} \PYG{n}{JUMP\PYGZus{}OK}\PYG{p}{,} \PYG{n}{NO\PYGZus{}TEL}\PYG{p}{,} \PYG{n}{NO\PYGZus{}WALL}\PYG{p}{,} \PYG{n}{NO\PYGZus{}EXAMINE}\PYG{p}{,}
\PYG{n}{NO\PYGZus{}MODIFY}\PYG{p}{,} \PYG{n}{NO\PYGZus{}CONNECT}\PYG{p}{,} \PYG{n}{NO\PYGZus{}POSSESS}\PYG{p}{,} \PYG{n}{NO\PYGZus{}PESTER}\PYG{p}{,}
\PYG{n}{NO\PYGZus{}OVERRIDE}\PYG{p}{,} \PYG{n}{NO\PYGZus{}USELOCK}\PYG{p}{,} \PYG{n}{NO\PYGZus{}MOVE}\PYG{p}{,} \PYG{n}{NO\PYGZus{}YELL}\PYG{p}{,} \PYG{n}{CLOAK}\PYG{p}{,}
\PYG{n}{SCLOAK}\PYG{p}{,} \PYG{n}{DARK}\PYG{p}{,} \PYG{n}{UNFINDABLE}\PYG{p}{,} \PYG{n}{SEE\PYGZus{}OEMIT}\PYG{p}{,} \PYG{n}{TELOK}\PYG{p}{,} \PYG{n}{SUSPECT}\PYG{p}{,}
\PYG{n}{SPAMMONITOR}
\end{sphinxVerbatim}


\subsubsection{TOGGLES (@toggle)}
\label{\detokenize{security:toggles-toggle}}
\begin{sphinxVerbatim}[commandchars=\\\{\}]
\PYG{n}{BRANDY\PYGZus{}MAIL}\PYG{p}{,} \PYG{n}{PENN\PYGZus{}MAIL}\PYG{p}{,} \PYG{n}{MUXPAGE}\PYG{p}{,} \PYG{n}{VPAGE}\PYG{p}{,} \PYG{n}{NOISY}\PYG{p}{,}
\PYG{n}{MONITOR}\PYG{o}{*} \PYG{p}{(}\PYG{n+nb}{all} \PYG{n}{monitor} \PYG{n}{toggles}\PYG{p}{)}\PYG{p}{,} \PYG{n}{MORTALREALITY}\PYG{p}{,}
\PYG{n}{NODEFAULT}\PYG{p}{,} \PYG{n}{NO\PYGZus{}FORMAT}\PYG{p}{,} \PYG{n}{PAGELOCK}\PYG{p}{,} \PYG{n}{SNUFFDARK}\PYG{p}{,} \PYG{n}{VARIABLE}
\end{sphinxVerbatim}


\subsubsection{@powers, @depowers, and @locks}
\label{\detokenize{security:powers-depowers-and-locks}}
\sphinxAtStartPar
Please review help (and wizhelp) for each of these items on how it can affect
a player, thing, exit, or room.  The help is quite verbose.


\subsection{Offline: Blocking twinks from being abusive on your game}
\label{\detokenize{security:offline-blocking-twinks-from-being-abusive-on-your-game}}
\begin{sphinxVerbatim}[commandchars=\\\{\}]
\PYG{n}{FLAGS}                   \PYG{p}{:} \PYG{n}{FUBAR}\PYG{p}{,} \PYG{n}{SLAVE}\PYG{p}{,} \PYG{n}{NO\PYGZus{}CONNECT}
\PYG{n}{Commands}\PYG{p}{:}               \PYG{p}{:} \PYG{n+nd}{@boot}\PYG{p}{,} \PYG{n+nd}{@nuke}\PYG{p}{,} \PYG{n+nd}{@toad}\PYG{p}{,} \PYG{n+nd}{@turtle}
\PYG{n}{Sitelocks}\PYG{p}{:} \PYG{p}{(}\PYG{n+nd}{@admin}\PYG{p}{)}     \PYG{p}{:} \PYG{n}{forbid\PYGZus{}host}\PYG{p}{,} \PYG{n}{forbid\PYGZus{}site}\PYG{p}{,} \PYG{n}{register\PYGZus{}host}\PYG{p}{,} \PYG{n}{register\PYGZus{}site}\PYG{p}{,} \PYG{n}{noguest\PYGZus{}host}\PYG{p}{,} \PYG{n}{noguest\PYGZus{}site}
\PYG{n}{Sitelock} \PYG{n}{by} \PYG{n}{player}      \PYG{p}{:} \PYG{n+nd}{@badsite}\PYG{p}{,} \PYG{n+nd}{@goodsite}\PYG{p}{,} \PYG{n}{NO\PYGZus{}CONNECT} \PYG{p}{(}\PYG{n}{flag}\PYG{p}{)}
\PYG{n}{Monitoring} \PYG{n}{player}       \PYG{p}{:} \PYG{n}{SUSPECT} \PYG{p}{(}\PYG{n}{flag}\PYG{p}{)}\PYG{p}{,} \PYG{n+nd}{@snoop}
\PYG{n}{TOR}\PYG{o}{/}\PYG{n}{Proxy} \PYG{n}{blocking}\PYG{p}{:}     \PYG{p}{:} \PYG{n+nd}{@blacklist} \PYG{p}{(}\PYG{n}{see} \PYG{n}{shell}\PYG{l+s+s1}{\PYGZsq{}}\PYG{l+s+s1}{s tor\PYGZus{}pull.sh), @admin proxy\PYGZus{}checker (see wizhelp), @tor (see wizhelp)}
\end{sphinxVerbatim}


\section{Extended lockdown of the mush and considerations}
\label{\detokenize{security:extended-lockdown-of-the-mush-and-considerations}}
\sphinxAtStartPar
These are flags, powers, toggles, and various conditions for consideration
when you decide to use some of the advanced features of RhostMUSH.
These are not all that is availble, but tend to be the juicier ones to consider.


\subsection{Attribute Restriction}
\label{\detokenize{security:attribute-restriction}}
\begin{sphinxVerbatim}[commandchars=\\\{\}]
\PYG{n+nd}{@attribute} \PYG{o}{\PYGZhy{}}\PYG{o}{\PYGZhy{}} \PYG{n}{Used} \PYG{k}{for} \PYG{n}{user}\PYG{o}{\PYGZhy{}}\PYG{n}{defined} \PYG{n}{attributes}
\PYG{n+nd}{@admin} \PYG{n}{attr\PYGZus{}access} \PYG{o}{\PYGZhy{}}\PYG{o}{\PYGZhy{}} \PYG{n}{used} \PYG{k}{for} \PYG{n}{built} \PYG{o+ow}{in} \PYG{n}{attributes} \PYG{p}{(}\PYG{n}{like} \PYG{n}{desc}\PYG{p}{)}
\PYG{n+nd}{@aflags} \PYG{o}{\PYGZhy{}}\PYG{o}{\PYGZhy{}} \PYG{n}{Used} \PYG{n}{to} \PYG{n+nb}{set} \PYG{n}{up} \PYG{n}{lovely} \PYG{n}{delicious} \PYG{n}{attribute} \PYG{n}{permission} \PYG{n}{masks}
\end{sphinxVerbatim}


\subsection{Command Restriction}
\label{\detokenize{security:command-restriction}}
\begin{sphinxVerbatim}[commandchars=\\\{\}]
\PYG{n+nd}{@icmd}    \PYG{o}{\PYGZhy{}} \PYG{n}{Very} \PYG{n}{useful}\PYG{o}{.}   \PYG{n}{Please} \PYG{n}{see} \PYG{n}{wizhelp} \PYG{n}{on} \PYG{n}{it}\PYG{o}{.}  \PYG{n}{It} \PYG{n}{disallows} \PYG{n}{commands} \PYG{k+kn}{from} \PYG{n+nn}{executing} \PYG{n}{including} \PYG{n}{overriding} \PYG{n}{them} \PYG{k}{with} \PYG{n}{softcode} \PYG{n}{alternatives}
\PYG{n+nd}{@admin} \PYG{n}{access} \PYG{o}{\PYGZhy{}} \PYG{n}{Changes} \PYG{n}{permissions}\PYG{p}{,} \PYG{n}{disables}\PYG{p}{,} \PYG{o+ow}{or} \PYG{n}{sets} \PYG{n}{to} \PYG{n}{be} \PYG{n}{overridden} \PYG{n}{a} \PYG{n}{command}\PYG{o}{.}  \PYG{n}{Useful} \PYG{n}{when} \PYG{n}{you} \PYG{n}{plan} \PYG{n}{to} \PYG{n}{override} \PYG{n}{commands} \PYG{k}{with} \PYG{n}{softcode}\PYG{o}{.}
\end{sphinxVerbatim}


\subsection{Flag/Toggle Restriction}
\label{\detokenize{security:flag-toggle-restriction}}
\begin{sphinxVerbatim}[commandchars=\\\{\}]
\PYG{n+nd}{@flagdef} \PYG{o}{\PYGZhy{}} \PYG{n}{Again}\PYG{p}{,} \PYG{n}{see} \PYG{n}{wizhelp} \PYG{n}{on} \PYG{n}{this}\PYG{o}{.}  \PYG{n}{There} \PYG{n}{are} \PYG{n}{also} \PYG{n}{netrhost}\PYG{o}{.}\PYG{n}{conf} \PYG{n}{options} \PYG{n}{so} \PYG{n}{you} \PYG{n}{can} \PYG{n}{have} \PYG{n}{them} \PYG{n}{loaded} \PYG{n}{at} \PYG{n}{start}\PYG{o}{.}  \PYG{n}{This} \PYG{n}{allows} \PYG{n}{tweaking} \PYG{n}{flags} \PYG{o+ow}{and} \PYG{n}{toggles} \PYG{n}{to} \PYG{n}{who} \PYG{n}{can} \PYG{n+nb}{set}\PYG{o}{/}\PYG{n}{unset}\PYG{o}{/}\PYG{n}{see} \PYG{k}{as} \PYG{n}{well} \PYG{k}{as} \PYG{n}{what} \PYG{n+nb}{type} \PYG{n}{can} \PYG{n}{use} \PYG{n}{it} \PYG{o+ow}{or} \PYG{n}{wha} \PYG{n+nb}{type} \PYG{n}{it} \PYG{n}{can} \PYG{n}{be} \PYG{n}{used} \PYG{n}{on}\PYG{o}{.}
\end{sphinxVerbatim}


\subsection{Config restrictions}
\label{\detokenize{security:config-restrictions}}
\begin{sphinxVerbatim}[commandchars=\\\{\}]
\PYG{n+nd}{@admin} \PYG{n}{config\PYGZus{}access} \PYG{o}{\PYGZhy{}} \PYG{n}{Changes} \PYG{n}{permission} \PYG{n}{of} \PYG{n}{who} \PYG{n}{can} \PYG{n+nb}{set} \PYG{n}{a} \PYG{n}{config} \PYG{n}{param}
\end{sphinxVerbatim}


\subsection{Function Restrictions}
\label{\detokenize{security:function-restrictions}}
\begin{sphinxVerbatim}[commandchars=\\\{\}]
\PYG{n+nd}{@function}\PYG{o}{/}\PYG{n+nd}{@lfunction} \PYG{o}{\PYGZhy{}}\PYG{o}{\PYGZhy{}} \PYG{n}{Allows} \PYG{n}{softcoded} \PYG{n}{functions} \PYG{n}{that} \PYG{n}{you} \PYG{n}{can} \PYG{n}{optionally} \PYG{n}{lock} \PYG{n}{down} \PYG{n}{at} \PYG{n}{your} \PYG{n}{leasure}
\PYG{n+nd}{@admin} \PYG{n}{function\PYGZus{}access} \PYG{o}{\PYGZhy{}}\PYG{o}{\PYGZhy{}} \PYG{n}{You} \PYG{n}{can} \PYG{n}{use} \PYG{n}{this} \PYG{n}{even} \PYG{n}{on} \PYG{n}{softcoded} \PYG{n}{functions} \PYG{k}{if} \PYG{n}{you} \PYG{n}{so} \PYG{n}{desired}\PYG{o}{.}
\end{sphinxVerbatim}


\subsection{RhostMUSH Flags and Descriptions}
\label{\detokenize{security:rhostmush-flags-and-descriptions}}
\begin{sphinxVerbatim}[commandchars=\\\{\}]
GUEST       \PYGZhy{} This is your guest flag, it should only be set on guests
WANDERER    \PYGZhy{} the WANDERER flag is default on new players.  This flag disables all building abilitites of the player.
NO\PYGZus{}COMMAND  \PYGZhy{} You can use this to stop a player from being able to connect without worrying about changing their password
FUBAR       \PYGZhy{} As the flag states, it f*\PYGZsq{}s them up beind all recognition.  It essentially stops them from doing absolutely anything in the mush but pose and say.  Yes, it even disables the quit command.
SLAVE       \PYGZhy{} Funny enough, slave allows anything but say and pose.  To ruin a troll\PYGZsq{}s life, set both FUBAR and SLAVE and sit back and smile.
NO\PYGZus{}TEL      \PYGZhy{} The target can\PYGZsq{}t teleport or be teleported
NO\PYGZus{}MOVE     \PYGZhy{} The target is locked at their location unable to move at all
NO\PYGZus{}WALL     \PYGZhy{} They do not see any @wall except a wizard @wall/no\PYGZus{}prefix.  This has the bonus of snuffing db save messages.
NO\PYGZus{}POSSESS  \PYGZhy{} Sometimes it\PYGZsq{}s useful to grant a builder character to multiple players.  The NO\PYGZus{}POSSESS flag makes it so that player can not be logged in more than 2 times.
NO\PYGZus{}MODIFY   \PYGZhy{} The target can not be modified (except by immortal/\PYGZsh{}1)
NO\PYGZus{}EXAMINE  \PYGZhy{} The target can not be examined/decompiled (except by immortal/\PYGZsh{}1)
STOP        \PYGZhy{} Once a matching \PYGZdl{}command is found on an object set STOP, it \PYGZsq{}stops\PYGZsq{} trying to find other \PYGZdl{}command matches.
NOSTOP      \PYGZhy{} If a target that is set STOP is also set NOSTOP, it will check the master room for a command and execute that as well if found.
NO\PYGZus{}PESTER   \PYGZhy{} Stops target from @pemit or whisper.  You may use @icmd as well.
NO\PYGZus{}OVERRIDE \PYGZhy{} Useful for immortals.  By default they override all locks, including attribute locks.  This makes it so an immortal\PYGZsq{}s passing of locks will behave like a mortals
NO\PYGZus{}USELOCK  \PYGZhy{} This is like NO\PYGZus{}OVERRIDE but only effects uselocks.  You likely want to set this on your immortal and wizard.
NO\PYGZus{}ANSINAME \PYGZhy{} stops a target from having an ansified name
NO\PYGZus{}CODE     \PYGZhy{} lock down advanced coding from a target
SPAMMONITOR \PYGZhy{} stop a target from issuing more than 60 commands a minute.
FREE        \PYGZhy{} Stop costing money for day to day processing of commands/building
\end{sphinxVerbatim}


\subsection{RhostMUSH Toggles and Descriptions}
\label{\detokenize{security:rhostmush-toggles-and-descriptions}}
\begin{sphinxVerbatim}[commandchars=\\\{\}]
\PYG{n}{MONITOR}            \PYG{o}{\PYGZhy{}} \PYG{n}{Enables} \PYG{n}{site} \PYG{n}{monitoring}\PYG{o}{.}  \PYG{n}{This} \PYG{o+ow}{is} \PYG{n}{the} \PYG{n}{main} \PYG{n}{toggle}
\PYG{n}{MONITOR\PYGZus{}SITE}       \PYG{o}{\PYGZhy{}} \PYG{n}{Adds} \PYG{n}{site} \PYG{n}{information} \PYG{n}{to} \PYG{n}{site} \PYG{n}{monitoring}
\PYG{n}{MONITOR\PYGZus{}USERID}     \PYG{o}{\PYGZhy{}} \PYG{n}{Adds} \PYG{n}{the} \PYG{n}{userid} \PYG{n}{to} \PYG{n}{site} \PYG{n}{monitoring}
\PYG{n}{MONITOR\PYGZus{}STATS}      \PYG{o}{\PYGZhy{}} \PYG{n}{Adds} \PYG{n}{connection} \PYG{n}{stats} \PYG{n}{to} \PYG{n}{site} \PYG{n}{monitoring}
\PYG{n}{MONITOR\PYGZus{}FAIL}       \PYG{o}{\PYGZhy{}} \PYG{n}{Adds} \PYG{n}{showing} \PYG{n}{failed} \PYG{n}{connections} \PYG{n}{to} \PYG{n}{site} \PYG{n}{monitoring}
\PYG{n}{MONITOR\PYGZus{}CONN}       \PYG{o}{\PYGZhy{}} \PYG{n}{Adds} \PYG{n}{connection} \PYG{n}{monitoring} \PYG{n}{to} \PYG{n}{site} \PYG{n}{monitoring}
\PYG{n}{MONITOR\PYGZus{}DISREASON}  \PYG{o}{\PYGZhy{}} \PYG{n}{Adds} \PYG{n}{disconnect} \PYG{n}{reasons} \PYG{n}{to} \PYG{n}{site} \PYG{n}{monitoring}
\PYG{n}{MONITOR\PYGZus{}TIME}       \PYG{o}{\PYGZhy{}} \PYG{n}{Adds} \PYG{n}{time} \PYG{n}{stamps} \PYG{n}{to} \PYG{n}{site} \PYG{n}{monitoring}
\PYG{n}{MONITOR\PYGZus{}BAD}        \PYG{o}{\PYGZhy{}} \PYG{n}{Shows} \PYG{n+nb}{all} \PYG{n}{bad} \PYG{n}{creation} \PYG{n}{attempts} \PYG{n}{to} \PYG{n}{site} \PYG{n}{monitoring}
\PYG{n}{MONITOR\PYGZus{}VLIMIT}     \PYG{o}{\PYGZhy{}} \PYG{n}{Shows} \PYG{n}{attempts} \PYG{n}{to} \PYG{n}{bypass} \PYG{n}{MAX} \PYG{n}{ATTRIBUTES}
\PYG{n}{MONITOR\PYGZus{}AREG}       \PYG{o}{\PYGZhy{}} \PYG{n}{Shows} \PYG{n+nb}{all} \PYG{n}{auto} \PYG{n}{registration} \PYG{n}{attempts}
\PYG{n}{MONITOR\PYGZus{}CPU}        \PYG{o}{\PYGZhy{}} \PYG{n}{Shows} \PYG{n+nb}{all} \PYG{n}{CPU} \PYG{n}{warnings} \PYG{o+ow}{and}\PYG{o}{/}\PYG{o+ow}{or} \PYG{n}{alerts} \PYG{n}{on} \PYG{n}{the} \PYG{n}{mush}
\PYG{n}{NO\PYGZus{}FORMAT}          \PYG{o}{\PYGZhy{}} \PYG{n}{Bypasses} \PYG{n+nd}{@conformat}\PYG{p}{,} \PYG{n+nd}{@exitformat}\PYG{p}{,} \PYG{o+ow}{and} \PYG{n}{other} \PYG{n}{formats}
\PYG{n}{SEE\PYGZus{}SUSPECT}        \PYG{o}{\PYGZhy{}} \PYG{n}{Allows} \PYG{n}{you} \PYG{n}{to} \PYG{n}{see} \PYG{n}{suspect} \PYG{n}{info} \PYG{o+ow}{in} \PYG{n}{the} \PYG{n}{WHO}\PYG{o}{/}\PYG{n}{DOING}
\PYG{n}{FORCEHALTED}        \PYG{o}{\PYGZhy{}} \PYG{n}{Allows} \PYG{n}{you} \PYG{n}{to} \PYG{n+nd}{@force}\PYG{o}{/}\PYG{n+nd}{@sudo} \PYG{n}{a} \PYG{n}{HALTED} \PYG{n}{target}
\PYG{n}{NOSHPROG}           \PYG{o}{\PYGZhy{}} \PYG{n}{Disallows} \PYG{n}{using} \PYG{l+s+s1}{\PYGZsq{}}\PYG{l+s+s1}{|}\PYG{l+s+s1}{\PYGZsq{}} \PYG{n}{to} \PYG{n}{execute} \PYG{n}{commands} \PYG{n}{outside} \PYG{n+nd}{@program}
\PYG{n}{PROG}               \PYG{o}{\PYGZhy{}} \PYG{n}{Allows} \PYG{n}{the} \PYG{n}{target} \PYG{n}{to} \PYG{n}{use} \PYG{n+nd}{@program}
\PYG{n}{IMMPROG}            \PYG{o}{\PYGZhy{}} \PYG{n}{Disables} \PYG{n}{the} \PYG{n}{ability} \PYG{n}{to} \PYG{n}{use} \PYG{n+nd}{@quitprogram} \PYG{k+kn}{from} \PYG{n+nn}{a} \PYG{n+nd}{@program}
\PYG{n}{PROG\PYGZus{}ON\PYGZus{}CONNECT}    \PYG{o}{\PYGZhy{}} \PYG{n}{Allows} \PYG{n}{a} \PYG{n+nd}{@program} \PYG{n}{to} \PYG{n}{resume} \PYG{k}{if} \PYG{n}{someone} \PYG{n}{reconnects}
\PYG{n}{IGNOREZONE}         \PYG{o}{\PYGZhy{}} \PYG{n}{Enables} \PYG{n}{a} \PYG{n}{zone} \PYG{n}{to} \PYG{n}{process} \PYG{n+nd}{@icmd}\PYG{l+s+s1}{\PYGZsq{}}\PYG{l+s+s1}{s}
\PYG{n}{PAGELOCK}           \PYG{o}{\PYGZhy{}} \PYG{n}{Enforces} \PYG{n}{target} \PYG{n}{to} \PYG{n}{require} \PYG{n}{passing} \PYG{n}{pagelocks}
\PYG{n}{MAIL\PYGZus{}LOCKDOWN}      \PYG{o}{\PYGZhy{}} \PYG{n}{Blocks} \PYG{n}{the} \PYG{n}{ability} \PYG{n}{of} \PYG{n}{a} \PYG{n}{wizard} \PYG{n}{to} \PYG{n}{check} \PYG{n}{another} \PYG{n}{player}\PYG{l+s+s1}{\PYGZsq{}}\PYG{l+s+s1}{s mail}
\PYG{n}{ATRUSE}             \PYG{o}{\PYGZhy{}} \PYG{n}{Enables} \PYG{n}{the} \PYG{n}{attribute} \PYG{n}{to} \PYG{n}{use} \PYG{n}{attribute} \PYG{n}{content} \PYG{n}{locking}
\PYG{n}{NOGLOBPARENT}       \PYG{o}{\PYGZhy{}} \PYG{n}{Disables} \PYG{n}{the} \PYG{n}{target} \PYG{k+kn}{from} \PYG{n+nn}{inheriting} \PYG{k}{global} \PYG{n}{parenting}
\PYG{n}{LOGROOM}            \PYG{o}{\PYGZhy{}} \PYG{n}{Enables} \PYG{n}{system} \PYG{n}{logs} \PYG{n}{on} \PYG{n}{the} \PYG{n}{room}
\PYG{n}{EXFULLWIZATTR}      \PYG{o}{\PYGZhy{}} \PYG{n}{Allows} \PYG{n}{target} \PYG{n}{to} \PYG{n}{examine} \PYG{n+nb}{all} \PYG{n}{wizard} \PYG{n}{attributes}
\PYG{n}{NODEFAULT}          \PYG{o}{\PYGZhy{}} \PYG{n}{Disables} \PYG{n}{attribute} \PYG{n}{formatting}\PYG{o}{/}\PYG{n}{handling} \PYG{n}{on} \PYG{n}{the} \PYG{n}{target}
\PYG{n}{CHKREALITY}         \PYG{o}{\PYGZhy{}} \PYG{n}{Enables} \PYG{n}{the} \PYG{n}{use} \PYG{n}{of} \PYG{n}{reality} \PYG{n}{locks} \PYG{n}{on} \PYG{n}{the} \PYG{n}{target}
\PYG{n}{HIDEIDLE}           \PYG{o}{\PYGZhy{}} \PYG{n}{Disables} \PYG{n}{deidling} \PYG{n}{when} \PYG{n}{you} \PYG{n}{execute} \PYG{n+nb}{any} \PYG{n}{command}
\PYG{n}{MORTALREALITY}      \PYG{o}{\PYGZhy{}} \PYG{n}{Enforces} \PYG{n}{a} \PYG{n}{wizard} \PYG{n}{to} \PYG{k}{pass} \PYG{n}{realities} \PYG{k}{as} \PYG{n}{a} \PYG{n}{mortal}
\PYG{n}{SNUFFDARK}          \PYG{o}{\PYGZhy{}} \PYG{n}{Hides} \PYG{n}{dark} \PYG{n}{exits} \PYG{k+kn}{from} \PYG{n+nn}{a} \PYG{n}{wizard}
\end{sphinxVerbatim}


\subsection{RhostMUSH @powers and Descriptions}
\label{\detokenize{security:rhostmush-powers-and-descriptions}}
\begin{sphinxVerbatim}[commandchars=\\\{\}]
\PYG{n}{WIZ\PYGZus{}WHO}            \PYG{o}{\PYGZhy{}} \PYG{n}{Allows} \PYG{n}{target} \PYG{n}{to} \PYG{n}{see} \PYG{n}{sites} \PYG{n}{ala} \PYG{n}{wizard} \PYG{n}{who}
\PYG{n}{NOFORCE}            \PYG{o}{\PYGZhy{}} \PYG{n}{target} \PYG{n}{an} \PYG{o+ow}{not} \PYG{n}{be} \PYG{n}{forced} \PYG{p}{(}\PYG{k}{except} \PYG{n}{by} \PYG{n}{immortal}\PYG{o}{/}\PYG{c+c1}{\PYGZsh{}1)}
\PYG{n}{FREE\PYGZus{}QUOTA}         \PYG{o}{\PYGZhy{}} \PYG{n}{Allow} \PYG{n}{target} \PYG{n}{to} \PYG{n}{have} \PYG{n}{unlimited} \PYG{n}{quota}
\PYG{n}{JOIN\PYGZus{}PLAYER}        \PYG{o}{\PYGZhy{}} \PYG{n}{Allow} \PYG{n}{to} \PYG{l+s+s1}{\PYGZsq{}}\PYG{l+s+s1}{join}\PYG{l+s+s1}{\PYGZsq{}} \PYG{n}{a} \PYG{n}{player}\PYG{l+s+s1}{\PYGZsq{}}\PYG{l+s+s1}{s location}
\PYG{n}{NO\PYGZus{}BOOT}            \PYG{o}{\PYGZhy{}} \PYG{n}{Player} \PYG{n}{can} \PYG{o+ow}{not} \PYG{n}{be} \PYG{n}{booted} \PYG{k}{except} \PYG{n}{by} \PYG{n}{immortal}\PYG{o}{/}\PYG{c+c1}{\PYGZsh{}1}
\PYG{n}{STEAL}              \PYG{o}{\PYGZhy{}} \PYG{n}{Player} \PYG{n}{can} \PYG{n}{give} \PYG{n}{negative} \PYG{n}{money}
\PYG{n}{TEL\PYGZus{}ANYWHERE}       \PYG{o}{\PYGZhy{}} \PYG{n}{Player} \PYG{n}{can} \PYG{n}{teleport} \PYG{n}{anywhere}
\PYG{n}{STAT\PYGZus{}ANY}           \PYG{o}{\PYGZhy{}} \PYG{n}{Player} \PYG{n}{can} \PYG{n+nd}{@search}\PYG{p}{,} \PYG{n+nd}{@stat}\PYG{p}{,} \PYG{o+ow}{or} \PYG{n+nd}{@find} \PYG{n}{things}
\PYG{n}{HALT\PYGZus{}QUEUE\PYGZus{}ALL}     \PYG{o}{\PYGZhy{}} \PYG{n}{Player} \PYG{n}{can} \PYG{n}{halt} \PYG{n}{the} \PYG{n}{queue}
\PYG{n}{SEARCH\PYGZus{}ANY}         \PYG{o}{\PYGZhy{}} \PYG{n}{Player} \PYG{n}{can} \PYG{n}{search} \PYG{k}{for} \PYG{n}{anything}
\PYG{n}{WHO\PYGZus{}UNFIND}         \PYG{o}{\PYGZhy{}} \PYG{n}{Player} \PYG{n}{can} \PYG{n}{see} \PYG{n}{hidden} \PYG{n}{player} \PYG{n}{on} \PYG{n}{WHO}
\PYG{n}{SHUTDOWN}           \PYG{o}{\PYGZhy{}} \PYG{n}{Player} \PYG{n}{can} \PYG{n+nd}{@shutdown} \PYG{n}{the} \PYG{n}{mush}
\PYG{n}{PURGE}              \PYG{o}{\PYGZhy{}} \PYG{n}{Player} \PYG{n}{can} \PYG{n}{use} \PYG{o}{/}\PYG{n}{purge} \PYG{n}{to} \PYG{n+nd}{@destroy} \PYG{o+ow}{and} \PYG{n+nd}{@nuke}
\PYG{n}{EXAMINE\PYGZus{}FULL}       \PYG{o}{\PYGZhy{}} \PYG{n}{Player} \PYG{n}{can} \PYG{n}{examine} \PYG{n}{anything} \PYG{p}{(}\PYG{o+ow}{not} \PYG{n+nb}{set} \PYG{n}{NO\PYGZus{}EXAMINE} \PYG{o+ow}{or} \PYG{n}{cloaked}\PYG{p}{)}
\PYG{n}{FORMATTING}         \PYG{o}{\PYGZhy{}} \PYG{o}{@}\PYG{o}{*}\PYG{n}{formats} \PYG{n}{allow} \PYG{n}{passing} \PYG{n}{what} \PYG{n}{a} \PYG{n}{person} \PYG{n}{sees} \PYG{k}{as} \PYG{o}{\PYGZpc{}}\PYG{l+m+mi}{0}\PYG{p}{,} \PYG{o}{\PYGZpc{}}\PYG{l+m+mi}{1}\PYG{p}{,} \PYG{n}{etc}
\PYG{n}{CHOWN\PYGZus{}ANYWHERE}     \PYG{o}{\PYGZhy{}} \PYG{n}{Chown} \PYG{n}{anything} \PYG{n}{anywhere} \PYG{n}{to} \PYG{n}{yourself}
\PYG{n}{CHOWN\PYGZus{}OTHER}        \PYG{o}{\PYGZhy{}} \PYG{n}{Chown} \PYG{n}{something} \PYG{n}{you} \PYG{n}{don}\PYG{l+s+s1}{\PYGZsq{}}\PYG{l+s+s1}{t own to yourself}
\PYG{n}{EXAMINE\PYGZus{}ALL}        \PYG{o}{\PYGZhy{}} \PYG{n}{Examine} \PYG{n}{other} \PYG{n}{things} \PYG{p}{(}\PYG{n}{tiered}\PYG{p}{)}
\PYG{n}{SEE\PYGZus{}QUEUE\PYGZus{}ALL}      \PYG{o}{\PYGZhy{}} \PYG{n}{Player} \PYG{n}{can} \PYG{n}{see} \PYG{n}{the} \PYG{n}{full} \PYG{n}{queue}
\PYG{n}{GRAB\PYGZus{}PLAYER}        \PYG{o}{\PYGZhy{}} \PYG{n}{Player} \PYG{n}{can} \PYG{n}{grab} \PYG{n}{a} \PYG{n}{remote} \PYG{n}{player} \PYG{o+ow}{and} \PYG{n}{pull} \PYG{n}{them} \PYG{n}{to} \PYG{n}{location}
\PYG{n}{LONG\PYGZus{}FINGERS}       \PYG{o}{\PYGZhy{}} \PYG{n}{Player} \PYG{o+ow}{is} \PYG{n}{granted} \PYG{n}{remote} \PYG{n}{control} \PYG{n}{of} \PYG{n}{things} \PYG{n}{they} \PYG{n}{own}
\PYG{n}{BOOT}               \PYG{o}{\PYGZhy{}} \PYG{n}{Player} \PYG{n}{can} \PYG{n+nd}{@boot}
\PYG{n}{SEE\PYGZus{}QUEUE}          \PYG{o}{\PYGZhy{}} \PYG{n}{Player} \PYG{n}{can} \PYG{n}{see} \PYG{n}{advanced} \PYG{n}{queue} \PYG{n}{features}
\PYG{n}{TEL\PYGZus{}ANYTHING}       \PYG{o}{\PYGZhy{}} \PYG{n}{Player} \PYG{n}{can} \PYG{n+nd}{@teleport} \PYG{n}{anything}
\PYG{n}{PCREATE}            \PYG{o}{\PYGZhy{}} \PYG{n}{Player} \PYG{n}{can} \PYG{n+nd}{@pcreate} \PYG{n}{players}
\PYG{n}{NOWHO}              \PYG{o}{\PYGZhy{}} \PYG{n}{Allows} \PYG{n}{the} \PYG{n}{use} \PYG{n}{of} \PYG{n+nd}{@hide}
\PYG{n}{HALT\PYGZus{}QUEUE}         \PYG{o}{\PYGZhy{}} \PYG{n}{Allows} \PYG{n}{halting} \PYG{n}{queue} \PYG{n}{by} \PYG{n}{specified} \PYG{n}{bitlevel}
\PYG{n}{SECURITY}           \PYG{o}{\PYGZhy{}} \PYG{n}{Allows} \PYG{n}{setting}  \PYG{n}{NOMOVE}    \PYG{n}{NO\PYGZus{}TEL}   \PYG{n}{SLAVE}   \PYG{n}{NO\PYGZus{}YELL}
\PYG{n}{WRAITH}             \PYG{o}{\PYGZhy{}} \PYG{n}{Allows} \PYG{n}{bypassing} \PYG{n}{exit} \PYG{n}{locks}
\PYG{n}{HIDEBIT}            \PYG{o}{\PYGZhy{}} \PYG{n}{Hides} \PYG{n}{your} \PYG{n}{admin} \PYG{n}{level} \PYG{k+kn}{from} \PYG{n+nn}{lower} \PYG{n}{levels}
\PYG{n}{FULLTEL}            \PYG{o}{\PYGZhy{}} \PYG{n}{Allows} \PYG{n}{full} \PYG{n}{immortal} \PYG{n}{level} \PYG{n}{teleportation}
\PYG{n}{EXECSCRIPT}         \PYG{o}{\PYGZhy{}} \PYG{n}{Allows} \PYG{n}{executing} \PYG{n}{external} \PYG{n}{scripts} \PYG{o+ow}{in} \PYG{o}{\PYGZti{}}\PYG{o}{/}\PYG{n}{game}\PYG{o}{/}\PYG{n}{scripts}
\end{sphinxVerbatim}


\subsection{RhostMUSH @depowers and Descriptions}
\label{\detokenize{security:rhostmush-depowers-and-descriptions}}
\begin{sphinxVerbatim}[commandchars=\\\{\}]
WALL               \PYGZhy{} Disables the ability to @wall
LONG\PYGZus{}FINGERS       \PYGZhy{} Disables remote access to things
STEAL              \PYGZhy{} Can not steal money
CREATE             \PYGZhy{} Can not create anything
WIZ\PYGZus{}WHO            \PYGZhy{} Can not access wizard who
CLOAK              \PYGZhy{} Can not wizard cloak
BOOT               \PYGZhy{} Can not boot
PAGE               \PYGZhy{} Can not page
FORCE              \PYGZhy{} Can not @force/@sudo
LOCKS              \PYGZhy{} Can not pass locks
COMMAND            \PYGZhy{} Can not execute any \PYGZdl{}command (including master room)
MASTER             \PYGZhy{} Can not use any master room \PYGZdl{}command
EXAMINE            \PYGZhy{} Lowers or disables the ability to examine/decompile
NUKE               \PYGZhy{} Can not nuke, toad, or turtle
FREE               \PYGZhy{} No longer has free money for anything
OVERRIDE           \PYGZhy{} No longer can override anything
TEL\PYGZus{}ANYWHERE       \PYGZhy{} No longer can teleport anywhere
TEL\PYGZus{}ANYTHING       \PYGZhy{} No longer can teleport anything other than themselves
POWER              \PYGZhy{} Can no longer use @power
MODIFY             \PYGZhy{} Can not modify things
CHOWN\PYGZus{}ME           \PYGZhy{} Can not chown anything to themselves
CHOWN\PYGZus{}OTHER        \PYGZhy{} Can not chown anything to others
ABUSE              \PYGZhy{} Can not use \PYGZdl{}commands on anything they do not own
UNL\PYGZus{}QUOTA          \PYGZhy{} No longer has infinite quota
SEARCH\PYGZus{}ANY         \PYGZhy{} Disables the ability to @search/@find things
GIVE               \PYGZhy{} Disables ability to give things/money
RECEIVE            \PYGZhy{} Disables the ability to recieve things/money
NOGOLD             \PYGZhy{} Limits (or disables) how much gold someone can give
NOSTEAL            \PYGZhy{} Can not give negative gold
PASSWORD           \PYGZhy{} Can not change password
MORTAL\PYGZus{}EXAMINE     \PYGZhy{} drops examine and all fetching to mortal only
PERSONAL\PYGZus{}COMMAND   \PYGZhy{} Disables all \PYGZdl{}commands on anything they own
\end{sphinxVerbatim}


\subsection{Site Restrictions}
\label{\detokenize{security:site-restrictions}}\begin{quote}

\sphinxAtStartPar
These are accessable via the @admin command, and the following options are
allowable.

\sphinxAtStartPar
You may see all site information at any time with: @list sites
\end{quote}


\subsubsection{IP based restrictions}
\label{\detokenize{security:ip-based-restrictions}}\begin{quote}

\sphinxAtStartPar
You may use CIDR notation such as /32 instead of 255.255.255.255.
Config file:  (see section on forbid\_site as it describes and gives examples)
Online Syntax: MASK:

\begin{sphinxVerbatim}[commandchars=\\\{\}]
                \PYG{n+nd}{@admin} \PYG{n}{forbid\PYGZus{}site}\PYG{o}{=}\PYG{l+m+mf}{123.123}\PYG{o}{.}\PYG{l+m+mf}{123.0} \PYG{l+m+mf}{255.255}\PYG{o}{.}\PYG{l+m+mf}{255.0}
                \PYG{n+nd}{@admin} \PYG{n}{forbid\PYGZus{}site}\PYG{o}{=}\PYG{l+m+mf}{123.123}\PYG{o}{.}\PYG{l+m+mf}{123.123} \PYG{l+m+mf}{255.255}\PYG{o}{.}\PYG{l+m+mf}{255.255}

\PYG{n}{CIDR}\PYG{p}{:}\PYG{p}{:}

                \PYG{n+nd}{@admin} \PYG{n}{forbid\PYGZus{}site}\PYG{o}{=}\PYG{l+m+mf}{123.123}\PYG{o}{.}\PYG{l+m+mf}{123.0} \PYG{o}{/}\PYG{l+m+mi}{24}
                \PYG{n+nd}{@admin} \PYG{n}{forbid\PYGZus{}site}\PYG{o}{=}\PYG{l+m+mf}{123.123}\PYG{o}{.}\PYG{l+m+mf}{123.123} \PYG{o}{/}\PYG{l+m+mi}{32}

\PYG{n}{REMOVING}\PYG{p}{:} \PYG{n}{MASK}\PYG{p}{:}\PYG{p}{:}

                \PYG{n+nd}{@site}\PYG{o}{/}\PYG{n+nb}{all} \PYG{l+m+mf}{123.123}\PYG{o}{.}\PYG{l+m+mf}{123.123}\PYG{o}{=}\PYG{l+m+mf}{255.255}\PYG{o}{.}\PYG{l+m+mf}{255.255}
                \PYG{n+nd}{@site}\PYG{o}{/}\PYG{n}{forbid} \PYG{l+m+mf}{123.123}\PYG{o}{.}\PYG{l+m+mf}{123.0}\PYG{o}{=}\PYG{l+m+mf}{254.255}\PYG{o}{.}\PYG{l+m+mf}{255.0}

          \PYG{n}{CIDR}\PYG{p}{:}\PYG{p}{:}

                \PYG{n+nd}{@site}\PYG{o}{/}\PYG{n+nb}{all} \PYG{l+m+mf}{123.123}\PYG{o}{.}\PYG{l+m+mf}{123.123}\PYG{o}{=}\PYG{o}{/}\PYG{l+m+mi}{32}
                \PYG{n+nd}{@site}\PYG{o}{/}\PYG{n}{forbid} \PYG{l+m+mf}{123.123}\PYG{o}{.}\PYG{l+m+mf}{123.0}\PYG{o}{=}\PYG{o}{/}\PYG{l+m+mi}{24}
\end{sphinxVerbatim}
\end{quote}

\begin{sphinxVerbatim}[commandchars=\\\{\}]
\PYG{n}{forbid\PYGZus{}site}      \PYG{o}{\PYGZhy{}} \PYG{n}{Set} \PYG{n}{the} \PYG{n}{specified} \PYG{n}{site} \PYG{n}{forbid} \PYG{n}{only}
\PYG{n}{register\PYGZus{}site}    \PYG{o}{\PYGZhy{}} \PYG{n}{Set} \PYG{n}{the} \PYG{n}{specified} \PYG{n}{site} \PYG{n}{register} \PYG{n}{only}
\PYG{n}{noguest\PYGZus{}site}     \PYG{o}{\PYGZhy{}} \PYG{n}{Set} \PYG{n}{the} \PYG{n}{specified} \PYG{n}{site} \PYG{n}{unable} \PYG{n}{to} \PYG{n}{connect} \PYG{n}{to} \PYG{n}{guests}
\PYG{n}{suspect\PYGZus{}site}     \PYG{o}{\PYGZhy{}} \PYG{n}{Set} \PYG{n}{the} \PYG{n}{specified} \PYG{n}{site} \PYG{n}{suspect} \PYG{n}{on} \PYG{n}{connect}
\PYG{n}{noautoreg\PYGZus{}site}   \PYG{o}{\PYGZhy{}} \PYG{n}{Set} \PYG{n}{the} \PYG{n}{specified} \PYG{n}{site} \PYG{n}{to} \PYG{o+ow}{not} \PYG{n}{allow} \PYG{n}{autoregistration}
\PYG{n}{trust\PYGZus{}site}       \PYG{o}{\PYGZhy{}} \PYG{n}{Allow} \PYG{n}{site} \PYG{n}{to} \PYG{n}{bypass} \PYG{n}{suspect} \PYG{n}{site} \PYG{n}{restrictions}
\PYG{n}{permit\PYGZus{}site}      \PYG{o}{\PYGZhy{}} \PYG{n}{Allow} \PYG{n}{site} \PYG{n}{to} \PYG{n}{bypass} \PYG{n}{sitelock} \PYG{n}{restrictions}
\PYG{n}{nodns\PYGZus{}site}       \PYG{o}{\PYGZhy{}} \PYG{n}{Site} \PYG{n}{will} \PYG{n}{no} \PYG{n}{longer} \PYG{n}{do} \PYG{n}{reverse} \PYG{n}{DNS} \PYG{n}{lookups}
\PYG{n}{noauth\PYGZus{}site}      \PYG{o}{\PYGZhy{}} \PYG{n}{Site} \PYG{n}{will} \PYG{n}{no} \PYG{n}{longer} \PYG{n}{do} \PYG{n}{AUTH} \PYG{n}{ident} \PYG{n}{lookups}
\end{sphinxVerbatim}


\subsubsection{DNS based restrictions}
\label{\detokenize{security:dns-based-restrictions}}\begin{quote}

\sphinxAtStartPar
These allow globbing wildcard matches.
The advanced feature is you can specify filtering on
when the condition is matched, such as allowing 2 players from a site to
be connected before disallowing anyone else to connect.
Config File: (see section on forbid\_host as it describes and gives examples)
Online Syntax:

\begin{sphinxVerbatim}[commandchars=\\\{\}]
ADD:      @admin forbid\PYGZus{}host=*.dsl*.comcast.net *.aol.com *another.site
DEL:      @admin forbid\PYGZus{}host=!*.aol.com
ADVANCED: @admin forbid\PYGZus{}host=mudconnect.com|3 (allow 3 at once only)
\end{sphinxVerbatim}
\end{quote}

\begin{sphinxVerbatim}[commandchars=\\\{\}]
\PYG{n}{forbid\PYGZus{}host}     \PYG{o}{\PYGZhy{}} \PYG{n}{Set} \PYG{n}{the} \PYG{n}{specified} \PYG{n}{site}\PYG{p}{(}\PYG{n}{s}\PYG{p}{)} \PYG{n}{forbid} \PYG{n}{only}
\PYG{n}{register\PYGZus{}host}   \PYG{o}{\PYGZhy{}} \PYG{n}{Set} \PYG{n}{the} \PYG{n}{specified} \PYG{n}{site}\PYG{p}{(}\PYG{n}{s}\PYG{p}{)} \PYG{n}{register} \PYG{n}{only}
\PYG{n}{noguest\PYGZus{}host}    \PYG{o}{\PYGZhy{}} \PYG{n}{Set} \PYG{n}{the} \PYG{n}{specified} \PYG{n}{site}\PYG{p}{(}\PYG{n}{s}\PYG{p}{)} \PYG{n}{unable} \PYG{n}{to} \PYG{n}{connect} \PYG{n}{to} \PYG{n}{guests}
\PYG{n}{suspect\PYGZus{}host}    \PYG{o}{\PYGZhy{}} \PYG{n}{Set} \PYG{n}{the} \PYG{n}{specified} \PYG{n}{site}\PYG{p}{(}\PYG{n}{s}\PYG{p}{)} \PYG{n}{suspect} \PYG{n}{on} \PYG{n}{connect}
\PYG{n}{noautoreg\PYGZus{}host}  \PYG{o}{\PYGZhy{}} \PYG{n}{Set} \PYG{n}{the} \PYG{n}{specified} \PYG{n}{site}\PYG{p}{(}\PYG{n}{s}\PYG{p}{)} \PYG{n}{to} \PYG{o+ow}{not} \PYG{n}{allow} \PYG{n}{autoregistration}
\PYG{n}{validate\PYGZus{}host}   \PYG{o}{\PYGZhy{}} \PYG{n}{Do} \PYG{o+ow}{not} \PYG{n}{allow} \PYG{n+nb}{any} \PYG{n}{autoregistration} \PYG{k+kn}{from} \PYG{n+nn}{emails} \PYG{n}{matching} \PYG{n}{site}
\PYG{n}{goodmail\PYGZus{}host}   \PYG{o}{\PYGZhy{}} \PYG{n}{Always} \PYG{n}{allow} \PYG{n}{autoregistration} \PYG{k+kn}{from} \PYG{n+nn}{emails} \PYG{n}{matching} \PYG{n}{site}
\PYG{n}{nobroadcast\PYGZus{}host} \PYG{o}{\PYGZhy{}} \PYG{n}{Snuff} \PYG{n}{online} \PYG{n}{site} \PYG{n}{broadcasts} \PYG{n}{via} \PYG{n}{MONITOR} \PYG{k}{for} \PYG{n}{specified} \PYG{n}{site}
\end{sphinxVerbatim}


\section{Methods to block anonymous connections and the pros and cons of doing so}
\label{\detokenize{security:methods-to-block-anonymous-connections-and-the-pros-and-cons-of-doing-so}}\begin{quote}

\sphinxAtStartPar
Now let\textquotesingle{}s assume you have some troll attempting to use proxies to connect.
There\textquotesingle{}s multiple ways to stop this.
\end{quote}


\subsection{Blacklisting through external tor\_pull.sh script}
\label{\detokenize{security:blacklisting-through-external-tor-pull-sh-script}}\begin{quote}

\sphinxAtStartPar
In \textasciitilde{}/Rhost/Server/game you will see a script called tor\_pull.sh
Execute this by running (from the game directory) ./tor\_pull.sh
This populates the blacklist with registered proxies from various sites
on the internet.  If you want specified ip\textquotesingle{}s added, feel free to add
them at the end of this file.
\end{quote}


\subsection{Blacklisting through internal @blacklist command}
\label{\detokenize{security:blacklisting-through-internal-blacklist-command}}\begin{quote}

\sphinxAtStartPar
On the mush, have as part of your startup @blacklist/load
This will load in the generated blacklist file for automatic forbid
sites based on the ip.
\end{quote}


\subsection{Blacklisting through internal @tor command}
\label{\detokenize{security:blacklisting-through-internal-tor-command}}\begin{quote}

\sphinxAtStartPar
@tor.  Please see \textquotesingle{}wizhelp tor\textquotesingle{} on how to set this up.  It in effect
will actively block all known exit nodes for TOR\textquotesingle{}s annonymous proxy
service.  It self\sphinxhyphen{}updates and will actively block TOR connections.
\end{quote}


\subsection{Blacklisting through internal @admin command}
\label{\detokenize{security:blacklisting-through-internal-admin-command}}\begin{quote}

\sphinxAtStartPar
@admin proxy\_checker (please see wizhelp)
This little doodad uses MTU checking against packet size which will
detect most methods of proxies.  Sadly, this also has false positives
because some situations require a differentating MTU value such as
briged network connect with things like cloud services, docker, or
similar encapsulated network services.  However, this option has
several settings from just monitoring/alerting of possible proxies
to downright forbidding them.  If you\textquotesingle{}re being actively attacked,
it may be worth considering adding this to add additional protection.
\end{quote}


\section{Setting up an SSL tunnel for secure connection options}
\label{\detokenize{security:setting-up-an-ssl-tunnel-for-secure-connection-options}}

\subsection{Quickstart for SSL setup with stunnel}
\label{\detokenize{security:quickstart-for-ssl-setup-with-stunnel}}\begin{enumerate}
\sphinxsetlistlabels{\arabic}{enumi}{enumii}{}{.}%
\item {} 
\sphinxAtStartPar
Modify your netrhost.conf file and add/change the following parameters:
\begin{enumerate}
\sphinxsetlistlabels{\arabic}{enumii}{enumiii}{}{.}%
\item {} 
\sphinxAtStartPar
sconnect\_reip 1

\item {} 
\sphinxAtStartPar
sconnect\_cmd SECRET\sphinxhyphen{}MAGIC\sphinxhyphen{}COOKIE
\begin{enumerate}
\sphinxsetlistlabels{\arabic}{enumiii}{enumiv}{}{.}%
\item {} 
\sphinxAtStartPar
SECRET\sphinxhyphen{}MAGIC\sphinxhyphen{}COOKIE is a case sensitive single word phrase. Any printable character other than the \textquotesingle{}\#\textquotesingle{} character is allowable.  You may use up to 30 characters.

\item {} 
\sphinxAtStartPar
Make sure the secret is a hard to guess phrase.  This is used by stunnel to forward on the originating IP address.

\end{enumerate}

\item {} 
\sphinxAtStartPar
sconnect\_host localhost 127.0.0.1 othersite.goes.here
\begin{enumerate}
\sphinxsetlistlabels{\arabic}{enumiii}{enumiv}{}{.}%
\item {} 
\sphinxAtStartPar
This is optional.

\item {} 
\sphinxAtStartPar
If you do not specify it it defaults to \textquotesingle{}localhost 127.0.0.1\textquotesingle{}.  If your domain has a unique name like \textquotesingle{}localhost.localdomain\textquotesingle{} like some ubuntu distributions, then you should customize your stunnel\_host.

\end{enumerate}

\end{enumerate}

\item {} 
\sphinxAtStartPar
go into the stunnel directory

\item {} 
\sphinxAtStartPar
./stunnel\_setup.sh
\begin{enumerate}
\sphinxsetlistlabels{\arabic}{enumii}{enumiii}{}{.}%
\item {} 
\sphinxAtStartPar
Choose the defaults or alter them based on preferences

\item {} 
\sphinxAtStartPar
Make sure to choose the warpbubble conf file

\end{enumerate}

\item {} 
\sphinxAtStartPar
./stunnel\_start.sh

\item {} 
\sphinxAtStartPar
Use ./stunnel\_stop.sh to stop the SSL tunnel at any time

\end{enumerate}

\sphinxAtStartPar
You do not need to shutdown the ssl handler if you shutdown the mush.  They
are entirely separate processes.


\subsection{Detailed SSL setup with stunnel}
\label{\detokenize{security:detailed-ssl-setup-with-stunnel}}
\sphinxAtStartPar
To setup SSL connectivity, we use the STUNNEL application to tunnel SSL to
the mush.  This acts a bit like a man in the middle but remains controlled
by the game owner which would have access to the end point anyway.

\begin{sphinxadmonition}{note}{Note:}
\sphinxAtStartPar
it is assumed you will have already initially set up your netrhost.conf.
\end{sphinxadmonition}


\subsubsection{stunnel directory}
\label{\detokenize{security:stunnel-directory}}
\sphinxAtStartPar
You would set up the stunnel from the \textquotesingle{}stunnel\textquotesingle{} directory.  There the following
files are of relevance:

\begin{sphinxVerbatim}[commandchars=\\\{\}]
\PYG{n}{README}                     \PYG{o}{\PYGZhy{}}\PYG{o}{\PYGZhy{}} \PYG{n}{a} \PYG{n}{readme} \PYG{n}{explaining} \PYG{n}{the} \PYG{n}{points} \PYG{n}{of} \PYG{n}{stunnel}
\PYG{n}{stunnel}\PYG{o}{.}\PYG{n}{conf}\PYG{o}{.}\PYG{n}{example}       \PYG{o}{\PYGZhy{}}\PYG{o}{\PYGZhy{}} \PYG{n}{The} \PYG{n}{example} \PYG{n}{stunnel}\PYG{o}{.}\PYG{n}{conf} \PYG{n}{file}\PYG{o}{.}  \PYG{n}{If} \PYG{n}{you} \PYG{n}{wish} \PYG{n}{to} \PYG{n}{create} \PYG{n}{this} \PYG{n}{manually} \PYG{n}{you}\PYG{l+s+s1}{\PYGZsq{}}\PYG{l+s+s1}{re welcome to.  Just make sure the end file is stunnel.conf}
\PYG{n}{stunnel\PYGZus{}setup}\PYG{o}{.}\PYG{n}{sh}           \PYG{o}{\PYGZhy{}}\PYG{o}{\PYGZhy{}} \PYG{n}{the} \PYG{n}{script} \PYG{n}{to} \PYG{n}{build} \PYG{n}{a} \PYG{n}{stunnel}\PYG{o}{.}\PYG{n}{conf} \PYG{n}{file} \PYG{k}{for} \PYG{n}{you} \PYG{n}{which} \PYG{n}{will} \PYG{n}{be} \PYG{n}{dropped} \PYG{n}{at} \PYG{n}{your} \PYG{n}{specified} \PYG{n}{location}\PYG{o}{.}
\PYG{n}{stunnel\PYGZus{}kill}\PYG{o}{.}\PYG{n}{sh}            \PYG{o}{\PYGZhy{}}\PYG{o}{\PYGZhy{}} \PYG{n}{Stop}\PYG{o}{/}\PYG{n}{terminate} \PYG{n}{the} \PYG{n}{stunnel} \PYG{n}{process}\PYG{o}{.}
\PYG{n}{stunnel\PYGZus{}start}\PYG{o}{.}\PYG{n}{sh}           \PYG{o}{\PYGZhy{}}\PYG{o}{\PYGZhy{}} \PYG{n}{Start} \PYG{n}{the} \PYG{n}{stunnel} \PYG{n}{process}\PYG{o}{.}
\PYG{n}{warpbubble}\PYG{o}{.}\PYG{n}{pl}              \PYG{o}{\PYGZhy{}}\PYG{o}{\PYGZhy{}} \PYG{n}{the} \PYG{n}{perl} \PYG{n}{script} \PYG{n}{that} \PYG{n}{handles} \PYG{n}{stunnel} \PYG{n}{to} \PYG{n}{mush} \PYG{n}{connectivity}\PYG{o}{.}
\PYG{n}{stunnel\PYGZus{}src}                \PYG{o}{\PYGZhy{}}\PYG{o}{\PYGZhy{}} \PYG{n}{If} \PYG{n}{you} \PYG{n}{do} \PYG{o+ow}{not} \PYG{n}{have} \PYG{n}{stunnel}\PYG{p}{,} \PYG{n}{this} \PYG{n}{directory} \PYG{n}{will} \PYG{n}{allow} \PYG{n}{you} \PYG{n}{to} \PYG{n}{download}\PYG{p}{,} \PYG{n+nb}{compile} \PYG{o+ow}{and} \PYG{n}{locally} \PYG{n}{install}\PYG{o}{.}
\end{sphinxVerbatim}


\subsubsection{Modifying netrhost.conf}
\label{\detokenize{security:modifying-netrhost-conf}}
\sphinxAtStartPar
To be able to utilize SSL, you first must set your netrhost.conf file with
the relevant information to enable SSL connectiions.  These three config
options must be set to be able to use SSL, however, sconnect\_host if
not set will default to \textquotesingle{}localhost 127.0.0.1\textquotesingle{}.

\begin{sphinxVerbatim}[commandchars=\\\{\}]
\PYG{n}{sconnect\PYGZus{}reip} \PYG{l+m+mi}{1}         \PYG{o}{\PYGZhy{}}\PYG{o}{\PYGZhy{}} \PYG{n}{This} \PYG{n}{enables} \PYG{n}{the} \PYG{n}{SSL} \PYG{n}{tunnel} \PYG{n}{layer} \PYG{n}{handler} \PYG{n}{within} \PYG{n}{rhost}\PYG{o}{.}
\PYG{n}{sconnect\PYGZus{}cmd} \PYG{n}{XYZZY}      \PYG{o}{\PYGZhy{}}\PYG{o}{\PYGZhy{}} \PYG{n}{this} \PYG{n}{will} \PYG{n+nb}{set} \PYG{n}{the} \PYG{n}{secret} \PYG{n}{SSL} \PYG{n}{command} \PYG{n}{handshake} \PYG{n}{command} \PYG{n}{to} \PYG{n}{XYZZY}\PYG{o}{.}  \PYG{n}{This} \PYG{o+ow}{is} \PYG{n}{case} \PYG{n}{sensitive} \PYG{o+ow}{and} \PYG{n}{can} \PYG{n}{be} \PYG{n}{up} \PYG{n}{to} \PYG{l+m+mi}{31} \PYG{n}{characters}\PYG{o}{.}  \PYG{n}{Please} \PYG{n}{make} \PYG{n}{sure} \PYG{n}{to} \PYG{n}{only} \PYG{n}{use} \PYG{n}{printable} \PYG{n}{non}\PYG{o}{\PYGZhy{}}\PYG{n}{whitespace} \PYG{n}{characters}\PYG{o}{.}  \PYG{n}{Ergo}\PYG{p}{:} \PYG{n}{one} \PYG{n}{word}
\PYG{n}{sconnect\PYGZus{}host} \PYG{n}{wildcards} \PYG{o}{\PYGZhy{}}\PYG{o}{\PYGZhy{}} \PYG{n}{This} \PYG{n}{allows} \PYG{n}{wildcarded} \PYG{n}{sites} \PYG{p}{(}\PYG{n}{one} \PYG{o+ow}{or} \PYG{n}{more}\PYG{p}{)} \PYG{n}{to} \PYG{n}{allow} \PYG{n}{to} \PYG{n}{access} \PYG{n}{the} \PYG{n}{sconnect}\PYG{o}{/}\PYG{n}{stunnel} \PYG{n}{handler}\PYG{o}{.}  \PYG{n}{This} \PYG{n}{defaults} \PYG{n}{to} \PYG{l+s+s1}{\PYGZsq{}}\PYG{l+s+s1}{localhost}\PYG{l+s+s1}{\PYGZsq{}} \PYG{o+ow}{and} \PYG{l+s+s1}{\PYGZsq{}}\PYG{l+s+s1}{127.0.0.1}\PYG{l+s+s1}{\PYGZsq{}} \PYG{n}{so} \PYG{k}{if} \PYG{n}{you} \PYG{n}{have} \PYG{l+s+s1}{\PYGZsq{}}\PYG{l+s+s1}{localhost.localdomain}\PYG{l+s+s1}{\PYGZsq{}} \PYG{n}{instead} \PYG{n}{then} \PYG{n}{you} \PYG{n}{must} \PYG{n+nb}{set} \PYG{n}{this} \PYG{n}{to} \PYG{n}{whatever} \PYG{o+ow}{is} \PYG{n}{seen} \PYG{k}{as} \PYG{l+s+s1}{\PYGZsq{}}\PYG{l+s+s1}{localhost}\PYG{l+s+s1}{\PYGZsq{}} \PYG{n}{to} \PYG{n}{you}\PYG{o}{.}  \PYG{n}{You} \PYG{n}{can} \PYG{n}{verify} \PYG{n}{this} \PYG{n}{by} \PYG{n}{checking} \PYG{n}{your} \PYG{o}{/}\PYG{n}{etc}\PYG{o}{/}\PYG{n}{hosts} \PYG{n}{file}\PYG{o}{.}
\end{sphinxVerbatim}

\begin{sphinxadmonition}{note}{Note:}
\sphinxAtStartPar
the sconnect\_host is optional.  If you do not specify it, it will default to two values:  \textquotesingle{}localhost\textquotesingle{} and \textquotesingle{}127.0.0.1\textquotesingle{}.
\end{sphinxadmonition}


\subsubsection{Running the stunnel setup program}
\label{\detokenize{security:running-the-stunnel-setup-program}}
\sphinxAtStartPar
At this point you\textquotesingle{}re ready to run the stunnel setup program.  So at this point type the following:

\begin{sphinxVerbatim}[commandchars=\\\{\}]
\PYG{o}{.}\PYG{o}{/}\PYG{n}{stunnel\PYGZus{}setup}\PYG{o}{.}\PYG{n}{sh}
\end{sphinxVerbatim}

\sphinxAtStartPar
This will prompt you through settings.  Most you can select the defaults to.
The SSL port you may need to change based on your administrative requirements.
It will prompt you with whatever you set for your mush name.  If you have not
selected a mush name at this point, you can select the defaults.

\sphinxAtStartPar
You will want to use the config file for warpbubble as this hides the secret.

\sphinxAtStartPar
Be aware that if you have DNS host lookups disabled on your mush, you
MUST have 127.0.0.1 as an entry for your sconnect\_host file.


\subsubsection{Starting the stunnel proxy}
\label{\detokenize{security:starting-the-stunnel-proxy}}
\sphinxAtStartPar
When you have your stunnel.conf file to the way you want, you then
issue the following command to run your SSL layer:

\begin{sphinxVerbatim}[commandchars=\\\{\}]
\PYG{o}{.}\PYG{o}{/}\PYG{n}{stunnel\PYGZus{}start}\PYG{o}{.}\PYG{n}{sh}
\end{sphinxVerbatim}


\subsubsection{Shutting down the stunnel proxy}
\label{\detokenize{security:shutting-down-the-stunnel-proxy}}
\sphinxAtStartPar
If ever you need to bring down the SSL layer, you may kill it with the command:

\begin{sphinxVerbatim}[commandchars=\\\{\}]
\PYG{o}{.}\PYG{o}{/}\PYG{n}{stunnel\PYGZus{}stop}\PYG{o}{.}\PYG{n}{sh}
\end{sphinxVerbatim}


\subsubsection{Configuring firewall on the host}
\label{\detokenize{security:configuring-firewall-on-the-host}}
\sphinxAtStartPar
Please be aware that the port that the SSL layer is on must be opened
in any firewall rule you specified to allow the connectivity.  This also must
not be the port the mush is running on and requires a separate port.


\chapter{Maintenance}
\label{\detokenize{maintenance:maintenance}}\label{\detokenize{maintenance::doc}}

\section{Note about Patching}
\label{\detokenize{maintenance:note-about-patching}}
\sphinxAtStartPar
There\textquotesingle{}s two ways you can look to patch the source.  If you plan to run the
RhostMUSH source from a git repository, then please use the git repo to
constantly update your code.  If you knew enough to want to set up a git repo
then we expect knowledge on how to keep source trees updated in the git repo
to be used the same as any other source distribution.

\sphinxAtStartPar
If, however, you have no idea what a git repo even is, or have no inclination
of using git to manage your RhostMUSH source, or just don\textquotesingle{}t care one way
or another, then you can use the included patch.sh routine (from under the
Server directory) to patch your source at any time.

\sphinxAtStartPar
From the server directory just type:

\begin{sphinxVerbatim}[commandchars=\\\{\}]
\PYG{o}{.}\PYG{o}{/}\PYG{n}{patch}\PYG{o}{.}\PYG{n}{sh}
\end{sphinxVerbatim}

\sphinxAtStartPar
That will auto\sphinxhyphen{}compile your source, auto make all your header files and
essentially keep everything up to date to the latest source.
Once that\textquotesingle{}s done, all you do from within the game is two commands:

\begin{sphinxVerbatim}[commandchars=\\\{\}]
\PYG{n+nd}{@reboot} \PYG{p}{(}\PYG{o+ow}{or} \PYG{n+nd}{@reboot}\PYG{o}{/}\PYG{n}{silent}\PYG{p}{)}  \PYG{o}{\PYGZhy{}}\PYG{o}{\PYGZhy{}} \PYG{n}{This} \PYG{n}{will} \PYG{n}{load} \PYG{o+ow}{in} \PYG{n}{the} \PYG{n}{new} \PYG{n}{binary}
\PYG{n+nd}{@readcache}  \PYG{o}{\PYGZhy{}}\PYG{o}{\PYGZhy{}} \PYG{n}{This} \PYG{n}{will} \PYG{n}{read} \PYG{o+ow}{in} \PYG{n+nb}{all} \PYG{n}{the} \PYG{o}{.}\PYG{n}{txt} \PYG{n}{file} \PYG{n}{changes}
\end{sphinxVerbatim}


\section{Daily Backups for RhostMUSH}
\label{\detokenize{maintenance:daily-backups-for-rhostmush}}
\sphinxAtStartPar
Make SURE YOU RUN DAILY Backups.  Rhost is very stable, but things outside the mush can damage the game. paranoia is fine, especially when they really are out to get you.  TO make the backups, do the following:

\begin{sphinxVerbatim}[commandchars=\\\{\}]
 \PYG{n+nd}{@dump}\PYG{o}{/}\PYG{n}{flat}      \PYG{o}{\PYGZhy{}}\PYG{o}{\PYGZhy{}} \PYG{n}{This} \PYG{n}{makes} \PYG{n}{a} \PYG{n}{flatfile} \PYG{n}{dump} \PYG{n}{of} \PYG{n}{the} \PYG{n}{main} \PYG{n}{database}\PYG{o}{.}  \PYG{n}{You} \PYG{n}{want} \PYG{n}{to} \PYG{n}{run} \PYG{n}{this} \PYG{n}{daily}\PYG{o}{.}
 \PYG{n}{wmail}\PYG{o}{/}\PYG{n}{unload}    \PYG{o}{\PYGZhy{}}\PYG{o}{\PYGZhy{}} \PYG{n}{This} \PYG{n}{makes} \PYG{n}{a} \PYG{n}{flatfile} \PYG{n}{dump} \PYG{n}{of} \PYG{n}{the} \PYG{n}{mail} \PYG{n}{database}\PYG{o}{.}  \PYG{n}{You} \PYG{n}{want} \PYG{n}{to} \PYG{n}{run} \PYG{n}{this} \PYG{n}{daily}\PYG{o}{.}
 \PYG{n+nd}{@areg}\PYG{o}{/}\PYG{n}{unload}    \PYG{o}{\PYGZhy{}}\PYG{o}{\PYGZhy{}} \PYG{n}{Only} \PYG{n}{worry} \PYG{n}{about} \PYG{n}{this} \PYG{k}{if} \PYG{n}{you} \PYG{n}{are} \PYG{n}{using} \PYG{n}{auto}\PYG{o}{\PYGZhy{}}\PYG{n}{registration} \PYG{n}{emailing}\PYG{o}{.}  \PYG{n}{Few} \PYG{n}{do}\PYG{o}{.}
 \PYG{n}{newsdb}\PYG{o}{/}\PYG{n}{unload}   \PYG{o}{\PYGZhy{}}\PYG{o}{\PYGZhy{}} \PYG{n}{Only} \PYG{n}{worry} \PYG{k}{if} \PYG{n}{you} \PYG{n}{use} \PYG{n}{the} \PYG{n}{hardcoded} \PYG{n}{bbs} \PYG{n}{system}\PYG{o}{.}  \PYG{n}{Most} \PYG{n}{don}\PYG{l+s+s1}{\PYGZsq{}}\PYG{l+s+s1}{t use it.}


\PYG{n}{Backups} \PYG{n}{are} \PYG{n}{already} \PYG{n}{handled} \PYG{o+ow}{and} \PYG{n}{integrated} \PYG{k}{with} \PYG{n}{a} \PYG{n}{script} \PYG{l+s+s1}{\PYGZsq{}}\PYG{l+s+s1}{backup\PYGZus{}flat.sh}\PYG{l+s+s1}{\PYGZsq{}}\PYG{o}{.}
\PYG{n}{If} \PYG{n}{you} \PYG{n}{wish} \PYG{n}{to} \PYG{n}{customize} \PYG{n}{this}\PYG{p}{,} \PYG{n}{feel} \PYG{n}{free}\PYG{o}{.}  \PYG{n}{Again}\PYG{p}{,} \PYG{n}{it} \PYG{o+ow}{is} \PYG{n}{well} \PYG{n}{documented} \PYG{o+ow}{and}
\PYG{n}{just} \PYG{n}{require} \PYG{n}{changing} \PYG{n}{settings} \PYG{n}{at} \PYG{n}{the} \PYG{n}{top} \PYG{n}{of} \PYG{n}{this} \PYG{n}{script}\PYG{o}{.}

\PYG{n}{By} \PYG{n}{default}\PYG{p}{,} \PYG{n}{it} \PYG{n}{does} \PYG{l+m+mi}{7} \PYG{n}{contiguous} \PYG{n}{backups}\PYG{o}{.}  \PYG{n}{You} \PYG{n}{may} \PYG{n}{increase} \PYG{o+ow}{or} \PYG{n}{decrease}
\PYG{n}{this} \PYG{n}{value} \PYG{n}{to} \PYG{n+nb}{any} \PYG{n}{value} \PYG{n}{you} \PYG{n}{want}\PYG{o}{.}

\PYG{n}{It} \PYG{n}{will}\PYG{p}{,} \PYG{n}{by} \PYG{n}{default}\PYG{p}{,} \PYG{n}{backup} \PYG{n+nb}{all} \PYG{n}{your} \PYG{n}{txt}\PYG{o}{/}\PYGZbs{}\PYG{o}{*}\PYG{o}{.}\PYG{n}{txt} \PYG{n}{files}\PYG{p}{,} \PYG{n}{your} \PYG{n}{netrhost}\PYG{o}{.}\PYG{n}{conf}
\PYG{n}{file}\PYG{p}{,} \PYG{n}{your} \PYG{n}{netrhost}\PYG{o}{.}\PYG{n}{db}\PYG{o}{.}\PYG{n}{flat} \PYG{p}{(}\PYG{n}{mush} \PYG{n}{db}\PYG{p}{)} \PYG{n}{file}\PYG{p}{,} \PYG{n}{your} \PYG{n}{RhostMUSH}\PYG{o}{.}\PYG{n}{dump}\PYG{o}{.}\PYGZbs{}\PYG{o}{*}
\PYG{p}{(}\PYG{n}{mail} \PYG{n}{db}\PYG{p}{)} \PYG{n}{files}\PYG{p}{,} \PYG{n}{your} \PYG{n}{RhostMUSH}\PYG{o}{.}\PYG{n}{news}\PYG{o}{.}\PYGZbs{}\PYG{o}{*} \PYG{p}{(}\PYG{n}{internal} \PYG{n}{news}\PYG{o}{/}\PYG{n}{bbs} \PYG{n}{db} \PYG{o}{\PYGZhy{}}\PYG{o}{\PYGZhy{}} \PYG{k}{if} \PYG{n}{used}\PYG{p}{)}\PYG{p}{,}
\PYG{n}{your} \PYG{n}{RhostMUSH}\PYG{o}{.}\PYG{n}{areg}\PYG{o}{.}\PYGZbs{}\PYG{o}{*} \PYG{p}{(}\PYG{n}{the} \PYG{n}{autoregistration} \PYG{n}{db} \PYG{o}{\PYGZhy{}}\PYG{o}{\PYGZhy{}} \PYG{k}{if} \PYG{n}{used}\PYG{p}{)}\PYG{p}{,} \PYG{o+ow}{and} \PYG{n+nb}{any} \PYG{n}{sqlite}
\PYG{n}{database} \PYG{n}{you} \PYG{n}{currently} \PYG{n}{may} \PYG{n}{be} \PYG{n}{using} \PYG{n}{which} \PYG{n}{are} \PYG{n}{OPTIONALLY} \PYG{n}{backed} \PYG{n}{up} \PYG{k}{if} \PYG{n}{you}
\PYG{n}{remove} \PYG{n}{the} \PYG{l+s+s1}{\PYGZsq{}}\PYG{l+s+s1}{\PYGZsh{}}\PYG{l+s+s1}{\PYGZsq{}} \PYG{k+kn}{from} \PYG{n+nn}{before} \PYG{n}{it}\PYG{o}{.}

\PYG{n}{The} \PYG{n}{backup} \PYG{n}{script} \PYG{n}{also} \PYG{n}{will} \PYG{n}{optionally} \PYG{n}{rcp}\PYG{o}{/}\PYG{n}{scp}\PYG{p}{,} \PYG{n}{ftp}\PYG{p}{,} \PYG{o+ow}{or} \PYG{n}{mail} \PYG{n+nb}{any} \PYG{n}{backups}
\PYG{n}{you} \PYG{n}{want} \PYG{n}{to} \PYG{n}{a} \PYG{n}{remote} \PYG{n}{destination}\PYG{o}{.}  \PYG{n}{Be} \PYG{n}{forewarned}\PYG{p}{,} \PYG{n}{the} \PYG{n}{backup} \PYG{n}{files} \PYG{n}{can}
\PYG{n}{potentially} \PYG{n}{get} \PYG{n}{rather} \PYG{n}{large} \PYG{k}{for} \PYG{n}{larger} \PYG{n}{games}\PYG{p}{,} \PYG{n}{even} \PYG{n}{compressed}\PYG{o}{.}  \PYG{n}{The}
\PYG{n}{average} \PYG{n}{size} \PYG{k}{for} \PYG{n}{these} \PYG{n}{files} \PYG{n}{will} \PYG{n}{be} \PYG{l+m+mi}{1}\PYG{o}{\PYGZhy{}}\PYG{l+m+mi}{5}\PYG{n}{MB}\PYG{o}{.}  \PYG{n}{It} \PYG{n}{could} \PYG{n}{potentially} \PYG{n}{get}
\PYG{n}{over} \PYG{l+m+mi}{10}\PYG{o}{\PYGZhy{}}\PYG{l+m+mi}{20}\PYG{n}{MB} \PYG{o+ow}{in} \PYG{n}{size} \PYG{k}{for} \PYG{n}{excessively} \PYG{n}{large} \PYG{n}{games}\PYG{p}{,} \PYG{n}{so} \PYG{n}{plan} \PYG{n}{accordingly}\PYG{o}{.}

\PYG{n}{Be} \PYG{n}{aware} \PYG{n}{that} \PYG{n}{the} \PYG{n}{backup} \PYG{n}{system} \PYG{n}{will} \PYG{n}{NOT} \PYG{n}{make} \PYG{n}{successful} \PYG{n}{backups} \PYG{k}{if} \PYG{n}{you}
\PYG{n}{run} \PYG{n}{out} \PYG{n}{of} \PYG{n}{disk} \PYG{n}{space}\PYG{o}{.}  \PYG{n}{This} \PYG{n}{includes} \PYG{n}{actually} \PYG{n}{running} \PYG{n}{out} \PYG{n}{of} \PYG{n}{disk} \PYG{n}{space}
\PYG{o+ow}{or} \PYG{n}{running} \PYG{n}{out} \PYG{n}{of} \PYG{n}{disk} \PYG{n}{quota}\PYG{o}{.}  \PYG{n}{There} \PYG{o+ow}{is} \PYG{n}{a} \PYG{n}{mechanism} \PYG{n}{inside} \PYG{n}{the} \PYG{n}{backup}
\PYG{n}{script} \PYG{n}{to} \PYG{n}{specify} \PYG{n}{an} \PYG{n}{email} \PYG{n}{address} \PYG{n}{that} \PYG{n}{you} \PYG{n}{wish} \PYG{n}{to} \PYG{n}{get} \PYG{n}{alerts} \PYG{k+kn}{from}
\PYG{n+nn}{in} \PYG{n}{these} \PYG{n}{instances}\PYG{o}{.}  \PYG{n}{I} \PYG{n}{recommend} \PYG{n}{using} \PYG{n}{it}\PYG{o}{.}

\PYG{n}{If} \PYG{n}{you} \PYG{n}{make} \PYG{n}{changes} \PYG{n}{to} \PYG{n}{your} \PYG{n}{backup\PYGZus{}flat}\PYG{o}{.}\PYG{n}{sh} \PYG{n}{script} \PYG{k}{with} \PYG{n}{an} \PYG{n}{already}
\PYG{n}{active} \PYG{o+ow}{and} \PYG{n}{running} \PYG{n}{mush} \PYG{o+ow}{and} \PYG{n}{wish} \PYG{n}{to} \PYG{n}{just} \PYG{n}{restart} \PYG{n}{the} \PYG{n}{backup} \PYG{n}{procedure}
\PYG{n}{just} \PYG{n}{issue}\PYG{p}{:}\PYG{p}{:}

   \PYG{o}{.}\PYG{o}{/}\PYG{n}{backup\PYGZus{}restart}\PYG{o}{.}\PYG{n}{sh}
\end{sphinxVerbatim}


\section{Signals and why you need them for control}
\label{\detokenize{maintenance:signals-and-why-you-need-them-for-control}}
\sphinxAtStartPar
Rhost by default allows signals at the shell to handle various processes in\sphinxhyphen{}game.

\sphinxAtStartPar
The following signals are useful.


\subsection{TERM (kill \sphinxhyphen{}TERM or kill \sphinxhyphen{}15)}
\label{\detokenize{maintenance:term-kill-term-or-kill-15}}\begin{itemize}
\item {} 
\sphinxAtStartPar
This will immediately terminate the mush, dumping a special flatfile called
netrhost.db.TERM and scramming the db in question by force\sphinxhyphen{}closing it
without any writes.  A TERM is the effort for the mush to shut down the
mush as fast as possible to avoid any db corruption if possible since
a TERM signal is common during a server shutdown, so time is paramount.

\end{itemize}


\subsection{USR1 (kill \sphinxhyphen{}USR1)}
\label{\detokenize{maintenance:usr1-kill-usr1}}\begin{itemize}
\item {} 
\sphinxAtStartPar
This by default issues a reboot on the server.  This is a special parameter
because this can actually be changed in\sphinxhyphen{}game to do any number of other
things.  Please refer to the RhostMUSH running in question if this is
the default behavior or if the method for USR1 is doing something else.

\end{itemize}


\subsection{USR2 (kill \sphinxhyphen{}USR2)}
\label{\detokenize{maintenance:usr2-kill-usr2}}\begin{itemize}
\item {} 
\sphinxAtStartPar
This will shutdown (cleanly) the mush.  This behaves as if you issued
a @shutdown from within the game, and follows all proper procedure
in bringing the game down cleanly and safely.  This shoudl be used
when doing maintenance on the game or if you need to bring it down
from the shell.

\end{itemize}


\subsection{KILL (kill \sphinxhyphen{}KILL or kill \sphinxhyphen{}9)}
\label{\detokenize{maintenance:kill-kill-kill-or-kill-9}}\begin{itemize}
\item {} 
\sphinxAtStartPar
This signal can not be caught and will immediately terminate the game
without any safty to the database at all.  Short of something horribly
wrong going on, this should never be used to bring down your mush.
Doing so will almost certainly corrupt your databases (ALL OF THEM)
that are open, including but not limited to your main database, your
mail database, your autoregistration database, and so forth.  So if
you do this, plan to do some database recovery from your flat files.
Also, when you bring down a mush in this manner, you need to issue
Startmush \sphinxhyphen{}f to bring it back up.

\end{itemize}


\section{Shutting down RhostMUSH gracefully}
\label{\detokenize{maintenance:shutting-down-rhostmush-gracefully}}

\subsection{RhostMUSH has many ways to shutdown the game cleanly}
\label{\detokenize{maintenance:rhostmush-has-many-ways-to-shutdown-the-game-cleanly}}\begin{enumerate}
\sphinxsetlistlabels{\arabic}{enumi}{enumii}{}{.}%
\item {} 
\sphinxAtStartPar
Log into the mush and issue @shutdown

\item {} 
\sphinxAtStartPar
Issue a kill \sphinxhyphen{}USR2 to the mush which issues an emergency @shutdown

\item {} 
\sphinxAtStartPar
Issue a kill \sphinxhyphen{}TERM to the mush which issues an emergency abort and clean shutdown.

\item {} 
\sphinxAtStartPar
Through the Autoshutdown script

\end{enumerate}

\begin{sphinxadmonition}{warning}{Warning:}
\sphinxAtStartPar
Never kill \sphinxhyphen{}9 RhostMUSH

\sphinxAtStartPar
Under NO CIRCUMSTANCES should you kill \sphinxhyphen{}9 your mush unless you don\textquotesingle{}t care for the
database.  The reason is if the mush happens to be saving, in any method, to the
database, especially a QDBM database, you will likely have just corrupted your
database, so pull out a flatfile to recover.

\sphinxAtStartPar
Sadly, this also may occur if the server hosting you takes a nose\sphinxhyphen{}dive in the middle
of a db write.  Rhost can recover corruption in\sphinxhyphen{}game while up, but if it bombs
in the middle of a write, all bets are off. :)
\end{sphinxadmonition}


\subsection{Autoshutdown script}
\label{\detokenize{maintenance:autoshutdown-script}}
\sphinxAtStartPar
The makefile will \textquotesingle{}make\textquotesingle{} the program that will STOP the mush.
Please tweek \textquotesingle{}autolog.h\textquotesingle{} with the proper parameters.

\sphinxAtStartPar
The \textquotesingle{}startup.sh\textquotesingle{} will START the mush.

\sphinxAtStartPar
Both of these are intended to be used for automations (automated processes)
like your rc.local file and/or startup scripts when you bring your server up.


\section{Network Port redirector}
\label{\detokenize{maintenance:network-port-redirector}}
\sphinxAtStartPar
This is a port redirector incase you decide to move your mush
to a new site/port.  To use, first, compile the code.  To do
this you would type the following:

\begin{sphinxVerbatim}[commandchars=\\\{\}]
\PYG{n}{cc} \PYG{n}{portmsg}\PYG{o}{.}\PYG{n}{c} \PYG{o}{\PYGZhy{}}\PYG{n}{o} \PYG{n}{portmsg}
\end{sphinxVerbatim}

\sphinxAtStartPar
if \textquotesingle{}cc\textquotesingle{} is not defined, try the following:

\begin{sphinxVerbatim}[commandchars=\\\{\}]
\PYG{n}{gcc} \PYG{n}{portmsg}\PYG{o}{.}\PYG{n}{c} \PYG{o}{\PYGZhy{}}\PYG{n}{o} \PYG{n}{portmsg}
\end{sphinxVerbatim}

\sphinxAtStartPar
Once compiled, you would then modify the file \textquotesingle{}file\textquotesingle{} to describe
the mush, what was done, where it\textquotesingle{}s moved to, then specify the
IP address and the PORT where specified.

\sphinxAtStartPar
To launch the application, you would then type:

\begin{sphinxVerbatim}[commandchars=\\\{\}]
\PYG{o}{.}\PYG{o}{/}\PYG{n}{portmsg} \PYG{n}{file} \PYG{o}{\PYGZlt{}}\PYG{n}{port}\PYG{o}{\PYGZgt{}}
\end{sphinxVerbatim}

\sphinxAtStartPar
Where \textless{}port\textgreater{} is the port where the mush used to run at.


\section{Using the built\sphinxhyphen{}in cron system for periodically running commands}
\label{\detokenize{maintenance:using-the-built-in-cron-system-for-periodically-running-commands}}

\subsection{Syntax for rhost.cron}
\label{\detokenize{maintenance:syntax-for-rhost-cron}}
\sphinxAtStartPar
The rhost.cron file will be in the syntax as follows:

\begin{sphinxVerbatim}[commandchars=\\\{\}]
\PYG{n}{username} \PYG{p}{(}\PYG{o+ow}{or} \PYG{n}{dbref}\PYG{c+c1}{\PYGZsh{})}
\PYG{n}{command1}\PYG{p}{;}\PYG{n}{command2}\PYG{p}{;}\PYG{n}{command3}\PYG{p}{;}\PYG{o}{.}\PYG{o}{.}\PYG{o}{.}\PYG{p}{;}\PYG{n}{commandN}
\PYG{n}{command}
\PYG{n}{command}
\PYG{n}{command1}\PYG{p}{;}\PYG{n}{command2}\PYG{p}{;}\PYG{n}{command3}\PYG{p}{;}\PYG{o}{.}\PYG{o}{.}\PYG{o}{.}\PYG{p}{;}\PYG{n}{commandN}
\end{sphinxVerbatim}

\sphinxAtStartPar
You can have commands strung together with a semicolon
on the same line.  This counts as a single line of input.
You can have at most 20 lines of commands after the target
you wish to execute the commands as.  The target may
be a player name OR a dbref\# of any valid dbref\# within
the game.  Invalid targets will abort the cron process.
Non\sphinxhyphen{}printable characters in the cron file will abort
the process.  Any aborts or warnings will be logged
in the netrhost.gamelog.

\sphinxAtStartPar
Here is a working example of the code cron file.
This example will perform dumps of the mush.


\subsubsection{Example syntaxt for rhost.cron}
\label{\detokenize{maintenance:example-syntaxt-for-rhost-cron}}
\begin{sphinxVerbatim}[commandchars=\\\{\}]
\PYG{c+c1}{\PYGZsh{}1}
\PYG{n+nd}{@dump}\PYG{o}{/}\PYG{n}{flat}\PYG{p}{;} \PYG{o}{@}\PYG{o}{@} \PYG{n}{dump} \PYG{n}{the} \PYG{n}{main} \PYG{n}{game} \PYG{n}{database} \PYG{n}{to} \PYG{n}{flatfile}
\PYG{n}{wmail}\PYG{o}{/}\PYG{n}{unload}\PYG{p}{;} \PYG{o}{@}\PYG{o}{@} \PYG{n}{dump} \PYG{n}{the} \PYG{n}{mail} \PYG{n}{database} \PYG{n}{to} \PYG{n}{flatfile}
\PYG{n+nd}{@areg}\PYG{o}{/}\PYG{n}{unload}\PYG{p}{;} \PYG{o}{@}\PYG{o}{@} \PYG{n}{dump} \PYG{n}{the} \PYG{n}{registration} \PYG{n}{database} \PYG{n}{to} \PYG{n}{flatfile}
\PYG{n}{newsdb}\PYG{o}{/}\PYG{n}{unload}\PYG{p}{;} \PYG{o}{@}\PYG{o}{@} \PYG{n}{dump} \PYG{n}{the} \PYG{n}{news} \PYG{n}{bbs} \PYG{n}{database} \PYG{n}{to} \PYG{n}{flatfile}
\end{sphinxVerbatim}


\section{The following scripts are used in the game directory}
\label{\detokenize{maintenance:the-following-scripts-are-used-in-the-game-directory}}
\begin{sphinxVerbatim}[commandchars=\\\{\}]
\PYG{n}{Startmush}               \PYG{o}{\PYGZhy{}}\PYG{o}{\PYGZhy{}} \PYG{n}{used} \PYG{n}{to} \PYG{n}{Start} \PYG{n}{up} \PYG{n}{the} \PYG{n}{mush}
\PYG{n}{backup\PYGZus{}flat}\PYG{o}{.}\PYG{n}{sh}          \PYG{o}{\PYGZhy{}}\PYG{o}{\PYGZhy{}} \PYG{n}{Used} \PYG{n}{to} \PYG{n}{run} \PYG{n}{backups} \PYG{k}{with} \PYG{n+nd}{@dump}\PYG{o}{/}\PYG{n}{flat} \PYG{n}{within} \PYG{n}{the} \PYG{n}{game} \PYG{p}{(}\PYG{n}{Started} \PYG{k}{with} \PYG{n}{Startmush} \PYG{n}{automatically}\PYG{p}{)}
\PYG{n}{backup\PYGZus{}restart}\PYG{o}{.}\PYG{n}{sh}       \PYG{o}{\PYGZhy{}}\PYG{o}{\PYGZhy{}} \PYG{n}{Restart} \PYG{n}{the} \PYG{n}{backup\PYGZus{}flat}\PYG{o}{.}\PYG{n}{sh} \PYG{k}{if} \PYG{n}{changes} \PYG{n}{are} \PYG{n}{done}
\PYG{n}{compress\PYGZus{}logs}\PYG{o}{.}\PYG{n}{sh}        \PYG{o}{\PYGZhy{}}\PYG{o}{\PYGZhy{}} \PYG{n}{Compress} \PYG{n}{logs} \PYG{o+ow}{in} \PYG{l+s+s1}{\PYGZsq{}}\PYG{l+s+s1}{oldlogs}\PYG{l+s+s1}{\PYGZsq{}}\PYG{o}{.}  \PYG{n}{Ran} \PYG{k}{with} \PYG{n}{Startmush}
\PYG{n}{findit}\PYG{o}{.}\PYG{n}{sh}               \PYG{o}{\PYGZhy{}}\PYG{o}{\PYGZhy{}} \PYG{n}{Internal} \PYG{n}{script} \PYG{n}{used} \PYG{n}{to} \PYG{n}{check} \PYG{k}{for} \PYG{n}{flatfile} \PYG{n}{validity}
\PYG{n}{mailhide}\PYG{o}{.}\PYG{n}{sh}             \PYG{o}{\PYGZhy{}}\PYG{o}{\PYGZhy{}} \PYG{n}{Wrapper} \PYG{n}{to} \PYG{n}{hide} \PYG{k+kn}{from} \PYG{n+nn}{address} \PYG{n}{using} \PYG{n}{the} \PYG{l+s+s1}{\PYGZsq{}}\PYG{l+s+s1}{mail}\PYG{l+s+s1}{\PYGZsq{}} \PYG{n}{progam}
\PYG{n}{minimal}\PYG{o}{.}\PYG{n}{sh}              \PYG{o}{\PYGZhy{}}\PYG{o}{\PYGZhy{}} \PYG{n}{Auto}\PYG{o}{\PYGZhy{}}\PYG{n}{load} \PYG{n}{the} \PYG{n}{minimal} \PYG{n}{db} \PYG{n}{into} \PYG{n}{the} \PYG{n}{mush}
\PYG{n}{proxysnarf}\PYG{o}{.}\PYG{n}{sha}          \PYG{o}{\PYGZhy{}}\PYG{o}{\PYGZhy{}} \PYG{n}{Internal} \PYG{n}{script} \PYG{k}{for} \PYG{n}{the} \PYG{n}{tor\PYGZus{}pull}\PYG{o}{.}\PYG{n}{sh} \PYG{n}{tor} \PYG{n}{proxy} \PYG{n}{blacklist}
\PYG{n}{tor\PYGZus{}pullit}\PYG{o}{.}\PYG{n}{sh}           \PYG{o}{\PYGZhy{}}\PYG{o}{\PYGZhy{}} \PYG{n}{Internal} \PYG{n}{script} \PYG{k}{for} \PYG{n}{the} \PYG{n}{tor\PYGZus{}pull}\PYG{o}{.}\PYG{n}{sh} \PYG{k}{for} \PYG{n}{proxy} \PYG{n}{blacklist}
\PYG{n}{recovery}\PYG{o}{.}\PYG{n}{sh}             \PYG{o}{\PYGZhy{}}\PYG{o}{\PYGZhy{}} \PYG{n}{If} \PYG{n}{your} \PYG{n}{db} \PYG{o+ow}{is} \PYG{n}{corrupt}\PYG{p}{,} \PYG{n}{run} \PYG{n}{this} \PYG{n}{to} \PYG{n}{auto}\PYG{o}{\PYGZhy{}}\PYG{n}{revert} \PYG{n}{to} \PYG{n}{an} \PYG{n}{earlier} \PYG{n}{flatfile}
\PYG{n}{tor\PYGZus{}pull}\PYG{o}{.}\PYG{n}{sh}             \PYG{o}{\PYGZhy{}}\PYG{o}{\PYGZhy{}} \PYG{n}{Create} \PYG{n}{a} \PYG{n}{blacklist}\PYG{o}{.}\PYG{n}{txt} \PYG{n}{file} \PYG{n}{that} \PYG{n}{can} \PYG{n}{be} \PYG{n}{loaded} \PYG{n}{via} \PYG{n}{the} \PYG{n}{internal} \PYG{n+nd}{@blacklist} \PYG{n}{command} \PYG{k}{for} \PYG{n}{proxy} \PYG{n}{handling}
\end{sphinxVerbatim}


\section{Textfiles for RhostMUSH}
\label{\detokenize{maintenance:textfiles-for-rhostmush}}
\begin{sphinxadmonition}{note}{\label{\detokenize{maintenance:id1}}Todo:}
\sphinxAtStartPar
Notate which need mkindx
\end{sphinxadmonition}

\begin{sphinxVerbatim}[commandchars=\\\{\}]
\PYG{n}{areghost}\PYG{o}{.}\PYG{n}{txt}           \PYG{o}{\PYGZhy{}} \PYG{n}{file} \PYG{n}{player} \PYG{n}{gets} \PYG{n}{when} \PYG{n}{autoregistration} \PYG{n}{on} \PYG{n}{registered} \PYG{n}{host}\PYG{o}{.}
\PYG{n}{autoreg}\PYG{o}{.}\PYG{n}{txt}            \PYG{o}{\PYGZhy{}} \PYG{n}{file} \PYG{n}{player} \PYG{n}{gets} \PYG{n}{when} \PYG{n}{autoregistration} \PYG{n}{on} \PYG{n}{non}\PYG{o}{\PYGZhy{}}\PYG{n}{registered} \PYG{n}{host}\PYG{o}{.}
\PYG{n}{autoreg\PYGZus{}include}\PYG{o}{.}\PYG{n}{txt}    \PYG{o}{\PYGZhy{}} \PYG{n}{file} \PYG{n}{player} \PYG{n}{receives} \PYG{o+ow}{in} \PYG{n}{email} \PYG{n}{when} \PYG{n}{they} \PYG{n}{autoregister} \PYG{n}{attached} \PYG{n}{to} \PYG{n}{login}\PYG{o}{/}\PYG{n}{passwd}
\PYG{n}{badsite}\PYG{o}{.}\PYG{n}{txt}            \PYG{o}{\PYGZhy{}} \PYG{n}{file} \PYG{n}{player} \PYG{n}{gets} \PYG{n}{when} \PYG{n}{site} \PYG{o+ow}{is} \PYG{o+ow}{not} \PYG{n}{allowed}\PYG{o}{.}
\PYG{n}{connect}\PYG{o}{.}\PYG{n}{txt}            \PYG{o}{\PYGZhy{}} \PYG{n}{file} \PYG{n}{player} \PYG{n}{gets} \PYG{n}{when} \PYG{n}{connect}
\PYG{n}{create\PYGZus{}reg}\PYG{o}{.}\PYG{n}{txt}         \PYG{o}{\PYGZhy{}} \PYG{n}{file} \PYG{n}{player} \PYG{n}{gets} \PYG{n}{when} \PYG{n}{their} \PYG{n}{site} \PYG{o+ow}{is} \PYG{n}{register} \PYG{o+ow}{and} \PYG{n}{they} \PYG{n}{can}\PYG{l+s+s1}{\PYGZsq{}}\PYG{l+s+s1}{t create.}
\PYG{n}{doorconf}\PYG{o}{.}\PYG{n}{txt}           \PYG{o}{\PYGZhy{}} \PYG{n}{file} \PYG{n}{that} \PYG{o+ow}{is} \PYG{n}{searched} \PYG{k}{for} \PYG{n}{information} \PYG{n}{regarding} \PYG{n+nd}{@door}\PYG{o}{.}
\PYG{n}{down}\PYG{o}{.}\PYG{n}{txt}               \PYG{o}{\PYGZhy{}} \PYG{n}{file} \PYG{n}{player} \PYG{n}{gets} \PYG{n}{when} \PYG{n}{the} \PYG{n}{mush} \PYG{n}{has} \PYG{n}{logins} \PYG{n}{disabled} \PYG{p}{(}\PYG{n+nd}{@disable} \PYG{n}{login}\PYG{p}{)}
\PYG{n}{error}\PYG{o}{.}\PYG{n}{txt}              \PYG{o}{\PYGZhy{}} \PYG{n}{the} \PYG{l+s+s1}{\PYGZsq{}}\PYG{l+s+s1}{Huh? (type help for help)}\PYG{l+s+s1}{\PYGZsq{}} \PYG{n}{messages}\PYG{o}{.}
\PYG{n}{full}\PYG{o}{.}\PYG{n}{txt}               \PYG{o}{\PYGZhy{}} \PYG{n}{file} \PYG{n}{player} \PYG{n}{gets} \PYG{n}{when} \PYG{n}{the} \PYG{n}{mush} \PYG{n}{can}\PYG{l+s+s1}{\PYGZsq{}}\PYG{l+s+s1}{t have any more players.}
\PYG{n}{guest}\PYG{o}{.}\PYG{n}{txt}              \PYG{o}{\PYGZhy{}} \PYG{n}{file} \PYG{n}{player} \PYG{n}{gets} \PYG{n}{when} \PYG{n}{they} \PYG{n}{connect} \PYG{k}{as} \PYG{n}{a} \PYG{n}{guest}\PYG{o}{.}
\PYG{n}{help}\PYG{o}{.}\PYG{n}{txt}               \PYG{o}{\PYGZhy{}} \PYG{n}{your} \PYG{n}{help} \PYG{n}{file}
\PYG{n}{motd}\PYG{o}{.}\PYG{n}{txt}               \PYG{o}{\PYGZhy{}} \PYG{n}{your} \PYG{n}{motd} \PYG{n}{file}
\PYG{n}{news}\PYG{o}{.}\PYG{n}{txt}               \PYG{o}{\PYGZhy{}} \PYG{n}{your} \PYG{n}{news} \PYG{n}{file}
\PYG{n}{newuser}\PYG{o}{.}\PYG{n}{txt}            \PYG{o}{\PYGZhy{}} \PYG{n}{file} \PYG{n}{newly} \PYG{n}{created} \PYG{n}{players} \PYG{n}{get} \PYG{n}{when} \PYG{n}{they} \PYG{n}{connect} \PYG{k}{for} \PYG{n}{the} \PYG{n}{first} \PYG{n}{time}\PYG{o}{.}
\PYG{n}{noguest}\PYG{o}{.}\PYG{n}{txt}            \PYG{o}{\PYGZhy{}} \PYG{n}{file} \PYG{n}{player} \PYG{n}{gets} \PYG{n}{when} \PYG{n}{they} \PYG{n}{are} \PYG{o+ow}{not} \PYG{n}{allowed} \PYG{n}{to} \PYG{n}{connect} \PYG{n}{to} \PYG{n}{a} \PYG{n}{guest}\PYG{o}{.}
\PYG{n}{plushelp}\PYG{o}{.}\PYG{n}{txt}           \PYG{o}{\PYGZhy{}} \PYG{n}{optional} \PYG{o}{+}\PYG{n}{help} \PYG{n}{file}\PYG{o}{.} \PYG{p}{(}\PYG{n}{needs} \PYG{n+nb}{compile} \PYG{n}{time} \PYG{n}{option}\PYG{p}{)}
\PYG{n}{quit}\PYG{o}{.}\PYG{n}{txt}               \PYG{o}{\PYGZhy{}} \PYG{n}{file} \PYG{n}{player} \PYG{n}{gets} \PYG{n}{when} \PYG{n}{they} \PYG{n}{disconnect}\PYG{o}{.}
\PYG{n}{register}\PYG{o}{.}\PYG{n}{txt}           \PYG{o}{\PYGZhy{}} \PYG{n}{file} \PYG{n}{player} \PYG{n}{gets} \PYG{n}{when} \PYG{n}{the} \PYG{n}{site} \PYG{o+ow}{is} \PYG{n}{locked} \PYG{n}{down} \PYG{k}{for} \PYG{n}{registration}\PYG{o}{.}
\PYG{n}{wizhelp}\PYG{o}{.}\PYG{n}{txt}            \PYG{o}{\PYGZhy{}} \PYG{n}{your} \PYG{n}{wizhelp} \PYG{n}{file}
\PYG{n}{wizmotd}\PYG{o}{.}\PYG{n}{txt}            \PYG{o}{\PYGZhy{}} \PYG{n}{your} \PYG{n}{wiz} \PYG{n}{motd} \PYG{n}{file}
\end{sphinxVerbatim}


\subsection{Textfile Frequently Asked Questions}
\label{\detokenize{maintenance:textfile-frequently-asked-questions}}
\sphinxAtStartPar
Q:  How do I put color in these files?

\sphinxAtStartPar
A1: Look at ansi.h and you need to put the literal ASCII codes.  They will look like: \textasciicircum{}{[}{[}0m (for ANSI\_NORMAL).  That\textquotesingle{}s \textless{}ESC\textgreater{}{[}

\sphinxAtStartPar
A2: You can enable ansi\_txtfiles then use \%c (or \%x/\%m) encoding for ansi, however you compiled your Rhost.

\sphinxAtStartPar
Q:  I want to design my own txt files to read in the mush.

\sphinxAtStartPar
A:  Easy.  Design them like help.txt would be set up, mkindx the file, then you can access it via @dynhelp online.

\sphinxAtStartPar
Q:  Do I have to mkindx these files whenever I make changes?

\sphinxAtStartPar
A:  Only the ones that have \textquotesingle{}\& \textquotesingle{} index. (help.txt, wizhelp.txt, news.txt, etc)

\sphinxAtStartPar
Q:  Do I have to @readcache in the game whenever I make a change?

\sphinxAtStartPar
A:  Only when you modify any of the files listed in README.TXTFILES.  Not the ones you use with @dynhelp.

\sphinxAtStartPar
Q:  Can\textquotesingle{}t I just make code in the mush that then is used for these silly txt files?

\sphinxAtStartPar
A:  Absolutely.  Check \textquotesingle{}wizhelp file\_object\textquotesingle{}.


\chapter{Troubleshooting}
\label{\detokenize{troubleshooting:troubleshooting}}\label{\detokenize{troubleshooting::doc}}

\section{Reporting bugs or getting help}
\label{\detokenize{troubleshooting:reporting-bugs-or-getting-help}}
\sphinxAtStartPar
If you find any bugs or problems, notify one of the developers of RhostMUSH and
a patch or workaround will be made available as soon as possible.  Current
developers are:  Seawolf, Thorin, Ashen\sphinxhyphen{}Shugar, Lensman, Kale, Mac, Zenty,
Ambrosia, Amos, and Morgan.  They can be found around the net.


\subsection{Troubleshooting issues with starting up}
\label{\detokenize{troubleshooting:troubleshooting-issues-with-starting-up}}

\subsubsection{Problem: If it says the shared ID is already in use}
\label{\detokenize{troubleshooting:problem-if-it-says-the-shared-id-is-already-in-use}}
\sphinxAtStartPar
A1: please verify that it is the right shared debug\_id in your netrhost.conf file

\sphinxAtStartPar
A2: Force a start by running:

\begin{sphinxVerbatim}[commandchars=\\\{\}]
\PYG{o}{.}\PYG{o}{/}\PYG{n}{Startmush} \PYG{o}{\PYGZhy{}}\PYG{n}{f}
\end{sphinxVerbatim}


\subsubsection{Problem: Your log file is massive and your mush is running}
\label{\detokenize{troubleshooting:problem-your-log-file-is-massive-and-your-mush-is-running}}
\sphinxAtStartPar
A1: To rotate this use the @logrotate command. See wizhelp on @logrotate


\subsubsection{Problem: The database flatfile you\textquotesingle{}re loading can\textquotesingle{}t load because a db is already defined}
\label{\detokenize{troubleshooting:problem-the-database-flatfile-you-re-loading-can-t-load-because-a-db-is-already-defined}}
\sphinxAtStartPar
A1: remove netrhost.db* and netrhost.gdbm* from your data directory


\subsubsection{Problem: The mail database won\textquotesingle{}t load and mail shows \textquotesingle{}offline\textquotesingle{}}
\label{\detokenize{troubleshooting:problem-the-mail-database-won-t-load-and-mail-shows-offline}}
\sphinxAtStartPar
A1: from within the MUSH run:

\begin{sphinxVerbatim}[commandchars=\\\{\}]
\PYG{n}{wmail}\PYG{o}{/}\PYG{n}{load}
\end{sphinxVerbatim}


\section{Stack limit and debugging}
\label{\detokenize{troubleshooting:stack-limit-and-debugging}}
\sphinxAtStartPar
Rhost uses a stack limit in the debug monitor.

\sphinxAtStartPar
This stack limit is set to a reasonable amount of 1000.
This is defined in the debug.h file in the hdrs directory.

\sphinxAtStartPar
This will directly impact the function\_recursion\_limit from being
set above 100.  If, for whatever reason, you really must have
a ridiculously high recursion limit, then it is a suggestion to
manually modify the stack limit in debug.h to a higher number.

\sphinxAtStartPar
We have reasonably set it to 10000 without too much issue, but keep
in mind, the overhead is higher for every stack you throw on the
process table.  Higher stack means more memory used.

\sphinxAtStartPar
Also be aware that your shell stack limit directly is affected
to this value.

\sphinxAtStartPar
Type:

\begin{sphinxVerbatim}[commandchars=\\\{\}]
\PYG{n}{ulimit} \PYG{o}{\PYGZhy{}}\PYG{n}{a}
\end{sphinxVerbatim}

\sphinxAtStartPar
This will show you your shell stack limits.  Do NOT set the
STACKMAX value higher than your shell\textquotesingle{}s stack value.

\sphinxAtStartPar
The value in \textasciitilde{}/Rhost/Server/hdrs/debug.h is currently set as:

\begin{sphinxVerbatim}[commandchars=\\\{\}]
\PYG{c+c1}{\PYGZsh{}define STACKMAX 1000}
\end{sphinxVerbatim}

\sphinxAtStartPar
Feel free to change this to a higher value if you wish.

\sphinxAtStartPar
The caveat.  This effects the debug stack daemon.  Meaning,
the only way for this to be updated is through @shutdown and
then a fresh ./Startmush.

\sphinxAtStartPar
A @reboot WILL NOT LOAD IN A NEW DEBUG MONITOR!!!!

\sphinxAtStartPar
You can issue @list stack to see the current stack ceiling ingame.


\section{How to reset the password for \#1}
\label{\detokenize{troubleshooting:how-to-reset-the-password-for-1}}
\begin{sphinxadmonition}{warning}{Warning:}
\sphinxAtStartPar
You can only use one of these options at a time. Make sure to change back your nerhost.conf and then reboot after making the changes.
\end{sphinxadmonition}


\subsection{Method 1: Reset to Default Password}
\label{\detokenize{troubleshooting:method-1-reset-to-default-password}}
\sphinxAtStartPar
in your netrhost.conf file add:

\begin{sphinxVerbatim}[commandchars=\\\{\}]
\PYG{n}{newpass\PYGZus{}god} \PYG{l+m+mi}{777}
\end{sphinxVerbatim}

\sphinxAtStartPar
This will reset \#1\textquotesingle{}s password to the default \textquotesingle{}Nyctasia\textquotesingle{}.


\subsection{Method 2: Increase Permissions of Immortals}
\label{\detokenize{troubleshooting:method-2-increase-permissions-of-immortals}}
\sphinxAtStartPar
in your netrhost.conf file add:

\begin{sphinxVerbatim}[commandchars=\\\{\}]
\PYG{n}{newpass\PYGZus{}god} \PYG{l+m+mi}{1}
\end{sphinxVerbatim}

\sphinxAtStartPar
This will allow IMMORTAL players to @newpassword \#1 upon reboot.


\section{Troubleshooting difficulties compiling RhostMUSH}
\label{\detokenize{troubleshooting:troubleshooting-difficulties-compiling-rhostmush}}

\subsection{Changes to conf for high\sphinxhyphen{}bit CPUs}
\label{\detokenize{troubleshooting:changes-to-conf-for-high-bit-cpus}}
\sphinxAtStartPar
RhostMUSH automatically detects 64\sphinxhyphen{}bit platforms, and should compile
cleanly on these.

\sphinxAtStartPar
In case you are trying to compile Rhost on some other crazy\sphinxhyphen{}wide CPUs
such as the PS2, PS3 or other 128 or 256 bit CPUs, you can easily do
so by changing a few lines of code in conf.c.

\sphinxAtStartPar
change:

\begin{sphinxVerbatim}[commandchars=\\\{\}]
\PYG{n}{typedef} \PYG{n}{unsigned} \PYG{n+nb}{int}    \PYG{n}{pmath1}\PYG{p}{;}
\PYG{n}{typedef} \PYG{n+nb}{int}             \PYG{n}{pmath2}\PYG{p}{;}
\PYG{c+c1}{\PYGZsh{}define ALLIGN1 4}
\end{sphinxVerbatim}

\sphinxAtStartPar
to:

\begin{sphinxVerbatim}[commandchars=\\\{\}]
\PYG{n}{typedef} \PYG{n}{unsigned} \PYG{n}{long}   \PYG{n}{pmath1}\PYG{p}{;}
\PYG{n}{typedef} \PYG{n}{long}            \PYG{n}{pmath2}\PYG{p}{;}
\PYG{c+c1}{\PYGZsh{}define ALLIGN1 8}
\end{sphinxVerbatim}

\begin{sphinxadmonition}{note}{Note:}
\sphinxAtStartPar
Replace 8 with the size of your CPU\textquotesingle{}s long integer. (4 for 32\sphinxhyphen{}bit,
8 for 64\sphinxhyphen{}bit, 16 for 128\sphinxhyphen{}bit, etc etc)
\end{sphinxadmonition}

\sphinxAtStartPar
RhostMUSH has only been tested to work on the AMD64, but there is no
reason to believe the same will not hold true for IA64.


\subsection{Changes to autconf for certain systems}
\label{\detokenize{troubleshooting:changes-to-autconf-for-certain-systems}}
\sphinxAtStartPar
You should not have to worry about this, but incase something really
weird occurs, you may need to look into these changes...

\sphinxAtStartPar
The autoconfig.h file needs to have modifications to it by hand.

\sphinxAtStartPar
There are three manual entries:

\sphinxAtStartPar
This one sets how it defines the int to character pointer.  It\textquotesingle{}s safe
to keep it as \textquotesingle{}unsigned int\textquotesingle{} for 32 bit platforms.  For non 32\sphinxhyphen{}bit,
define it to  how an int is defined on that system:

\begin{sphinxVerbatim}[commandchars=\\\{\}]
\PYG{n}{typedef} \PYG{n}{unsigned} \PYG{n+nb}{int}    \PYG{n}{pmath1}\PYG{p}{;}
\end{sphinxVerbatim}

\sphinxAtStartPar
This one sets how it defines the signed int to character pointer.  Same
restrictions apply as above for unsigned int:

\begin{sphinxVerbatim}[commandchars=\\\{\}]
\PYG{n}{typedef} \PYG{n+nb}{int}     \PYG{n}{pmath2}\PYG{p}{;}
\end{sphinxVerbatim}

\sphinxAtStartPar
This sets the allignment for the given platform.  4 represents a 32 bit
platform.  8 would represent a 64 bit platform, etc.  Change accordingly:

\begin{sphinxVerbatim}[commandchars=\\\{\}]
\PYG{c+c1}{\PYGZsh{}define ALLIGN1 4}
\end{sphinxVerbatim}

\begin{sphinxadmonition}{warning}{Warning:}
\sphinxAtStartPar
Make sure these three entries are defined in your autoconf.h file else
the mush will not compile.
\end{sphinxadmonition}


\section{Dealing with DB Corruption}
\label{\detokenize{troubleshooting:dealing-with-db-corruption}}
\sphinxAtStartPar
Ok.  Your database won\textquotesingle{}t come up.

\sphinxAtStartPar
If you are reading this, then likely the scenerio is one of the following:
\begin{enumerate}
\sphinxsetlistlabels{\arabic}{enumi}{enumii}{}{.}%
\item {} 
\sphinxAtStartPar
The mush says it can\textquotesingle{}t find your database files.

\item {} 
\sphinxAtStartPar
The mush says it can\textquotesingle{}t read or load your database files.

\item {} 
\sphinxAtStartPar
The mush seems to load fine but I can\textquotesingle{}t log in anyone and when I do
all the names and attributes of things seem to be gone!

\item {} 
\sphinxAtStartPar
Bringing up your mail database

\end{enumerate}

\sphinxAtStartPar
First thing is first.  Don\textquotesingle{}t have a panic attack.


\subsection{If the mush says it can\textquotesingle{}t find your database files}
\label{\detokenize{troubleshooting:if-the-mush-says-it-can-t-find-your-database-files}}

\subsubsection{Check the names of the database files in your \textquotesingle{}data\textquotesingle{} directory}
\label{\detokenize{troubleshooting:check-the-names-of-the-database-files-in-your-data-directory}}
\sphinxAtStartPar
They should be named something like:

\begin{sphinxVerbatim}[commandchars=\\\{\}]
\PYG{n}{netrhost}\PYG{o}{.}\PYG{n}{db}
\PYG{n}{netrhost}\PYG{o}{.}\PYG{n}{db}\PYG{o}{.}\PYG{n}{old}
\PYG{n}{netrhost}\PYG{o}{.}\PYG{n}{db}\PYG{o}{.}\PYG{n}{old}\PYG{o}{.}\PYG{n}{prev}
\PYG{n}{netrhost}\PYG{o}{.}\PYG{n}{gdbm}\PYG{o}{.}\PYG{n}{dir}
\PYG{n}{netrhost}\PYG{o}{.}\PYG{n}{gdbm}\PYG{o}{.}\PYG{n}{pag}
\end{sphinxVerbatim}

\sphinxAtStartPar
And you may see a netrhost.db.flat


\subsubsection{Check your netrhost.conf file}
\label{\detokenize{troubleshooting:check-your-netrhost-conf-file}}
\sphinxAtStartPar
If you never touched the *database or muddb\_name params, you should be good.

\sphinxAtStartPar
Verify your *database params (and muddb\_name) are still set to \textquotesingle{}netrhost\textquotesingle{} as
part of the name.  Ergo, the default values and you didn\textquotesingle{}t change them.
These should match up with the filenames in your data directory.

\sphinxAtStartPar
If these names do not match up, it can\textquotesingle{}t find the database files to load.
So you shouldn\textquotesingle{}t have to change these names, ever. :)


\subsubsection{Check your mush.config file}
\label{\detokenize{troubleshooting:check-your-mush-config-file}}
\sphinxAtStartPar
If you never modified this file, you should be good.

\sphinxAtStartPar
The gamename should be \textquotesingle{}netrhost\textquotesingle{} for this file.  This does NOT control
the name of your game.  This controls the name of all the files
as associated to the mush.  So changing this means the netrhost.conf
file, all your database files, and so forth.  Please don\textquotesingle{}t change this :)


\subsection{If the mush says it can\textquotesingle{}t read or load your database files}
\label{\detokenize{troubleshooting:if-the-mush-says-it-can-t-read-or-load-your-database-files}}
\sphinxAtStartPar
Double check everything for the previous issue. Make sure everything is named properly.


\subsubsection{Verify you have enough disk space. (quota)}
\label{\detokenize{troubleshooting:verify-you-have-enough-disk-space-quota}}
\sphinxAtStartPar
Some account have a limited quota to run in.  If you reached or exceed
your disk quota, you can have a corrupted database.  So always keep
your eye on the size.  quota \sphinxhyphen{}s to see a human readable format to see
how much quota you have left.  You want to make sure current in use is
below the \textquotesingle{}grace\textquotesingle{} and soft/hard limits shown.  If not, you\textquotesingle{}re out of
space.

\sphinxAtStartPar
You will need to remove some files before you repair and bring up your
mush again.  Try to keep your quota at least 200 megs free to allow
plenty of wonderful growth space for the mush.


\subsubsection{Verify you have enough disk space.  (system)}
\label{\detokenize{troubleshooting:verify-you-have-enough-disk-space-system}}
\sphinxAtStartPar
The second way you can run out of disk space is by the filesystem itself.
do a df \sphinxhyphen{}h . in your \textquotesingle{}game\textquotesingle{} directory\textquotesingle{}.  That is df \sphinxhyphen{}h \textless{}period\textgreater{}.
This will return how much disk space is being used and how much remains.
If it shows 100\% used, you\textquotesingle{}re out of disk space and the db is corrupt.

\sphinxAtStartPar
At this point, you\textquotesingle{}re pretty screwed.  You can see if anything exists
in your system to free up some space, but if the filesystem itself
is filled, reach out to the owner of the server and let them know.
It\textquotesingle{}s a much bigger deal than just your mush if that\textquotesingle{}s the case.

\sphinxAtStartPar
Until this issue is resolved, you can not repair and bring up your mush.
No disk to run the game.


\subsection{If the mush seems to load fine but I can\textquotesingle{}t log in anyone and when I do all the names and attributes of things seem to be gone!}
\label{\detokenize{troubleshooting:if-the-mush-seems-to-load-fine-but-i-can-t-log-in-anyone-and-when-i-do-all-the-names-and-attributes-of-things-seem-to-be-gone}}
\sphinxAtStartPar
Ok, at this point you likely had your mush up when the physical server
went down hard.  Weither through an emergency shutdown or a physical
power outage, your db likely was brought down hard during a write,
so it left it in a corrupt state.  These things happen.  This is
why we always strongly request you make daily flatfile dumps.

\sphinxAtStartPar
So, to recover your database.


\subsubsection{The bad news}
\label{\detokenize{troubleshooting:the-bad-news}}
\sphinxAtStartPar
If you have no flatfile backup or never bothered with backups?
I\textquotesingle{}m sorry, at this point you\textquotesingle{}re SOA.  There\textquotesingle{}s no easy way to
recover a corrupted binary database.  If you absolutely need
data out of it we may be able to help you to piece meal things
out of it, but otherwise it\textquotesingle{}s a lost cause.  You\textquotesingle{}ll have to start
over from scratch.  I\textquotesingle{}m sorry.


\subsubsection{The good news}
\label{\detokenize{troubleshooting:the-good-news}}
\sphinxAtStartPar
If you made backups, or if the server had a normal shutdown, you
likely have a flatfile backup.  You will see a netrhost.db.flat
in either the \textquotesingle{}data\textquotesingle{} directory or \textquotesingle{}prevflat\textquotesingle{} directory.  That
is your manual flatfile backup.

\sphinxAtStartPar
If the server had a normal shutdown, you will see a file called
netrhost.db.TERMFLAT.  This is a scram\sphinxhyphen{}emergency db flatfile.
It attempts to write this at the time of server shutdown to
hopefully keep a clean backup in the case of issues since
it identifies the server is coming down hard.  Make sure
if you plan to use the TERMFLAT as your recovery flatfile
that the very last line shows something like ** END OF DUMP **.
That shows you had a successful backup.


\subsubsection{Now, to restore your database?}
\label{\detokenize{troubleshooting:now-to-restore-your-database}}
\sphinxAtStartPar
Please refer to the file \textquotesingle{}README.DBLOADING\textquotesingle{}.


\subsection{Bringing up your mail database}
\label{\detokenize{troubleshooting:bringing-up-your-mail-database}}
\sphinxAtStartPar
Your mail db may or may not come up at this point.


\subsubsection{If after restoring main database your mail database works}
\label{\detokenize{troubleshooting:if-after-restoring-main-database-your-mail-database-works}}
\sphinxAtStartPar
If your mail database came up and does not show
\textquotesingle{}Mail: mail is currently off\textquotesingle{} then you should be good to go.

\sphinxAtStartPar
Please issue on the MUSH:

\begin{sphinxVerbatim}[commandchars=\\\{\}]
\PYG{n}{wmail}\PYG{o}{/}\PYG{n}{fix}
\PYG{n}{wmail}\PYG{o}{/}\PYG{n}{lfix}
\end{sphinxVerbatim}

\sphinxAtStartPar
This will put your mail system in sync with your current database and
fix up any errors that may exist.

\sphinxAtStartPar
wmail/fix fixes the mail.

\sphinxAtStartPar
wmail/lfix loads in the fixes.


\subsubsection{If after restoring main database your mail database does not work}
\label{\detokenize{troubleshooting:if-after-restoring-main-database-your-mail-database-does-not-work}}
\sphinxAtStartPar
If your mail database is not up and shows \textquotesingle{}Mail: mail is currently off\textquotesingle{} then your mail db is currupt.


\section{Dealing with a corrupt mail database}
\label{\detokenize{troubleshooting:dealing-with-a-corrupt-mail-database}}
\sphinxAtStartPar
It says when you try to access mail that mail is disabled and/or off.

\sphinxAtStartPar
Nothing you do can bring it on line.  Well, this is bad, but not horrible.

\sphinxAtStartPar
The mail db is totally separate from the main game database.  This means
that it in no way damaged or corrupted your main mush database.

\sphinxAtStartPar
The bad news?  Yes.  Your mail database is corrupt.  To bring it back,
is it hopes that you read ahead of time about how to backup your mush,
which would include the mail database.


\subsection{Backing up your mail database}
\label{\detokenize{troubleshooting:backing-up-your-mail-database}}
\sphinxAtStartPar
You should be making a flatifile dump of mail db daily by running on the MUSH:

\begin{sphinxVerbatim}[commandchars=\\\{\}]
\PYG{n}{wmail}\PYG{o}{/}\PYG{n}{unload}
\end{sphinxVerbatim}

\sphinxAtStartPar
To recover your mail, it assumes you have a mail flatfile in either the
\textasciitilde{}/Server/game/data directory or the \textasciitilde{}/Server/game/prevflat directory.  The
latter directory is used in junction to the backup\_flat.sh and will always
house the latest flatfile if not one recently dumped in your data directory.


\subsection{Automatically recovering your mail database}
\label{\detokenize{troubleshooting:automatically-recovering-your-mail-database}}
\sphinxAtStartPar
If you have a flatfile dump of your mail db, run this command on the MUSH:

\begin{sphinxVerbatim}[commandchars=\\\{\}]
\PYG{n}{wmail}\PYG{o}{/}\PYG{n}{load}
\end{sphinxVerbatim}

\sphinxAtStartPar
Yup, that\textquotesingle{}s it.  It\textquotesingle{}ll take care of everything else.  Isn\textquotesingle{}t automation grand?

\sphinxAtStartPar
Doesn\textquotesingle{}t even require a reboot :)

\begin{sphinxadmonition}{note}{Note:}
\sphinxAtStartPar
You may at this point wish to run the following:
wmail/fix  \sphinxhyphen{}\sphinxhyphen{} this fixes the mail database and sync\textquotesingle{}s it to the mush db.
wmail/lfix \sphinxhyphen{}\sphinxhyphen{} this loads in the fixed mail database
\end{sphinxadmonition}

\sphinxAtStartPar
If you have a very old mail database, this is likely going to be required
to sync against nuked players and other changes to the game since the flatfile.

\sphinxAtStartPar
If this is a new db that you have, you can skip the fixing.


\subsection{Manually recovering your mail database}
\label{\detokenize{troubleshooting:manually-recovering-your-mail-database}}
\sphinxAtStartPar
To recover your mail manually, you need to delete your mail databases,
reboot, then reload your mail flatfiles.  If you have no mail flatfiles,
well, you\textquotesingle{}re going to have to start over with the mail database.  Sorry.

\sphinxAtStartPar
First, go into the \textquotesingle{}game\textquotesingle{} subdirectory.  Inside that directory is a \textquotesingle{}data\textquotesingle{}
directory.

\sphinxAtStartPar
You will be deleting all the files with the following names:

\begin{sphinxVerbatim}[commandchars=\\\{\}]
\PYG{n}{RhostMUSH}\PYG{o}{.}\PYG{n}{mail}\PYG{o}{.}\PYG{o}{*}                \PYG{p}{(}\PYG{n}{like} \PYG{n}{RhostMUSH}\PYG{o}{.}\PYG{n}{mail}\PYG{o}{.}\PYG{n}{dir}\PYG{o}{/}\PYG{n}{RhostMUSH}\PYG{o}{.}\PYG{n}{mail}\PYG{o}{.}\PYG{n}{pag}\PYG{p}{)}
\PYG{n}{RhostMUSH}\PYG{o}{.}\PYG{n}{folder}\PYG{o}{.}\PYG{o}{*}              \PYG{p}{(}\PYG{n}{like} \PYG{n}{RhostMUSH}\PYG{o}{.}\PYG{n}{folder}\PYG{o}{.}\PYG{n}{dir}\PYG{o}{/}\PYG{n}{RhostMUSH}\PYG{o}{.}\PYG{n}{folder}\PYG{o}{.}\PYG{n}{pag}\PYG{p}{)}
\end{sphinxVerbatim}

\begin{sphinxadmonition}{warning}{Warning:}
\sphinxAtStartPar
DO NOT DELETE OTHER NAMED FILES!!!
\end{sphinxadmonition}

\sphinxAtStartPar
Once these files are deleted, you may issue a @reboot to restart the mush.
This will unlock the mail system and load in a fresh db.

\sphinxAtStartPar
Now, if you have flatfiles of the old mail database, you will see in either
the \textquotesingle{}data\textquotesingle{} directory or the \textquotesingle{}prevflat\textquotesingle{} directory files that are called:

\begin{sphinxVerbatim}[commandchars=\\\{\}]
\PYG{n}{RhostMUSH}\PYG{o}{.}\PYG{n}{dump}\PYG{o}{.}\PYG{n}{folder}
\PYG{n}{RhostMUSH}\PYG{o}{.}\PYG{n}{dump}\PYG{o}{.}\PYG{n}{mail}
\end{sphinxVerbatim}

\sphinxAtStartPar
Make sure these two files are in the \textquotesingle{}data\textquotesingle{} subdirectory.  Copy them in
if they exist in your \textquotesingle{}prevflat\textquotesingle{} directory.

\sphinxAtStartPar
Once they are in the \textquotesingle{}data\textquotesingle{} directory, within the mush type: wmail/load

\sphinxAtStartPar
This loads in the flatfile and recover the mail database.

\sphinxAtStartPar
Now, at this point the mail db may not be 100\% in\sphinxhyphen{}sync with the game db.

\sphinxAtStartPar
So let\textquotesingle{}s fix it.

\sphinxAtStartPar
wmail/fix   \sphinxhyphen{}\sphinxhyphen{} this will run a fix on the mail db and repair any issues.

\sphinxAtStartPar
wmail/lfix  \sphinxhyphen{}\sphinxhyphen{} this will load the fixed flatfile back into the mush.

\sphinxAtStartPar
At this point you should be good to go.


\chapter{Advanced Features}
\label{\detokenize{advanced:advanced-features}}\label{\detokenize{advanced::doc}}

\section{Installing using an ansible playbook}
\label{\detokenize{advanced:installing-using-an-ansible-playbook}}\label{\detokenize{advanced:ansible-install}}
\sphinxAtStartPar
To begin, you will run the following command in a directory that will house your game:

\begin{sphinxVerbatim}[commandchars=\\\{\}]
\PYG{n}{git} \PYG{n}{clone} \PYG{n}{https}\PYG{p}{:}\PYG{o}{/}\PYG{o}{/}\PYG{n}{github}\PYG{o}{.}\PYG{n}{com}\PYG{o}{/}\PYG{n}{RhostMUSH}\PYG{o}{/}\PYG{n}{trunk} \PYG{n}{Rhost}
\end{sphinxVerbatim}

\sphinxAtStartPar
You may also just run the yml file and ansible (ansible\sphinxhyphen{}playbook) to install your RhostMUSH engine:

\begin{sphinxVerbatim}[commandchars=\\\{\}]
\PYG{n}{wget} \PYG{n}{https}\PYG{p}{:}\PYG{o}{/}\PYG{o}{/}\PYG{n}{raw}\PYG{o}{.}\PYG{n}{githubusercontent}\PYG{o}{.}\PYG{n}{com}\PYG{o}{/}\PYG{n}{RhostMUSH}\PYG{o}{/}\PYG{n}{trunk}\PYG{o}{/}\PYG{n}{master}\PYG{o}{/}\PYG{n}{rhostinstall}\PYG{o}{.}\PYG{n}{yml}
\PYG{n}{ansible}\PYG{o}{\PYGZhy{}}\PYG{n}{playbook} \PYG{n}{rhostinstall}\PYG{o}{.}\PYG{n}{yml}
\end{sphinxVerbatim}

\sphinxAtStartPar
This downloads the latest stable version of the code, bringing with it all patches and scripts, documentation and support tools that you will need.


\section{Adding hardcoded modules}
\label{\detokenize{advanced:adding-hardcoded-modules}}
\sphinxAtStartPar
RhostMUSH does support module writing.


\subsection{Modifying sourcode to add a module}
\label{\detokenize{advanced:modifying-sourcode-to-add-a-module}}
\sphinxAtStartPar
This requires hooking your changes into local.c, then modifying the Makefile (in the src directory)
for any new source files that you wish to add.

\sphinxAtStartPar
Something to be aware of is that all localized data is ran after the system cache subroutine.


\subsection{Adding an @startup to make use of modules}
\label{\detokenize{advanced:adding-an-startup-to-make-use-of-modules}}
\sphinxAtStartPar
This means that if your code is depending on @startups, you need to put a delay in the @startup
so that your local code can be loaded in as modules prior to the @startup execution.

\sphinxAtStartPar
Something that will not work:

\begin{sphinxVerbatim}[commandchars=\\\{\}]
\PYG{n+nd}{@startup} \PYG{n}{me}\PYG{o}{=}\PYG{n+nd}{@superhappyfuncommand} \PYG{n}{loadmeup}\PYG{o}{=}\PYG{n}{now}
\end{sphinxVerbatim}

\sphinxAtStartPar
A small alteration that will likely make this work fine:

\begin{sphinxVerbatim}[commandchars=\\\{\}]
\PYG{n+nd}{@startup} \PYG{n}{me}\PYG{o}{=}\PYG{n+nd}{@wait} \PYG{l+m+mi}{1}\PYG{o}{=}\PYG{n+nd}{@superhappyfuncommand} \PYG{n}{loadmeup}\PYG{o}{=}\PYG{n}{now}
\end{sphinxVerbatim}

\sphinxAtStartPar
That 1 second delay for the queue will give the game engine time to load in your module code.


\subsection{Contributing your module back to Rhost}
\label{\detokenize{advanced:contributing-your-module-back-to-rhost}}
\sphinxAtStartPar
If you wish your modules to be part of the main Rhost distribution you have two options:
\begin{enumerate}
\sphinxsetlistlabels{\arabic}{enumi}{enumii}{}{.}%
\item {} 
\sphinxAtStartPar
Attempt to hack the bin/asksource.sh and bin/asksource.blank files.

\item {} 
\sphinxAtStartPar
Ask one of the Rhost devs to do it for you :)

\end{enumerate}


\section{Reality Levels Setup}
\label{\detokenize{advanced:reality-levels-setup}}
\sphinxAtStartPar
Reality levels are a means to forbid (or allow) interaction between objects
in the same location.


\subsection{Reality Levels Visibility}
\label{\detokenize{advanced:reality-levels-visibility}}
\sphinxAtStartPar
Each object (player, room, exit, thing) has two lists of reality levels.
An Rx list, which describe what it can see and a Tx list, which describe
where it can be seen. Those are bitfields. In order for X to see Y a bitwise
\textquotesingle{}and\textquotesingle{} is performed between X\textquotesingle{}s RxLevel and Y\textquotesingle{}s TxLevel. If the result is not
0, then X sees Y. If the result is 0, as far as X is concerned, Y doesn\textquotesingle{}t
exist. This affects contents lists, exit lists, look, say, pose, @emit,
@verb, connect/disconnect, has arrived/has left messages, exit and object
matching. \textquotesingle{}here\textquotesingle{} and \textquotesingle{}me\textquotesingle{} match always.

\sphinxAtStartPar
It doesn\textquotesingle{}t affect @remit, @pemit, page, WHO or channels.
By default, all new objects are created with an RxLevel of 1 and TxLevel of
1. Rooms are an exception, created with an RxLevel of 1 and a TxLevel of
0xFFFFFFFF. Those default levels can be changed with configuration
parameters.

\sphinxAtStartPar
An object is always visible to itself, even if its Rx and Tx levels don\textquotesingle{}t
match. (See examples below)


\subsection{Reality Levels Descriptions}
\label{\detokenize{advanced:reality-levels-descriptions}}
\sphinxAtStartPar
For every reality level defined, you can define an attribute that serves as
description. If you look at something and match more than one of its
TxLevels, you\textquotesingle{}ll see all the corresponding descriptions on the target
object. If the object doesn\textquotesingle{}t have any descriptions for the matching levels,
you\textquotesingle{}ll see the regular @desc.

\sphinxAtStartPar
The @adesc attribute on the target is only triggered if the target can see
the looker in turn. It\textquotesingle{}s only triggered once, no matter how many descs the
looker sees. The @odesc is shown only to those people that see /both/ the
looker and the target.

\sphinxAtStartPar
Through extension, @afail/@ofail and similar pairs (@adrop/@odrop,
@asucc/@osucc etc) work in the same way. @verb commands are similary
affected.

\sphinxAtStartPar
Softcoded commands are only matched on the objects that can see the player.
The player doesn\textquotesingle{}t have to see the object. This includes commands in the
Master Room.

\sphinxAtStartPar
Exits are treated specially. In order to be able to use an exit name (or to
use the \textquotesingle{}move \textless{}exit\textgreater{}\textquotesingle{} command) the exit must be visible to the enactor. In
order to pass through the exit, the exit must see the enactor in turn. There
are reasons for this, which will become evident in the examples below.


\subsection{Reality Levels Configuration parameters}
\label{\detokenize{advanced:reality-levels-configuration-parameters}}
\sphinxAtStartPar
A few configuration parameters have been introduced to deal with the reality
levels:

\begin{sphinxVerbatim}[commandchars=\\\{\}]
\PYG{n}{reality\PYGZus{}level} \PYG{o}{\PYGZlt{}}\PYG{n}{name}\PYG{o}{\PYGZgt{}} \PYG{o}{\PYGZlt{}}\PYG{n}{value}\PYG{o}{\PYGZgt{}} \PYG{p}{[}\PYG{o}{\PYGZlt{}}\PYG{n}{desc} \PYG{n}{attribute} \PYG{n}{name}\PYG{o}{\PYGZgt{}}\PYG{p}{]}
\end{sphinxVerbatim}

\sphinxAtStartPar
This directive can only be used in the config file (not with the @admin
command) and should be repeated for each reality level you want to define.
It defines a new level named \textless{}name\textgreater{} with a bitvalue of \textless{}value\textgreater{} and an
optional desc attribute. There is a limit of 8 characters on \textless{}name\textgreater{}, a
32\sphinxhyphen{}bit value on \textless{}value\textgreater{} (basically an unsigned long) and 32 characters on
the attribute name. A maximum of 32 reality levels can be defined:

\begin{sphinxVerbatim}[commandchars=\\\{\}]
\PYG{n}{def\PYGZus{}exit\PYGZus{}tx} \PYG{o}{\PYGZlt{}}\PYG{n}{value}\PYG{o}{\PYGZgt{}}
\PYG{n}{def\PYGZus{}exit\PYGZus{}rx} \PYG{o}{\PYGZlt{}}\PYG{n}{value}\PYG{o}{\PYGZgt{}}
\PYG{n}{def\PYGZus{}room\PYGZus{}tx} \PYG{o}{\PYGZlt{}}\PYG{n}{value}\PYG{o}{\PYGZgt{}}
\PYG{n}{def\PYGZus{}room\PYGZus{}rx} \PYG{o}{\PYGZlt{}}\PYG{n}{value}\PYG{o}{\PYGZgt{}}
\PYG{n}{def\PYGZus{}player\PYGZus{}rx} \PYG{o}{\PYGZlt{}}\PYG{n}{value}\PYG{o}{\PYGZgt{}}
\PYG{n}{def\PYGZus{}player\PYGZus{}tx} \PYG{o}{\PYGZlt{}}\PYG{n}{value}\PYG{o}{\PYGZgt{}}
\PYG{n}{def\PYGZus{}thing\PYGZus{}rx} \PYG{o}{\PYGZlt{}}\PYG{n}{value}\PYG{o}{\PYGZgt{}}
\PYG{n}{def\PYGZus{}thing\PYGZus{}tx} \PYG{o}{\PYGZlt{}}\PYG{n}{value}\PYG{o}{\PYGZgt{}}
\end{sphinxVerbatim}

\sphinxAtStartPar
These 8 directives define the default reality levels of newly created
objects. They can be set in the config file or with the @admin command.
Like above, \textless{}value\textgreater{} must be a decimal number:

\begin{sphinxVerbatim}[commandchars=\\\{\}]
\PYG{n}{wiz\PYGZus{}always\PYGZus{}real} \PYG{o}{\PYGZlt{}}\PYG{l+m+mi}{0}\PYG{o}{|}\PYG{l+m+mi}{1}\PYG{o}{\PYGZgt{}}
\end{sphinxVerbatim}

\sphinxAtStartPar
If this parameter is set to 1 then wizards (and immortals on Rhost) will see
everything and will be visible to everyone. Their effective Rx and Tx levels
will always be 0xFFFFFFFF. Also settable in the config file and with the
@admin command.

\sphinxAtStartPar
Compile with \sphinxhyphen{}DREALITY\_LEVELS compile time option to enable \textquotesingle{}Real\textquotesingle{} needs to be \textquotesingle{}1\textquotesingle{}
This is an example file only to be added to the mush.conf file Format:

\begin{sphinxVerbatim}[commandchars=\\\{\}]
\PYG{n}{reality\PYGZus{}level} \PYG{o}{\PYGZlt{}}\PYG{l+m+mi}{8} \PYG{n}{char} \PYG{n}{name}\PYG{o}{\PYGZgt{}} \PYG{o}{\PYGZlt{}}\PYG{n+nb}{hex}\PYG{o}{\PYGZhy{}}\PYG{n}{byte}\PYG{o}{\PYGZhy{}}\PYG{n}{mask}\PYG{o}{\PYGZgt{}} \PYG{o}{\PYGZlt{}}\PYG{n}{optional}\PYG{o}{\PYGZhy{}}\PYG{n}{desc}\PYG{p}{:} \PYG{n}{DESC} \PYG{n}{default}\PYG{o}{\PYGZgt{}}
\end{sphinxVerbatim}


\subsection{Reality Levels Example mush.conf}
\label{\detokenize{advanced:reality-levels-example-mush-conf}}
\begin{sphinxVerbatim}[commandchars=\\\{\}]
\PYG{n}{reality\PYGZus{}level} \PYG{n}{Real} \PYG{l+m+mi}{1}
\PYG{n}{reality\PYGZus{}level} \PYG{n}{Obf1} \PYG{l+m+mi}{2}
\PYG{n}{reality\PYGZus{}level} \PYG{n}{Obf2} \PYG{l+m+mi}{4}
\PYG{n}{reality\PYGZus{}level} \PYG{n}{Obf3} \PYG{l+m+mi}{8} \PYG{n}{OBFDESC}
\PYG{n}{reality\PYGZus{}level} \PYG{n}{Obf4} \PYG{l+m+mi}{16} \PYG{n}{OBFDESC}
\PYG{n}{reality\PYGZus{}level} \PYG{n}{Obf5} \PYG{l+m+mi}{32} \PYG{n}{OBFDESC}
\PYG{n}{reality\PYGZus{}level} \PYG{n}{Obf6} \PYG{l+m+mi}{64} \PYG{n}{OBFDESC}
\PYG{n}{reality\PYGZus{}level} \PYG{n}{Obf7} \PYG{l+m+mi}{128} \PYG{n}{OBFDESC}
\PYG{n}{reality\PYGZus{}level} \PYG{n}{Obf8} \PYG{l+m+mi}{256} \PYG{n}{OBFDESC}
\PYG{n}{reality\PYGZus{}level} \PYG{n}{Obf9} \PYG{l+m+mi}{512} \PYG{n}{OBFDESC}
\PYG{n}{reality\PYGZus{}level} \PYG{n}{Obf10} \PYG{l+m+mi}{1024} \PYG{n}{OBFDESC}
\PYG{n}{reality\PYGZus{}level} \PYG{n}{Umbra} \PYG{l+m+mi}{2048} \PYG{n}{UMBRADESC}
\PYG{n}{reality\PYGZus{}level} \PYG{n}{Fae} \PYG{l+m+mi}{4096} \PYG{n}{FAEDESC}
\PYG{n}{reality\PYGZus{}level} \PYG{n}{Shadow} \PYG{l+m+mi}{8192} \PYG{n}{SHADOWDESC}
\PYG{n}{reality\PYGZus{}level} \PYG{n}{Spy} \PYG{l+m+mi}{16384}
\PYG{n}{reality\PYGZus{}level} \PYG{n}{Death} \PYG{l+m+mi}{32768} \PYG{n}{DEATHDESC}
\PYG{n}{reality\PYGZus{}level} \PYG{n}{All} \PYG{l+m+mi}{4294967295}
\end{sphinxVerbatim}


\subsection{Reality Levels Commands}
\label{\detokenize{advanced:reality-levels-commands}}
\sphinxAtStartPar
Two wiz\sphinxhyphen{}only commands are used to set the reality levels of an object:

\begin{sphinxVerbatim}[commandchars=\\\{\}]
\PYG{n+nd}{@rxlevel} \PYG{o}{\PYGZlt{}}\PYG{n+nb}{object}\PYG{o}{\PYGZgt{}}\PYG{o}{=}\PYG{o}{\PYGZlt{}}\PYG{n+nb}{list}\PYG{o}{\PYGZgt{}}
\PYG{n+nd}{@txlevel} \PYG{o}{\PYGZlt{}}\PYG{n+nb}{object}\PYG{o}{\PYGZgt{}}\PYG{o}{=}\PYG{o}{\PYGZlt{}}\PYG{n+nb}{list}\PYG{o}{\PYGZgt{}}
\end{sphinxVerbatim}

\sphinxAtStartPar
\textless{}list\textgreater{} is a space\sphinxhyphen{}separated list of level names that have to be set on the
object. If a level name is prefixed with an exclamation mark (!) that level
will be cleared from the object.

\begin{sphinxadmonition}{warning}{Warning:}
\sphinxAtStartPar
Changing the Tx levels of an object might make it invisible to you.
In this case, you can still manipulate it by using his \#dbref (or *player
for players).
\end{sphinxadmonition}


\subsection{Functions}
\label{\detokenize{advanced:functions}}
\sphinxAtStartPar
There are five functions that deal with reality levels:

\begin{sphinxVerbatim}[commandchars=\\\{\}]
\PYG{n}{hasrxlevel}\PYG{p}{(}\PYG{o}{\PYGZlt{}}\PYG{n+nb}{object}\PYG{o}{\PYGZgt{}}\PYG{p}{,}\PYG{o}{\PYGZlt{}}\PYG{n}{level}\PYG{o}{\PYGZgt{}}\PYG{p}{)}
\PYG{n}{hastxlevel}\PYG{p}{(}\PYG{o}{\PYGZlt{}}\PYG{n+nb}{object}\PYG{o}{\PYGZgt{}}\PYG{p}{,}\PYG{o}{\PYGZlt{}}\PYG{n}{level}\PYG{o}{\PYGZgt{}}\PYG{p}{)}
\end{sphinxVerbatim}

\sphinxAtStartPar
These two functions check if an object has the specified Rx or Tx level.
You must control \textless{}object\textgreater{}. They return 0 or 1 and \#\sphinxhyphen{}1 in case the object
does not exist or you don\textquotesingle{}t have permissions:

\begin{sphinxVerbatim}[commandchars=\\\{\}]
\PYG{n}{rxlevel}\PYG{p}{(}\PYG{o}{\PYGZlt{}}\PYG{n+nb}{object}\PYG{o}{\PYGZgt{}}\PYG{p}{)}
\PYG{n}{txlevel}\PYG{p}{(}\PYG{o}{\PYGZlt{}}\PYG{n+nb}{object}\PYG{o}{\PYGZgt{}}\PYG{p}{)}
\end{sphinxVerbatim}

\sphinxAtStartPar
These two functions return a space\sphinxhyphen{}separated list of the object\textquotesingle{}s Rx or Tx
levels. Again, you must control the object:

\begin{sphinxVerbatim}[commandchars=\\\{\}]
\PYG{n}{cansee}\PYG{p}{(}\PYG{o}{\PYGZlt{}}\PYG{n}{obj1}\PYG{o}{\PYGZgt{}}\PYG{p}{,}\PYG{o}{\PYGZlt{}}\PYG{n}{obj2}\PYG{o}{\PYGZgt{}}\PYG{p}{)}
\end{sphinxVerbatim}

\sphinxAtStartPar
A wiz\sphinxhyphen{}only function, returns 1 of \textless{}obj1\textgreater{} can see \textless{}obj2\textgreater{} from a reality
levels point of view. It doesn\textquotesingle{}t check if the objects are in the same
location, the DARK/CLOAKED flags and so on. Just \textless{}obj1\textgreater{}\textquotesingle{}s Rx level against
\textless{}obj2\textgreater{}\textquotesingle{}s Tx level.

\begin{sphinxadmonition}{warning}{Warning:}
\sphinxAtStartPar
If you are using it on MUX2.0 with /both/ reality levels and Wod
Realms enabled, the function will perform both checks and the Wod Realms
version checks against the DARK flag.
\end{sphinxadmonition}


\subsection{Example 1: A simplified Witchcraft setup}
\label{\detokenize{advanced:example-1-a-simplified-witchcraft-setup}}
\sphinxAtStartPar
In Witchcraft, besides the various Gifted classes, characters can be spirits
There are spirit realms to which the mundane can not travel. Therefore we
will use 2 reality levels: Real and Ghost. Since some spirits can become
solid for a limited period of time, we will also use an additional desc for
the Ghost level, called GHOSTDESC. Therefore in the config file we will
have:

\begin{sphinxVerbatim}[commandchars=\\\{\}]
\PYG{n}{reality\PYGZus{}level} \PYG{n}{Real} \PYG{l+m+mi}{1}
\PYG{n}{reality\PYGZus{}level} \PYG{n}{Ghost} \PYG{l+m+mi}{2} \PYG{n}{GHOSTDESC}
\end{sphinxVerbatim}

\sphinxAtStartPar
Ghosts can pass through most mundane locks, so the exists should allows the
ghosts to pass:

\begin{sphinxVerbatim}[commandchars=\\\{\}]
\PYG{n}{def\PYGZus{}exit\PYGZus{}rx} \PYG{l+m+mi}{3}
\end{sphinxVerbatim}

\sphinxAtStartPar
Note that def\_exit\_tx isn\textquotesingle{}t set. Why? Because ghosts see the mundane world
anyway, so a spirit character will have:

\begin{sphinxVerbatim}[commandchars=\\\{\}]
@txlevel \PYGZlt{}player\PYGZgt{}=!Real Ghost
@rxlevel \PYGZlt{}player\PYGZgt{}=Real Ghost
\end{sphinxVerbatim}

\sphinxAtStartPar
Let\textquotesingle{}s assume 3 players:

\begin{sphinxVerbatim}[commandchars=\\\{\}]
\PYG{n}{John} \PYG{o+ow}{is} \PYG{n}{a} \PYG{n}{Mundane}\PYG{o}{.} \PYG{n}{He} \PYG{n}{won}\PYG{l+s+s1}{\PYGZsq{}}\PYG{l+s+s1}{t see spirits.}
\PYG{n}{John}\PYG{l+s+s1}{\PYGZsq{}}\PYG{l+s+s1}{s Rx: Real}
\PYG{n}{John}\PYG{l+s+s1}{\PYGZsq{}}\PYG{l+s+s1}{s Tx: Real}
\PYG{n}{John}\PYG{l+s+s1}{\PYGZsq{}}\PYG{l+s+s1}{s @desc: This is John.}
\PYG{n}{John}\PYG{l+s+s1}{\PYGZsq{}}\PYG{l+s+s1}{s \PYGZam{}GHOSTDESC: (Not important, since it}\PYG{l+s+s1}{\PYGZsq{}}\PYG{n}{s} \PYG{n}{never} \PYG{n}{visible}\PYG{p}{)}
\PYG{n}{Johh}\PYG{l+s+s1}{\PYGZsq{}}\PYG{l+s+s1}{s @adesc: }\PYG{l+s+s1}{\PYGZpc{}}\PYG{l+s+s1}{N has looked at you.}
\PYG{n}{John}\PYG{l+s+s1}{\PYGZsq{}}\PYG{l+s+s1}{s @odesc: has looked at John.}
\end{sphinxVerbatim}

\sphinxAtStartPar
Jack is a Gifted. He will sense spirits, but is still made from flesh and blood so visible to mundanes:

\begin{sphinxVerbatim}[commandchars=\\\{\}]
\PYG{n}{Jack}\PYG{l+s+s1}{\PYGZsq{}}\PYG{l+s+s1}{s Rx: Real Ghost}
\PYG{n}{Jack}\PYG{l+s+s1}{\PYGZsq{}}\PYG{l+s+s1}{s Tx: Real}
\PYG{n}{Jack}\PYG{l+s+s1}{\PYGZsq{}}\PYG{l+s+s1}{s @desc: This is Jack.}
\PYG{n}{Jack}\PYG{l+s+s1}{\PYGZsq{}}\PYG{l+s+s1}{s \PYGZam{}GHOSTDESC: (Not important, since it}\PYG{l+s+s1}{\PYGZsq{}}\PYG{n}{s} \PYG{n}{never} \PYG{n}{visible}\PYG{p}{)}
\PYG{n}{Jack}\PYG{l+s+s1}{\PYGZsq{}}\PYG{l+s+s1}{s @adesc: }\PYG{l+s+s1}{\PYGZpc{}}\PYG{l+s+s1}{N has looked at you.}
\PYG{n}{Jack}\PYG{l+s+s1}{\PYGZsq{}}\PYG{l+s+s1}{s @odesc: has looked at Jack.}
\end{sphinxVerbatim}

\sphinxAtStartPar
Frank is a ghost. He will see other spirits as well as mundanes, but won\textquotesingle{}t be visible to mundanes. He can also become visible to everybody:

\begin{sphinxVerbatim}[commandchars=\\\{\}]
\PYG{n}{Frank}\PYG{l+s+s1}{\PYGZsq{}}\PYG{l+s+s1}{s Rx: Real Ghost}
 \PYG{n}{Frank}\PYG{l+s+s1}{\PYGZsq{}}\PYG{l+s+s1}{s Tx: Ghost}
 \PYG{n}{Frank}\PYG{l+s+s1}{\PYGZsq{}}\PYG{l+s+s1}{s @desc: This is Frank, looking human.}
 \PYG{n}{Frank}\PYG{l+s+s1}{\PYGZsq{}}\PYG{l+s+s1}{s \PYGZam{}GHOSTDESC: This is Frank}\PYG{l+s+s1}{\PYGZsq{}}\PYG{n}{s} \PYG{n}{ghostly} \PYG{n}{shape}\PYG{o}{.}
 \PYG{n}{Frank}\PYG{l+s+s1}{\PYGZsq{}}\PYG{l+s+s1}{s @adesc: }\PYG{l+s+s1}{\PYGZpc{}}\PYG{l+s+s1}{N has looked at you.}
 \PYG{n}{Frank}\PYG{l+s+s1}{\PYGZsq{}}\PYG{l+s+s1}{s @odesc: has looked at Frank.}
\end{sphinxVerbatim}

\sphinxAtStartPar
Following are commands that each of the players enter and what they see.  I\textquotesingle{}ll assume the +materialize command is defined like:

\begin{sphinxVerbatim}[commandchars=\\\{\}]
\PYGZam{}CMD\PYGZus{}MATERIALIZE \PYGZlt{}cmdobject\PYGZgt{}=\PYGZdl{}+materialize:@txlevel \PYGZpc{}\PYGZsh{}=Real; @pemit \PYGZpc{}\PYGZsh{}=You are now material.
\end{sphinxVerbatim}

\begin{sphinxVerbatim}[commandchars=\\\{\}]
\PYG{o}{|}         \PYG{n}{John}           \PYG{o}{|}         \PYG{n}{Jack}          \PYG{o}{|}         \PYG{n}{Frank}
                         \PYG{o}{|}                       \PYG{o}{|}
 \PYG{o}{\PYGZgt{}} \PYG{n}{l}                     \PYG{o}{|}                       \PYG{o}{|}
 \PYG{n}{A} \PYG{n}{room}                  \PYG{o}{|}                       \PYG{o}{|}
 \PYG{n}{This} \PYG{o+ow}{is} \PYG{n}{a} \PYG{n}{bare} \PYG{n}{room}\PYG{o}{.}    \PYG{o}{|}                       \PYG{o}{|}
 \PYG{n}{Contents}\PYG{p}{:}               \PYG{o}{|}                       \PYG{o}{|}
 \PYG{n}{Jack}                    \PYG{o}{|}                       \PYG{o}{|}
 \PYG{n}{Obvious} \PYG{n}{exits}\PYG{p}{:}          \PYG{o}{|}                       \PYG{o}{|}
 \PYG{n}{Out} \PYG{o}{\PYGZlt{}}\PYG{n}{O}\PYG{o}{\PYGZgt{}}                 \PYG{o}{|}                       \PYG{o}{|}
                         \PYG{o}{|}\PYG{o}{\PYGZgt{}} \PYG{n}{l}                    \PYG{o}{|}
                         \PYG{o}{|}\PYG{n}{A} \PYG{n}{room}                 \PYG{o}{|}
                         \PYG{o}{|}\PYG{n}{This} \PYG{o+ow}{is} \PYG{n}{a} \PYG{n}{bare} \PYG{n}{room}\PYG{o}{.}   \PYG{o}{|}
                         \PYG{o}{|}\PYG{n}{Contents}\PYG{p}{:}              \PYG{o}{|}
                         \PYG{o}{|}\PYG{n}{John} \PYG{n}{Frank}             \PYG{o}{|}
                         \PYG{o}{|}\PYG{n}{Obvious} \PYG{n}{exits}\PYG{p}{:}         \PYG{o}{|}
                         \PYG{o}{|}\PYG{n}{Out} \PYG{o}{\PYGZlt{}}\PYG{n}{O}\PYG{o}{\PYGZgt{}}                \PYG{o}{|}
                         \PYG{o}{|}                       \PYG{o}{|}\PYG{o}{\PYGZgt{}} \PYG{n}{l}
                         \PYG{o}{|}                       \PYG{o}{|}\PYG{n}{A} \PYG{n}{room}
                         \PYG{o}{|}                       \PYG{o}{|}\PYG{n}{This} \PYG{o+ow}{is} \PYG{n}{a} \PYG{n}{bare} \PYG{n}{room}\PYG{o}{.}
                         \PYG{o}{|}                       \PYG{o}{|}\PYG{n}{Contents}\PYG{p}{:}
                         \PYG{o}{|}                       \PYG{o}{|}\PYG{n}{John} \PYG{n}{Jack}
                         \PYG{o}{|}                       \PYG{o}{|}\PYG{n}{Obvious} \PYG{n}{exits}\PYG{p}{:}
                         \PYG{o}{|}                       \PYG{o}{|}\PYG{n}{Out} \PYG{o}{\PYGZlt{}}\PYG{n}{O}\PYG{o}{\PYGZgt{}}
 \PYG{o}{\PYGZgt{}}\PYG{n}{l} \PYG{n}{Jack}                 \PYG{o}{|}                       \PYG{o}{|}
 \PYG{n}{Jack}                    \PYG{o}{|}\PYG{n}{John} \PYG{n}{has} \PYG{n}{looked} \PYG{n}{at} \PYG{n}{you}\PYG{o}{.}\PYG{o}{|}\PYG{n}{John} \PYG{n}{has} \PYG{n}{looked} \PYG{n}{at} \PYG{n}{Jack}\PYG{o}{.}
 \PYG{n}{This} \PYG{o+ow}{is} \PYG{n}{Jack}\PYG{o}{.}           \PYG{o}{|}                       \PYG{o}{|}
 \PYG{o}{\PYGZgt{}}\PYG{n}{l} \PYG{n}{Frank}                \PYG{o}{|}                       \PYG{o}{|}
 \PYG{n}{I} \PYG{n}{don}\PYG{l+s+s1}{\PYGZsq{}}\PYG{l+s+s1}{t see that here.  |                       |}
                         \PYG{o}{|}\PYG{o}{\PYGZgt{}}\PYG{n}{l} \PYG{n}{Frank}               \PYG{o}{|}
                         \PYG{o}{|}\PYG{n}{Frank}                  \PYG{o}{|}\PYG{n}{Jack} \PYG{n}{has} \PYG{n}{looked} \PYG{n}{at} \PYG{n}{you}\PYG{o}{.}
                         \PYG{o}{|}\PYG{n}{This} \PYG{o+ow}{is} \PYG{n}{Frank}\PYG{l+s+s1}{\PYGZsq{}}\PYG{l+s+s1}{s ghostly|}
                         \PYG{o}{|}\PYG{n}{shape}\PYG{o}{.}                 \PYG{o}{|}
                         \PYG{o}{|}                       \PYG{o}{|}\PYG{o}{\PYGZgt{}}\PYG{n}{l} \PYG{n}{John}
                         \PYG{o}{|}\PYG{n}{Frank} \PYG{n}{has} \PYG{n}{looked} \PYG{n}{at}    \PYG{o}{|}\PYG{n}{John}
                         \PYG{o}{|}\PYG{n}{John}\PYG{o}{.}                  \PYG{o}{|}\PYG{n}{This} \PYG{o+ow}{is} \PYG{n}{John}\PYG{o}{.}
                         \PYG{o}{|}                       \PYG{o}{|}\PYG{o}{\PYGZgt{}}\PYG{o}{+}\PYG{n}{materialize}
                         \PYG{o}{|}                       \PYG{o}{|}\PYG{n}{You} \PYG{n}{are} \PYG{n}{now} \PYG{n}{material}\PYG{o}{.}
 \PYG{o}{\PYGZgt{}}\PYG{n}{l} \PYG{n}{Frank}                \PYG{o}{|}                       \PYG{o}{|}
 \PYG{n}{Frank}                   \PYG{o}{|}\PYG{n}{John} \PYG{n}{has} \PYG{n}{looked} \PYG{n}{at}     \PYG{o}{|}\PYG{n}{Frank} \PYG{n}{has} \PYG{n}{looked} \PYG{n}{at} \PYG{n}{you}\PYG{o}{.}
 \PYG{n}{This} \PYG{o+ow}{is} \PYG{n}{Frank}\PYG{p}{,} \PYG{n}{looking}  \PYG{o}{|}\PYG{n}{Frank}\PYG{o}{.}                 \PYG{o}{|}
 \PYG{n}{human}\PYG{o}{.}                  \PYG{o}{|}                       \PYG{o}{|}
                         \PYG{o}{|}\PYG{o}{\PYGZgt{}}\PYG{n}{l} \PYG{n}{Frank}               \PYG{o}{|}
 \PYG{n}{Jack} \PYG{n}{has} \PYG{n}{looked} \PYG{n}{at}      \PYG{o}{|}\PYG{n}{Frank}                  \PYG{o}{|}\PYG{n}{John} \PYG{n}{has} \PYG{n}{looked} \PYG{n}{at} \PYG{n}{you}\PYG{o}{.}
 \PYG{n}{Frank}\PYG{o}{.}                  \PYG{o}{|}\PYG{n}{This} \PYG{o+ow}{is} \PYG{n}{Frank}\PYG{p}{,} \PYG{n}{looking} \PYG{o}{|}
                         \PYG{o}{|}\PYG{n}{human}\PYG{o}{.}                 \PYG{o}{|}
                         \PYG{o}{|}\PYG{n}{This} \PYG{o+ow}{is} \PYG{n}{Frank}\PYG{l+s+s1}{\PYGZsq{}}\PYG{l+s+s1}{s ghostly|}
                         \PYG{o}{|}\PYG{n}{shape}\PYG{o}{.}                 \PYG{o}{|}
\end{sphinxVerbatim}


\subsection{Example 2: A WoD setup}
\label{\detokenize{advanced:example-2-a-wod-setup}}
\sphinxAtStartPar
The reality levels will be defined like this:

\begin{sphinxVerbatim}[commandchars=\\\{\}]
\PYG{n}{reality\PYGZus{}level}         \PYG{n}{Real} \PYG{l+m+mi}{1}
\PYG{n}{reality\PYGZus{}level}         \PYG{n}{Obf1} \PYG{l+m+mi}{2}
\PYG{n}{reality\PYGZus{}level}         \PYG{n}{Obf2} \PYG{l+m+mi}{4}
\PYG{n}{reality\PYGZus{}level}         \PYG{n}{Obf3} \PYG{l+m+mi}{8} \PYG{n}{OBFDESC}
\PYG{n}{reality\PYGZus{}level}         \PYG{n}{Obf4} \PYG{l+m+mi}{16} \PYG{n}{OBFDESC}
\PYG{n}{reality\PYGZus{}level}         \PYG{n}{Obf5} \PYG{l+m+mi}{32} \PYG{n}{OBFDESC}
\PYG{n}{reality\PYGZus{}level}         \PYG{n}{Umbra} \PYG{l+m+mi}{64} \PYG{n}{UMBRADESC}
\PYG{n}{reality\PYGZus{}level}         \PYG{n}{Fae} \PYG{l+m+mi}{128} \PYG{n}{FAEDESC}
\PYG{n}{reality\PYGZus{}level}         \PYG{n}{Shadow} \PYG{l+m+mi}{256} \PYG{n}{SHADOWDESC}
\PYG{n}{reality\PYGZus{}level}         \PYG{n}{All} \PYG{l+m+mi}{511}
\end{sphinxVerbatim}

\sphinxAtStartPar
5 levels of Obfuscation, Umbra, Dreaming, Wraiths. \textquotesingle{}All\textquotesingle{} is a handy
replacement for all levels, useful for wizards and wizobjects that should
be visible on all levels. Also useful when you want to set an object\textquotesingle{}s
levels to something without knowing what he had before:

\begin{sphinxVerbatim}[commandchars=\\\{\}]
\PYG{n+nd}{@rxlevel} \PYG{c+c1}{\PYGZsh{}276=!All Real}
\end{sphinxVerbatim}

\sphinxAtStartPar
!All will clear all levels, then the object will gain the Real level.
There is more than one Obfuscation level because of the relation between
Auspex and Obfuscation.

\sphinxAtStartPar
A vampire with Obfuscate 2, should not be visible by one with Auspex 1.
However one with Auspex 3 should see another vampire with Obfuscate 1, 2
/or/ 3.

\sphinxAtStartPar
Obfuscated players can move if they have Obf \textgreater{} 1. Umbral and Shadow players
should also be able to see most of the exits. So the exits at creation
should have default levels of Real + Obf2 + Obf3 + Obf4 + Obf5 + Umbra +
Shadow = 1 + 4 + 8 + 16 + 32 + 64 + 256 = 381:

\begin{sphinxVerbatim}[commandchars=\\\{\}]
\PYG{n}{def\PYGZus{}exit\PYGZus{}rx} \PYG{l+m+mi}{381}
\PYG{n}{def\PYGZus{}exit\PYGZus{}tx} \PYG{l+m+mi}{381}
\end{sphinxVerbatim}

\sphinxAtStartPar
Obf1 is not included since an Obfuscated vampire should not be able to move
if it only has Obf1. Therefore they won\textquotesingle{}t see the exits. If you want them
to be able to see the exits, but not to use them, change the default Tx of
the exits:

\begin{sphinxVerbatim}[commandchars=\\\{\}]
\PYG{n}{def\PYGZus{}exit\PYGZus{}rx} \PYG{l+m+mi}{381}
\PYG{n}{def\PYGZus{}exit\PYGZus{}tx} \PYG{l+m+mi}{383}
\end{sphinxVerbatim}

\sphinxAtStartPar
Joe the Mortal will have an RxLevel: Real and a TxLevel: Real
Jack the Malk, who likes to walk around Obfuscated and has Obfuscate 2 will
have an RxLevel: Real (he sees what the mortals see) but a TxLevel: Obf2
Jimmy the Nossie, who is using the Mask and has Obfuscate 4, but doesn\textquotesingle{}t
try to make himself invisible will have an RxLevel: Real (as Jack)
and a TxLevel: Real Obf4. He will also set his @desc to what the mortals see and
\&OBFDESC to his real slimy desc. Simply put, he will be visible to mortals,
but not with his real desc.

\sphinxAtStartPar
Aldrin the Gangrel, has Auspex 4 and activates it. Therefore, his TxLevel
will still be Real, but RxLevel: Real Obf1 Obf2 Obf3 Obf4 (all of them). So
he can see Joe, Jack and Jimmy\textquotesingle{}s both descs.
Joe, on the other hand, won\textquotesingle{}t see Jack at all. He will still see Jimmy, but
only Jimmy\textquotesingle{}s @desc, not the OBFDESC

\sphinxAtStartPar
Frida the Fae... will have RxLevel: Real Fae and TxLevel: Real Fae. @desc
set to the mundane desc, \&FAEDESC set to the Chimerical desc.
Emily the Enchanted will have an RxLevel: Real Fae, but the TxLevel: Real.
No \&FAEDESC on her, although she\textquotesingle{}ll be able to see it the one on Frida.
Gil the Garou, while travelling through the Umbra, will have RxLevel: Umbra
and TxLevel: Umbra. \&UMBRADESC is his friend. He won\textquotesingle{}t see mortals or other
characters who are not in the Umbra.

\sphinxAtStartPar
Barbie the Bastet, who\textquotesingle{}s only peeking in the Umbra, without going there,
will have RxLevel: Umbra, TxLevel: Real. Dangerous position since she
can\textquotesingle{}t see the things that see her.

\sphinxAtStartPar
Deanna the Drake, who activates her spirit vision, will have
RxLevel: Real Umbra and TxLevel: Real. She will see characters in Umbra and
real world at the same time and perceive the desc appropiate to the realm
the ohter character is in.

\sphinxAtStartPar
Wanda the Wraith: RxLevel: Real Shadow, TxLevel: Shadow. Her @desc
would be empty, but the \&SHADOWDESC should be set.
Marie the Mortal+ Medium: RxLevel: Real Shadow, TxLevel: Real

\sphinxAtStartPar
Global code objects that all characters should be able to use:

\begin{sphinxVerbatim}[commandchars=\\\{\}]
\PYG{n}{RxLevel}\PYG{p}{:} \PYG{n}{All}\PYG{p}{,} \PYG{n}{TxLevel}\PYG{p}{:} \PYG{n}{All}
\end{sphinxVerbatim}


\subsection{Example 3: Softcode}
\label{\detokenize{advanced:example-3-softcode}}
\sphinxAtStartPar
Considering the config directives from example 2 and assuming a function
getstat(\textless{}dbref\textgreater{},\textless{}stat\textgreater{}) that will return the value of a player\textquotesingle{}s stat from
the sheet here are softcode examples that implement some of the WoD powers.
In a real game you would have to use some more checks, of course.

\begin{sphinxVerbatim}[commandchars=\\\{\}]
@create Reality Levels Commands (RLS)
\PYGZam{}CMD\PYGZus{}OBFON rls=\PYGZdl{}+obf/on:@switch [setr(0, getstat(\PYGZpc{}\PYGZsh{},Obfuscate))]=0, @pemit \PYGZpc{}\PYGZsh{}=You don\PYGZsq{}t have Obfuscate!, \PYGZob{}@txlevel \PYGZpc{}\PYGZsh{}=!All Obf\PYGZpc{}q0; @pemit \PYGZpc{}\PYGZsh{}=You are now invisible.\PYGZcb{}
\PYGZam{}CMD\PYGZus{}OBFOFF rls=\PYGZdl{}+obf/off:@txlevel \PYGZpc{}\PYGZsh{}=Real; @pemit \PYGZpc{}\PYGZsh{}=You are now visible.
@@ Note: +obf/on clears all TxLevels before setting the appropiate Obf
@@ This is necesary, because a character might advance from Obf2 to
@@ Obf3 and he should be visible /only/ on the Obf3 level.
@@ +obf/off simply sets the Real Tx level, without clearing the Obf. The
@@ reason is the Mask. Players with Obf3 or higher who use the Mask should
@@ +obf/on, then +obf/off after approval and everything is set.
\PYGZam{}CMD\PYGZus{}AUSPEXON rls=\PYGZdl{}+auspex/on:@switch [setr(0, getstat(\PYGZpc{}\PYGZsh{}, Auspex))]=0, @pemit \PYGZpc{}\PYGZsh{}=You don\PYGZsq{}t have Auspex!, \PYGZob{}@rxlevel \PYGZpc{}\PYGZsh{}=[iter(lnum(1, \PYGZpc{}q0), Obf\PYGZsh{}\PYGZsh{})]; @pemit \PYGZpc{}\PYGZsh{}=Auspex enabled.\PYGZcb{}
\PYGZam{}CMD\PYGZus{}AUSPEXOFF rls=\PYGZdl{}+auspex/off:@switch [hasrxlevel(\PYGZpc{}\PYGZsh{}, Obf1)]=0, @pemit \PYGZpc{}\PYGZsh{}= You don\PYGZsq{}t have Auspex enabled!, \PYGZob{}@rxlevel \PYGZpc{}\PYGZsh{}=[iter(lnum(1, 5), !Obf\PYGZsh{}\PYGZsh{})]; @pemit \PYGZpc{}\PYGZsh{}=Auspex disabled.\PYGZcb{}
\PYGZam{}CMD\PYGZus{}UMBRAENTER rls=\PYGZdl{}+umbra/enter:@rxlevel \PYGZpc{}\PYGZsh{}=!Real Umbra; @txlevel \PYGZpc{}\PYGZsh{}= !Real Umbra; @pemit \PYGZpc{}\PYGZsh{}=You are now in the Umbra.
\PYGZam{}CMD\PYGZus{}UMBRALEAVE rls=\PYGZdl{}+umbra/leave:@rxlevel \PYGZpc{}\PYGZsh{}=Real !Umbra; @txlevel \PYGZpc{}\PYGZsh{}= Real !Umbra; @pemit \PYGZpc{}\PYGZsh{}=You left the Umbra.
\PYGZam{}CMD\PYGZus{}PEEKON rls=\PYGZdl{}+peek/on:@switch hastxlevel(\PYGZpc{}\PYGZsh{},Umbra)=1, \PYGZob{}@rxlevel \PYGZpc{}\PYGZsh{}=Real !Umbra; @pemit \PYGZpc{}\PYGZsh{}=You are now peeking in the real world\PYGZcb{}, \PYGZob{}@rxlevel \PYGZpc{}\PYGZsh{}=!Real Umbra; @pemit \PYGZpc{}\PYGZsh{}=You are now peeking into the Umbra\PYGZcb{}
\PYGZam{}CMD\PYGZus{}PEEKOFF rls=\PYGZdl{}+peek/off:@rxlevel \PYGZpc{}\PYGZsh{}=!Real !Umbra [setinter(Real Umbra, txlevel(\PYGZpc{}\PYGZsh{}))]; @pemit \PYGZpc{}\PYGZsh{}=You are no longer peeking.
\end{sphinxVerbatim}


\section{Execscript and external programs and scripts}
\label{\detokenize{advanced:execscript-and-external-programs-and-scripts}}

\subsection{Execscript variables}
\label{\detokenize{advanced:execscript-variables}}

\subsubsection{Execscript Built in variables}
\label{\detokenize{advanced:execscript-built-in-variables}}

\begin{savenotes}\sphinxattablestart
\centering
\begin{tabulary}{\linewidth}[t]{|T|T|}
\hline
\sphinxstyletheadfamily 
\sphinxAtStartPar
Variable
&\sphinxstyletheadfamily 
\sphinxAtStartPar
Description
\\
\hline
\sphinxAtStartPar
MUSH\_PLAYER
&
\sphinxAtStartPar
player dbref\#
\\
\hline
\sphinxAtStartPar
MUSH\_CAUSE
&
\sphinxAtStartPar
cause dbref\#
\\
\hline
\sphinxAtStartPar
MUSH\_CALLER
&
\sphinxAtStartPar
caller dbref\#
\\
\hline
\sphinxAtStartPar
MUSH\_OWNER
&
\sphinxAtStartPar
owner of player dbref\#
\\
\hline
\sphinxAtStartPar
MUSH\_FLAGS
&
\sphinxAtStartPar
space delimited list of flags on player
\\
\hline
\sphinxAtStartPar
MUSH\_TOGGLES
&
\sphinxAtStartPar
space delimited list of toggles on player
\\
\hline
\sphinxAtStartPar
MUSH\_OFLAGS
&
\sphinxAtStartPar
space delimited list of flags of player owner
\\
\hline
\sphinxAtStartPar
MUSH\_OTOGGLES
&
\sphinxAtStartPar
space delimited list of toggles of player owner
\\
\hline
\sphinxAtStartPar
MUSHL\_VARS
&
\sphinxAtStartPar
space delimited list of MUSH attributes from player
This is passed from the mush\textquotesingle{}s EXECSCRIPT\_VARS attr
\\
\hline
\end{tabulary}
\par
\sphinxattableend\end{savenotes}


\subsubsection{Execscript Dynamic variables}
\label{\detokenize{advanced:execscript-dynamic-variables}}

\begin{savenotes}\sphinxattablestart
\centering
\begin{tabulary}{\linewidth}[t]{|T|T|}
\hline
\sphinxstyletheadfamily 
\sphinxAtStartPar
Variable
&\sphinxstyletheadfamily 
\sphinxAtStartPar
Description
\\
\hline
\sphinxAtStartPar
MUSHV\_\textless{}arg\textgreater{}
&
\sphinxAtStartPar
\textless{}arg\textgreater{} variable passed from MUSHL\_VARS
These are the attributes from EXECSCRIPT\_VARS
\\
\hline
\end{tabulary}
\par
\sphinxattableend\end{savenotes}


\subsubsection{Execscript Register variables}
\label{\detokenize{advanced:execscript-register-variables}}

\begin{savenotes}\sphinxattablestart
\centering
\begin{tabulary}{\linewidth}[t]{|T|T|}
\hline
\sphinxstyletheadfamily 
\sphinxAtStartPar
Variable
&\sphinxstyletheadfamily 
\sphinxAtStartPar
Description
\\
\hline
\sphinxAtStartPar
MUSHQ\_\textless{}arg\textgreater{}
&
\sphinxAtStartPar
setq registers 0\sphinxhyphen{}9 and a\sphinxhyphen{}z
\\
\hline
\sphinxAtStartPar
MUSHQN\_\textless{}arg\textgreater{}
&
\sphinxAtStartPar
labels that are assigned the setq vars
\\
\hline
\sphinxAtStartPar
MUSHN\_\textless{}arg\textgreater{}
&
\sphinxAtStartPar
The labels that were defined by any register
Note: they must be ASCII\sphinxhyphen{}7 clean and contain no white spaces
\\
\hline
\end{tabulary}
\par
\sphinxattableend\end{savenotes}


\subsection{Execsript set object}
\label{\detokenize{advanced:execsript-set-object}}
\sphinxAtStartPar
The script executed with execscript() will read in a file with the same name
as the script ending in \textquotesingle{}.set\textquotesingle{}.  This is a loader object that will set attributes
and registers back into the mush that you wish to pass from the script. The
fields are SPACE SEPARATED.  The values are NOT evaluated.


\subsubsection{Execscript set object field format}
\label{\detokenize{advanced:execscript-set-object-field-format}}

\paragraph{Execscript set object Dynamic variables}
\label{\detokenize{advanced:execscript-set-object-dynamic-variables}}
\begin{sphinxVerbatim}[commandchars=\\\{\}]
\PYG{n}{VARNAME}        \PYG{n}{OWNER}        \PYG{n}{CONTENTS} \PYG{p}{(}\PYG{o+ow}{or} \PYG{n}{leave} \PYG{n}{null} \PYG{n}{to} \PYG{n}{clear}\PYG{p}{)}
\end{sphinxVerbatim}


\subparagraph{Execscript set object Dynamic variables Examples}
\label{\detokenize{advanced:execscript-set-object-dynamic-variables-examples}}
\begin{sphinxVerbatim}[commandchars=\\\{\}]
\PYG{n}{SEX} \PYG{c+c1}{\PYGZsh{}123 Male}
\PYG{n}{DESC} \PYG{c+c1}{\PYGZsh{}123 \PYGZpc{}r\PYGZpc{}tThis is a willow tree of unique description\PYGZpc{}r\PYGZpc{}rIt sways in the wind.}
\PYG{n}{RED} \PYG{c+c1}{\PYGZsh{}123 This is the color \PYGZpc{}ch\PYGZpc{}crred\PYGZpc{}cn.}
\PYG{n}{WIPETHISATTR} \PYG{c+c1}{\PYGZsh{}123}
\PYG{n}{MULTILINE} \PYG{c+c1}{\PYGZsh{}123 This is a line}
\PYG{n}{that} \PYG{n}{continues} \PYG{n}{on}
\PYG{n}{because} \PYG{n}{of} \PYG{n}{the} \PYG{n}{line} \PYG{n}{feed} \PYG{p}{(}\PYG{n}{a} \PYG{n}{control}\PYG{o}{\PYGZhy{}}\PYG{n}{M}\PYG{p}{)} \PYG{n}{on} \PYG{n}{each} \PYG{n}{line}
\PYG{n}{on} \PYG{n}{the} \PYG{n}{lines} \PYG{n}{above}
\end{sphinxVerbatim}


\paragraph{Execscript set object Register variables}
\label{\detokenize{advanced:execscript-set-object-register-variables}}
\begin{sphinxVerbatim}[commandchars=\\\{\}]
\PYG{n}{REGISTER}       \PYG{n}{Q}            \PYG{n}{CONTENTS} \PYG{p}{(}\PYG{o+ow}{or} \PYG{n}{leave} \PYG{n}{null} \PYG{n}{to} \PYG{n}{clear}\PYG{p}{)}
\end{sphinxVerbatim}


\subparagraph{Execscript set object Register variables Examples}
\label{\detokenize{advanced:execscript-set-object-register-variables-examples}}
\begin{sphinxadmonition}{note}{Note:}
\sphinxAtStartPar
The last exammple clears register 0
\end{sphinxadmonition}

\begin{sphinxVerbatim}[commandchars=\\\{\}]
\PYG{n}{W} \PYG{n}{Q} \PYG{n}{This} \PYG{o+ow}{is} \PYG{n}{stored} \PYG{o+ow}{in} \PYG{n}{register} \PYG{n}{W}
\PYG{l+m+mi}{1} \PYG{n}{Q} \PYG{n}{This} \PYG{o+ow}{is} \PYG{n}{stored} \PYG{o+ow}{in} \PYG{n}{register} \PYG{l+m+mi}{1}
\PYG{l+m+mi}{0} \PYG{n}{Q}
\PYG{n}{foo} \PYG{n}{QN} \PYG{n}{this} \PYG{n}{sets} \PYG{n}{register} \PYG{k}{with} \PYG{n}{label} \PYG{l+s+s1}{\PYGZsq{}}\PYG{l+s+s1}{foo}\PYG{l+s+s1}{\PYGZsq{}}
\end{sphinxVerbatim}


\subsection{Execscript Example bash script}
\label{\detokenize{advanced:execscript-example-bash-script}}
\begin{sphinxVerbatim}[commandchars=\\\{\}]
\PYG{c+ch}{\PYGZsh{}!/bin/bash}
\PYG{n+nb}{echo} \PYG{l+s+s2}{\PYGZdq{}}\PYG{l+s+s2}{This was called by player }\PYG{l+s+si}{\PYGZdl{}\PYGZob{}}\PYG{n+nv}{MUSH\PYGZus{}PLAYER}\PYG{l+s+si}{\PYGZcb{}}\PYG{l+s+s2}{ that is owned by }\PYG{l+s+si}{\PYGZdl{}\PYGZob{}}\PYG{n+nv}{MUSH\PYGZus{}OWNER}\PYG{l+s+si}{\PYGZcb{}}\PYG{l+s+s2}{\PYGZdq{}}
\PYG{n+nb}{echo} \PYG{l+s+s2}{\PYGZdq{}Displaying Registers:\PYGZdq{}}
\PYG{n+nv}{regs}\PYG{o}{=}\PYG{l+s+s2}{\PYGZdq{}0 1 2 3 4 5 6 7 8 9 A B C D E F G H I J K L M N O P Q R S T U V W X Y Z\PYGZdq{}}
\PYG{k}{for} list \PYG{k}{in} \PYG{l+s+si}{\PYGZdl{}\PYGZob{}}\PYG{n+nv}{regs}\PYG{l+s+si}{\PYGZcb{}}
\PYG{k}{do}
    \PYG{n+nb}{eval} \PYG{n+nb}{echo} \PYG{l+s+s2}{\PYGZdq{}}\PYG{l+s+s2}{Register }\PYG{l+s+si}{\PYGZdl{}\PYGZob{}}\PYG{n+nv}{list}\PYG{l+s+si}{\PYGZcb{}}\PYG{l+s+s2}{: \PYGZbs{}\PYGZdl{}\PYGZob{}MUSHQ\PYGZus{}}\PYG{l+s+si}{\PYGZdl{}\PYGZob{}}\PYG{n+nv}{list}\PYG{l+s+si}{\PYGZcb{}}\PYG{l+s+s2}{\PYGZcb{}}\PYG{l+s+s2}{\PYGZdq{}}
\PYG{k}{done}
\PYG{n+nb}{echo} \PYG{l+s+s2}{\PYGZdq{}Displaying variables:\PYGZdq{}}
\PYG{k}{for} vars \PYG{k}{in} \PYG{l+s+si}{\PYGZdl{}\PYGZob{}}\PYG{n+nv}{MUSHL\PYGZus{}VARS}\PYG{l+s+si}{\PYGZcb{}}
\PYG{k}{do}
    \PYG{n+nb}{eval} \PYG{n+nb}{echo} \PYG{l+s+s2}{\PYGZdq{}}\PYG{l+s+s2}{Variable }\PYG{l+s+si}{\PYGZdl{}\PYGZob{}}\PYG{n+nv}{vars}\PYG{l+s+si}{\PYGZcb{}}\PYG{l+s+s2}{: \PYGZbs{}\PYGZdl{}\PYGZob{}MUSHV\PYGZus{}}\PYG{l+s+si}{\PYGZdl{}\PYGZob{}}\PYG{n+nv}{vars}\PYG{l+s+si}{\PYGZcb{}}\PYG{l+s+s2}{\PYGZcb{}}\PYG{l+s+s2}{\PYGZdq{}}
\PYG{k}{done}
\end{sphinxVerbatim}


\subsection{Exescript Notes and warnings}
\label{\detokenize{advanced:exescript-notes-and-warnings}}
\sphinxAtStartPar
While MUSHL\_VARS are sanitized on what is allowable as a mush variable, this
is not necessarilly sanitized on what the calling script can fetch as a valid
variable.  Of note, you can not set environment variables that contain an
equals sign.  Be aware of this limitation.

\sphinxAtStartPar
Remember, MUSHL\_VARS is the environment variable seen by the script.
This is EXECSCRIPT\_VARS on the mush itself, that is the attribute set
on the target that contains the execscript() that is being executed.


\subsection{Scripts to be used with execscript}
\label{\detokenize{advanced:scripts-to-be-used-with-execscript}}
\begin{sphinxVerbatim}[commandchars=\\\{\}]
\PYG{n}{account}\PYG{o}{/}                       \PYG{o}{\PYGZhy{}}\PYG{o}{\PYGZhy{}} \PYG{n}{Directory} \PYG{k}{for} \PYG{n}{execscripts} \PYG{n}{relating} \PYG{n}{to} \PYG{n}{account} \PYG{n}{creation}
\PYG{n}{compile39}\PYG{o}{.}\PYG{n}{sh}                   \PYG{o}{\PYGZhy{}}\PYG{o}{\PYGZhy{}} \PYG{n}{Script} \PYG{k}{for} \PYG{n}{patching} \PYG{o+ow}{and} \PYG{n}{compiling} \PYG{n}{RhostMUSH} \PYG{l+m+mf}{3.9}
\PYG{n+nb}{compile}\PYG{o}{.}\PYG{n}{sh}                     \PYG{o}{\PYGZhy{}}\PYG{o}{\PYGZhy{}} \PYG{n}{Script} \PYG{k}{for} \PYG{n}{patching} \PYG{o+ow}{and} \PYG{n}{compiling} \PYG{n}{RhostMUSH}
\PYG{n}{config}\PYG{o}{.}\PYG{n}{sh}                      \PYG{o}{\PYGZhy{}}\PYG{o}{\PYGZhy{}} \PYG{n}{Script} \PYG{k}{for} \PYG{n}{setting} \PYG{n+nb}{compile} \PYG{n}{time} \PYG{n}{options} \PYG{k}{for} \PYG{n}{RhostMUSH}
\PYG{n}{debug}\PYG{o}{.}\PYG{n}{sh}                       \PYG{o}{\PYGZhy{}}\PYG{o}{\PYGZhy{}} \PYG{n}{Script} \PYG{k}{for} \PYG{n}{debugging} \PYG{n}{RhostMUSH}
\PYG{n+nb}{dict}\PYG{o}{.}\PYG{n}{sh}                        \PYG{o}{\PYGZhy{}}\PYG{o}{\PYGZhy{}} \PYG{n}{Script} \PYG{k}{for} \PYG{n}{querying} \PYG{n}{a} \PYG{n}{dictionary}
\PYG{n}{diff}\PYG{o}{.}\PYG{n}{sh}                        \PYG{o}{\PYGZhy{}}\PYG{o}{\PYGZhy{}} \PYG{n}{Script} \PYG{k}{for} \PYG{n}{querying} \PYG{n}{differences} \PYG{n}{between} \PYG{n}{two} \PYG{n}{arguments}
\PYG{n}{fortune}\PYG{o}{.}\PYG{n}{sh}                     \PYG{o}{\PYGZhy{}}\PYG{o}{\PYGZhy{}} \PYG{n}{Script} \PYG{k}{for} \PYG{n}{querying} \PYG{n}{fortune} \PYG{n}{program}
\PYG{n}{fullweather}\PYG{o}{.}\PYG{n}{sh}                 \PYG{o}{\PYGZhy{}}\PYG{o}{\PYGZhy{}} \PYG{n}{Script} \PYG{k}{for} \PYG{n}{querying} \PYG{n}{a} \PYG{n}{graphical} \PYG{n}{weather} \PYG{n}{forecast} \PYG{p}{(}\PYG{n}{alternative}\PYG{p}{)}
\PYG{n}{git}\PYG{o}{.}\PYG{n}{sh}                         \PYG{o}{\PYGZhy{}}\PYG{o}{\PYGZhy{}} \PYG{n}{Script} \PYG{k}{for} \PYG{n}{querying} \PYG{n}{git} \PYG{n}{version} \PYG{n}{of} \PYG{n}{RhostMUSH}
\PYG{n}{hello}\PYG{o}{.}\PYG{n}{sh}                       \PYG{o}{\PYGZhy{}}\PYG{o}{\PYGZhy{}} \PYG{n}{Script} \PYG{k}{for} \PYG{n}{teaching} \PYG{n}{execscript} \PYG{k}{for} \PYG{l+s+s1}{\PYGZsq{}}\PYG{l+s+s1}{Hello World}\PYG{l+s+s1}{\PYGZsq{}}
\PYG{n}{iostat}\PYG{o}{.}\PYG{n}{sh}                      \PYG{o}{\PYGZhy{}}\PYG{o}{\PYGZhy{}} \PYG{n}{Script} \PYG{k}{for} \PYG{n}{querying} \PYG{n}{server} \PYG{n}{stats} \PYG{n}{of} \PYG{n}{RhostMUSH}
\PYG{n}{jsonvalidate}\PYG{o}{.}\PYG{n}{sh}                \PYG{o}{\PYGZhy{}}\PYG{o}{\PYGZhy{}} \PYG{n}{Python} \PYG{n}{Script} \PYG{k}{for} \PYG{n}{validating} \PYG{n}{JSON}
\PYG{n}{logsearch}\PYG{o}{.}\PYG{n}{sh}                   \PYG{o}{\PYGZhy{}}\PYG{o}{\PYGZhy{}} \PYG{n}{Script} \PYG{k}{for} \PYG{n}{searching} \PYG{n}{throgh} \PYG{n}{logfiles} \PYG{k}{for} \PYG{n}{RhostMUSH}
\PYG{n}{math\PYGZus{}example}\PYG{o}{.}\PYG{n}{sh}                \PYG{o}{\PYGZhy{}}\PYG{o}{\PYGZhy{}} \PYG{n}{Examples} \PYG{n}{of} \PYG{n}{math} \PYG{n}{operations} \PYG{n}{to} \PYG{n}{be} \PYG{n}{used} \PYG{k}{with} \PYG{n}{math}\PYG{o}{.}\PYG{n}{sh}
\PYG{n}{math}\PYG{o}{.}\PYG{n}{sh}                        \PYG{o}{\PYGZhy{}}\PYG{o}{\PYGZhy{}} \PYG{n}{Script} \PYG{k}{for} \PYG{n}{mathematical} \PYG{n}{operations}
\PYG{n}{memory}\PYG{o}{.}\PYG{n}{sh}                      \PYG{o}{\PYGZhy{}}\PYG{o}{\PYGZhy{}} \PYG{n}{Script} \PYG{k}{for} \PYG{n}{querying} \PYG{n}{memory} \PYG{n}{usage} \PYG{n}{of} \PYG{n}{RhostMUSH}
\PYG{n}{mkindx}\PYG{o}{.}\PYG{n}{sh}                      \PYG{o}{\PYGZhy{}}\PYG{o}{\PYGZhy{}} \PYG{n}{Script} \PYG{k}{for} \PYG{n}{indexing} \PYG{n}{RhostMSH} \PYG{n}{helpfiles}
\PYG{n}{pastebinread}\PYG{o}{.}\PYG{n}{sh}                \PYG{o}{\PYGZhy{}}\PYG{o}{\PYGZhy{}} \PYG{n}{Script} \PYG{k}{for} \PYG{n}{reading} \PYG{n}{data} \PYG{k+kn}{from} \PYG{n+nn}{a} \PYG{n}{pastebin} \PYG{n}{URL}
\PYG{n}{pastebinwrite}\PYG{o}{.}\PYG{n}{sh}               \PYG{o}{\PYGZhy{}}\PYG{o}{\PYGZhy{}} \PYG{n}{Script} \PYG{k}{for} \PYG{n}{writing} \PYG{n}{data} \PYG{n}{to} \PYG{n}{a} \PYG{n}{pastebin}
\PYG{n}{qspell}\PYG{o}{.}\PYG{n}{sh}                      \PYG{o}{\PYGZhy{}}\PYG{o}{\PYGZhy{}} \PYG{n}{Script} \PYG{k}{for} \PYG{n}{checking} \PYG{n}{spelling} \PYG{p}{(}\PYG{n}{alternative}\PYG{p}{)}
\PYG{n}{quota}\PYG{o}{.}\PYG{n}{sh}                       \PYG{o}{\PYGZhy{}}\PYG{o}{\PYGZhy{}} \PYG{n}{Script} \PYG{k}{for} \PYG{n}{checking} \PYG{n}{disk} \PYG{n}{quote} \PYG{o+ow}{and} \PYG{n}{usage}
\PYG{n}{random}\PYG{o}{.}\PYG{n}{sh}                      \PYG{o}{\PYGZhy{}}\PYG{o}{\PYGZhy{}} \PYG{n}{Script} \PYG{k}{for} \PYG{n}{getting} \PYG{n}{a} \PYG{n}{random} \PYG{n}{number}
\PYG{n}{roomlog}\PYG{o}{.}\PYG{n}{sh}                     \PYG{o}{\PYGZhy{}}\PYG{o}{\PYGZhy{}} \PYG{n}{Script} \PYG{k}{for} \PYG{n}{viewing} \PYG{n}{logs} \PYG{o+ow}{in} \PYG{n}{roomlog} \PYG{n}{directory}
\PYG{n}{spell}\PYG{o}{.}\PYG{n}{sh}                       \PYG{o}{\PYGZhy{}}\PYG{o}{\PYGZhy{}} \PYG{n}{Script} \PYG{k}{for} \PYG{n}{checking} \PYG{n}{spelling}
\PYG{n}{stats}\PYG{o}{.}\PYG{n}{sh}                       \PYG{o}{\PYGZhy{}}\PYG{o}{\PYGZhy{}} \PYG{n}{Script} \PYG{k}{for} \PYG{n}{querying} \PYG{n}{server} \PYG{o+ow}{and} \PYG{n}{process} \PYG{n}{stats} \PYG{k}{for} \PYG{n}{RhostMUSH}
\PYG{n}{thes}\PYG{o}{.}\PYG{n}{sh}                        \PYG{o}{\PYGZhy{}}\PYG{o}{\PYGZhy{}} \PYG{n}{Script} \PYG{k}{for} \PYG{n}{adding} \PYG{n}{a} \PYG{n}{word} \PYG{n}{to} \PYG{n}{the} \PYG{n}{dictionary} \PYG{k}{for} \PYG{n}{spell} \PYG{n}{scripts}
\PYG{n}{tinyurl}\PYG{o}{.}\PYG{n}{sh}                     \PYG{o}{\PYGZhy{}}\PYG{o}{\PYGZhy{}} \PYG{n}{Script} \PYG{k}{for} \PYG{n}{shortening} \PYG{n}{a} \PYG{n}{URL}
\PYG{n}{weather}\PYG{o}{.}\PYG{n}{sh}                     \PYG{o}{\PYGZhy{}}\PYG{o}{\PYGZhy{}} \PYG{n}{Script} \PYG{k}{for} \PYG{n}{querying} \PYG{n}{a} \PYG{n}{graphical} \PYG{n}{weather} \PYG{n}{forecast}
\PYG{n}{web}\PYG{o}{.}\PYG{n}{sh}                         \PYG{o}{\PYGZhy{}}\PYG{o}{\PYGZhy{}} \PYG{n}{Script} \PYG{k}{for} \PYG{n}{querying} \PYG{n}{an} \PYG{n}{arbitary} \PYG{n}{website}
\end{sphinxVerbatim}


\section{Using printf() for advanced text output}
\label{\detokenize{advanced:using-printf-for-advanced-text-output}}
\sphinxAtStartPar
The function printf() in Rhost can be used to greatly reduce coding in efforts for outputs,
screens and data display.  It can automatically center, justify and wrap the text parameters given to it.


\subsection{Printf Example one}
\label{\detokenize{advanced:printf-example-one}}
\begin{sphinxVerbatim}[commandchars=\\\{\}]
@emit printf(|\PYGZdl{}\PYGZhy{}12s|\PYGZdl{}12s|\PYGZdl{}\PYGZca{}12s\PYGZdl{}\PYGZam{}14s\PYGZdl{}\PYGZus{}12s|,a b c, d e f, g h i, wrap(lnum(20),12, l, |, |), j k l)

|a b c       |       d e f|   g h i    |0 1 2 3 4 5 |j     k    l|
                                       |6 7 8 9 10  |
                                       |11 12 13 14 |
                                       |15 16 17 18 |
                                       |19          |
\end{sphinxVerbatim}


\subsection{Printf Example two}
\label{\detokenize{advanced:printf-example-two}}
\begin{sphinxVerbatim}[commandchars=\\\{\}]
@emit printf(\PYGZdl{}14\PYGZam{}s \PYGZdl{}\PYGZca{}4\PYGZam{}s \PYGZdl{}\PYGZhy{}3\PYGZam{}s \PYGZdl{}15\PYGZam{}s,
iter(Bruised|Hurt|Injured|Wounded|Mauled|Crippled|Incapacitated,\PYGZsh{}\PYGZsh{},|,\PYGZpc{}R),
iter(|\PYGZhy{}1|\PYGZhy{}1|\PYGZhy{}2|\PYGZhy{}2|\PYGZhy{}5|,\PYGZsh{}\PYGZsh{},|,\PYGZpc{}r),iter(lnum(1,7),\PYGZpc{}[[if(gte(get(\PYGZpc{}\PYGZsh{}/damage),\PYGZsh{}\PYGZsh{}),X,\PYGZpc{}b)]\PYGZpc{}],,\PYGZpc{}r),
* Aggravated\PYGZpc{}RX Lethal\PYGZpc{}R/ Bashing)

         Bruised      [ ]    * Aggravated
            Hurt  \PYGZhy{}1  [ ]        X Lethal
         Injured  \PYGZhy{}1  [ ]       / Bashing
         Wounded  \PYGZhy{}2  [ ]
          Mauled  \PYGZhy{}2  [ ]
        Crippled  \PYGZhy{}5  [ ]
   Incapacitated      [ ]
\end{sphinxVerbatim}


\subsection{Printf Example three}
\label{\detokenize{advanced:printf-example-three}}
\begin{sphinxVerbatim}[commandchars=\\\{\}]
@emit [printf(\PYGZdl{}\PYGZhy{}10|\PYGZdq{}\PYGZsq{}s\PYGZdl{}\PYGZhy{}60|\PYGZdq{}s,a b c d e f g h i j k l m n o p q r s t u v w x y z,
this is a test a groovy test blah blah blah [repeat(blah\PYGZpc{}b,100)])]END!

a b c d e this is a test a groovy test blah blah blah blah blah blah
f g h i j blah blah blah blah blah blah blah blah blah blah blah blah
k l m n o blah blah blah blah blah blah blah blah blah blah blah blah
p q r s t blah blah blah blah blah blah blah blah blah blah blah blah
u v w x y blah blah blah blah blah blah blah blah blah blah blah blah
z         blah blah blah blah blah blah blah blah blah blah blah blah
blah blah blah blah blah blah blah blah blah blah blah blah blah blah
blah blah blah blah blah blah blah blah blah blah blah blah blah blah
blah blah blah blah blah blah blah                                    END!
\end{sphinxVerbatim}


\chapter{Format for image files}
\label{\detokenize{advanced:format-for-image-files}}
\sphinxAtStartPar
The image format goes like this:


\begin{savenotes}\sphinxatlongtablestart\begin{longtable}[c]{|l|l|l|}
\hline
\sphinxstyletheadfamily 
\sphinxAtStartPar
Data Type
&\sphinxstyletheadfamily 
\sphinxAtStartPar
Example
&\sphinxstyletheadfamily 
\sphinxAtStartPar
Description
\\
\hline
\endfirsthead

\multicolumn{3}{c}%
{\makebox[0pt]{\sphinxtablecontinued{\tablename\ \thetable{} \textendash{} continued from previous page}}}\\
\hline
\sphinxstyletheadfamily 
\sphinxAtStartPar
Data Type
&\sphinxstyletheadfamily 
\sphinxAtStartPar
Example
&\sphinxstyletheadfamily 
\sphinxAtStartPar
Description
\\
\hline
\endhead

\hline
\multicolumn{3}{r}{\makebox[0pt][r]{\sphinxtablecontinued{continues on next page}}}\\
\endfoot

\endlastfoot

\sphinxAtStartPar
INT
&
\sphinxAtStartPar
3
&
\sphinxAtStartPar
TYPE: room 0, thing 1, exit 2, player 3, zone 4, garbage 5
\\
\hline
\sphinxAtStartPar
STRING
&
\sphinxAtStartPar
Wizard
&
\sphinxAtStartPar
NAME: of the target.  Verbatum, no quotes surround it
\\
\hline
\sphinxAtStartPar
*INT
&
\sphinxAtStartPar
123
&
\sphinxAtStartPar
LOCATION: dbref\# of the target.  No prepending \textquotesingle{}\#\textquotesingle{} used.
\\
\hline
\sphinxAtStartPar
*INT
&
\sphinxAtStartPar
234
&
\sphinxAtStartPar
CONTENTS: The first content in a linked list content table (\sphinxhyphen{}1 if none)
\\
\hline
\sphinxAtStartPar
*INT
&
\sphinxAtStartPar
345
&
\sphinxAtStartPar
EXITS: The first exit in a linked list exit table (\sphinxhyphen{}1 if none)
\\
\hline
\sphinxAtStartPar
*INT
&
\sphinxAtStartPar
0
&
\sphinxAtStartPar
LINK: This is the \textquotesingle{}home\textquotesingle{} of the object or what it\textquotesingle{}s linked to (\sphinxhyphen{}1 for none)
\\
\hline
\sphinxAtStartPar
*INT
&
\sphinxAtStartPar
123
&
\sphinxAtStartPar
NEXT: The next thing after this item for a content holder
\\
\hline
\sphinxAtStartPar
STRING
&
\sphinxAtStartPar
\#123
&
\sphinxAtStartPar
LOCK: The boolean string lock if it exists.  (empty if no lock)
\\
\hline
\sphinxAtStartPar
*INT
&
\sphinxAtStartPar
1
&
\sphinxAtStartPar
OWNER: The dbref\# owner of the target.  For players same dbref as player.
\\
\hline
\sphinxAtStartPar
INT
&
\sphinxAtStartPar
789
&
\sphinxAtStartPar
PARENT: The parent of the target.  (\sphinxhyphen{}1 if none)
\\
\hline
\sphinxAtStartPar
*INT
&
\sphinxAtStartPar
99999
&
\sphinxAtStartPar
MONEY: The int value of the money the players has.
\\
\hline
\sphinxAtStartPar
INT
&
\sphinxAtStartPar
194592
&
\sphinxAtStartPar
FLAGS1: The first word of flags (@set flags) on a player      (see FLAGS)
\\
\hline
\sphinxAtStartPar
INT
&
\sphinxAtStartPar
194222
&
\sphinxAtStartPar
FLAGS2: The second word of flags (@set flags) on a player     (see FLAGS)
\\
\hline
\sphinxAtStartPar
INT
&
\sphinxAtStartPar
199999
&
\sphinxAtStartPar
FLAGS3: The third word of flags (@set flags {[}{]}) on a player   (see FLAGS)
\\
\hline
\sphinxAtStartPar
INT
&
\sphinxAtStartPar
1582958
&
\sphinxAtStartPar
FLAGS4: The forth word of flags (@set flags {[}{]}) on a player   (see FLAGS)
\\
\hline
\sphinxAtStartPar
INT
&
\sphinxAtStartPar
159955
&
\sphinxAtStartPar
TOGGLES1: The first word of toggles (@toggle) on a player    (see TOGGLES)
\\
\hline
\sphinxAtStartPar
INT
&
\sphinxAtStartPar
159958
&
\sphinxAtStartPar
TOGGLES2: The second word of toggles (@toggle) on a player   (see TOGGLES)
\\
\hline
\sphinxAtStartPar
INT
&
\sphinxAtStartPar
159958
&
\sphinxAtStartPar
POWER1: The first word of powers (@power) on a player         (see POWERS)
\\
\hline
\sphinxAtStartPar
INT
&
\sphinxAtStartPar
159958
&
\sphinxAtStartPar
POWER2: The second word of powers (@power) on a player        (see POWERS)
\\
\hline
\sphinxAtStartPar
INT
&
\sphinxAtStartPar
159958
&
\sphinxAtStartPar
POWER3: The third word of powers (@power) on a player         (see POWERS)
\\
\hline
\sphinxAtStartPar
INT
&
\sphinxAtStartPar
159958
&
\sphinxAtStartPar
DEPOWER1: The first word of depowers (@depower) on a player  (see DEPOWERS)
\\
\hline
\sphinxAtStartPar
INT
&
\sphinxAtStartPar
159958
&
\sphinxAtStartPar
DEPOWER2: The second word of depowers (@depower) on a player (see DEPOWERS)
\\
\hline
\sphinxAtStartPar
INT
&
\sphinxAtStartPar
159958
&
\sphinxAtStartPar
DEPOWER3: The third word of depowers (@depower) on a player  (see DEPOWERS)
\\
\hline
\sphinxAtStartPar
INT
&
\sphinxAtStartPar
\sphinxhyphen{}1
&
\sphinxAtStartPar
ZONE(S): The list of zones starting here and ending with \textquotesingle{}\sphinxhyphen{}1\textquotesingle{}. (see ZONES)
\\
\hline
\sphinxAtStartPar
\textgreater{}STRING
&
\sphinxAtStartPar
\textgreater{}VA
&
\sphinxAtStartPar
ATTRIBUTENAME: Attribute name to store, starts with \textgreater{} identifier
\\
\hline
\sphinxAtStartPar
STRING
&
\sphinxAtStartPar
Wheee
&
\sphinxAtStartPar
ATTRIBUTECONTENTS: Contents of attribute.  Multi\sphinxhyphen{}lines seperate with \textasciicircum{}M (control\sphinxhyphen{}M)
\\
\hline
\sphinxAtStartPar
\textgreater{}STRING
&
\sphinxAtStartPar
\textgreater{}Desc
&
\sphinxAtStartPar
ATTRIBUTENAME: Another attribute to chain in
\\
\hline
\sphinxAtStartPar
STRING
&
\sphinxAtStartPar
Ugly
&
\sphinxAtStartPar
ATTRIBUTECONTENTS: Contents of the next attribute
\\
\hline
\sphinxAtStartPar
\textgreater{}STRING
&
\sphinxAtStartPar
*Password
&
\sphinxAtStartPar
PASSWORDATTRIB: Special password attribute.  Attribute name is \textquotesingle{}*Password\textquotesingle{}
\\
\hline
\sphinxAtStartPar
STRING
&
\sphinxAtStartPar
\$6\$xy\$xy
&
\sphinxAtStartPar
PASSWORDCONTENTS: The SHA512 password (if glibc 2.7+ supported on system) (see PASS)
\\
\hline
\sphinxAtStartPar
\textless{}
&
\sphinxAtStartPar
\textless{}
&
\sphinxAtStartPar
This is the marker to specify the end of the attribute contents.  This is always the last line
\\
\hline
\end{longtable}\sphinxatlongtableend\end{savenotes}

\begin{sphinxadmonition}{note}{Note:}
\sphinxAtStartPar
Any Data type starting with \textquotesingle{}*\textquotesingle{} is ignored when @snapshot/loading.
\end{sphinxadmonition}

\sphinxAtStartPar
The structure above with the examples would look like this in the file:

\begin{sphinxVerbatim}[commandchars=\\\{\}]
3
Wizard
123
234
345
0
123
\PYGZsh{}123
1
789
99999
194592
194222
199999
1582958
159955
159958
159958
159958
159958
159958
159958
159958
\PYGZhy{}1
\PYGZgt{}VA
Wheee
\PYGZgt{}Desc
Ugly
\PYGZgt{}*Password
\PYGZdl{}6\PYGZdl{}xy\PYGZdl{}xy
\PYGZlt{}
\end{sphinxVerbatim}


\section{HELP key indexes for the values}
\label{\detokenize{advanced:help-key-indexes-for-the-values}}
\sphinxAtStartPar
FLAGS: The following flags are to be used.  They are BITWISE masks that you
need to add together for the values that you apply

\begin{sphinxVerbatim}[commandchars=\\\{\}]
\PYG{c+cm}{/* First word of flags */}
\PYG{c+cp}{\PYGZsh{}}\PYG{c+cp}{define SEETHRU         0x00000008      }\PYG{c+cm}{/* Can see through to the other side */}
\PYG{c+cp}{\PYGZsh{}}\PYG{c+cp}{define WIZARD          0x00000010      }\PYG{c+cm}{/* gets automatic control */}
\PYG{c+cp}{\PYGZsh{}}\PYG{c+cp}{define LINK\PYGZus{}OK         0x00000020      }\PYG{c+cm}{/* anybody can link to this room */}
\PYG{c+cp}{\PYGZsh{}}\PYG{c+cp}{define DARK            0x00000040      }\PYG{c+cm}{/* Don\PYGZsq{}t show contents or presence */}
\PYG{c+cp}{\PYGZsh{}}\PYG{c+cp}{define JUMP\PYGZus{}OK         0x00000080      }\PYG{c+cm}{/* Others may @tel here */}
\PYG{c+cp}{\PYGZsh{}}\PYG{c+cp}{define STICKY          0x00000100      }\PYG{c+cm}{/* Object goes home when dropped */}
\PYG{c+cp}{\PYGZsh{}}\PYG{c+cp}{define DESTROY\PYGZus{}OK      0x00000200      }\PYG{c+cm}{/* Others may @destroy */}
\PYG{c+cp}{\PYGZsh{}}\PYG{c+cp}{define HAVEN           0x00000400      }\PYG{c+cm}{/* No killing here, or no pages */}
\PYG{c+cp}{\PYGZsh{}}\PYG{c+cp}{define QUIET           0x00000800      }\PYG{c+cm}{/* Prevent \PYGZsq{}feelgood\PYGZsq{} messages */}
\PYG{c+cp}{\PYGZsh{}}\PYG{c+cp}{define HALT            0x00001000      }\PYG{c+cm}{/* object cannot perform actions */}
\PYG{c+cp}{\PYGZsh{}}\PYG{c+cp}{define TRACE           0x00002000      }\PYG{c+cm}{/* Generate evaluation trace output */}
\PYG{c+cp}{\PYGZsh{}}\PYG{c+cp}{define GOING           0x00004000      }\PYG{c+cm}{/* object is available for recycling */}
\PYG{c+cp}{\PYGZsh{}}\PYG{c+cp}{define MONITOR         0x00008000      }\PYG{c+cm}{/* Process \PYGZca{}x:action listens on obj? */}
\PYG{c+cp}{\PYGZsh{}}\PYG{c+cp}{define MYOPIC          0x00010000      }\PYG{c+cm}{/* See things as nonowner/nonwizard */}
\PYG{c+cp}{\PYGZsh{}}\PYG{c+cp}{define PUPPET          0x00020000      }\PYG{c+cm}{/* Relays ALL messages to owner */}
\PYG{c+cp}{\PYGZsh{}}\PYG{c+cp}{define CHOWN\PYGZus{}OK        0x00040000      }\PYG{c+cm}{/* Object may be @chowned freely */}
\PYG{c+cp}{\PYGZsh{}}\PYG{c+cp}{define ENTER\PYGZus{}OK        0x00080000      }\PYG{c+cm}{/* Object may be ENTERed */}
\PYG{c+cp}{\PYGZsh{}}\PYG{c+cp}{define VISUAL          0x00100000      }\PYG{c+cm}{/* Everyone can see properties */}
\PYG{c+cp}{\PYGZsh{}}\PYG{c+cp}{define IMMORTAL        0x00200000      }\PYG{c+cm}{/* Object can\PYGZsq{}t be killed */}
\PYG{c+cp}{\PYGZsh{}}\PYG{c+cp}{define HAS\PYGZus{}STARTUP     0x00400000      }\PYG{c+cm}{/* Load some attrs at startup */}
\PYG{c+cp}{\PYGZsh{}}\PYG{c+cp}{define OPAQUE          0x00800000      }\PYG{c+cm}{/* Can\PYGZsq{}t see inside */}
\PYG{c+cp}{\PYGZsh{}}\PYG{c+cp}{define VERBOSE         0x01000000      }\PYG{c+cm}{/* Tells owner everything it does. */}
\PYG{c+cp}{\PYGZsh{}}\PYG{c+cp}{define INHERIT         0x02000000      }\PYG{c+cm}{/* Gets owner\PYGZsq{}s privs. (i.e. Wiz) */}
\PYG{c+cp}{\PYGZsh{}}\PYG{c+cp}{define NOSPOOF         0x04000000      }\PYG{c+cm}{/* Report originator of all actions. */}
\PYG{c+cp}{\PYGZsh{}}\PYG{c+cp}{define ROBOT           0x08000000      }\PYG{c+cm}{/* Player is a ROBOT */}
\PYG{c+cp}{\PYGZsh{}}\PYG{c+cp}{define SAFE            0x10000000      }\PYG{c+cm}{/* Need /override to @destroy */}
\PYG{c+cp}{\PYGZsh{}}\PYG{c+cp}{define CONTROL\PYGZus{}OK      0x20000000      }\PYG{c+cm}{/* ControlLk specifies who ctrls me */}
\PYG{c+cp}{\PYGZsh{}}\PYG{c+cp}{define HEARTHRU        0x40000000      }\PYG{c+cm}{/* Can hear out of this obj or exit */}
\PYG{c+cp}{\PYGZsh{}}\PYG{c+cp}{define TERSE           0x80000000      }\PYG{c+cm}{/* Only show room name on look */}

\PYG{c+cm}{/* Second word of flags */}
\PYG{c+cp}{\PYGZsh{}}\PYG{c+cp}{define KEY             0x00000001      }\PYG{c+cm}{/* No puppets */}
\PYG{c+cp}{\PYGZsh{}}\PYG{c+cp}{define ABODE           0x00000002      }\PYG{c+cm}{/* May @set home here */}
\PYG{c+cp}{\PYGZsh{}}\PYG{c+cp}{define FLOATING        0x00000004      }\PYG{c+cm}{/* Inhibit Floating room.. msgs */}
\PYG{c+cp}{\PYGZsh{}}\PYG{c+cp}{define UNFINDABLE      0x00000008      }\PYG{c+cm}{/* Cant loc() from afar */}
\PYG{c+cp}{\PYGZsh{}}\PYG{c+cp}{define PARENT\PYGZus{}OK       0x00000010      }\PYG{c+cm}{/* Others may @parent to me */}
\PYG{c+cp}{\PYGZsh{}}\PYG{c+cp}{define LIGHT           0x00000020      }\PYG{c+cm}{/* Visible in dark places */}
\PYG{c+cp}{\PYGZsh{}}\PYG{c+cp}{define HAS\PYGZus{}LISTEN      0x00000040      }\PYG{c+cm}{/* Internal: LISTEN attr set */}
\PYG{c+cp}{\PYGZsh{}}\PYG{c+cp}{define HAS\PYGZus{}FWDLIST     0x00000080      }\PYG{c+cm}{/* Internal: FORWARDLIST attr set */}
\PYG{c+cp}{\PYGZsh{}}\PYG{c+cp}{define ADMIN           0x00000100      }\PYG{c+cm}{/* Player has admin privs */}
\PYG{c+cp}{\PYGZsh{}}\PYG{c+cp}{define GUILDOBJ        0x00000200}
\PYG{c+cp}{\PYGZsh{}}\PYG{c+cp}{define GUILDMASTER     0x00000400      }\PYG{c+cm}{/* Player has gm privs */}
\PYG{c+cp}{\PYGZsh{}}\PYG{c+cp}{define NO\PYGZus{}WALLS        0x00000800      }\PYG{c+cm}{/* So to stop normal walls */}
\PYG{c+cp}{\PYGZsh{}}\PYG{c+cp}{define REQUIRE\PYGZus{}TREES   0x00001000      }\PYG{c+cm}{/* Trees are required on this target for attrib sets */}
\PYG{c+cm}{/* \PYGZhy{}\PYGZhy{}\PYGZhy{}\PYGZhy{}FREE\PYGZhy{}\PYGZhy{}\PYGZhy{}\PYGZhy{}         0x00002000 */}   \PYG{c+cm}{/* \PYGZsh{}define OLD\PYGZus{}NOROBOT  0x00002000 */}
\PYG{c+cp}{\PYGZsh{}}\PYG{c+cp}{define SCLOAK          0x00004000}
\PYG{c+cp}{\PYGZsh{}}\PYG{c+cp}{define CLOAK           0x00008000}
\PYG{c+cp}{\PYGZsh{}}\PYG{c+cp}{define FUBAR           0x00010000}
\PYG{c+cp}{\PYGZsh{}}\PYG{c+cp}{define INDESTRUCTABLE  0x00020000      }\PYG{c+cm}{/* object can\PYGZsq{}t be nuked */}
\PYG{c+cp}{\PYGZsh{}}\PYG{c+cp}{define NO\PYGZus{}YELL         0x00040000      }\PYG{c+cm}{/* player can\PYGZsq{}t @wall */}
\PYG{c+cp}{\PYGZsh{}}\PYG{c+cp}{define NO\PYGZus{}TEL          0x00080000      }\PYG{c+cm}{/* player can\PYGZsq{}t @tel or be @tel\PYGZsq{}d */}
\PYG{c+cp}{\PYGZsh{}}\PYG{c+cp}{define FREE            0x00100000      }\PYG{c+cm}{/* object/player has unlim money */}
\PYG{c+cp}{\PYGZsh{}}\PYG{c+cp}{define GUEST\PYGZus{}FLAG      0x00200000}
\PYG{c+cp}{\PYGZsh{}}\PYG{c+cp}{define RECOVER         0x00400000}
\PYG{c+cp}{\PYGZsh{}}\PYG{c+cp}{define BYEROOM         0x00800000}
\PYG{c+cp}{\PYGZsh{}}\PYG{c+cp}{define WANDERER        0x01000000}
\PYG{c+cp}{\PYGZsh{}}\PYG{c+cp}{define ANSI            0x02000000}
\PYG{c+cp}{\PYGZsh{}}\PYG{c+cp}{define ANSICOLOR       0x04000000}
\PYG{c+cp}{\PYGZsh{}}\PYG{c+cp}{define NOFLASH         0x08000000}
\PYG{c+cp}{\PYGZsh{}}\PYG{c+cp}{define SUSPECT         0x10000000      }\PYG{c+cm}{/* Report some activities to wizards */}
\PYG{c+cp}{\PYGZsh{}}\PYG{c+cp}{define BUILDER         0x20000000      }\PYG{c+cm}{/* Player has architect privs */}
\PYG{c+cp}{\PYGZsh{}}\PYG{c+cp}{define CONNECTED       0x40000000      }\PYG{c+cm}{/* Player is connected */}
\PYG{c+cp}{\PYGZsh{}}\PYG{c+cp}{define SLAVE           0x80000000      }\PYG{c+cm}{/* Disallow most commands */}

\PYG{c+cm}{/* Third word of flags \PYGZhy{} Thorin 3/95 */}
\PYG{c+cp}{\PYGZsh{}}\PYG{c+cp}{define NOCONNECT       0x00000001}
\PYG{c+cp}{\PYGZsh{}}\PYG{c+cp}{define DPSHIFT         0x00000002}
\PYG{c+cp}{\PYGZsh{}}\PYG{c+cp}{define NOPOSSESS       0x00000004}
\PYG{c+cp}{\PYGZsh{}}\PYG{c+cp}{define COMBAT          0x00000008}
\PYG{c+cp}{\PYGZsh{}}\PYG{c+cp}{define IC              0x00000010}
\PYG{c+cp}{\PYGZsh{}}\PYG{c+cp}{define ZONEMASTER      0x00000020}
\PYG{c+cp}{\PYGZsh{}}\PYG{c+cp}{define ALTQUOTA        0x00000040}
\PYG{c+cp}{\PYGZsh{}}\PYG{c+cp}{define NOEXAMINE       0x00000080}
\PYG{c+cp}{\PYGZsh{}}\PYG{c+cp}{define NOMODIFY        0x00000100}
\PYG{c+cp}{\PYGZsh{}}\PYG{c+cp}{define CMDCHECK        0x00000200}
\PYG{c+cp}{\PYGZsh{}}\PYG{c+cp}{define DOORRED         0x00000400}
\PYG{c+cp}{\PYGZsh{}}\PYG{c+cp}{define PRIVATE         0x00000800      }\PYG{c+cm}{/* For exits only */}
\PYG{c+cp}{\PYGZsh{}}\PYG{c+cp}{define NOMOVE          0x00001000}
\PYG{c+cp}{\PYGZsh{}}\PYG{c+cp}{define STOP            0x00002000}
\PYG{c+cp}{\PYGZsh{}}\PYG{c+cp}{define NOSTOP          0x00004000}
\PYG{c+cp}{\PYGZsh{}}\PYG{c+cp}{define NOCOMMAND       0x00008000}
\PYG{c+cp}{\PYGZsh{}}\PYG{c+cp}{define AUDIT           0x00010000}
\PYG{c+cp}{\PYGZsh{}}\PYG{c+cp}{define SEE\PYGZus{}OEMIT       0x00020000}
\PYG{c+cp}{\PYGZsh{}}\PYG{c+cp}{define NO\PYGZus{}GOBJ         0x00040000}
\PYG{c+cp}{\PYGZsh{}}\PYG{c+cp}{define NO\PYGZus{}PESTER       0x00080000}
\PYG{c+cp}{\PYGZsh{}}\PYG{c+cp}{define LRFLAG          0x00100000}
\PYG{c+cp}{\PYGZsh{}}\PYG{c+cp}{define TELOK           0x00200000}
\PYG{c+cp}{\PYGZsh{}}\PYG{c+cp}{define NO\PYGZus{}OVERRIDE     0x00400000}
\PYG{c+cp}{\PYGZsh{}}\PYG{c+cp}{define NO\PYGZus{}USELOCK      0x00800000}
\PYG{c+cp}{\PYGZsh{}}\PYG{c+cp}{define DR\PYGZus{}PURGE        0x01000000      }\PYG{c+cm}{/* For rooms only...internal */}
\PYG{c+cp}{\PYGZsh{}}\PYG{c+cp}{define NO\PYGZus{}ANSINAME     0x02000000      }\PYG{c+cm}{/* Remove the ability to set @ansiname */}
\PYG{c+cp}{\PYGZsh{}}\PYG{c+cp}{define SPOOF           0x04000000}
\PYG{c+cp}{\PYGZsh{}}\PYG{c+cp}{define SIDEFX          0x08000000      }\PYG{c+cm}{/* Allow enactor to use side\PYGZhy{}effects */}
\PYG{c+cp}{\PYGZsh{}}\PYG{c+cp}{define ZONECONTENTS    0x10000000      }\PYG{c+cm}{/* Search contents of zonemaster for \PYGZdl{}commands */}
\PYG{c+cp}{\PYGZsh{}}\PYG{c+cp}{define NOWHO           0x20000000      }\PYG{c+cm}{/* Player in WHO doesn\PYGZsq{}t show up \PYGZhy{} use with @hide */}
\PYG{c+cp}{\PYGZsh{}}\PYG{c+cp}{define ANONYMOUS       0x40000000      }\PYG{c+cm}{/* Player set shows up as \PYGZsq{}Someone\PYGZsq{} when talking */}
\PYG{c+cp}{\PYGZsh{}}\PYG{c+cp}{define BACKSTAGE       0x80000000      }\PYG{c+cm}{/* Immortal toggle for items on control */}

\PYG{c+cm}{/* Forth word of flags \PYGZhy{} Thorin 3/95 */}
\PYG{c+cp}{\PYGZsh{}}\PYG{c+cp}{define NOBACKSTAGE     0x00000001      }\PYG{c+cm}{/* Immortal toggle to control no\PYGZhy{}backstage */}
\PYG{c+cp}{\PYGZsh{}}\PYG{c+cp}{define LOGIN           0x00000002      }\PYG{c+cm}{/* Enable player to login past @disable logins */}
\PYG{c+cp}{\PYGZsh{}}\PYG{c+cp}{define INPROGRAM       0x00000004      }\PYG{c+cm}{/* Player is inside a program */}
\PYG{c+cp}{\PYGZsh{}}\PYG{c+cp}{define COMMANDS        0x00000008      }\PYG{c+cm}{/* Optional define for \PYGZdl{}commands */}
\PYG{c+cp}{\PYGZsh{}}\PYG{c+cp}{define MARKER0         0x00000010      }\PYG{c+cm}{/* TM 3.0 marker flags */}
\PYG{c+cp}{\PYGZsh{}}\PYG{c+cp}{define MARKER1         0x00000020}
\PYG{c+cp}{\PYGZsh{}}\PYG{c+cp}{define MARKER2         0x00000040}
\PYG{c+cp}{\PYGZsh{}}\PYG{c+cp}{define MARKER3         0x00000080}
\PYG{c+cp}{\PYGZsh{}}\PYG{c+cp}{define MARKER4         0x00000100}
\PYG{c+cp}{\PYGZsh{}}\PYG{c+cp}{define MARKER5         0x00000200}
\PYG{c+cp}{\PYGZsh{}}\PYG{c+cp}{define MARKER6         0x00000400}
\PYG{c+cp}{\PYGZsh{}}\PYG{c+cp}{define MARKER7         0x00000800}
\PYG{c+cp}{\PYGZsh{}}\PYG{c+cp}{define MARKER8         0x00001000}
\PYG{c+cp}{\PYGZsh{}}\PYG{c+cp}{define MARKER9         0x00002000}
\PYG{c+cp}{\PYGZsh{}}\PYG{c+cp}{define BOUNCE          0x00004000      }\PYG{c+cm}{/* That lovly TM 3.0 Bouncey thingy */}
\PYG{c+cp}{\PYGZsh{}}\PYG{c+cp}{define SHOWFAILCMD     0x00008000      }\PYG{c+cm}{/* Show failed \PYGZdl{}commands defauilt error */}
\PYG{c+cp}{\PYGZsh{}}\PYG{c+cp}{define NOUNDERLINE     0x00010000      }\PYG{c+cm}{/* Strip UNDERLINE character from ANSI */}
\PYG{c+cp}{\PYGZsh{}}\PYG{c+cp}{define NONAME          0x00020000      }\PYG{c+cm}{/* Target does not display name with look */}
\PYG{c+cp}{\PYGZsh{}}\PYG{c+cp}{define ZONEPARENT      0x00040000      }\PYG{c+cm}{/* Target zone allows inheritance of attribs */}
\PYG{c+cp}{\PYGZsh{}}\PYG{c+cp}{define SPAMMONITOR     0x00080000      }\PYG{c+cm}{/* Monitor the target for spam */}
\PYG{c+cp}{\PYGZsh{}}\PYG{c+cp}{define BLIND           0x00100000      }\PYG{c+cm}{/* Exits and locations snuff arrived/left */}
\PYG{c+cp}{\PYGZsh{}}\PYG{c+cp}{define NOCODE          0x00200000      }\PYG{c+cm}{/* Players may not code */}
\PYG{c+cp}{\PYGZsh{}}\PYG{c+cp}{define HAS\PYGZus{}PROTECT     0x00400000      }\PYG{c+cm}{/* Player target has protect name data */}
\PYG{c+cp}{\PYGZsh{}}\PYG{c+cp}{define XTERMCOLOR      0x00800000      }\PYG{c+cm}{/* Extended AnSI Xterm colors */}
\PYG{c+cp}{\PYGZsh{}}\PYG{c+cp}{define HAS\PYGZus{}ATTRPIPE    0x01000000      }\PYG{c+cm}{/* Attribute piping via @pipe */}
\PYG{c+cm}{/* 0x02000000 free */}
\PYG{c+cm}{/* 0x04000000 free */}
\PYG{c+cm}{/* 0x08000000 free */}
\PYG{c+cm}{/* 0x10000000 free */}
\PYG{c+cm}{/* 0x20000000 free */}
\PYG{c+cm}{/* 0x40000000 free */}
\PYG{c+cm}{/* 0x80000000 free */}
\end{sphinxVerbatim}

\sphinxAtStartPar
TOGGLES: Toggles are BITWISE masks taht need to be applied for each word like
the flags above.  They are added together for each word type

\begin{sphinxVerbatim}[commandchars=\\\{\}]
\PYG{c+cm}{/* First word of toggles \PYGZhy{} Thorin 3/95 */}
\PYG{c+cp}{\PYGZsh{}}\PYG{c+cp}{define TOG\PYGZus{}MONITOR             0x00000001      }\PYG{c+cm}{/* Active monitor on player */}
\PYG{c+cp}{\PYGZsh{}}\PYG{c+cp}{define TOG\PYGZus{}MONITOR\PYGZus{}USERID      0x00000002      }\PYG{c+cm}{/* show userid */}
\PYG{c+cp}{\PYGZsh{}}\PYG{c+cp}{define TOG\PYGZus{}MONITOR\PYGZus{}SITE        0x00000004      }\PYG{c+cm}{/* show site */}
\PYG{c+cp}{\PYGZsh{}}\PYG{c+cp}{define TOG\PYGZus{}MONITOR\PYGZus{}STATS       0x00000008      }\PYG{c+cm}{/* show stats */}
\PYG{c+cp}{\PYGZsh{}}\PYG{c+cp}{define TOG\PYGZus{}MONITOR\PYGZus{}FAIL        0x00000010      }\PYG{c+cm}{/* show fails */}
\PYG{c+cp}{\PYGZsh{}}\PYG{c+cp}{define TOG\PYGZus{}MONITOR\PYGZus{}CONN        0x00000020}
\PYG{c+cp}{\PYGZsh{}}\PYG{c+cp}{define TOG\PYGZus{}VANILLA\PYGZus{}ERRORS      0x00000040      }\PYG{c+cm}{/* show normal error msg */}
\PYG{c+cp}{\PYGZsh{}}\PYG{c+cp}{define TOG\PYGZus{}NO\PYGZus{}ANSI\PYGZus{}EX          0x00000080      }\PYG{c+cm}{/* supress ansi stuff in ex */}
\PYG{c+cp}{\PYGZsh{}}\PYG{c+cp}{define TOG\PYGZus{}CPUTIME             0x00000100      }\PYG{c+cm}{/* show cpu time for cmds */}
\PYG{c+cp}{\PYGZsh{}}\PYG{c+cp}{define TOG\PYGZus{}MONITOR\PYGZus{}DISREASON   0x00000200}
\PYG{c+cp}{\PYGZsh{}}\PYG{c+cp}{define TOG\PYGZus{}MONITOR\PYGZus{}VLIMIT      0x00000400}
\PYG{c+cp}{\PYGZsh{}}\PYG{c+cp}{define TOG\PYGZus{}NOTIFY\PYGZus{}LINK         0x00000800}
\PYG{c+cp}{\PYGZsh{}}\PYG{c+cp}{define TOG\PYGZus{}MONITOR\PYGZus{}AREG        0x00001000}
\PYG{c+cp}{\PYGZsh{}}\PYG{c+cp}{define TOG\PYGZus{}MONITOR\PYGZus{}TIME        0x00002000}
\PYG{c+cp}{\PYGZsh{}}\PYG{c+cp}{define TOG\PYGZus{}CLUSTER             0x00004000      }\PYG{c+cm}{/* Object is part of a cluster */}
\PYG{c+cp}{\PYGZsh{}}\PYG{c+cp}{define TOG\PYGZus{}SNUFFDARK           0x00008000      }\PYG{c+cm}{/* Snuff Dark Exit Viewing */}
\PYG{c+cp}{\PYGZsh{}}\PYG{c+cp}{define TOG\PYGZus{}NOANSI\PYGZus{}PLAYER       0x00010000      }\PYG{c+cm}{/* Do not show ansi player names */}
\PYG{c+cp}{\PYGZsh{}}\PYG{c+cp}{define TOG\PYGZus{}NOANSI\PYGZus{}THING        0x00020000      }\PYG{c+cm}{/* ... things */}
\PYG{c+cp}{\PYGZsh{}}\PYG{c+cp}{define TOG\PYGZus{}NOANSI\PYGZus{}ROOM         0x00040000      }\PYG{c+cm}{/* ... rooms */}
\PYG{c+cp}{\PYGZsh{}}\PYG{c+cp}{define TOG\PYGZus{}NOANSI\PYGZus{}EXIT         0x00080000      }\PYG{c+cm}{/* ... exits */}
\PYG{c+cp}{\PYGZsh{}}\PYG{c+cp}{define TOG\PYGZus{}NO\PYGZus{}TIMESTAMP        0x00100000      }\PYG{c+cm}{/* Do not modify timestamps on target */}
\PYG{c+cp}{\PYGZsh{}}\PYG{c+cp}{define TOG\PYGZus{}NO\PYGZus{}FORMAT           0x00200000      }\PYG{c+cm}{/* Override @conformat/@exitformat */}
\PYG{c+cp}{\PYGZsh{}}\PYG{c+cp}{define TOG\PYGZus{}ZONE\PYGZus{}AUTOADD        0x00400000      }\PYG{c+cm}{/* Automatically add FIRST zone in list */}
\PYG{c+cp}{\PYGZsh{}}\PYG{c+cp}{define TOG\PYGZus{}ZONE\PYGZus{}AUTOADDALL     0x00800000      }\PYG{c+cm}{/* Automatically add ALL zones in list */}
\PYG{c+cp}{\PYGZsh{}}\PYG{c+cp}{define TOG\PYGZus{}WIELDABLE           0x01000000      }\PYG{c+cm}{/* Marker to specify if object is wieldable */}
\PYG{c+cp}{\PYGZsh{}}\PYG{c+cp}{define TOG\PYGZus{}WEARABLE            0x02000000      }\PYG{c+cm}{/* Marker to specify if object is wearable */}
\PYG{c+cp}{\PYGZsh{}}\PYG{c+cp}{define TOG\PYGZus{}SEE\PYGZus{}SUSPECT         0x04000000      }\PYG{c+cm}{/* Specify who sees suspect in WHO/MONITOR */}
\PYG{c+cp}{\PYGZsh{}}\PYG{c+cp}{define TOG\PYGZus{}MONITOR\PYGZus{}CPU         0x08000000      }\PYG{c+cm}{/* Specify who sees CPU overflow allerts */}
\PYG{c+cp}{\PYGZsh{}}\PYG{c+cp}{define TOG\PYGZus{}BRANDY\PYGZus{}MAIL         0x10000000      }\PYG{c+cm}{/* Define brandy like mail interface */}
\PYG{c+cp}{\PYGZsh{}}\PYG{c+cp}{define TOG\PYGZus{}FORCEHALTED         0x20000000      }\PYG{c+cm}{/* The item toggled can @force halted things */}
\PYG{c+cp}{\PYGZsh{}}\PYG{c+cp}{define TOG\PYGZus{}PROG                0x40000000      }\PYG{c+cm}{/* Can use @program on other people/things */}
\PYG{c+cp}{\PYGZsh{}}\PYG{c+cp}{define TOG\PYGZus{}NOSHELLPROG         0x80000000      }\PYG{c+cm}{/* Target can not issue commands inside a prog */}

\PYG{c+cm}{/* Second word of toggles \PYGZhy{}\PYGZhy{} Ash */}
\PYG{c+cp}{\PYGZsh{}}\PYG{c+cp}{define TOG\PYGZus{}EXTANSI             0x00000001      }\PYG{c+cm}{/* Specify if target can used extended ansi naming */}
\PYG{c+cp}{\PYGZsh{}}\PYG{c+cp}{define TOG\PYGZus{}IMMPROG             0x00000002      }\PYG{c+cm}{/* Only an immortal can @quitprogram them */}
\PYG{c+cp}{\PYGZsh{}}\PYG{c+cp}{define TOG\PYGZus{}MONITOR\PYGZus{}BFAIL       0x00000004      }\PYG{c+cm}{/* Monitor if a failed connect on bad character */}
\PYG{c+cp}{\PYGZsh{}}\PYG{c+cp}{define TOG\PYGZus{}PROG\PYGZus{}ON\PYGZus{}CONNECT     0x00000008      }\PYG{c+cm}{/* Reverse logic of program on connect */}
\PYG{c+cp}{\PYGZsh{}}\PYG{c+cp}{define TOG\PYGZus{}MAIL\PYGZus{}STRIPRETURN    0x00000010      }\PYG{c+cm}{/* Strip carrage return in mail combining */}
\PYG{c+cp}{\PYGZsh{}}\PYG{c+cp}{define TOG\PYGZus{}PENN\PYGZus{}MAIL           0x00000020      }\PYG{c+cm}{/* Use PENN style syntax */}
\PYG{c+cp}{\PYGZsh{}}\PYG{c+cp}{define TOG\PYGZus{}SILENTEFFECTS       0x00000040      }\PYG{c+cm}{/* Silents did\PYGZus{}it() functionality on target */}
\PYG{c+cp}{\PYGZsh{}}\PYG{c+cp}{define TOG\PYGZus{}IGNOREZONE          0x00000080      }\PYG{c+cm}{/* Target is set to @icmd zones */}
\PYG{c+cp}{\PYGZsh{}}\PYG{c+cp}{define TOG\PYGZus{}VPAGE               0x00000100      }\PYG{c+cm}{/* Target sees alias in pages */}
\PYG{c+cp}{\PYGZsh{}}\PYG{c+cp}{define TOG\PYGZus{}PAGELOCK            0x00000200      }\PYG{c+cm}{/* Target issues pagelocks as normal */}
\PYG{c+cp}{\PYGZsh{}}\PYG{c+cp}{define TOG\PYGZus{}MAIL\PYGZus{}NOPARSE        0x00000400      }\PYG{c+cm}{/* Don\PYGZsq{}t parse \PYGZpc{}t/\PYGZpc{}b/\PYGZpc{}r in mail */}
\PYG{c+cp}{\PYGZsh{}}\PYG{c+cp}{define TOG\PYGZus{}MAIL\PYGZus{}LOCKDOWN       0x00000800      }\PYG{c+cm}{/* Mortal\PYGZhy{}accessed mail/number and mail/check */}
\PYG{c+cp}{\PYGZsh{}}\PYG{c+cp}{define TOG\PYGZus{}MUXPAGE             0x00001000      }\PYG{c+cm}{/* Have \PYGZsq{}page\PYGZsq{} work like MUX */}
\PYG{c+cp}{\PYGZsh{}}\PYG{c+cp}{define TOG\PYGZus{}NOZONEPARENT        0x00002000      }\PYG{c+cm}{/* Zone Child does NOT inherit parent attribs */}
\PYG{c+cp}{\PYGZsh{}}\PYG{c+cp}{define TOG\PYGZus{}ATRUSE              0x00004000      }\PYG{c+cm}{/* Enactor can use Attribute based USELOCKS */}
\PYG{c+cp}{\PYGZsh{}}\PYG{c+cp}{define TOG\PYGZus{}VARIABLE            0x00008000      }\PYG{c+cm}{/* Set exit to be variable */}
\PYG{c+cp}{\PYGZsh{}}\PYG{c+cp}{define TOG\PYGZus{}KEEPALIVE           0x00010000      }\PYG{c+cm}{/* Send \PYGZsq{}keepalives\PYGZsq{} to the target player */}
\PYG{c+cp}{\PYGZsh{}}\PYG{c+cp}{define TOG\PYGZus{}CHKREALITY          0x00020000      }\PYG{c+cm}{/* Target checks @lock/user for Reality passes */}
\PYG{c+cp}{\PYGZsh{}}\PYG{c+cp}{define TOG\PYGZus{}NOISY               0x00040000      }\PYG{c+cm}{/* Always do noisy sets */}
\PYG{c+cp}{\PYGZsh{}}\PYG{c+cp}{define TOG\PYGZus{}ZONECMDCHK          0x00080000      }\PYG{c+cm}{/* Zone commands checked on target like @parent */}
\PYG{c+cp}{\PYGZsh{}}\PYG{c+cp}{define TOG\PYGZus{}HIDEIDLE            0x00100000      }\PYG{c+cm}{/* Allow wizards/immortals to hide their idle time */}
\PYG{c+cp}{\PYGZsh{}}\PYG{c+cp}{define TOG\PYGZus{}MORTALREALITY       0x00200000      }\PYG{c+cm}{/* Override the wiz\PYGZus{}always\PYGZus{}real setting */}
\PYG{c+cp}{\PYGZsh{}}\PYG{c+cp}{define TOG\PYGZus{}ACCENTS             0x00400000      }\PYG{c+cm}{/* Accents being displayed */}
\PYG{c+cp}{\PYGZsh{}}\PYG{c+cp}{define TOG\PYGZus{}PREMAILVALIDATE     0x00800000      }\PYG{c+cm}{/* Pre\PYGZhy{}Validate the mail send list before sending mail */}
\PYG{c+cp}{\PYGZsh{}}\PYG{c+cp}{define TOG\PYGZus{}SAFELOG             0x01000000      }\PYG{c+cm}{/* Allow \PYGZsq{}clean logging\PYGZsq{} by the player */}
\PYG{c+cp}{\PYGZsh{}}\PYG{c+cp}{define TOG\PYGZus{}UTF8                0x02000000      }\PYG{c+cm}{/* UTF8 being displayed */}
\PYG{c+cm}{/* 0x04000000 free */}
\PYG{c+cp}{\PYGZsh{}}\PYG{c+cp}{define TOG\PYGZus{}NODEFAULT           0x08000000      }\PYG{c+cm}{/* Allow target to inherit default attribs */}
\PYG{c+cp}{\PYGZsh{}}\PYG{c+cp}{define TOG\PYGZus{}EXFULLWIZATTR       0x10000000      }\PYG{c+cm}{/* Examine Wiz attribs */}
\PYG{c+cp}{\PYGZsh{}}\PYG{c+cp}{ifdef ENH\PYGZus{}LOGROOM}
\PYG{c+cp}{\PYGZsh{}}\PYG{c+cp}{define TOG\PYGZus{}LOGROOMENH          0x20000000      }\PYG{c+cm}{/* Enhanced Room Logging */}
\PYG{c+cp}{\PYGZsh{}}\PYG{c+cp}{endif}
\PYG{c+cp}{\PYGZsh{}}\PYG{c+cp}{define TOG\PYGZus{}LOGROOM             0x40000000      }\PYG{c+cm}{/* Log Room\PYGZsq{}s location/contents */}
\PYG{c+cp}{\PYGZsh{}}\PYG{c+cp}{define TOG\PYGZus{}NOGLOBPARENT        0x80000000      }\PYG{c+cm}{/* Target does not inherit global attributes */}
\end{sphinxVerbatim}

\sphinxAtStartPar
POWERS:  Powers are handled a bit differently.  They\textquotesingle{}re used as BITWISE shift
markers that you would have to compute the shift then add it after the fact.

\begin{sphinxVerbatim}[commandchars=\\\{\}]
\PYG{c+cm}{/* First word of power positions.  Each position is 2 bits so the}
\PYG{c+cm}{   number here is how far over to shift the 2 bit pattern         */}
\PYG{c+cp}{\PYGZsh{}}\PYG{c+cp}{define POWER\PYGZus{}CHANGE\PYGZus{}QUOTAS     0}
\PYG{c+cp}{\PYGZsh{}}\PYG{c+cp}{define POWER\PYGZus{}CHOWN\PYGZus{}ME          2}
\PYG{c+cp}{\PYGZsh{}}\PYG{c+cp}{define POWER\PYGZus{}CHOWN\PYGZus{}ANYWHERE    4}
\PYG{c+cp}{\PYGZsh{}}\PYG{c+cp}{define POWER\PYGZus{}CHOWN\PYGZus{}OTHER       6}
\PYG{c+cp}{\PYGZsh{}}\PYG{c+cp}{define POWER\PYGZus{}WIZ\PYGZus{}WHO           8}
\PYG{c+cp}{\PYGZsh{}}\PYG{c+cp}{define POWER\PYGZus{}EX\PYGZus{}ALL            10}
\PYG{c+cp}{\PYGZsh{}}\PYG{c+cp}{define POWER\PYGZus{}NOFORCE           12}
\PYG{c+cp}{\PYGZsh{}}\PYG{c+cp}{define POWER\PYGZus{}SEE\PYGZus{}QUEUE\PYGZus{}ALL     14}
\PYG{c+cp}{\PYGZsh{}}\PYG{c+cp}{define POWER\PYGZus{}FREE\PYGZus{}QUOTA        16}
\PYG{c+cp}{\PYGZsh{}}\PYG{c+cp}{define POWER\PYGZus{}GRAB\PYGZus{}PLAYER       18}
\PYG{c+cp}{\PYGZsh{}}\PYG{c+cp}{define POWER\PYGZus{}JOIN\PYGZus{}PLAYER       20}
\PYG{c+cp}{\PYGZsh{}}\PYG{c+cp}{define POWER\PYGZus{}LONG\PYGZus{}FINGERS      22}
\PYG{c+cp}{\PYGZsh{}}\PYG{c+cp}{define POWER\PYGZus{}NO\PYGZus{}BOOT           24}
\PYG{c+cp}{\PYGZsh{}}\PYG{c+cp}{define POWER\PYGZus{}BOOT              26}
\PYG{c+cp}{\PYGZsh{}}\PYG{c+cp}{define POWER\PYGZus{}STEAL             28}
\PYG{c+cp}{\PYGZsh{}}\PYG{c+cp}{define POWER\PYGZus{}SEE\PYGZus{}QUEUE         30}

\PYG{c+cm}{/* Second word of power positions. */}
\PYG{c+cp}{\PYGZsh{}}\PYG{c+cp}{define POWER\PYGZus{}SHUTDOWN          0}
\PYG{c+cp}{\PYGZsh{}}\PYG{c+cp}{define POWER\PYGZus{}TEL\PYGZus{}ANYWHERE      2}
\PYG{c+cp}{\PYGZsh{}}\PYG{c+cp}{define POWER\PYGZus{}TEL\PYGZus{}ANYTHING      4}
\PYG{c+cp}{\PYGZsh{}}\PYG{c+cp}{define POWER\PYGZus{}PCREATE           6}
\PYG{c+cp}{\PYGZsh{}}\PYG{c+cp}{define POWER\PYGZus{}STAT\PYGZus{}ANY          8}
\PYG{c+cp}{\PYGZsh{}}\PYG{c+cp}{define POWER\PYGZus{}FREE\PYGZus{}WALL         10}
\PYG{c+cp}{\PYGZsh{}}\PYG{c+cp}{define POWER\PYGZus{}EXECSCRIPT        12}
\PYG{c+cp}{\PYGZsh{}}\PYG{c+cp}{define POWER\PYGZus{}FREE\PYGZus{}PAGE         14}
\PYG{c+cp}{\PYGZsh{}}\PYG{c+cp}{define POWER\PYGZus{}HALT\PYGZus{}QUEUE        16}
\PYG{c+cp}{\PYGZsh{}}\PYG{c+cp}{define POWER\PYGZus{}HALT\PYGZus{}QUEUE\PYGZus{}ALL    18}
\PYG{c+cp}{\PYGZsh{}}\PYG{c+cp}{define POWER\PYGZus{}FORMATTING        20}
\PYG{c+cp}{\PYGZsh{}}\PYG{c+cp}{define POWER\PYGZus{}NOKILL            22}
\PYG{c+cp}{\PYGZsh{}}\PYG{c+cp}{define POWER\PYGZus{}SEARCH\PYGZus{}ANY        24}
\PYG{c+cp}{\PYGZsh{}}\PYG{c+cp}{define POWER\PYGZus{}SECURITY          26}
\PYG{c+cp}{\PYGZsh{}}\PYG{c+cp}{define POWER\PYGZus{}WHO\PYGZus{}UNFIND        28}

\PYG{c+cm}{/* Third word of power positions. */}
\PYG{c+cp}{\PYGZsh{}}\PYG{c+cp}{define POWER\PYGZus{}OPURGE            0}
\PYG{c+cp}{\PYGZsh{}}\PYG{c+cp}{define POWER\PYGZus{}HIDEBIT           2}
\PYG{c+cp}{\PYGZsh{}}\PYG{c+cp}{define POWER\PYGZus{}NOWHO             4}
\PYG{c+cp}{\PYGZsh{}}\PYG{c+cp}{define POWER\PYGZus{}FULLTEL\PYGZus{}ANYWHERE  6}
\PYG{c+cp}{\PYGZsh{}}\PYG{c+cp}{define POWER\PYGZus{}EX\PYGZus{}FULL           8}
\PYG{c+cp}{\PYGZsh{}}\PYG{c+cp}{define POWER\PYGZus{}API               10}
\PYG{c+cp}{\PYGZsh{}}\PYG{c+cp}{define POWER\PYGZus{}MONITORAPI        12}
\PYG{c+cp}{\PYGZsh{}}\PYG{c+cp}{define POWER\PYGZus{}WIZ\PYGZus{}IDLE          14}
\PYG{c+cp}{\PYGZsh{}}\PYG{c+cp}{define POWER\PYGZus{}WIZ\PYGZus{}SPOOF         16}
\PYG{c+cm}{/* 18 free */}
\PYG{c+cm}{/* 20 free */}
\PYG{c+cm}{/* 22 free */}
\PYG{c+cm}{/* 24 free */}
\PYG{c+cm}{/* 26 free */}
\PYG{c+cm}{/* 28 free */}
\PYG{c+cm}{/* 30 free */}
\end{sphinxVerbatim}

\sphinxAtStartPar
DEPOWERS:  like @powers they are handled with a BITWISE offshift that you
will have to calculate then add

\begin{sphinxVerbatim}[commandchars=\\\{\}]
\PYG{c+cm}{/* First word */}
\PYG{c+cp}{\PYGZsh{}}\PYG{c+cp}{define DP\PYGZus{}WALL                 0}
\PYG{c+cp}{\PYGZsh{}}\PYG{c+cp}{define DP\PYGZus{}LONG\PYGZus{}FINGERS         2}
\PYG{c+cp}{\PYGZsh{}}\PYG{c+cp}{define DP\PYGZus{}STEAL                4}
\PYG{c+cp}{\PYGZsh{}}\PYG{c+cp}{define DP\PYGZus{}CREATE               6}
\PYG{c+cp}{\PYGZsh{}}\PYG{c+cp}{define DP\PYGZus{}WIZ\PYGZus{}WHO              8}
\PYG{c+cp}{\PYGZsh{}}\PYG{c+cp}{define DP\PYGZus{}CLOAK                10}
\PYG{c+cp}{\PYGZsh{}}\PYG{c+cp}{define DP\PYGZus{}BOOT                 12}
\PYG{c+cp}{\PYGZsh{}}\PYG{c+cp}{define DP\PYGZus{}PAGE                 14}
\PYG{c+cp}{\PYGZsh{}}\PYG{c+cp}{define DP\PYGZus{}FORCE                16}
\PYG{c+cp}{\PYGZsh{}}\PYG{c+cp}{define DP\PYGZus{}LOCKS                18}
\PYG{c+cp}{\PYGZsh{}}\PYG{c+cp}{define DP\PYGZus{}COM                  20}
\PYG{c+cp}{\PYGZsh{}}\PYG{c+cp}{define DP\PYGZus{}COMMAND              22}
\PYG{c+cp}{\PYGZsh{}}\PYG{c+cp}{define DP\PYGZus{}MASTER               24}
\PYG{c+cp}{\PYGZsh{}}\PYG{c+cp}{define DP\PYGZus{}EXAMINE              26}
\PYG{c+cp}{\PYGZsh{}}\PYG{c+cp}{define DP\PYGZus{}NUKE                 28}
\PYG{c+cp}{\PYGZsh{}}\PYG{c+cp}{define DP\PYGZus{}FREE                 30}

\PYG{c+cm}{/* Second word */}
\PYG{c+cp}{\PYGZsh{}}\PYG{c+cp}{define DP\PYGZus{}OVERRIDE             0}
\PYG{c+cp}{\PYGZsh{}}\PYG{c+cp}{define DP\PYGZus{}TEL\PYGZus{}ANYWHERE         2}
\PYG{c+cp}{\PYGZsh{}}\PYG{c+cp}{define DP\PYGZus{}TEL\PYGZus{}ANYTHING         4}
\PYG{c+cp}{\PYGZsh{}}\PYG{c+cp}{define DP\PYGZus{}PCREATE              6}
\PYG{c+cp}{\PYGZsh{}}\PYG{c+cp}{define DP\PYGZus{}POWER                8}
\PYG{c+cp}{\PYGZsh{}}\PYG{c+cp}{define DP\PYGZus{}QUOTA                10}
\PYG{c+cp}{\PYGZsh{}}\PYG{c+cp}{define DP\PYGZus{}MODIFY               12}
\PYG{c+cp}{\PYGZsh{}}\PYG{c+cp}{define DP\PYGZus{}CHOWN\PYGZus{}ME             14}
\PYG{c+cp}{\PYGZsh{}}\PYG{c+cp}{define DP\PYGZus{}CHOWN\PYGZus{}OTHER          16}
\PYG{c+cp}{\PYGZsh{}}\PYG{c+cp}{define DP\PYGZus{}ABUSE                18}
\PYG{c+cp}{\PYGZsh{}}\PYG{c+cp}{define DP\PYGZus{}UNL\PYGZus{}QUOTA            20}
\PYG{c+cp}{\PYGZsh{}}\PYG{c+cp}{define DP\PYGZus{}SEARCH\PYGZus{}ANY           22}
\PYG{c+cp}{\PYGZsh{}}\PYG{c+cp}{define DP\PYGZus{}GIVE                 24}
\PYG{c+cp}{\PYGZsh{}}\PYG{c+cp}{define DP\PYGZus{}RECEIVE              26}
\PYG{c+cp}{\PYGZsh{}}\PYG{c+cp}{define DP\PYGZus{}NOGOLD               28}
\PYG{c+cp}{\PYGZsh{}}\PYG{c+cp}{define DP\PYGZus{}NOSTEAL              30}
\PYG{c+cm}{/* Third word...and there was much rejoicing */}
\PYG{c+cp}{\PYGZsh{}}\PYG{c+cp}{define DP\PYGZus{}PASSWORD             0}
\PYG{c+cp}{\PYGZsh{}}\PYG{c+cp}{define DP\PYGZus{}MORTAL\PYGZus{}EXAMINE       2}
\PYG{c+cp}{\PYGZsh{}}\PYG{c+cp}{define DP\PYGZus{}PERSONAL\PYGZus{}COMMANDS    4}
\PYG{c+cm}{/* 6  free */}
\PYG{c+cp}{\PYGZsh{}}\PYG{c+cp}{define DP\PYGZus{}DARK                 8}
\PYG{c+cm}{/* 10 free */}
\PYG{c+cm}{/* 12 free */}
\PYG{c+cm}{/* 14 free */}
\PYG{c+cm}{/* 16 free */}
\PYG{c+cm}{/* 18 free */}
\PYG{c+cm}{/* 20 free */}
\PYG{c+cm}{/* 22 free */}
\PYG{c+cm}{/* 24 free */}
\PYG{c+cm}{/* 26 free */}
\PYG{c+cm}{/* 28 free */}
\PYG{c+cm}{/* 30 free */}
\end{sphinxVerbatim}

\sphinxAtStartPar
ZONES:  Zones are special.  If there are no zones, the value will be \textquotesingle{}\sphinxhyphen{}1\textquotesingle{}.

\sphinxAtStartPar
So entering zones if there are no zones:

\begin{sphinxVerbatim}[commandchars=\\\{\}]
\PYG{o}{\PYGZhy{}}\PYG{l+m+mi}{1}
\end{sphinxVerbatim}

\sphinxAtStartPar
Entering zones if it has three zones (\#123, \#456, and \#789):

\begin{sphinxVerbatim}[commandchars=\\\{\}]
\PYG{l+m+mi}{123}
\PYG{l+m+mi}{456}
\PYG{l+m+mi}{789}
\PYG{o}{\PYGZhy{}}\PYG{l+m+mi}{1}
\end{sphinxVerbatim}

\sphinxAtStartPar
As you see, the last value of the zone \sphinxstyleemphasis{MUST} be \sphinxhyphen{}1.  This tells it
that there are no more zones to add.
\phantomsection\label{\detokenize{install/windows:install-windows}}
\index{Windows@\spxentry{Windows}}\index{Microsoft Windows@\spxentry{Microsoft Windows}}\index{Install \sphinxhyphen{} Windows@\spxentry{Install \sphinxhyphen{} Windows}}\ignorespaces 

\chapter{Requirements if Using Windows}
\label{\detokenize{install/windows:requirements-if-using-windows}}\label{\detokenize{install/windows:index-0}}\label{\detokenize{install/windows::doc}}\begin{itemize}
\item {} 
\sphinxAtStartPar
(BETA ONLY) cygwin under Windows.  It requires the entire base development set and Requirements below.

\end{itemize}


\section{Installing on Windows 10 with BASH}
\label{\detokenize{install/windows:installing-on-windows-10-with-bash}}
\sphinxAtStartPar
Rhost can be compiled and run under the new Bash on Ubuntu on Windows.
This has been tested with the Preview build 14342.
\begin{itemize}
\item {} 
\sphinxAtStartPar
After installing Bash you will need to install the following packages:
\begin{itemize}
\item {} 
\sphinxAtStartPar
gcc

\item {} 
\sphinxAtStartPar
git

\item {} 
\sphinxAtStartPar
make

\end{itemize}

\item {} 
\sphinxAtStartPar
Optional packages:
\begin{itemize}
\item {} 
\sphinxAtStartPar
libpcre3

\item {} 
\sphinxAtStartPar
libpcre3\sphinxhyphen{}dev

\item {} 
\sphinxAtStartPar
openssl

\end{itemize}

\item {} 
\sphinxAtStartPar
When configuring rhost (using confsource) select the Disable Debugmon

\end{itemize}

\sphinxAtStartPar
option.


\section{Installing on Windows with Cygwin}
\label{\detokenize{install/windows:installing-on-windows-with-cygwin}}
\sphinxAtStartPar
Rhost does work under windows using the cygwin package.
\begin{itemize}
\item {} 
\sphinxAtStartPar
When you do install cygwin, the following packages must be added:
\begin{itemize}
\item {} 
\sphinxAtStartPar
bash

\item {} 
\sphinxAtStartPar
crypt

\item {} 
\sphinxAtStartPar
gcc

\item {} 
\sphinxAtStartPar
gdbm

\item {} 
\sphinxAtStartPar
git

\item {} 
\sphinxAtStartPar
make

\end{itemize}

\item {} 
\sphinxAtStartPar
Optional packages:
\begin{itemize}
\item {} 
\sphinxAtStartPar
openssl

\end{itemize}

\item {} 
\sphinxAtStartPar
The src/Makefile has to manually have the CYGWIN line uncommented.

\end{itemize}


\section{Startig RhostMUSH on Windows}
\label{\detokenize{install/windows:startig-rhostmush-on-windows}}
\sphinxAtStartPar
When you issue Startmush, you must pass it the \sphinxhyphen{}cyg option.


\chapter{Upgrading a Legacy RhostMUSH Installation}
\label{\detokenize{legacy:upgrading-a-legacy-rhostmush-installation}}\label{\detokenize{legacy::doc}}

\section{Converting database betwen GDBM and QDBM}
\label{\detokenize{legacy:converting-database-betwen-gdbm-and-qdbm}}
\sphinxAtStartPar
Ok, if you plan to recompile your game that is using GDBM to QDBM, or visa versa
some bad news.

\sphinxAtStartPar
The databases are NOT COMPATIBLE to each other, at least in the binary form.


\subsection{Downgrading QDBM to GDBM}
\label{\detokenize{legacy:downgrading-qdbm-to-gdbm}}
\begin{sphinxadmonition}{warning}{Warning:}
\sphinxAtStartPar
I would NEVER change from QDBM back to GDBM, but if you\textquotesingle{}re set on it these steps:

\sphinxAtStartPar
You would use the same steps if you plan to move QDBM to GDBM.  I however would
not do this.  Moving from QDBM to GDBM is a huge step backwards.  Seriously,
don\textquotesingle{}t do it unless you have absolutely no other recourse.

\sphinxAtStartPar
IF you plan (for whatever reason) to move from QDBM to GDBM, you should verify
the following
\end{sphinxadmonition}
\begin{enumerate}
\sphinxsetlistlabels{\arabic}{enumi}{enumii}{}{.}%
\item {} 
\sphinxAtStartPar
You have on a 64 bit system, no object that has more than 400 attributes on it.

\item {} 
\sphinxAtStartPar
You have on a 32 bit system, no object that has more than 750 attributes on it.

\item {} 
\sphinxAtStartPar
Any CONTENT of any attribute must be below 4000 characters in length.

\item {} 
\sphinxAtStartPar
Once you have that done, you may follow the procedures below on converting (upgrade) from GDBM to QDBM.  This works the same as converting (downgrading) QDBM back down to GDBM

\end{enumerate}


\subsection{Upgradging GDBM to QDBM}
\label{\detokenize{legacy:upgradging-gdbm-to-qdbm}}
\sphinxAtStartPar
Now, if you\textquotesingle{}ve kept reading and plan to convert your GDBM database to QDBM great news!
It\textquotesingle{}s more stable, it\textquotesingle{}s faster, and lets you have far more flexibility.

\sphinxAtStartPar
So, BEFORE YOU RECOMPILE YOUR CODE.  This is what you have to do.


\subsubsection{While logged in to your mush, issue the following commands}
\label{\detokenize{legacy:while-logged-in-to-your-mush-issue-the-following-commands}}\begin{enumerate}
\sphinxsetlistlabels{\arabic}{enumi}{enumii}{}{.}%
\item {} 
\sphinxAtStartPar
@dump/flat    \sphinxhyphen{}\sphinxhyphen{} This will make a flatfile dump of your MUSH database

\item {} 
\sphinxAtStartPar
wmail/unload  \sphinxhyphen{}\sphinxhyphen{} This will make a flatfile dump of your MAIL database

\item {} 
\sphinxAtStartPar
@areg/unload  \sphinxhyphen{}\sphinxhyphen{} If you use the AutoRegistration engine, this dumps it

\item {} 
\sphinxAtStartPar
newsdb/unload \sphinxhyphen{}\sphinxhyphen{} If you use the hardcoded news/bbs engine.  This dumps it

\end{enumerate}


\subsubsection{Verify the files exist}
\label{\detokenize{legacy:verify-the-files-exist}}\begin{enumerate}
\sphinxsetlistlabels{\arabic}{enumi}{enumii}{}{.}%
\item {} 
\sphinxAtStartPar
Server/game/data/netrhost.db.flat

\item {} 
\sphinxAtStartPar
Server/game/data/RhostMUSH.dump.folder

\item {} 
\sphinxAtStartPar
Server/game/data/RhostMUSH.dump.mail

\item {} 
\sphinxAtStartPar
(Optional) Server/game/data/RhostMUSH.areg.dump

\item {} 
\sphinxAtStartPar
(Optional) Server/game/data/RhostMUSH.news.flat

\end{enumerate}


\subsubsection{Shutdown the MUSH}
\label{\detokenize{legacy:shutdown-the-mush}}
\sphinxAtStartPar
@shutdown your mush


\subsubsection{From the Server directory}
\label{\detokenize{legacy:from-the-server-directory}}\begin{enumerate}
\sphinxsetlistlabels{\arabic}{enumi}{enumii}{}{.}%
\item {} 
\sphinxAtStartPar
make clean

\item {} 
\sphinxAtStartPar
make confsource

\end{enumerate}
\begin{enumerate}
\sphinxsetlistlabels{\arabic}{enumi}{enumii}{}{.}%
\item {} 
\sphinxAtStartPar
Select QDBM and if you wish at this time increase your LBUF size

\item {} 
\sphinxAtStartPar
Select any other options you may want

\end{enumerate}
\begin{enumerate}
\sphinxsetlistlabels{\arabic}{enumi}{enumii}{}{.}%
\item {} 
\sphinxAtStartPar
(r)un and let it compile.

\end{enumerate}

\begin{sphinxadmonition}{note}{\label{\detokenize{legacy:id1}}Todo:}
\sphinxAtStartPar
Figure out why that asterisk is there.
\end{sphinxadmonition}
\begin{enumerate}
\sphinxsetlistlabels{\arabic}{enumi}{enumii}{}{.}%
\item {} 
\sphinxAtStartPar
Main DB: Delete (rm) the following files (from Rhost/Server/game/data):

\begin{sphinxVerbatim}[commandchars=\\\{\}]
\PYG{n}{netrhost}\PYG{o}{.}\PYG{n}{gdbm}\PYG{o}{*}
\PYG{n}{netrhost}\PYG{o}{.}\PYG{n}{db}
\PYG{n}{netrhost}\PYG{o}{.}\PYG{n}{db}\PYG{o}{.}\PYG{n}{new}
\PYG{n}{netrhost}\PYG{o}{.}\PYG{n}{db}\PYG{o}{.}\PYG{n}{new}\PYG{o}{.}\PYG{n}{prev}
\end{sphinxVerbatim}

\item {} 
\sphinxAtStartPar
Mail DB: Delete (rm) the following files (from Rhost/Server/game/data):

\begin{sphinxVerbatim}[commandchars=\\\{\}]
\PYG{n}{RhostMUSH}\PYG{o}{.}\PYG{n}{folder}\PYG{o}{.}\PYG{n}{dir}
\PYG{n}{RhostMUSH}\PYG{o}{.}\PYG{n}{folder}\PYG{o}{.}\PYG{n}{pag}
\PYG{n}{RhostMUSH}\PYG{o}{.}\PYG{n}{mail}\PYG{o}{.}\PYG{n}{dir}
\PYG{n}{RhostMUSH}\PYG{o}{.}\PYG{n}{mail}\PYG{o}{.}\PYG{n}{pag}
\end{sphinxVerbatim}

\item {} 
\sphinxAtStartPar
(Optional) AutoReg DB: Delete (rm) the following files (from Rhost/Server/game/data):

\begin{sphinxVerbatim}[commandchars=\\\{\}]
\PYG{n}{RhostMUSH}\PYG{o}{.}\PYG{n}{areg}\PYG{o}{.}\PYG{n}{dir}
\PYG{n}{RhostMUSH}\PYG{o}{.}\PYG{n}{areg}\PYG{o}{.}\PYG{n}{pag}
\end{sphinxVerbatim}

\item {} 
\sphinxAtStartPar
(Optional) News/BBS DB: Delete (rm) the following files (from Rhost/Server/game/data):

\begin{sphinxVerbatim}[commandchars=\\\{\}]
\PYG{n}{RhostMUSH}\PYG{o}{.}\PYG{n}{news}\PYG{o}{.}\PYG{n}{dir}
\PYG{n}{RhostMUSH}\PYG{o}{.}\PYG{n}{news}\PYG{o}{.}\PYG{n}{pag}
\end{sphinxVerbatim}

\end{enumerate}


\subsubsection{From the Server/game directory}
\label{\detokenize{legacy:from-the-server-game-directory}}\begin{enumerate}
\sphinxsetlistlabels{\arabic}{enumi}{enumii}{}{.}%
\item {} 
\sphinxAtStartPar
Load the database:

\begin{sphinxVerbatim}[commandchars=\\\{\}]
\PYG{o}{.}\PYG{o}{/}\PYG{n}{db\PYGZus{}load} \PYG{n}{data}\PYG{o}{/}\PYG{n}{netrhost}\PYG{o}{.}\PYG{n}{gdbm} \PYG{n}{data}\PYG{o}{/}\PYG{n}{netrhost}\PYG{o}{.}\PYG{n}{db}\PYG{o}{.}\PYG{n}{flat} \PYG{n}{data}\PYG{o}{/}\PYG{n}{netrhost}\PYG{o}{.}\PYG{n}{db}\PYG{o}{.}\PYG{n}{new}
\end{sphinxVerbatim}

\item {} 
\sphinxAtStartPar
Start the MUSH back:

\begin{sphinxVerbatim}[commandchars=\\\{\}]
\PYG{o}{.}\PYG{o}{/}\PYG{n}{Startmush}
\end{sphinxVerbatim}

\end{enumerate}


\subsubsection{While logged into the mush issue the following commands}
\label{\detokenize{legacy:while-logged-into-the-mush-issue-the-following-commands}}\begin{enumerate}
\sphinxsetlistlabels{\arabic}{enumi}{enumii}{}{.}%
\item {} 
\sphinxAtStartPar
Load in the mail database:

\begin{sphinxVerbatim}[commandchars=\\\{\}]
\PYG{n}{wmail}\PYG{o}{/}\PYG{n}{load}
\end{sphinxVerbatim}

\item {} 
\sphinxAtStartPar
(optional) Load in the autoreg database:

\begin{sphinxVerbatim}[commandchars=\\\{\}]
\PYG{n+nd}{@areg}\PYG{o}{/}\PYG{n}{load}
\end{sphinxVerbatim}

\item {} 
\sphinxAtStartPar
(optional) Load in the news/bbs database:

\begin{sphinxVerbatim}[commandchars=\\\{\}]
\PYG{n}{newsdb}\PYG{o}{/}\PYG{n}{load}
\end{sphinxVerbatim}

\end{enumerate}


\subsubsection{Verify that you have QDBM running and your valid values}
\label{\detokenize{legacy:verify-that-you-have-qdbm-running-and-your-valid-values}}\begin{enumerate}
\sphinxsetlistlabels{\arabic}{enumi}{enumii}{}{.}%
\item {} 
\sphinxAtStartPar
@list options system\#.  @list options (spammy)

\end{enumerate}


\section{Updating RhostMUSH prior to 3.9.5p2}
\label{\detokenize{legacy:updating-rhostmush-prior-to-3-9-5p2}}
\sphinxAtStartPar
Ok.

\sphinxAtStartPar
So you\textquotesingle{}re running an old RhostMUSH.

\sphinxAtStartPar
One prior to 3.9.5p2 and want to take advantage of the new
format of the Makefile and the automated mysql stuff and
all the other goodies that isn\textquotesingle{}t really (easilly) done
with just patch.sh.

\sphinxAtStartPar
Well, you\textquotesingle{}re in luck.  It is actually fairly easy to do.

\sphinxAtStartPar
This is what you have to do.

\sphinxAtStartPar
First thing\textquotesingle{}s first.
\begin{enumerate}
\sphinxsetlistlabels{\arabic}{enumi}{enumii}{}{.}%
\item {} 
\sphinxAtStartPar
Log into your existing mush.  Let\textquotesingle{}s make current backups
of all your flatfiles.  Issue:

\begin{sphinxVerbatim}[commandchars=\\\{\}]
\PYG{n+nd}{@dump}\PYG{o}{/}\PYG{n}{flat}
\PYG{n}{wmail}\PYG{o}{/}\PYG{n}{unload}
\PYG{n+nd}{@areg}\PYG{o}{/}\PYG{n}{unload}
\PYG{n}{newsdb}\PYG{o}{/}\PYG{n}{unload}
\end{sphinxVerbatim}

\item {} 
\sphinxAtStartPar
Shutdown your game:

\begin{sphinxVerbatim}[commandchars=\\\{\}]
\PYG{n+nd}{@shutdown}
\end{sphinxVerbatim}

\item {} 
\sphinxAtStartPar
Make an image of all your current backed up files.  From The Server/game directory you would type:

\begin{sphinxVerbatim}[commandchars=\\\{\}]
\PYG{o}{.}\PYG{o}{/}\PYG{n}{backup\PYGZus{}flat}\PYG{o}{.}\PYG{n}{sh} \PYG{o}{\PYGZhy{}}\PYG{n}{s}
\end{sphinxVerbatim}

\end{enumerate}

\begin{sphinxadmonition}{note}{Note:}
\sphinxAtStartPar
Please remember the \textquotesingle{}\sphinxhyphen{}s\textquotesingle{} to the ./backup\_flat.sh.
\end{sphinxadmonition}
\begin{enumerate}
\sphinxsetlistlabels{\arabic}{enumi}{enumii}{}{.}%
\item {} 
\sphinxAtStartPar
Make note of the most recently created file in the directory Server/game/oldflat.  It\textquotesingle{}s usually named something like:

\begin{sphinxVerbatim}[commandchars=\\\{\}]
\PYG{n}{RhostMUSH}\PYG{o}{.}\PYG{n}{dbflat1}\PYG{o}{.}\PYG{n}{tar}\PYG{o}{.}\PYG{n}{gz}
\end{sphinxVerbatim}

\end{enumerate}

\begin{sphinxadmonition}{note}{Note:}
\sphinxAtStartPar
You will need this file later.
\end{sphinxadmonition}
\begin{enumerate}
\sphinxsetlistlabels{\arabic}{enumi}{enumii}{}{.}%
\item {} 
\sphinxAtStartPar
Rename your \textquotesingle{}Rhost\textquotesingle{} directory to something else.  This is the directory that you have containing the \textquotesingle{}Server\textquotesingle{} directory.  Name it anything you want other than \textquotesingle{}Rhost\textquotesingle{}.  For those not used to unix you would type:

\begin{sphinxVerbatim}[commandchars=\\\{\}]
\PYG{n}{mv} \PYG{n}{Rhost} \PYG{n}{Rhost\PYGZus{}old}
\end{sphinxVerbatim}

\item {} 
\sphinxAtStartPar
Pull in the latest Rhost.  You would type:

\begin{sphinxVerbatim}[commandchars=\\\{\}]
\PYG{n}{git} \PYG{n}{clone} \PYG{n}{https}\PYG{p}{:}\PYG{o}{/}\PYG{o}{/}\PYG{n}{github}\PYG{o}{.}\PYG{n}{com}\PYG{o}{/}\PYG{n}{RhostMUSH}\PYG{o}{/}\PYG{n}{trunk} \PYG{n}{Rhost}
\end{sphinxVerbatim}

\end{enumerate}

\begin{sphinxadmonition}{note}{Note:}
\sphinxAtStartPar
You would type this in the same directory you have renamed your old \textquotesingle{}Rhost\textquotesingle{} directory
\end{sphinxadmonition}
\begin{enumerate}
\sphinxsetlistlabels{\arabic}{enumi}{enumii}{}{.}%
\item {} 
\sphinxAtStartPar
go into the Rhost/Server directory.   Type:

\begin{sphinxVerbatim}[commandchars=\\\{\}]
\PYG{n}{make} \PYG{n}{confsource}
\end{sphinxVerbatim}

\end{enumerate}

\begin{sphinxadmonition}{note}{Note:}
\sphinxAtStartPar
Select what options you want (including the mysql and other goodies) then compile it (also within the menu, it\textquotesingle{}s the \textquotesingle{}r\textquotesingle{} option).
\end{sphinxadmonition}
\begin{enumerate}
\sphinxsetlistlabels{\arabic}{enumi}{enumii}{}{.}%
\item {} 
\sphinxAtStartPar
Once your game is compiled and ready to go you need to copy in the data from your old game.  Copy the RhostMUSH.dbflat1.tar.gz we mentioned in step \#4 to the Server/game directory of your NEW GAME DIRECTORY.  From within the \textquotesingle{}game\textquotesingle{} directory of your current game you should be able to issue (if you named the old one Rhost\_old). Again this needs to be done FROM YOUR Server/game directory!!!

\end{enumerate}
\begin{quote}
\begin{enumerate}
\sphinxsetlistlabels{\arabic}{enumi}{enumii}{}{.}%
\item {} 
\sphinxAtStartPar
cp netrhost.conf netrhost.conf.orig

\item {} 
\sphinxAtStartPar
cp ../../Rhost\_old/Server/game/RhostMUSH.dbflat1.tar.gz .

\item {} 
\sphinxAtStartPar
tar \sphinxhyphen{}zxvf RhostMUSH.dbflat1.tar.gz

\item {} 
\sphinxAtStartPar
Compare your current netrhost.conf to the default one that came with the source (that you renamed to netrhost.conf.orig).  Likely the only sections you have to add to your current netrhost.conf (that came with your RhostMUSH.dbflat1.tar.gz archive), will be toward the end, with the include rhost\_ingame.conf and rhost\_mysql.conf.  Depending on how old your game is coming from you may need to add more options.  Any config option that is the same between the netrhost.conf files do not have to be copied over, and you want to keep your custom settings (like don\textquotesingle{}t port or other stuff you have already customized).

\item {} 
\sphinxAtStartPar
Load in your flatfile information:

\begin{sphinxVerbatim}[commandchars=\\\{\}]
\PYG{o}{.}\PYG{o}{/}\PYG{n}{db\PYGZus{}load} \PYG{n}{data}\PYG{o}{/}\PYG{n}{netrhost}\PYG{o}{.}\PYG{n}{gdbm} \PYG{n}{data}\PYG{o}{/}\PYG{n}{netrhost}\PYG{o}{.}\PYG{n}{db}\PYG{o}{.}\PYG{n}{flat} \PYG{n}{data}\PYG{o}{/}\PYG{n}{netrhost}\PYG{o}{.}\PYG{n}{db}\PYG{o}{.}\PYG{n}{new}
\end{sphinxVerbatim}

\item {} 
\sphinxAtStartPar
Your ./Startmush should re\sphinxhyphen{}index all your txt files you originally made:

\begin{sphinxVerbatim}[commandchars=\\\{\}]
\PYG{o}{.}\PYG{o}{/}\PYG{n}{Startmush}
\end{sphinxVerbatim}

\item {} 
\sphinxAtStartPar
In your game type the following as an immortal or as \#1.

\end{enumerate}
\begin{enumerate}
\sphinxsetlistlabels{\arabic}{enumi}{enumii}{}{.}%
\item {} 
\sphinxAtStartPar
Load in your mail flatfile:

\begin{sphinxVerbatim}[commandchars=\\\{\}]
\PYG{n}{wmail}\PYG{o}{/}\PYG{n}{load}
\end{sphinxVerbatim}

\item {} 
\sphinxAtStartPar
Load in your autoregistration flatfile (if available):

\begin{sphinxVerbatim}[commandchars=\\\{\}]
\PYG{n+nd}{@areg}\PYG{o}{/}\PYG{n}{load}
\end{sphinxVerbatim}

\item {} 
\sphinxAtStartPar
Load in your hardcoded bbs flatfile (if available):

\begin{sphinxVerbatim}[commandchars=\\\{\}]
\PYG{n}{newsdb}\PYG{o}{/}\PYG{n}{load}
\end{sphinxVerbatim}

\end{enumerate}
\end{quote}
\begin{enumerate}
\sphinxsetlistlabels{\arabic}{enumi}{enumii}{}{.}%
\item {} 
\sphinxAtStartPar
You should be good to go on a current directory structure for Rhost.  Enjoy!

\end{enumerate}


\section{Adding MySQL to RhostMUSH older than 3.9.5p2}
\label{\detokenize{legacy:adding-mysql-to-rhostmush-older-than-3-9-5p2}}
\sphinxAtStartPar
MySQL is now native in RhostMUSH as of 3.9.5p2.

\begin{sphinxadmonition}{warning}{Warning:}
\sphinxAtStartPar
To autodetect it, YOU MUST HAVE mysql\_config installed and running on your server.  Without this, even if you have mysql db installed it won\textquotesingle{}t be able to recognize the parameters you will need for it and will thus fail.  Please check your linux distribution to see what packages are needed to install mysql\_config.
\end{sphinxadmonition}

\sphinxAtStartPar
Download the git repository to a seperate directory so that you can
copy over the files that it requires you to.

\sphinxAtStartPar
Suggestion:

\begin{sphinxVerbatim}[commandchars=\\\{\}]
\PYG{n}{git} \PYG{n}{clone} \PYG{n}{https}\PYG{p}{:}\PYG{o}{/}\PYG{o}{/}\PYG{n}{github}\PYG{o}{.}\PYG{n}{com}\PYG{o}{/}\PYG{n}{RhostMUSH}\PYG{o}{/}\PYG{n}{trunk} \PYG{o}{\PYGZti{}}\PYG{o}{/}\PYG{n}{tmprho}
\end{sphinxVerbatim}

\sphinxAtStartPar
If you are patching UP from an older version, you need to update
the following files:
\begin{enumerate}
\sphinxsetlistlabels{\arabic}{enumi}{enumii}{}{.}%
\item {} 
\sphinxAtStartPar
update your src/Makefile to the one in the 3.9.5p2+ repo
( cp \textasciitilde{}/tmprho/Server/src/Makefile \textasciitilde{}/Rhost/Server/src/Makefile )

\item {} 
\sphinxAtStartPar
update your bin/asksource.* files to the one in the 3.9.5p2+ repo
( cp \textasciitilde{}/tmprho/Server/bin/asksource.* \textasciitilde{}/Rhost/Server/bin/ )

\item {} 
\sphinxAtStartPar
append \textquotesingle{}include rhost\_mysql.conf\textquotesingle{} BEFORE the rhost\_ingame.conf file
and before the section that says \textquotesingle{}define local aliases\textquotesingle{} toward the end of
your netrhost.conf file.
( edit your \textasciitilde{}/Rhost/Server/game/netrhost.conf file )

\item {} 
\sphinxAtStartPar
copy the game/rhost\_mysql.conf file from the 3.9.5p2+ repo
( cp \textasciitilde{}/tmprho/Server/game/rhost\_mysql.conf \textasciitilde{}/Rhost/Server/game/ )

\item {} 
\sphinxAtStartPar
The following lines have to be REPLACED/CHANGED in local.c ( toward the top ):
( you may edit this or copy the one from the other distro )
( do either:  edit \textasciitilde{}/Rhost/Server/src/local.c )
(        or:  cp \textasciitilde{}/tmprho/Server/src/local.c \textasciitilde{}/Rhost/Server/src/local.c )

\end{enumerate}

\begin{sphinxadmonition}{note}{Note:}
\sphinxAtStartPar
IF REPLACING/CHANGING local.c COPY BELOW
\end{sphinxadmonition}

\begin{sphinxVerbatim}[commandchars=\\\{\}]
\PYG{c+cm}{/* Called when the mush starts up, immediatly prior to the main game}
\PYG{c+cm}{* loop being entered. By this point all databases are loaded and}
\PYG{c+cm}{* all variables configured.}
\PYG{c+cm}{*/}
\PYG{c+cp}{\PYGZsh{}}\PYG{c+cp}{ifdef MYSQL\PYGZus{}VERSION}
 \PYG{k}{extern} \PYG{k+kt}{void} \PYG{n}{local\PYGZus{}mysql\PYGZus{}init}\PYG{p}{(}\PYG{k+kt}{void}\PYG{p}{)}\PYG{p}{;}
 \PYG{k}{extern} \PYG{k+kt}{int} \PYG{n}{sql\PYGZus{}shutdown}\PYG{p}{(}\PYG{n}{dbref} \PYG{n}{player}\PYG{p}{)}\PYG{p}{;}
\PYG{c+cp}{\PYGZsh{}}\PYG{c+cp}{endif}

\PYG{c+cp}{\PYGZsh{}}\PYG{c+cp}{ifdef SQLITE}
 \PYG{k}{extern} \PYG{k+kt}{void} \PYG{n}{local\PYGZus{}sqlite\PYGZus{}init}\PYG{p}{(}\PYG{k+kt}{void}\PYG{p}{)}\PYG{p}{;}
\PYG{c+cp}{\PYGZsh{}}\PYG{c+cp}{endif }\PYG{c+cm}{/* SQLITE */}

\PYG{k+kt}{void} \PYG{n+nf}{local\PYGZus{}startup}\PYG{p}{(}\PYG{k+kt}{void}\PYG{p}{)} \PYG{p}{\PYGZob{}}
\PYG{c+cp}{\PYGZsh{}}\PYG{c+cp}{ifdef SQLITE}
 \PYG{n}{local\PYGZus{}sqlite\PYGZus{}init}\PYG{p}{(}\PYG{p}{)}\PYG{p}{;}
\PYG{c+cp}{\PYGZsh{}}\PYG{c+cp}{endif }\PYG{c+cm}{/* SQLITE */}
\PYG{c+cp}{\PYGZsh{}}\PYG{c+cp}{ifdef MYSQL\PYGZus{}VERSION}
 \PYG{n}{local\PYGZus{}mysql\PYGZus{}init}\PYG{p}{(}\PYG{p}{)}\PYG{p}{;}
\PYG{c+cp}{\PYGZsh{}}\PYG{c+cp}{endif}
 \PYG{n}{load\PYGZus{}regexp\PYGZus{}functions}\PYG{p}{(}\PYG{p}{)}\PYG{p}{;}
\PYG{p}{\PYGZcb{}}

\PYG{c+cm}{/* Called immediatly after the main game loop exits. At this point}
\PYG{c+cm}{* all databases and variables are still configured}
\PYG{c+cm}{*/}
\PYG{k+kt}{void} \PYG{n+nf}{local\PYGZus{}shutdown}\PYG{p}{(}\PYG{k+kt}{void}\PYG{p}{)} \PYG{p}{\PYGZob{}}
\PYG{c+cp}{\PYGZsh{}}\PYG{c+cp}{ifdef MYSQL\PYGZus{}VERSION}
 \PYG{n}{sql\PYGZus{}shutdown}\PYG{p}{(}\PYG{l+m+mi}{\PYGZhy{}1}\PYG{p}{)}\PYG{p}{;}
\PYG{c+cp}{\PYGZsh{}}\PYG{c+cp}{endif}
\PYG{p}{\PYGZcb{}}
\end{sphinxVerbatim}
\begin{enumerate}
\sphinxsetlistlabels{\arabic}{enumi}{enumii}{}{.}%
\item {} 
\sphinxAtStartPar
Issue \textquotesingle{}make clean\textquotesingle{} then make confsource to rebuild using the latest
builder script to build in the mysql changes.

\end{enumerate}
\phantomsection\label{\detokenize{comparison/flags:comparison-flags}}
\index{Comparison \sphinxhyphen{} Flags@\spxentry{Comparison \sphinxhyphen{} Flags}}\index{Flags \sphinxhyphen{} Comparison@\spxentry{Flags \sphinxhyphen{} Comparison}}\ignorespaces 

\chapter{Comparison of Flags}
\label{\detokenize{comparison/flags:comparison-of-flags}}\label{\detokenize{comparison/flags:index-0}}\label{\detokenize{comparison/flags::doc}}

\begin{savenotes}\sphinxatlongtablestart\begin{longtable}[c]{|l|l|l|}
\sphinxthelongtablecaptionisattop
\caption{Comparison of Flags\strut}\label{\detokenize{comparison/flags:id1}}\\*[\sphinxlongtablecapskipadjust]
\hline
\sphinxstyletheadfamily 
\sphinxAtStartPar
PENN/MUX Flag
&\sphinxstyletheadfamily 
\sphinxAtStartPar
RhostMUSH Flag
&\sphinxstyletheadfamily 
\sphinxAtStartPar
Commentary on Difference Between Flags
\\
\hline
\endfirsthead

\multicolumn{3}{c}%
{\makebox[0pt]{\sphinxtablecontinued{\tablename\ \thetable{} \textendash{} continued from previous page}}}\\
\hline
\sphinxstyletheadfamily 
\sphinxAtStartPar
PENN/MUX Flag
&\sphinxstyletheadfamily 
\sphinxAtStartPar
RhostMUSH Flag
&\sphinxstyletheadfamily 
\sphinxAtStartPar
Commentary on Difference Between Flags
\\
\hline
\endhead

\hline
\multicolumn{3}{r}{\makebox[0pt][r]{\sphinxtablecontinued{continues on next page}}}\\
\endfoot

\endlastfoot

\sphinxAtStartPar
ABODE
&
\sphinxAtStartPar
ABODE
&\\
\hline
\sphinxAtStartPar
BLIND
&
\sphinxAtStartPar
BLIND
&\\
\hline
\sphinxAtStartPar
CHOWN\_OK
&
\sphinxAtStartPar
CHOWN\_OK
&\\
\hline
\sphinxAtStartPar
DARK
&
\sphinxAtStartPar
DARK
&\\
\hline
\sphinxAtStartPar
FREE
&
\sphinxAtStartPar
FREE
&\\
\hline
\sphinxAtStartPar
GOING
&
\sphinxAtStartPar
GOING/BYEROOM
&\\
\hline
\sphinxAtStartPar
HAVEN
&
\sphinxAtStartPar
HAVEN
&\\
\hline
\sphinxAtStartPar
INHERIT
&
\sphinxAtStartPar
INHERIT
&\\
\hline
\sphinxAtStartPar
JUMP\_OK
&
\sphinxAtStartPar
JUMP\_OK
&\\
\hline
\sphinxAtStartPar
KEY
&
\sphinxAtStartPar
KEY
&\\
\hline
\sphinxAtStartPar
LINK\_OK
&
\sphinxAtStartPar
LINK\_OK
&\\
\hline
\sphinxAtStartPar
MONITOR
&
\sphinxAtStartPar
MONITOR
&\\
\hline
\sphinxAtStartPar
NOSPOOF
&
\sphinxAtStartPar
NOSPOOF
&\\
\hline
\sphinxAtStartPar
OPAQUE
&
\sphinxAtStartPar
OPAQUE
&\\
\hline
\sphinxAtStartPar
QUIET
&
\sphinxAtStartPar
QUIET
&\\
\hline
\sphinxAtStartPar
STICKY
&
\sphinxAtStartPar
STICKY
&\\
\hline
\sphinxAtStartPar
TRACE
&
\sphinxAtStartPar
TRACE
&\\
\hline
\sphinxAtStartPar
UNFINDABLE
&
\sphinxAtStartPar
UNFINDABLE
&\\
\hline
\sphinxAtStartPar
VISUAL
&
\sphinxAtStartPar
VISUAL
&\\
\hline
\sphinxAtStartPar
WIZARD
&
\sphinxAtStartPar
ROYALTY
&\\
\hline
\sphinxAtStartPar
ANSI
&
\sphinxAtStartPar
ANSI/ANSICOLOR
&\\
\hline
\sphinxAtStartPar
PARENT\_OK
&
\sphinxAtStartPar
PARENT\_OK
&\\
\hline
\sphinxAtStartPar
ROYALTY
&
\sphinxAtStartPar
COUNCILOR/ARCHITECT
&\\
\hline
\sphinxAtStartPar
AUDIBLE
&
\sphinxAtStartPar
AUDIBLE
&\\
\hline
\sphinxAtStartPar
BOUNCE
&
\sphinxAtStartPar
BOUNCE
&\\
\hline
\sphinxAtStartPar
CONNECTED
&
\sphinxAtStartPar
CONNECTED
&\\
\hline
\sphinxAtStartPar
DESTROY\_OK
&
\sphinxAtStartPar
DESTROY\_OK
&\\
\hline
\sphinxAtStartPar
ENTER\_OK
&
\sphinxAtStartPar
ENTER\_OK
&\\
\hline
\sphinxAtStartPar
FIXED
&
\sphinxAtStartPar
NO\_TEL
&\\
\hline
\sphinxAtStartPar
UNINSPECTED
&
\sphinxAtStartPar
Not Available
&
\sphinxAtStartPar
Just a marker flag
\\
\hline
\sphinxAtStartPar
HALTED
&
\sphinxAtStartPar
HALTED
&\\
\hline
\sphinxAtStartPar
IMMORTAL
&
\sphinxAtStartPar
GUILDMASTER
&
\sphinxAtStartPar
You don\textquotesingle{}t want IMMORTAL
\\
\hline
\sphinxAtStartPar
GAGGED
&
\sphinxAtStartPar
FUBAR
&\\
\hline
\sphinxAtStartPar
CONSTANT
&
\sphinxAtStartPar
NO\_MODIFY
&\\
\hline
\sphinxAtStartPar
LIGHT
&
\sphinxAtStartPar
LIGHT
&\\
\hline
\sphinxAtStartPar
MYOPIC
&
\sphinxAtStartPar
MYOPIC
&\\
\hline
\sphinxAtStartPar
AUDITORIUM
&
\sphinxAtStartPar
AUDITORIUM
&\\
\hline
\sphinxAtStartPar
ZONE
&
\sphinxAtStartPar
Use @zone
&\\
\hline
\sphinxAtStartPar
PUPPET
&
\sphinxAtStartPar
PUPPET
&\\
\hline
\sphinxAtStartPar
TERSE
&
\sphinxAtStartPar
TERSE
&\\
\hline
\sphinxAtStartPar
ROBOT
&
\sphinxAtStartPar
ROBOT
&\\
\hline
\sphinxAtStartPar
SAFE
&
\sphinxAtStartPar
SAFE
&\\
\hline
\sphinxAtStartPar
TRANSPARENT
&
\sphinxAtStartPar
TRANSPARENT
&\\
\hline
\sphinxAtStartPar
SUSPECT
&
\sphinxAtStartPar
SUSPECT
&\\
\hline
\sphinxAtStartPar
VERBOSE
&
\sphinxAtStartPar
VERBOSE
&\\
\hline
\sphinxAtStartPar
STAFF
&
\sphinxAtStartPar
Not Available
&
\sphinxAtStartPar
Just a marker flag
\\
\hline
\sphinxAtStartPar
SLAVE
&
\sphinxAtStartPar
SLAVE
&\\
\hline
\sphinxAtStartPar
ORPHAN
&
\sphinxAtStartPar
Not Available
&
\sphinxAtStartPar
@lock/use the parent instead
\\
\hline
\sphinxAtStartPar
CONTROL\_OK
&
\sphinxAtStartPar
Not Available
&
\sphinxAtStartPar
Use @lock/ZoneWizLock
\\
\hline
\sphinxAtStartPar
STOP
&
\sphinxAtStartPar
STOP
&
\sphinxAtStartPar
See also NOSTOP
\\
\hline
\sphinxAtStartPar
COMMANDS
&
\sphinxAtStartPar
COMMANDS
&\\
\hline
\sphinxAtStartPar
PRESENCE
&
\sphinxAtStartPar
Not Available
&
\sphinxAtStartPar
See: Reality Levels
\\
\hline
\sphinxAtStartPar
NOBLEED
&
\sphinxAtStartPar
Not Needed
&
\sphinxAtStartPar
Rhost doesn\textquotesingle{}t bleed ANSI
\\
\hline
\sphinxAtStartPar
VACATION
&
\sphinxAtStartPar
Not Available
&
\sphinxAtStartPar
Just a marker flag
\\
\hline
\sphinxAtStartPar
HEAD
&
\sphinxAtStartPar
Not Available
&
\sphinxAtStartPar
Just a marker flag
\\
\hline
\sphinxAtStartPar
WATCHER
&
\sphinxAtStartPar
Not Available
&
\sphinxAtStartPar
@toggle MONITOR
\\
\hline
\sphinxAtStartPar
HTML
&
\sphinxAtStartPar
Not Available
&
\sphinxAtStartPar
Rhost doesn\textquotesingle{}t support Pueblo
\\
\hline
\sphinxAtStartPar
REDIR\_OK
&
\sphinxAtStartPar
Not Available
&
\sphinxAtStartPar
Rhost doesn\textquotesingle{}t support @redirect
\\
\hline
\sphinxAtStartPar
SPEECHMOD
&
\sphinxAtStartPar
Not Available
&
\sphinxAtStartPar
Rhost doesn\textquotesingle{}t support @speechmod \sphinxhyphen{} use @icmd
\\
\hline
\sphinxAtStartPar
MARKER0\sphinxhyphen{}MARKER9
&
\sphinxAtStartPar
MARKER0\sphinxhyphen{}MARKER9
&\\
\hline
\end{longtable}\sphinxatlongtableend\end{savenotes}
\phantomsection\label{\detokenize{comparison/powers:comparison-powers}}
\index{Comparison \sphinxhyphen{} Powers@\spxentry{Comparison \sphinxhyphen{} Powers}}\index{Powers \sphinxhyphen{} Comparison@\spxentry{Powers \sphinxhyphen{} Comparison}}\ignorespaces 

\chapter{Comparison of powers}
\label{\detokenize{comparison/powers:comparison-of-powers}}\label{\detokenize{comparison/powers:index-0}}\label{\detokenize{comparison/powers::doc}}
\begin{sphinxVerbatim}[commandchars=\\\{\}]
announce              Can use the @wall command.
Rhost Equiv: \PYGZhy{} FREE\PYGZus{}WALL (@power)

boot                  Can use the @boot command.
Rhost Equiv: BOOT (@power)

builder               Can build, if the builder power is enabled.
Rhost Equiv: ARCHITECT (flag)

chown\PYGZus{}anything        Can @chown anything to anyone.
Rhost Equiv: CHOWN\PYGZus{}OTHER (@power)

comm\PYGZus{}all              Like a wizard with respect to channels.
Rhost has no hardcoded comsystem.  You can tweek the softcode.

control\PYGZus{}all           Can modify any object in the database. (God\PYGZhy{}set only.)
Rhost Equiv: TwinkLock (@lock)

expanded\PYGZus{}who          Sees the wizard WHO, and SESSION commands.
Rhost Equiv: WIZ\PYGZus{}WHO (@power)

find\PYGZus{}unfindable       Can locate unfindable people.
see\PYGZus{}hidden            Can see hidden (DARK) players on WHO, etc.
Rhost Equiv: WHO\PYGZus{}UNFIND (@power)

free\PYGZus{}money            Unlimited money.
Rhost Equiv: FREE (flag)

free\PYGZus{}quota            Unlimited quota.
Rhost Equiv: FREE\PYGZus{}QUOTA (@power)

guest                 Is this a guest character?
Rhost Equiv: GUEST (flag)

halt                  Can @halt anything, and @halt/all.
Rhost Equiv: HALT\PYGZus{}QUEUE (@power) or HALT\PYGZus{}QUEUE\PYGZus{}ALL (@power)

hide                  Can set themselves DARK.
Rhost Equiv: NOWHO (@power)

idle                  No idle timeout.
Rhost Equiv: @timeout \PYGZbs{}*player=\PYGZhy{}1

link\PYGZus{}variable         Can @link an exit to \PYGZdq{}variable\PYGZdq{}.
Rhost Equiv: Anyone can do this.  VARIABLE (@toggle)

link\PYGZus{}to\PYGZus{}anything      Can @link an exit to any (non\PYGZhy{}variable) destination.
Rhost Equiv: @lock/link (@lock)

long\PYGZus{}fingers          Can get, look, whisper, etc from a distance.
Rhost Equiv: LONG\PYGZus{}FINGERS (@power)

no\PYGZus{}destroy            Cannot be @toad\PYGZsq{}ed.
Rhost Equiv: INDESTRUCTABLE (flag)

open\PYGZus{}anywhere         Can @open an exit from any location.
Rhost Equiv: @lock/open (@lock)

poll                  Can set the @poll.
Rhost has nothing equivelant.  Just softcode a +poll, or @hook it for permissions.

prog                  Can use @program on players other than themself.
Rhost Equiv: PROG (@toggle)

search                Can @search anyone.
Rhost Equiv: SEARCH\PYGZus{}ANY (@power)

see\PYGZus{}all               Can examine and see attributes like a wizard.
Rhost Equiv: EXAMINE\PYGZus{}FULL (@power) (and EXFULLWIZATTR (@toggle) for wiz only attribs)

see\PYGZus{}queue             Can @ps/all or @ps any player.
Rhost Equiv: SEE\PYGZus{}QUEUE (@power) or SEE\PYGZus{}QUEUE\PYGZus{}ALL (@power)

stat\PYGZus{}any              Can @stat any player.
Rhost Equiv: STAT\PYGZus{}ANY (@power)

steal\PYGZus{}money           Can give negative money.
Rhost Equiv: STEAL (@power)

tel\PYGZus{}anywhere          Can teleport anywhere.
Rhost Equiv: TEL\PYGZus{}ANYWHERE (@power) or FULL\PYGZus{}TEL (@power)

tel\PYGZus{}anything          Can teleport anything (includes tel\PYGZus{}anywhere)
Rhost Equiv: TEL\PYGZus{}ANYTHING (@power)

unkillable            Cannot be killed with the \PYGZsq{}kill\PYGZsq{} command.
Rhost Equiv: NOKILL (@power)

use\PYGZus{}sql               Can call the SQL() function. (God\PYGZhy{}set only.)
Rhost Equiv: SQL is a 3rd party patch.

watch\PYGZus{}logins          Can set or reset the WATCHER flag on themselves.
Rhost Equiv: MONITOR (@toggle)
\end{sphinxVerbatim}
\phantomsection\label{\detokenize{historical/nda:historical-nda}}
\index{NDA@\spxentry{NDA}}\index{Non\sphinxhyphen{}Disclosure Agreement@\spxentry{Non\sphinxhyphen{}Disclosure Agreement}}\ignorespaces 

\chapter{Historical Non\sphinxhyphen{}Disclosure Agreement}
\label{\detokenize{historical/nda:historical-non-disclosure-agreement}}\label{\detokenize{historical/nda:index-0}}\label{\detokenize{historical/nda::doc}}
\sphinxAtStartPar
The following NDA comes from the time when RhostMUSH was not publially
available. It is preserved here for historical reasons. We are suckers for
looking back at things. :)
\begin{quote}
\begin{quote}

\sphinxAtStartPar
\sphinxhyphen{}\sphinxhyphen{}Ambrosia
\end{quote}
\begin{enumerate}
\sphinxsetlistlabels{\arabic}{enumi}{enumii}{}{.}%
\item {} 
\sphinxAtStartPar
I agree, to not give out the code, in part or in full, in any form of
medium, to anyone or anything not previously allowed by the developers.

\item {} 
\sphinxAtStartPar
I agree, to not let others look at the code, in part or in full, in
any form of medium, to anyone or anything not previously allowed by the
developers.

\item {} 
\sphinxAtStartPar
I am aware that any modifications I make to the code is \_FULLY\_
permitted, and that I do \_NOT\_ have to return said patches to the
developers.

\end{enumerate}
\end{quote}

\begin{sphinxadmonition}{note}{Note:}
\sphinxAtStartPar
The Rhost developers would like to see what was added, and possibly
look at adding them to the main distro if we see others would like
it (with full credits to you), but we belive once you have the code,
you should be allowed to play with it fully as long as the first two
rules are kept.
\end{sphinxadmonition}
\phantomsection\label{\detokenize{meta/copyright:copyright}}
\index{Copyright@\spxentry{Copyright}}\ignorespaces 

\chapter{Copyright}
\label{\detokenize{meta/copyright:index-0}}\label{\detokenize{meta/copyright:id1}}\label{\detokenize{meta/copyright::doc}}
\sphinxAtStartPar
Copyright© 1990, 1991, 1992, 1993, 1994, 1995, 1996, 1997, 1998, 1999,
2000, 2001, 2002, 2003, 2004, 2005, 2006, 2007, 2008, 2009, 2010, 2011, 2012,
2013, 2014, 2015, 2016, 2017, 2018, 2019, 2020, 2021

\sphinxAtStartPar
Seawolf, Thorin, Ashen\sphinxhyphen{}Shugar, Kale, Lensman, Morgan, Odin, Kage, Ambrosia, Rook

\sphinxAtStartPar
All rights, reserved.

\sphinxAtStartPar
The copyright includes but is not limited to all changes and modifications to
the code, the help files, and all information included in this code.
Copying of these changes is not permitted without prior approval.
Borrowing of ideas require notification of where idea originated.
Please use \textquotesingle{}RhostMUSH\textquotesingle{} when identifying source. Modification of code is allowed
as long as contact and acceptance is made prior to changes with one (or more) of
the original writers of the code in writing.

\sphinxAtStartPar
This copyright does not include the original code that was given as GNU freeware

\sphinxAtStartPar
This copyright information may not be changed, altered, or omitted.
\phantomsection\label{\detokenize{txtfiles/help:txtfiles-help}}
\index{help.txt@\spxentry{help.txt}}\index{txtfiles \sphinxhyphen{} help.txt@\spxentry{txtfiles \sphinxhyphen{} help.txt}}\ignorespaces 

\chapter{RhostMUSH Internal Help Files}
\label{\detokenize{txtfiles/help:rhostmush-internal-help-files}}\label{\detokenize{txtfiles/help:index-0}}\label{\detokenize{txtfiles/help::doc}}

\phantomsection\label{\detokenize{txtfiles/wizhelp:txtfiles-wizhelp}}
\index{wizhelp.txt@\spxentry{wizhelp.txt}}\index{txtfiles \sphinxhyphen{} wizhelp.txt@\spxentry{txtfiles \sphinxhyphen{} wizhelp.txt}}\ignorespaces 

\chapter{RhostMUSH Internal WizHelp Files}
\label{\detokenize{txtfiles/wizhelp:rhostmush-internal-wizhelp-files}}\label{\detokenize{txtfiles/wizhelp:index-0}}\label{\detokenize{txtfiles/wizhelp::doc}}




\renewcommand{\indexname}{Index}
\printindex
\end{document}